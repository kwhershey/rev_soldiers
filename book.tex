ABNEY, WILLIAM 
Buried: Probably Gallatin County, Illinois 
Spouse: Judith 
Residence: Listed in 1830 census, Gallatin County 
Service: Soldier; Virginia Continental. Served from Virginia; continued in the service of the United States after the close of the war. 
Pension: Judith W1202 Contl. Va. 
Sources: CR, PENSION, W, Virginia Records 
ADAMS, WILLIAM 
Born: Born in Virginia 
Buried: Martin Cemetery, Newell Township, Vermilion County, Illinois 
Residences: Moved to Kentucky following the war and in 1825 removed to Illinois where he settled in Newell Township, Vermilion County. 
Service: Virginia 
Marker: His name is inscribed on a granite tablet in floor of fountain in front of Post Office, Danville. The memorial consists of a granite shaft with a foun­tain, topped with a bronze wreath of laurel leaves, the date "1776" and a four foot bronze statue of a Minute Man. The marker was placed by Governor Bradford Chapter DAR, and dedicated on September 3, 1915. 
Sources: DAR, W, Virginia Records 
ALLEN, ARCHIBALD 
Born: 1749 in Virginia 
Buried: Putnam County, Illinois 
Residences: From Virginia he moved to Maryland, then to Kentucky, then to 
Indiana and from there to Putnam County, Illinois. 
Service: Soldier; Virginia. He served two months in 1781 with Capt. Charles Shelton and Col. Elias Edmonds, in the Virginia troops. 
Pension: RI02: Va. Residence Hennepin, Putnam County, Illinois, when pension claim was rejected (Act June 7,1832) "Not SLX months service." 
Sources: DAR, PENSION, W, Virginia Records 
ALLEN, DANIEL 
Died: After 1835 
Buried: Greene County, Illinois 
Service: Pvt., North Carolina Continental troops 
Pension: N.C.: S16601. Pension roll, Greene County, Ill., Oct. 23, 1833 age 71; also listed 1835. 
Sources: DAR, HR, PENSION
ALLEN, ELEAZER 
Born: 1755, Connecticut 
Died: 1828 
Buried: Shiloh Cemetery, Shiloh, St. Clair County, Illinois 
Service: Private; Connecticut. Enlisted May 1, 1775 for eight months with Capt. James Chapman; again Jan. 1, 1776 for one year under the same Captain and with Col. Samuel Par sons in what was known as "Parson's Continen­tals." He was in the battles of New York, King's Bridge, and White Plains. 
Pension: S35171: Conn. Illinois pension roll, St. Clair County, May 12, 1825 
Marker: Marker placed June, 1942 by SAR, Belleville Chapter DAR, and Ameri­can Legion. 
Sources: DAR, PENSION, W 
ALLEN, WILLIAM 
Born: Pennsylvania Died: After 1850 
Buried : Probably Gallatin County, Illinois 
Residences: Came to Illinois from Orange County, N.C. Listed in 1850 census for Gallatin County. 
Service: Sergeant: North Carolina. Enlisted N.C. serving in Cavalry and Infantry. Sergeant, 1781 under Lt. John Campbell and Robert Scoby, with Col. Archi­bald Lytle. Taken prisoner at Hillsburg; exchanged Aug. 11, 1782. 
Pension: S30822. Pension Roll, Gallatin County. April 9, 1833, age 73 
Sources: CR, PENSION, W 
ALLERE, BASELLE 
Buried: In or near Kaskaskia, Randolph County, Illinois 
Service: Patriot. A loyal French subject in Company raised by Capt. Francis Charleville under Col. George Rogers Clark for conquest of Vincennes. En­listed in January 1779 for 8 months. 
Marker: His name is on bronze marker placed by Fort Chartres Chapter DAR, June 24, 1934, on the grounds of Sparta High School. 
Sources: DAR, NSDAR, W 
ALLERE, JOSEPH 
Born: Virginia Buried : Randol ph County, Illinois 
Service: Patriot: Virginia. He served under Col. George Rogers Clark and is listed in the non-commissioned officers and soldiers of the Illinois Regiment and the Western Army entitled to land for war service. 
Marker: His name is on a bronze marker placed by Fort Chartres Chapter DAR, Sparta, on June 24, 1934, on the grounds of Sparta High School, Randolph County. 
Sources: DAR, NSDAR , W 
ANDERSON, JOSEPH 
Died: Nine-Mile Creek, about 5 miles from Kaskaskia 
Buried: Randolph County, Illinois 
Service: Soldier; Virginia. Solder under Col. George Rogers Clark 
Pension: Applied for pension 
Marker: His name is on a bronze marker placed by Fort Chartres Chapter DAR, Sparta, on June 24, 1934 at Sparta High School. 
Sources: DAR, NSDAR, W 
ANTERE, MICHAEL 
Buried: In or near Kaskaskia, Randolph County, Illinois 
Service: Frenchman in Company raised by Capt. Francis Charleville under Col. George Rogers Clark for conquest of Vincennes. Enlisted in January 1779 for 8 months. 
Pension: Applied for pension. 
Marker: His name is on bronze marker placed by Fort Chartres Chapter DAR, Sparta, on June 24, 1934, at Sparta High School. 
Sources: DAR, NSDAR, W 
ARCHER, ZACHARIAH 
Born: 1752 in Down County, Ireland 
Died: July 5, 1822 
Buried: Walnut Prairie Cemetery, West Union, Clark Co., IIlinob-, The grave is marked 
Spouse: Jane Kilgore 
Residences: He came to Illinois in 1819. 
Service: Private; Pennsylvania. He enlisted in 1776from Northumberland County, Pa., in Capt. William Peebles' Company, Rifle Regt., commanded by Col. Samuel Miles. He was transferred to the Pennsylvania state regiment with Capt. Matthew Scott and Col. Walter Stewart. He was in the battles of Long Island, Trenton and Brandywine, also in camp at Valley Forge. 
Pension: Jane W23462(Pa) 
Sources: HR, PI, PENSION, W 
ARMSTRONG, JOHN 
Buried: Wabash County, Illinois 
Residences: Residence during the war was North Carolina. He moved to Ken­tucky, and from there to Tennessee and in 1815 he came to Wabash County, Illinois, settling on land purchased from Levi Compton. 
Service: Soldier; North Carolina. 
Sources: W, North Carolina Records; County History ARMSTRONG, JOSHUA 
Born: August1,1756inPennsylvania 
Died: September 25, 1845 
Buried: Armstrong Cemetery, near Jerseyville, Jersey County,Illinois (also known as Richl and Cemetery) 
Spouse: Sarah Morris 
Service: Artificer: Pennsylvania. Private; North Carolina Continental troops. He was a soldier in Col. George Rogers Clark's expedition. 
Pension: Sarah W23461 (Pa). He was on Pension Roll, Greene County, January 31, 1834, age 79. Sarah, widow, on pension roll, Greene County, September 25, 1845. 
Marker: Placed by Ninian Edwards Chapter DAR on October 26, 1930 
Sources: DAR, HR, NSDAR, PI, PENSION 
ARMSTRONG, ROBERT 
Born: May 28, 1760 in North Carolina Died: September 9, 1834 
Buried: Old Concord Cemetery, Menard County, Illinois Spouse: Nancy Green 
Service: Private; NC:SC. He served in the North Carolina State Militia from Crowder's Creek, under Col. Little and Gen. Ashe. He was in the battle of Briar Creek, 1779. 
Pension: R259 (NC:SC). His widow, Nancy, applied for pension after 1836. Resi­dence, Petersburg, Menard County, when pension claim suspended (Act July 7,1838)"For further proof of service." 
Marker: Marker placed by Pierre Menard Chapter DAR, Petersburg, in April 1932. 
Sources: DAR, NSDAR, PI, PENSION 
ASHE, THOMAS 
Buried: Probably Marion County, Illinois 
Residence: Came to Marion County, Illinois before 1825 from North Carolina. 
Service: Served from North Carolina. Pension: He applied for Pension in 1833 in Marion County. 
Sources: W 
AUGUSTINE, JOSEPH T. 
Died: March 12, 18-­-
Buried: Braceville-Gardner Cemetery, old section, Braceville, Grundy County, Illinois 
Pension: Applied for pension. 
Sources: HR 

AUR, JOHN (Job) 
Buried: Marshall Cemetery, Marshall, Clark County, Illinois 
Service: Served in German Crown Prince Army 
Pension: Applied for pension 
Sources: HR 
AUSTIN, ELIJAH 
Buried: North Arm Cemetery, Edgar County, Illinois 
Service: Soldier; Massachusetts. He enlisted July IS, 1776 in Capt. John King's Company, Col. Mark Hopkins' Regt. and served 16 days in Berkshire County, Mass. 
Pension: He applied for a pension. 
Sources: W, Mass. Soldiers and Sailors in the Revolutionary War 
AUSTIN, JEREMIAH 
Born: December 30, 1760 in Charleston, Rhode Island 
Died: May 20, 1846 Plainfield, Will County, Illinois 
Spouse: Mrs. Esther Colgrove 
Service: Private; Rhode Island. He served Jan. I, 1777 to March 16, 1779. En­gaged in Quaker Hill Battle, Rhode Island. Served under Capt. Maxon, Capt. Benjamin Hoppin and Col. Thomas Potter, Col. Joseph Stanton and Col. John Topham. 
Pension: BLWT-67608-160-55. Esther W810S (RI) 
Sources: DAR, PI, PENSION 


BACKUS (Bacus), BENJAMIN 
Buried: Old Cemetery, Carmi, White County, Illinois. Government Headstone 
Service: Sergeant in Wylie's Company 
Marker: A granite monument in city park, inscribed with the names of Revolu­tionary soldiers buried in White County, was dedicated by Wabash Chapter DAR, Carmi, in October 1936. 
Sources: DAR 
BADGETT, THOMAS 
Buried: Pleasant Grove Cemetery, Mt , Vernon, Jefferson County, Illinois 
Service: Soldier; South Carolina. He also served in the WAR of 1812. 
Sources: DAR, HR 
BAILEY, ROBERT 
Born: About 1761 
Died: After 1840 
Buried: Vermilion County, Illinois 
Service: Private; NC:GA. He enlisted July 20, 1778 and served in the Tenth Regt ., 
North Carolina troops for 9 months; Pvt. in Lt. Col. Baker's Regt. 
Pension: S32108 GA:NC. Illinois pension roll, Vermilion County, August 14, 1833, age72. Illinois Pension Census, June 1, 1840, Vermilion County, age 85. 
Sources: HS, PENSION 
BAIN, DANIEL (David) 
Born: 1760 in Virginia 
Died: 1838 
Buried: Franklin County, Illinois 
Pension: Applied for pension 
Sources: DAR, NSDAR 
BAIRD, DAVID 
Born: March4,1760inNewJersey 
Died: February 20, 1837 
Buried: Lebanon Cemetery, Indianola, Vermilion County, Illinois 
Spouse: Elinor Barkelow 
Service: Quartermaster: New Jersey. He enlisted in Monmouth County in the 
First New Jersey Militia, September 1776. He re-enlisted, serving for different periods each year until the close of the war, serving under Capt. David Gordon. Capt. Kenneth Harrison, Capt. Coons, Capt. Samue Carhart, Capt. John Price, and Capt. Cornelius Schanch; Col. Asher Holmes, Col. Thomas Henderson, and Col. Cahart Walton. He served as Private, Sergeant, Ensign, Lieutenant. and Quartermaster. 
Pension: Pensioned Marker: His name is inscribed on granite tablet set in floor of drinking fountain in front of Post Office, Danville. The marker was placed by Governor Brad­ford Chapter DAR in 1915. 
Sources: DAR, NSDAR, PI, PENSION, W 
BAITH (BAYTHE), GEORGE 
Born: 1762 Lancaster County, Pa. 
Died: October 4, 1844 
Buried: Honey Creek Township, near Villas, Crawford County, Illinois 
Spouse: Susanna 
Residences: After the war he moved to Crawford County, III. and settled in Palestine. 
Service: Private; Pennsylvania: Virginia. He enlisted in 1777 in the Pennsylvania troops and was taken prisoner. He was hospitalized because of severe wounds. 
When he recovered he enlisted in Berkeley County, Virginia in 1780 for three months, and again in 1781 for three months, serving as baggage wagon driver in Gen. Anthony Wayne's army. 
Pension: Pension Roll, August 14, 1833, age 72. Pension Census, June 1, 1840, Crawford County, age 85. Susanna, widow, R572 Residence Robinson, Craw­ford County when pension suspended (Act July 7,1838) "Six months service allowed-proof of marriage deficient." 
Sources: DAR, PENSION, W 
BAKER, ABSOLOM 
Born: North Carolina 
Died: 1833 
Buried: Probably Madison County, Illinois 
Residences: He came to Illinois in 1824, settling in Sangamon County. After a short period he removed to Madison County. His name appears in the census records of Madison County 1818, and 1820. 
Service: Soldier; North Carolina. He enlisted in May, 1775 in North Carolina, under Capt. John Brannon, serving until 1781. He was in the battles of Stono and Bacon's (Biggins) Bridge; was taken prisoner at the siege of Charleston and held thirty days; was in Buford's defeat, the battle of Ramsour's Mills, Sumter's defeat at Hanging Rock; was wounded in the battle of Gates' defeat near Camden, and was in the battles of King's Mountain, Monk's Corner, Guilford Court House and Eutaw Springs. 
Pension: S35184: NC. Madison County Pension Roll May 21, 1830 
Sources: CR, DAR, W 
BAKER, FRANCIS, JR, 
Born: January 1, 1762 
Died: April 20, 1846 
Buried: Rushville Cemetery, Rushville, Schuyler County, Illinois 
Spouses: (1) Nancy Davis 
     (2) Mary Magdelina Brandenburg 
Service: Private; Militia: Maryland: New York*, He served in the New York troops in Capt. Ebenezer Newell's Company, Col. Symonds' Regt., enlisting August 4, 1777 and again in October 13, 1781 in Capt. Timothy Reed's Company, Col. Barnes' Regt.

*New York service from Journal of the Illinois State Historical Society 
Marker: Marker placed by Miss Florence Baker and General Macomb Chapter, DAR October 23, 1965 
Sources: DAR, HS, PI 
BAKER, ISAAC 
Born: August 6, 1758 in Fredericktown, Maryland 
Died: September, 1849 
Buried: Rochester Cemetery, Sangamon County, Illinois 
Spouse: Phoebe Waddell 
Residences: He came to Illinois in 1828, settling in Rochester Township. During the Harrison campaign in 1840, Isaac Baker, over 80 years of age, rode in a parade through Springfield in a log cabin drawn by thirty-two yoke of oxen. The cabin was lined with deer and coon skins and a barrel of cider quenched the thirst of the campaigners. 
Service: Private; Fifer: Maryland. His service was in Maryland as a fifer during the last two years of the war. 
Pension: R414 (Md) Residence Springfield, Sangamon County, Illinois when pen­sion claim suspended (Act June 7, 1832) "for further specifications of each tour and service." 
Marker: His name is on a bronze marker in the south mall of Old State Capitol, Springfield, placed by Springfield Chapter DAR, and SAR, on October 19, 1911. 
Sources: DAR, PI, PENSION, W 
BAKER, MICHAEL M. 
Born: March 27,1753 in the Borough of Southwark, London , England 
Died: September 13, 1831 
Buried: White Hall Cemetery, White Hall Township, Greene County, Illinois. Government Headstone 
Spouse: Mrs. Mary McGahe 
Residences: He came to the United States in 1773. After the war he moved to Ohio and from there to Greene County. 
Service: Sergeant: Pennsylvania. He enlisted in 1779in Major Frederick Vernon's Company, Co!. David Broadherdiri's Eighth Pennsylvania Regiment. 
Pension: Pa S46700 BLWT 616-100. Pension Roll, May 13, 1819. Transferred from Ohio (widow, Mary Baker, was pensioned as former widow of Andrew McGahe (McGahy) of Pa. W23530). 
Sources: DAR, HR, PI, PENSION, W 
BALDWIN, DAVID 
Born: May 5,1761 in Dutchess County, New York Died: April 29, 1847 
Buried: Moss Ridge Cemetery, Carthage, Hancock County, Illinois 
Spouse: Hannah Bull 
Service: Private; New York. He enlisted when fifteen years of age, serving as Private in the Third Regiment, under Col. John Field in the New York line. He served ten months, from February to December. 
Pension: R444 (NY) Residence Carthage, Hancock County, when pension claim suspended (Act June 7, 1832) "Not six months service." 
Marker: Monument at Carthage placed by Shadrach Bond Chapter DAR, 1902­1903. Listed on marker at court house, Carthage, placed by Chapter July 10, 1910. 
Sources: DAR, PI, PENSION, W 
BANES, JOHN 
Born: About 1760 in Virginia 
Died: September 2, 1840 
Buried: Perry County, Illinois 
Spouse: Susannah
Residences: From Virginia he moved to Sumner County, Tennessee, and then to Perry County, Illinois Service: Private; NC:SC. Enlisted at Mecklenburg in 1779 and served six times, with Capt. Peter Bennett, and Capt. George Ferringot, Col. William Moore, Col. Ambrose Ramsey, Col. Joseph Taylor, and Major Joel Lewis. He was in the battle of Camden. 
Pension: Listed as pensioner in 1840 Census, Perry County, age 80. Susannah R644 Residence Perry County when pension claim rejected (Act July 7, 1838) "Not a widow at time of Act." She died before August 23, 1842. 
Sources: PENSION, W 
BARBAN (BARBAU), JEAN BAPTISTE, SR. 
Born: 1722 in New Orleans 
Died: 1810 
Buried: Randolph County, Illinois 
Service: Commandant at Prairie du Rocher; a Justice and Deputy-County Lieu­tenant. 
Marker: He is listed on a marker placed at Sparta High School in 1934 by Ft. Chartres Chapter DAR, Sparta. 
Sources: DAR, NSDAR, W 
BARBEE, JOHN 
Born: May 5,1754 in Culpepper County, Virginia 
Died: August 21, 1824 
Buried: Probably Barbee Cemetery, Lamotte Twp., Crawford County, Illinois 
Married: January 24,1782 in Lincoln County, Kentucky 
Spouse: Martha "Patty" Gaines, born November 25,1762 in Mercer County, Ky. Died June 25, 1826; probably buried Barbee Cemetery, Crawford County. 
Children: Nancy (Lyle), James, Joshua, Elizabeth (Marrs), Lucy (Young), William, 
Sarah (Hill), Frances (Wright), Polly (d. young), Thomas, John. 
Residences: Residence in 1779, Mercer County, Kentucky; to Shelby County, Ken­tucky; to Crawford County, Illinois around 1820. 
Service: Private; Virginia. He served in Capt. Robert Todd's Company of Foot, Col. George Rogers Clark's Illinois Regiment. 
Pension: His right to Bounty Land was not established until August 15, 1834. 
Sources: DAR 
BARBER, EZEKIAL 
Born: April 12, 1750 
Died: 1806 
Buried: Florence, now Ellis Grove, Randolph County, Illinois on Ephraim Bilder­ back's farm (cemetery no longer exists). 
Spouse: Elizabeth Goddard (Gozzard) 
Service: Soldier; Connecticut. Served in 22nd Regt. Continental troops, Muster Roll of Capt. Abel Pettibone's Company in the 22nd Regt. of foot commanded by Samuel Wyllys, Esq., May 10, 1776. Roll endorsed Stamford, Connecticut, September 19, 1776. 
Sources: PI National Archives, County History, family records 
BARKER, JACOB 
Born: 1764 
Died: March 2, 1853 
Buried: Barker Cemetery, McLeansboro, Hamilton County, Illinois 
Spouse: Mary Price 
Residences: Lived for many years on his farm in Hamilton County between the towns of Walpole and Broughton. 
Service: Private; SC:PA. He served in the. Pennsylvania troops in Capt. John Rea's Company. His service is also listed as South Carolina. According to family tradition, he was wounded twice. 
Pension: R497 (SC) Residence Hamilton County, Illinois when pension claim was rejected (Act June 7,1832) "Not six months service.” 
Sources: DAR, NSDAR, PI , PENSION, W 
BARKER, ZEBEDIAH 
Born: November 25,1750 at Methuen, Essex County, Massachusetts 
Died: October 10, 1819 
Buried: Barker-Tolin Cemetery, between Burkesville and Waterloo, Monroe County, Illinois, Also recorded as New Design Cemetery, Waterloo Township (probably on his farm). 
Spouse: Susanna Messer 
Residences: He came to Illinois in 1818, settling in New Design, Monroe County. 
Service: Orderly Sergeant: Mass. Enlisted June, 1776 as a "Minute Man ." Served as an Orderly Sergeant under Capt. Maloon, Capt. David Whittier and Col. Thomas Poor. He was in the battle of Stillwater. 
Pension: R505 (Mass) Susannah, widow of Zebediah, residence Monroe County when pension claim rejected (Act July 4, 1836) "For further proof." Papers sent to Hon. J. Carr May 27, 1840. 
Marker: He is listed on marker placed at Sparta High School in 1934 by Ft. Chartres Chapter DAR, Sparta. 
Sources: DAR, NSDAR, PI, PENSION, W 
BARNES, EBENEZER 
Born: February 3, 1759 in Boston, Massachusetts 
Died: May 17, 1836 
Buried: Stout's Grove Cemetery, Danvers Township, McLean County, Illinois. Private Headstone 
Spouse: Ruth Children: Harrison, Ebenezer Residences: He came to Danvers, McLean County, Illinois in 1829. 
Service: Private; Sergeant: Mass:VT. He enlisted in 1775 from Worcester County, Massachusetts, serving as Corporal under Capt. Archelaus Batchelder, Col. Joseph Read; the same year as Sergeant with Capt. Aldrich; in 1776 for nine months as Sergeant under Capt. Gideon Foster, Col. Ebenezer Smith. Re­enlisting in 1777, he was made Lieutenant with Capt. Samuel Fletcher, Col. Timothy Bedel's Regiment, serving four months; in 1778, he served ten months as Lieutenant with Capt. John Tyler, Col. Joseph Fay, all in the Massachusetts line. Total service, two years, seven months. 
Pension: S32106 (Mass: VT). Illinois pension roll, McLean County, February 7, 1834, age 74. 
Marker: Grave marked by Letitia Green Stevenson Chapter DAR in 1928. His name is also engraved on bronze tablet on soldiers and sailors monument in Miller Park, Bloomington, dedicated Memorial Day, May 30, 1913. 
Sources: DAR, PI, PENSION, W 
BARR (O'Barr), HUGH 
Born: 1757/58 
Died: April 24, 1842 
Buried: Barr-Johnson Cemetery, near Flemington, Grandview Twp., Edgar Co., Illinois. 
Spouses: (1) Priscilla James 
    (2) unknown Residences: He was a minister. 
Service: Private; N.C.*. Enlisted in Capt. William Dix' Company, and served from April 1780 to October 1781. He was in a battle near Camden, South Carolina under General Gates and in the battle of Guilford Court House under General Greene. 
*Mrs. Walker shows Massachusetts service. 
Pension: S32110 (NC:VA) 
Marker: Marker placed by Madam Rachel Edgar Chapter DAR, Paris, November 10, 1957. 
Sources: DAR, NSDAH, PI, PENSION, W 
BARRACK, PETER 
Buried: Ed Eavis farm, Robinson, Crawford County, Illinois 
Spouse: Margaret 
Service: Soldier; Maryland. Served in the Maryland troops. 
Pension: R555 (Md). 
Residence Palestine, Crawford County, Illinois, when pen­sion claim rejected (under Act June 7,1832) " For additional evidence." 
Sources: DAR, PENSION, W 
BARROW (Banow), DANIEL 
Born: December 6,1757 in Brunswick County, Virginia 
Died: November 8, 1837 
Buried: Private cemetery, Bradley Township, Jackson County, Illinois 
Spouse: Hannah Stone 
Service: Private; NC:VA. He enlisted in 1776 under Capt. John Williamson in the Virginia troops. He enlisted again in the North Carolina troops, with Capt. John Hill, Capt. Francis Tartanson, and Col. James Hogan. He was in the battle of Guilford Court House. 
Pension: S32104 (NC:SC). Illinois pension roll, Jackson County, May 3, 1834, age 76. 
Sources: DAR, PI , PENSION, W 
BARTHOLOMEW, JOSEPH 
Born: March 15, 1766 in New Jersey 
Died: November 3. 1840 
Buried: Clarksville Cemetery, near Lexington, Money Creek Twp., McLean, County, Illinois 
Spouses: (1) Christiana Pickenpaw 
    (2) Elizabeth McNaught 
Children: (first marriage): Sarah Espey, Catharine McNaught, Amelia Hopkins, Mary Hopkins, Martha Vail, Elizabeth Epler, Christiana Epler, Marston C., James C., John (second marriage): Nancy Bradley, Angela Merryman, George M., William M. 
Residences: Settled in Money Creek Township in 1830. 
Service: Scout: Pennsylvania. Private in Capt. Jonathan Rowland's Company, Tradyffren, Pennsylvania line of troops in 1780. He was a Major General in the War of 1812. 
Marker: Marker placed on grave by Letitia Green Stevenson Chapter DAR, Bloomington in 1941. His name is on a bronze tablet in soldiers and sailors monument , Miller Park, Bloomington, erected in 1913. 
Sources: DAR, NSDAR, PI, W 
BARTLETT,EBENEZER 
Born: October 4, 1759 Newburg, Orange County, New York 
Died: December 3, 1838 
Buried: York Cemetery, West Union, Clark County, Illinois. Government Head­stone 
Spouse: Jane Sairs 
Residences: Arrived in Clark County in 1838 
Service: Private; New York. Enlisted in December 1775 in Capt. Thomas Moffett's Company, and served continuously to the Fall of 1781. He was in the battles of Harlem, White Plains and Ft. Montgomery. 
Pension: S32109 (NY ) 
Marker: Marker placed by Walter Burdick Chapter DAR, Marshall. 
Sources: DAR, PI, PENSION, W 
BATEMAN, ASA 
Buried: Pleasant Grove Cemetery, Mt. Vernon, Jefferson County, Illinois 
Service: Soldier in the war from South Carolina. He also served in the War of 1812. 
Sources: HR 
BATES, WILLIAM 
Born: 1759 in Pennsylvania 
Died: February 1848 in Madison County 
Buried: Lock Haven Cemetery, Jerseyville, Jersey County, Illinois 
Service: Private; SC:NC. He served in the First South Carolina Regiment, com­manded by Col. Charles Pinckney, from April 14, 1776, to December 1776. 
Pension: S32105 (NC). Illinois pension roll, Greene County, May 28, 1833, age 77. Madison County Pension Census, June 1,1840, age 82. 
Sources: DAR, HR, PENSION, W 
BEAN, NICHOLAS 
Born: March 16, 1760 in Bucks County, Pennsylvania Died: August 21, 1838 
Buried: Bean Cemetery, Parker Township, Clark County, Illinois 
Spouse: Ann Wilbourn 
Residences: He lived in North Carolina before moving to Illinois in 1830. Service: Private; Pennsylvania Continental. He served under Capt. John Lacy and Col. John Bull. After serving five months, enlisted in Col. Henry Lee's regiment , serving two years. He was in the battles of Guilford Court House, Cowpens, Eutaw Springs and the siege of Yorktown. He was wounded. He also served in the War of 1812. 
Pension: S32112(Pa). Pension roll, Clark County, May 3,1834, age 70. 
Marker: His grave is marked. 
Sources: DAR, PI, PENSION, W 
BEARD, JAMES 
Born: Pennsylvania 
Buried: Heathville, Crawford County, Illinois, near the Lawrence Countyline 
Married: Never married 
Residences: From Pennsylvania, he moved to Kentucky, and in 1810 came to Lawrence County. 
Service: Soldier; Pennsylvania. He served from Cumberland County, Pa., in the 2nd Company, 4th Battalion, under Col. Samuel Culbertson in 1782, and the same year with Capt. John McConnell. Soon after the close of the War of 1812, he was plowing in a field when an Indian who had a fancied grievance against him, stole up behind him and shot him, killing him instantly. 
Marker: His name is on a bronze tablet at the Lawrenceville Court House, placed by Toussaint Du Bois Chapter DAR in 1921. 
Sources: DAR, W 
BEAUUEU, MICHAEL 
Died: After 1779 at Cahokia, St. Clair County, Illinois 
Service: Frenchman who served under Col. Ceorge Rogers Clark. He was a Jus­tice in Clark's Court and later was elected Justice in the Court of the District in 1779. This may be the "Beauvennue" who was a volunteer in Capt. Charloville's Regiment entitled to 200 acres of land. 
Sources: W 
BECKWITH, SILAS 
Born: 1746 in New Hampshire 
Died: October 13, 1835 
Buried: Beckwith Cemetery (or Swearinger), Flat Rock, Crawford County, Illinois. 
Private Headstone Spouses: (1) Anna Reeves 
			  (2) Anna Smith 
Service: Teamster: New York: Connecticut: Private; Massachusetts". First Lt. in Capt. Andrew Lusk's Co., in the Second Berkshire Regiment. He was com­missioned February 14, 1780. 

Mrs. Walker gives Massachusetts service. 
Pension: S32115 (Conn: N.Y.). Pension Roll, May 30, 1833, Crawford County, age 88. Marker: Bronze marker placed in Beckwith Cemetery by James Halstead, Sr. DAR Chapter, June 1925. 
Sources: DAR, HR, NSDAR, PI, PENSION, W 
BEELER, SAMUEL, JR. 
Born: January 27, 1760 in Virginia 
Died: January 14, 1840 
Buried: East Twin Grove Cemetery, Dry Grove Township, McLean County, Illinois. Private Headstone. 
Married: June 11,1790 
Spouse: Mary Graves 
Children: Samuel, Jr., William, MaIY (Polly), Henry, Elizabeth Rockhold, George, Amelia Kelly, Nancy Hill, Emmeline Dollauhan, Isabella Rockhold, Lavina Rynearson . 
Residences: He came to McLean County about 1830. 
Service: Captain: Virginia. He served in the Virginia line of troops 158 days and was paid at Romney or Winchester in October 1775. 
Marker: Bronze marker placed by Letitia Green Stevenson Chapter DAR in 1926. His name is on the soldiers and sailors monument in Miller Park, Bloomington. 
Sources: DAR, NSDAR, PI , W 
BEER, ROBERT 
Born: April 21, 1750 in Ireland 
Died: March 26, 1842 
Buried: Beer Cemetery, three miles west of Fairview, Fulton County, Illinois. 
Private Headstone 
Married: 1776 
Spouse: Nancy Quinn 
Children: William, Elizabeth, Daniel, Robert, Mary, Charles, Thomas, Nancy, John, Martha, Mathew 
Residences: Came to the United States in 1765, settling in Pennsylvania. Resi­dence was Beaver County, Pa. in 1832. He came to Fulton County in 1841. 
Service: Ensign: Pennsylvania. Drafted at Easton, Northampton County, Pa. in 1777 and served for two months in the sixth company, sixth Battalion, com­manded by Col. Jacob Stroud. He was also an Indian spy. He also served in the Pennsylvania Militia, Capt. Mack's Company, during the spring and fall of 1778, two months each time. 
Pension: S12185(Pa) June 1, 1840 Pension Census, Fulton Co., age 97. 
Marker: Marker placed by Jonathan Latimer DAR Chapter, October 16, 1916. 
Sources: DAR, HR, PI, PENSION, W 
BEGGS, ALEXANDER 
Born: May 30, 1754in Antrim County, Ireland 
Died: February 4, 1837 
Buried: Union County, Illinois 
Service: Private, Pennsylvania: North Carolina. He served in 1776 for four months in Henry Lee's Artillery Company; ten months in 1777; three months in 1778 and 1781. He was captured at Brandywine but escaped that night. He was in the battle of Stono. 
Pension: S32114 N.C.:Pa. Union County, Illinois pension roll, July 16, 1833, age 80 
Sources: DAR, PENSION, W 
BELL, NATHANIEL 
Born: March 15, 1755 in Warren County, North Carolina 
Died: January 17, 1835 
Buried: Probably St. Clair County, Illinois. 
Residences: Settled in St. Clair County but was living in McLeanCountyin1834. 
Service: Private; North Carolina: South Carolina. He enlisted in Anson County, North Carolina, April 1, 1776, serving fourteen months under Capt. Thomas Potts, Col. Isaac Huger, South Carolina troops. He enlisted in September, 1781 for two months with Capt. Thomas Harris, Col. William Laxteri's North Carolina troops. 
Pension: S32113 N.C:S.C. He was a resident of St. Clair County in 1830. McLean County, Illinois pension roll, February 4, 1834, age 79. 
Sources: CR, DAR, PI , PENSION, W 
BELL,ROBERT 
Born: Northern Ireland 
Died: 1837 
Buried: Friendsville, near Mt. Carmel, Wabash County, Illinois 
Residences: He came from Rockbridge County, Virginia, to Wabash County in 1818, settling in Friendsville Precinct. 
Service: Private; Virginia. He enlisted when sixteen years of age, serving seven years in the light artillery under Lafayette in the Virginia line of troops. 
Pension: S8065(Va) 
Sources: DAR, NSDAR, PENSION, W 
BEMAN (BEEMAN), WILLIAM 
Born: May 2, 1758 in Kent, Connecticut 
Died: October 21, 1837 
Buried: Seely Mill Cemetery, White Hall, Greene County, Illinois 
Service: Private; Pennsylvania Continental troops: Connecticut. He enlisted June 1, 1776 for six months with Col. Heman Swift and Capt. Ebenezer Couch. In 1777 he served two months with Capt. Sackett and Col. Hooker, and the same year with Capt. Peter Mills and Col. John Wood. In 1779 he served six months with Col. Heman Swift. 
Pension: S32111(Conn.) Pension Roll, January 9, 1834, Greene County, age 75. 
Sources: DAR, HR, PENSION, W 
BENNETI, WILLIAM 
Born: May 9, 1758 at Sandown, New Hampshire 
Died: February 15, 1846 
Buried: Vanderhoff Cemetery, near Wasco, Kane County, Illinois 
Residences: He moved to Genesee County, New York, and from there to Kane County, Illinois, in 1836. 
Service: Soldier; Massachusetts: New Hampshire. He enlisted in August, 1776, under Capt. Nathan Brown and Col. Pierce Long, New Hampshire troops and again in 1779 with the same officers. He served in 1780 under Col. Timothy Bedel, in Massachusetts troops, with Capt. Charles Johnson and Col. James Wadsworth. His fourth period of service was in September, 1782, with Capt. Cutting Farror, New Hampshire troops. He was in the battle of Fort Ann. 
Pension: S30854 Mass:NH. Pension Census, June 1, 1840, Kane County, age 86.  
Marker: Marker placed by Elgin Chapter DAR in 1900. 
Sources: CR, DAR, NSDAR, PENSION, W 
BENSON, JAMES 
Born: Talbot County, Maryland 
Buried: Probably Edgar County, Illinois 
Residences. Settled in Virginia after the war; came to Edgar County with son in 1824. Son removed to Jasper County in 1851. 
Service: Served with Maryland troops as a Sailor. 
Sources: DAR, W 
BETTISWORTH, CHARLES
Born: November 4, 1761 
Died: June 12, 1842  
Buried: Bethel Cemetery, Basco, Hancock County, Illinois. Private Headstone 
Spouse: Mary 
Service: Corporal, Virginia. He enlisted when eighteen years of age and served until the close of the war in the Virginia line of troops. 
Pension: 532117. Pension Roll, November 15, 1832, age 71. Transferred from Kentucky March 4, 1834. Pension Roll, June 1, 1840, Hancock County, age 79, residing with D. Bettsworth, head of family. 
Marker: His name is on a bronze tablet in the Court House, honoring the Sol­diers and Sailors of the Revolutionary War buried in Hancock County, erected by Shadrach Bond Chapter DAR in 1910. 
Sources: DAR, HR, PI, PENSION,W 
BEVANS (Bevens, Bivens), JOHN 
Born: September 15, 1760 in Middletown, Connecticut 
Died: February 24, 1839 
Buried: Marietta Cemetery, Marietta, Fulton County, Illinois. Private Headstone 
Married: September 18, 1779 
Spouse: Hannah Owen b. 1763; d. November 1843, Lorrain County, Ohio 
Children: (children living in 1843) Benjamin 0 ., Mary, John, Fanny, Ira, Henry, Harriet , Lorenzo, Milton 
Residences: He removed to Wayne County, Ohio, and from there to Fulton Count y. 
Service: Private; Fifer; Mass; NY. He enlisted in March, 1775, for nine months as a fifer in Capt. Jacob Allen's Company, Col. Jonathan Brewer's Massa­chusetts Regiment. In December, 1775 he enlisted for one year with Capt. Aaron Haynes and Col. Asa Whitcomb. In August, 1777 he enlisted for two years under Capt. Ebenezer Webber, and in 1779 he served for three months with Capt. Gideon King in the New York line of troops. 
Pension: Hannah R875. Her residence was Fulton County when pension rejected (Act July 7, 1838)" Not a widow at date of Act." 
Marker: His grave has a DAR marker. 
Sources: DAR,HR,NSDAR,PI, PENSION.W 
BIDWELL, DANIEL 
Born: April 5, 1761 
Died: January 25, 1839 
Buried: Charles Cemetery, Carmi, White County, Illinois 
Service: Sergeant: N.Y. He served from New York in the Albany County Militia, Thirteenth Regiment, with Capt. John McCrea and in Capt. Joel Wright's Company. 
Pension: S32118 N.Y. Illinois pension roll, White County, age 73. 
Marker: His name is on a marker placed in 1936 by Wabash Chapter DAR, Carmi, honoring 35 Revolutionary soldiers buried in White County. A gov­ernment headstone was placed in the old cemetery, Carmi, by the Chapter on Sept ember 21,1964. 
Sources: DAR. PENSION, W 
BIENVENUE, ANTOINE, SR. 
Buried: Probably near Kaskaskia, Randolph County, Illinois 
Residences: He was listed in Kaskaskia in the 1810 Census 
Service: He enlisted in January, 1779 for eight months to serve under Capt. Fran­cis Charleville and Col. George Rogers Clark. He was one of 50 Frenchmen recruited to aid in the conquest of Vincennes. 
Marker: His name is on a bronze marker on the grounds of Sparta High School placed by Fort Chartres Chapter DAR on June 24, 1934. 
Sources: CR, NSDAR, W 
BIGGS, ELIJAH (Elisha) 
Buried: Wade Copeland Cemetery, Windsor, Shelby County, Illinois
Service: Captain: North Carolina. Served from North Caroiina under Capt. Wil­liam Hall and Col. Kenan. 
Pension: R829 N.C. Resident of Shelbyville, Shelby County when pension claim rejected (Act June 7, 1832) "not 6 months service." 
Marker: The grave is reported to be marked. Sources: DAR, NSDAR, W 
BIGGS, WILLIAM 
Born: In Maryland in 1753/55 
Died: 1827at the residence of Major Samuel Judy 
Buried: Peters Station (Waterloo), Madison County, Illinois 
Spouse: Nancy Residences: Following his service in Illinois, he returned to Virginia; returning to Illinois in 1784, he settled near Bellefontaine, Monroe County. He re­moved to Madison County where he probably died. He was twice elected to represent Illinois territory at Vincennes, 1812-1814. He was later elected Senator. In 1789 he was captured by the Kickapoo Indians, but was released by paying a heavy ransom. 
Service: Lieutenant: Md.:Va. He served as a Lieutenant under Col. George Rogers Clark for the conquest of Illinois. 
Pension: Rl9360(Va) Half Pay 
Sources: DAR, NSDAR, PI, PENSION, W 
BLAIR, WILLIAM 
Born: December, 1760 in Lancaster County, Pennsylvania 
Died: April, 1840 
Buried: Probably Schuyler County, Illinois 
Spouse: Catherine Evans 
Service: Private; Pennsylvania. He enlisted from Cumberland County, Pa., in May, 1778, serving two months as a substitute for his father, Alexander. In May, 1779 he enlisted under General John Sullivan; and again enlisted for seven months on the frontier until 1781. 
Pension: S2I647 (Pa) Illinois pension roll, Schuyler County, April 6, 1833, age 73 
Sources: PI, PENSION, W 
BLANKENSHIP, BENJAMIN 
Born: 1760 in Hampton, Virginia 
Died: 1844 
Buried: Probably Henderson County, Illinois 
Spouse: Patience Jackson 
Residences: He removed to Ohio and in 1836 came to Warren County, Illinois. (Henderson County was separated from Warren County in 1841) 
Service: Private; Virginia Continental. He enlisted in 1777, serving in Capt. Anthony Singleton's Company, and Col. Charles Harrison's Regiment. He was in the battle of Camden. 
Pension: S30870 (Va) Pension Census, Warren County, June 1,1840, age 80 
Sources: DAR, PI, PENSION, W 
BLEVINS, WILLIAM 
Buried: Probably Vermilion County, Illinois 
Service: Soldier; Virginia Pension: R945 (Va). 
Residence Danville, Illinois, when pension claim rejected (Act June 7, 1832) "not under military authority or organization ." 
Sources: PENSION 
BLOUIN, DANIEL (David) 
Born: About 1740 
Died: After February 23, 1782 
Buried: Near Kaskaskia, Randolph County, lllinois 
Spouse: Helene Charleville 
Service: Private; Virginia. Enlisted in January, 1779 for eight months, one of 50 Frenchmen serving under Col. George Rogers Clark. 
Marker: His name is on a bronze marker on the grounds of Sparta High School, placed by Fort Chartres Chapter DAR on June 24, 1934. 
Sources: DAR, NSDAR, PI, W 
BOBBITf, ISHAM (Ishmail) 
Born: March 3, 1754 in North Carolina 
Died: March 6, 1836 
Buried: Paschal Farm Cemetery, Jacksonville, Morgan County, Illinois. Private Headstone 
Spouse: Elizabeth 
Service: Sergeant: North Carolina. Served under Capt. Farley, and was at the siege of Yorktown. 
Pension: Illinois pension roll, Morgan County, May 29, 1833, age 79. Elizabeth W24709 (NC). 
Marker: His name is on a plaque in front of the Morgan County Courthouse, placed by Rev. James Caldwell Chapter DAR, Jacksonville in 1914. 
Sources: DAR, HR, NSDAR, PI, PENSION, W 
BOISMENUE, M. 
Buried: Cahokia, St. Clair County, Illinois 
Service: Soldier with Thomas Brady in the expedition against St. Joseph, Michi­gan. He also served with Col. George Rogers Clark. He was wounded and remained with the Indians all winter, returning to Cahokia in the spring. 
Sources: W 
BOND, SHADRACH, SR. 
Born: Near Baltimore, Maryland 
Buried: In the old cemetery on the bluff above his residence, Monroe County, Illinois. (Monroe County formed from St. Clair 1816) 
Residences: Came to Illinois in 1781 from Virginia. He was a member of the first General Assembly of the territory which convened west of the Ohio River after the Revolutionary War, serving four times. He was also elected Justice of St. Clair County. He was the uncle of Shadrach Bond, the first Governor of the State of Illinois. 
Service: Sergeant in Company of Col. George Rogers Clark. 
Sources: DAR, W 
BOON, THOMAS 
Born: 1760 in North Carolina 
Died: 1836 in Clark County, Illinois 
Spouse: Mary; died January 29, 1847 
Service: Private in South Carolina troops. 
Pension: Pension Roll, July 16, 1833, age 73. Mary W23656 (SC); widow, Mary, on pension list, May 10, 1836. 
Sources: PENSION, W 
BORDERS, PETER 
Born: 1764 in Lancaster, Pennsylvania 
Died: March 9, 1859 
Buried: Irish Grove Cemetery, between Greenview and Middletown, Menard County, Illinois 
Residences: In 1803 moved to Clark County, Kentucky and in 1804 to Green County, Ohio. In 1825 he moved to what is now Menard County, being among the first settlers. 
Service: Private; Virginia. At the age of 17 years he enlisted from Loudon County, Virginia, in the Continental Army and served for six months. He served as Private in Simerals' Company of Col. Woods Virginia Regiment. He was stationed at Winchester Barracks to guard prisoners. 
Pension: RI029 (Va). 
Residence Mount Pulaski, Logan County, when pension claim suspended (Act June 7, 1832) "For proof of service and officers names." 
Residence Springfield, Sangamon County, when pension claim suspended (Act June 7,1832) " Not six months service." 
Marker: Placed by Pierre Menard Chapter DAR, September 15, 1935, in Irish Grove Cemetery. 
Sources: DAR, HR, NSDAR , PENSION, W 
BOSlWICH (Bostic), EZRA 
Born: 1753 in Queen Anne County, Maryland 
Died: January 1843 ~ 
Buried: McCord Cemetery, near Irving, Montgomery County, Illinois 
Spouse: Drucilla 
Residences: Came to Illinois in 1818, settling in the Bostwick settlement, near the present Village of Irving. 
Service: Private; North Carolina. Enlisted October 15, 1780 under Capt. Patrick Began, North Carolina troops, serving until the close of the war. 
Pension: Illinois pension census, June 1, 1840, Montgomery County. age 86. Drucilla W23653 (NC) 
Sources: DAR, HR, PENSION, W 
BOURN,EBENEZER 
Born: 1753 in Lebanon, Connecticut 
Died: August 29, 1839 
Buried: Harrisonville, Monroe County, Illinois 
Service: Soldier; Virginia. Enlisted under Col. George Rogers Clark in 1778 for fourteen days, while on an exploring and hunting expedition on the Ohio River. Enlisted for fourteen months with Capt. John Williams, Col. Mont­gomery, and Col. William Lynn. He also served in Capt. William Harrod's Company. 
Pension: S32129 (Va) Pension Roll December 17,1833, Monroe County, age 81. 
Sources: DAR, HR, PENSION, W 
BOUTWELL, STEPHEN 
Born: 1753 
Died: 1835 
Buried: Old Bradley Cemetery, near Junction, Gallatin County, Illinois 
Service: Corporal: Virginia troops. Served under Capt. Samuel Hawes and Col. Alexander Spotswood from January 1, 1777 to June of the same year. 
Pension: R1061 (SC) Residence Equality, Gallatin County, when pension claim suspended "For further proof and information." 
Sources: DAR, PENSION, W 
BRADSHAW, JONATHAN 
Born : 1757 
Died: January 22, 1834 
Buried: Probably in White County, Illinois 
Service: Private in the New Jersey troops on the Continental line. Sources : DAR 
BRADY, THOMAS 
Born: In Pennsylvania 
Died: Cahokia, St. Clair County, Illinois Spouse: Mrs. LaCompt (nee LaFlamme), born in 1734 at St. Joseph, Michigan. She came to Cahokia in 1770 and was a widow when she married M. La­Compt, which marriage produced a large French family. After the death of her second husband, LaCompt, she married Thomas Brady at Cahokia. After his death, she assumed the name LaCompt. She died in 1843 at the age of 109. Mrs. Brady understood the Indian language and was able to communi­cate with the Indians. She is credited with preventing attacks on the white settlers during the period 1771 to 1795 by walking out to meet the warring Pottawatamies, Kickapoos, and other tribes, often at night, persuading them to dismiss their hostile intentions. 
Residences: Thomas Brady was a resident of Cahokia before the Revolutionary War. Hew as the first sheriff of St. Clair County. 
Service: In 1777 he raised a small company of men from the villages of Cahokia and Prairie du Pont and marched through the wilderness to St. Joseph, Michigan. They captured the garrison, but returning, were overtaken at Calumet. Two men were killed and Brady was taken prisoner. The following year he escaped and reached Cahokia. He served under Colonel George Rogers Clark. 
Sources: DAR, W, County History 
BRAMBLETT (Bramlet , Bromlet), REUBEN 
Born: Probably Virginia 
Died: After 1840 
Buried: Bromlet Cemetery, Raleigh Township, Saline County, Illinois (created 22 from Gallatin County in 1847) 
Spouse: Nancy Adams, former widow of Rev. War soldier George Adams, Va. 
Residences: He came to Illinois in 1819 settling in Raleigh Township, Gallatin County, now Saline. 
Service: Private, Virginia Continental troops 
Pension: S30896 (Va) Pension roll, April 6, 1833, Gallatin County, age 76. In Pension Census of June 1, 1840, Gallatin County, age 82. Nancy W8393. 
Marker: His name is on a marker on Courthouse lawn in Harrisburg, placed by Michael Hillegas Chapter DAR 1931. 
Sources: DAR, PENSION 
BRANDT, REUBEN 
Buried: Crowder Cemetery, Illinois Country Club grounds, Springfield, Sangamon County, Illinois 
Service: Sergeant from Virginia 
Sources: NSD AR 
BRATNEY, ROBERT 
Born: Ireland 
Died: 1832 Kaskaskia 
Buried: Randolph County, Illinois 
Residences: Came to America and settled in Tennessee. In 1820 he removed to Illinois, settling near the mouth of Little Plum Creek in Evansville Town­ship, Randolph County. 
Service: Enlisted in Tennessee 
Marker: His name is on a bronze marker on the grounds of Sparta High School placed by Ford Chartres Chapter DAR, Sparta in 1934. 
Sources: DAR, W 
BRIANCE, HENRY, SR. 
Born: About 1762 in North Carolina 
Died: August 19, 1833 
Buried: Clear Springs Cemetery, Montgomery County, Illinois 
Service: Private; North Carolina Continental troops. He enlisted in 1777, under Col. Wade Hampton, Gen. Thomas Sumpter, and Gen. Francis Marion. He was in the battles of Eutaw Springs, Friday Fort, Thompson's Fort, Monk's Corner and Monroe Field . 
Pension: S32133(NC) Illinois pension roll, Montgomery County, August 14, 1833, age 71 
Sources: DAR, PENSION, W 
BRIDGES, ALLEN J. 
Born: 1756, Wake County, North Carolina 
Died: After 1840 
Buried : White Hall Cemetery, White Hall, Greene County, Illinois. Government Headstone 
Spouse: Elizabeth Irwin 
Service: Private; North Carolina Continental troops. He enlisted in Rowan County and served seven months in Capt. Simeon Alexander's Company, Col. Joseph McDowell's Regiment. He was in the battles of Ramsour and Salisbury. He served in the War of 1812. 
Pension: Pension Roll, April 8, 1832, age 77; 1840 Federal Census, Greene Co., age 80. Elizabeth W8159(NC) 
Sources: DAR, HR, PENSION, W 
BRIDGES, GEORGE 
Born: February 12, 1762 near Elizabeth, on Cape Fear River, North Carolina 
Died: After October 20, 1834 
Buried: Near Troy, Madison County, Illinois 
Spouse: Nancy Edwards 
Residences: He came to Illinois in 1808 and settled near Troy, Madison County. 
Service: Private and Drummer; North Carolina. He enlisted at Salisbury, March 10, 1777, under Captains Griffith McCrea and Christopher Goodwin, serving nineteen months; enlisted June, 1780 for three months under Capt. James Craig and Col. John Fifer; enlisted November, 1780, for three months; during a subsequent three month enlistment, he was taken prisoner by the British. He enlisted for his fifth period of service in May, 1781. 
Pension: S32139 N.C. Illinois pension roll June 25,1834, Madison County, age 71. Marker: His name is listed on a tablet on the Edwardsville Court House placed by Ninian Edwards Chapter DAR, September 1912. 
Sources: DAR, PI, PENSION, W 
BRISCOE, HENRY 
Born: February 3, 1763 in Maryland 
Died: September 26, 1839 
Buried: Family cemetery one mile east of Westfield, Clark County, Illinois Spouse: Katharina Brookhart 
Residences: He removed to Kentucky and from there to Clark County. 
Service: Private, Maryland. He enlisted in 1781, serving until December under Capt. David Lynn and Major Alexander Roxburg. He was in the siege of Yorktown, 1781. 
Pension: S30888 (Md) Marker: 
Marker placed by Walter Burdick Chapter DAR. 
Sources: DAR, PI, PENSION, W 
BROADWELL, MOSES 
Born: November 14, 1764 in Elizabethtown, New Jersey 
Died: April 10, 1827 
Buried: Oak Ridge Cemetery, Springfield, Sangamon County, Illinois. Government 
Headstone (placed by Springfield Chapter DAR in 1923) Spouse: Jane Broadwell 
Residences: He came to Illinois in 1820, settling near Pleasant Plains, Sangamon County. 
Service: Private, New Jersey. He entered the service at an early age, serving near the close of the war in the Third New Jersey Regiment under Co. Elias Dayton, 1780. 
Marker: His name is on a bronze plaque in the south mall of the Old State Capitol placed by Springfield Chapters DAR and SAR on October 19, 1911. 
Sources: DAR, HR, NSDAR, PI, W 
BROCKETT, THOMAS 
Died: Before 1840 in Effingham County, Illinois 
Spouse: Martha J. 
Service: Soldier; Pennsylvania. He served from Pennsylvania in Lt. Talmage Hall's Company, Col. Lewis Nicola's Regiment. 
Pension: His widow, Martha J. Brockett, received pension in Effingham County in 1840, age 85. 
Sources: HS, PENSION 
BROCKMAN (Breckman), THOMAS 
Born: 1759 in Albemarle County, Virginia 
Died: About 1838 
Buried: Abandoned cemetery on "Spinner's Farm ," Montgomery County, Illinois 
Spouse: Elizabeth Burress 
Service: Private, Virginia. He entered the service early in 1776 under Capt. John Marks, Col. Charles Lewis' Regiment, in General Nathaniel Greene's Division, serving for three years; he also served under Capt. Archibald Moon, and was in the battles of Brandywine, Germantown, and Stony Point. Pension: RI233 (Va). Residence Hillsboro, Montgomery County when pension claim suspended (Act June 7.,1832) " for further proof from Virginia records." 
Sources: DAR, PI, PENSION, W 
BRONSON (Brunson), PHINEAS 
Born: November9,1764in Enfield, Connecticut 
Died: October 25, 1844 
Buried: Princeville Cemetery, Princeville, Peoria County, Illinois 
Spouse: Isabel Wright 
Children: Heil 
Service: Private and Army Wagon Master, Connecticut. He served in the Third Company of the Second Regiment under Major Benjamin Walbridge and Col. Zebulon Butler. 
Pension: S32137(Conn.) Pension Census, Peoria County, June1,1840, age 76 
Marker: Placed by Peoria Chapter DAR November 20, 1973. 
Sources: DAR, NSDAR, PI, PENSION, W 
BROWN, DANIEL 
Born: October 1757 in Bucks County, Pennsylvania 
Died: After 1833 
Buried: Wanda Cemetery, Roxana, Madison County, Illinois 
Spouse: Elizabeth 
Service: Private and Sergeant, Virginia Continental troops. He enlisted in Augusta County, August 8, 1776, under Capt. John Gilmore, Col. William Russell and Col. William Christian, serving three months; enlisted for six weeks under Capt. Charles Cadliff, six weeks under Capt. John Martin; one month from May, 1782, was made Sergeant under Capt. Robert McBride, Col. Stephen Trigg; October, 1782, he served for one month under Capt. Samuel Kirkham, Col. Benjamin Logan. He re-enlisted in 1786 following the war for a short term of service. 
Pension: S32132 (Va). He was pensioned and his widow, Elizabeth, received a pension. Illinois Pension Roll, Madison County, October 14, 18:5:5 , age 75. Elizabeth W5907 (Va) 
Marker: His name is on a bronze tablet in the Madison County Court House, Edwardsville, placed by Ninian Edwards Chapter DAR Alton, September 16, 1912. 
Sources: DAR, PENSION, W 
BROWN, EBENEZER 
Died: 1834 
Buried: Monroe County, Illinois 
Service: Soldier; Connecticut. He served from Connecticut in Capt. Henry Champion's Company, Col. Willis' Regiment from Waterbury, 1777 to 1781. 
Sources: HS 
BROWN, GEORGE 
Born: 1752 in Chesterfield County, Virginia 
Died: March 24, 1842 
Buried: Nashville, Washington County, Illinois 
Service: Soldier;Virginia. He enlisted March, 1780, in Charlotte County, Virginia, serving two months with Capt. Thomas Williams; two months in 1781 under Capt. Dudley Barrel and Col. Peter Muhlenberg; two months under Capt. Pickeway and Col. Holt Richardson. 
Pension: 532134 (Va). Resident of Nashville, Washington County when pension claim suspended "for further proof and specification" (Act June 7, 1832). Pension Census June 1,1840, Washington County, age 88. 
Sources: CR, DAR, PENSION, W 
BROWN, GEORGE 
Born: 1755 in Rowan County, North Carolina 
Died: December 10, 1846 
Buried: Brown Homestead Cemetery, Beech Ridge, Alexander County, Illinois 
Spouses: (1) Barbara Wasnbouoy 
    (2) Mrs. Margaret Sowers 
Residences: Following the war he removed to Tennessee and then to Union County, Illinois. 
Service: Private, N.C. He enlisted in Rowan County, N.C. in the fall of 1776, serving two months under Capt. James Montgomery and Col. Francis Locke. He enlisted in the summer of 1779, serving five and one-half months under Quartermaster Yarberry. 
Pension: Pension Census, June I, 1840, Union County, age 78. Margaret W25274 (NC) 
Sources: DAR, PI , PENSION, W 
BROWN, JOHN 
Died: May 6, 1832 
Buried: Fairview Cemetery, Sparta, Randolph County, Illinois Service: Private 
Marker: His name is on a bronze marker on the grounds of Sparta High School, Sparta, placed in 1934 by Ft. Chartres Chapter DAR. 
Sources: DAR , HR 
BROWN, NATHANIEL (Nathan) 
Born: About 1753 in New York 
Died: After 1840 
Buried: Batavia Township, Kane County, Illinois 
Service: Soldier; New York. He enlisted in Westchester County, New York, under Capt. Benjamin Chapin and Col. Thaddeus Crane. 
Pension: Age 87 in the Pension Census of June I, 1840, Kane County. 
Sources : PENSION, W 
BROWN, SAMUEL 
Born: About 1764, Pittsylvania County, Virginia 
Died: After 1840 
Buried: Macoupin County, Illinois Residences: He first settled in Morgan County but removed to Macoupin County. 
Service: Private; Virginia. He served from Pittsylvania County, Virginia. Enlisted 
September or October, 1780, for duration of war and served in Capt. Wm . Barrett's Company; 1st Lt. John Linton, Third Regiment Cavalry commanded by Col. William Washington and Lt. Col. Richard Watts. Was at Eutaw Springs battle where Col. Washington was wounded and taken prisoner; in siege of Charleston; served two years, eight months. Discharged June 1783. 
Pension: S32138 (Va) BL REG 2326:50-1855. Applied June 1, 1835 from Macoupin County; Pension Census June 1, 1840, Macoupin County, age 76, residing with Alexander Montgomery, head of family. 
Sources: PENSION, W 
BROWN, WILUAM 
Died: 1841 
Buried: Cherry Point Cemetery, Wenona, Marshall County, Illinois 
Service: Private, Virginia Militia; also served in the War of 1812. 
Sources : DAR, HR 
BROWNFIELD (Bromfield), ROBERT 
Born: June 4, 1760 in Pennsylvania 
Died: June 17,1841 
Buried: Rheinhart Private Cemetery, Somer Township, Champaign County, Illinois 
Spouse: Ramsey Residences: He removed to Kentucky, and from there, in 1833, to Champaign County, Somer Township. Resident, 1840 census. 
Service: Soldier; Pennsylvania. He enlisted in Mayor June,1780, from Cumber­land County, Pennsylvania; served as a Ranger under Capt. Thomas Camp­bell; guarded baggage wagons, serving until 1781. The Pennsylvania Archives state that he also served in the Westmoreland County Militia. 
Pension: R1353 (Pa). Applied for pension, August 28, 1835, while resident of Champaign County, age 75. Claim rejected as "he did not render actual service in an embodied corps as required by the pension laws." Residence Danville, Vermilion County when pension claim rejected (Act June 7, 1832) "Not military service -rangers service requires proof of authority." 
Sources: HR, NSDAR, PI, PENSION, W 
BRUNER, ADAM 
Born: May 8, 1763 in Maryland 
Died: October 19, 1846 
Buried: Bruner Cemetery, southeast of Rio, Knox County, Illinois. Private Head ­stone Married: June 7,1784 
Spouse: Elizabeth Rice 
Children: Michael, Adam, Jr., David, (Brother) Peter 
Service: Private; Pennsylvania. He enlisted February 4, 1781, in Capt. John Cayers' Company, Major Richard Salter's Third Regiment of Foot, Philadel­phia Militia. 
Marker: Placed by Mildred Warner Washington Chapter DAR, Monmouth on Octob er 11, 1918. 
Sources: DAR, NSDAR, PI , W 
BRUNER, PETER 
Born: In 1760 in Maryland 
Died: July 17,1850 
Buried: Bruner Cemetery, southeast of Rio, Knox County, Illinois. Private Head­stone 
Married : 1790 
Spouse: Sarah 
Children: Peter, (Brother) Adam 
Service: Private; Pennsylvania. He enlisted in Capt. George Feathers Company, Ninth Battalion, Lancaster County Militia, Col. John Huber's Regiment in 1779; served in Capt. John Smuller's Company in 1780 and 1781; in 1782 in Capt. James Patten's Company. 
Pension: S5303 (Pa) Marker: Placed by Mildred Warner Washington Chapter DAR, Monmouth on October 11, 1918. 
Sources: DAR, HR, PI, PENSION, W 
BRUNSON, MAXWELL S. 
Buried: Brunson Cemetery, Lockport, Will County, Illinois 
Service: He served in the Revolutionary War in 1776 and in the War of 1812. 
Sources: DAR, HR 
BRYAN, GEORGE 
Born: February 15, 1758 in Rowan County, North Carolina Died: November 22, 1845 
Buried: Old Harmony Cemetery, Woodside Township, Sangarnon County, Illinois 
Spouses: (1) Elizabeth Ragan 
    (2) Cassandra (Wright) Miller 
Residences: While young he moved with his parents to Virginia, and from there to Kentucky in 1781. He moved to Sangamon County, Illinois, in 1834. 
Service: Private and Spy; North Carolina Militia. He rendered service in defend­ing the Fort, which was named in his honor, against an attack by Indians. The bravery of one of the young Indian maidens won Bryan's heart and he married her in the autumn. 
Pension: S32142 (NC) Marker: His name is on a bronze plaque in the south mall, Old State Capitol, Springfield, placed by Springfield Chapters DAR and SAR, October 19, 1911. 
Sources: DAB., PI , PE NSION, W 
BURNHAM, GURDON (Gurdin) 
Born: February 20, 1757, probably Connecticut 
Died: After 1834 
Buried: Edgar County, Illinois 
Spouse: Martha Cahoon 
Service: Private; Connecticut Continental. He enlisted in Connecticut in 1776 and served on board the ship Alfred as a drummer, was captured in an engage­ment off Barbadoes and was exchanged in 1778. 
Pension: S31585 Connecticut Navy. Pension Roll, June 4, 1834, Edgar County, age 75 
Sources: PI, PENSION, W 
BURRIS, MARTIN 
Born: 1754 in Pennsylvania 
Died: 1839 
Buried: Morgan County, Illinois 
Service: Soldier; Virginia Pension: S32151 (Va) Morgan County, Illinois pension roll, May 29, 1833, age 78 
Marker: His name is on a marker placed in the Morgan County Court House at Jacksonville in 1914 by the Reverend James Caldwell Chapter DAR. 
Sources: DAR, PENSION, W 
BURROUGHS, DANIEL, SR. 
Born: 1756 in New York 
Died: October 18, 1843 
Buried: Griswold Cemetery, Plano, Kendall County, Illinois 
Spouses: (1) Mary Crane 
    (2) Olive Carpenter 
Service: Private; New Hampshire: Patriotic Service: New York. He enlisted in the Charlotte County Militia with Capt. Elsharna Tozer, Col. Alexander Webster, and Col. Thomas Armstrong in the Dorset Regiment.
Pension: S32144 (NH) Pension Census, Kane County, June 1, 1840, age 85. (Ken­dall County formed from Kane in 1841) 
Marker: Placed by Fort Payne Chapter DAR, Naperville, May 1, 1970. 
Sources: DAR, HR, PI , PENSION, W 
BURTON, JOHN 
Born: 1761 in Mecklenburg County, Virginia 
Died: 1839 
Buried: McDevitt Cemetery, Chatham Township, Sangamon County, Illinois 
Service: Private; Virginia Continentals. He enlisted in Mecklenburg County, Virginia in 1780 for three months in Capt. Asa Oliver's Company, Col. Char­les Fleming's Regiment; in 1781 for three months in Capt. Stephen A. Berry's Company, Virginia troops. He was at the siege of Yorktown. 
Pension: S32146; S46831. Pension roll for Sangamon County, May 3, 1834, age 72 
Marker: His name is on a bronze plaque in the south mall, Old State Capitol, Springfield, placed by Springfield Chapters DAR and SAR, October 19, 1911. 
Sources: DAR, NSDAR, PENSION, W 
BUSHBY (Busby), ROBERT 
Born: July, 1759 in Hanover County, Virginia 
Died: After 1840 
Buried: Macoupin County, Illinois 
Residences: He removed to Morgan County, Illinois, but died in Macoupin County. 
Service: He served in the Virginia troops. 
Pension: S30904 (Va) Pension Census, June 1, 1840, Macoupin County, age 84, residing with Lewis Harmon, head of family. 
Sources: PENSION, W 
BUTLER, JAMES F. 
Died: August 1, 1821 
Buried: Old Homer Cemetery, Homer, Champaign County, Illinois 
Service: Revolutionary War and War of 1812 
Sources: HR 
BYRUM, BENJAMIN J. 
Born: In New Castle, Pennsylvania 
Died: After 1781 
Buried: Kaskaskia, Randolph County, Illinois 
Spouse: Mary H
Residences: Arrived in Kaskaskia in the spring of 1781. 
Service: Soldier; Virginia. Had discharge papers from the service and Oath of Fidelity taken at Fort Pitt. 
Pension: Mary Squire, former widow of Benjamin J. Byrum, Va. Pension File RI0027 (Va.) 
Sources: PENSION, W 


CABOT, FRANCIS 
Born: November 16,1752 
Died: February 20, 1831 
Buried: Marion Cemetery, Williamson County, Illinois 
Spouse: Marcy Hodgman 
Service: Patriotic Service; Vermont 
Marker: Boulder placed in Marion Churchyard, October 13, 1936, by Egyptian and Wabash Chapters DAR honoring Francis Cabot, Revolutionary soldier and patriot. 
Sources: DAR 
CALDWELL (Colwell), SAMUEL 
Born: 1749 near Wheeling, Virginia 
Died: 1850, aged 101 
Buried: On the Brenneman farm between Chili and Stillwell, Hancock County, Illinois. Government Headstone 
Service: Scout, Virginia; Army Chief of Scouts,Virginia Line Troops
Pension: S32168 (Va.) Pension Census, June I, 1840, Hancock County, age 77 
Marker: His name appears on a tablet in the corridor of the second floor of the Court House in Carthage placed by Shadrach Bond Chapter DAR, July 10, 1910. 
Sources: DAR, HR, PENSION, W 
CALLIS (Callois, Collis, Collins), DAVID 
Buried: McCord Cemetery, Perry, Pike County, Illinois 
Service: Soldier; Virginia. He served in the war from Virginia, and again served in the United States troops after the Revolutionary War. 
Pension: He was pensioned in Pike County. 
Sources: HR, W 
CAMPBELL, ENOS 
Born: 1753 
Died: June 2, 1838 
Buried: Old Sackett Cemetery, Salisbury Township, Sangamon County, Illinois 
Spouse: Darnarius Nowe 
Residences: A Scotchman, following the war he moved to Pennsylvania, and from there to Ohio and to Sangamon County in 1835, settling in Gardner Town­ship. 
Service: Private; New Jersey. He enlisted in New Jersey, serving six years in Capt. William Helm's Company of Col. Israel Shriver's 2nd N. J. Regiment. 
Pension: S35205 (NJ) 
Marker: His name is on a bronze plaque in the south mall of the Old State Capitol, Springfield, placed by Springfield Chapters DAR and SAR, October 19, 1911. A bronze marker was placed on the grave by Springfield Chapter DAR, January 12, 1973. 
Sources: DAR, PI, PENSION, W 
CAMPBELL, JAMES 
Born: April 4, 1763 
Died: 1858, Williamson County, Illinois 
Service: Private 
Marker: Reported in old records by General Francis Marion Chapter, Marion, Williamson County. 
Sources: NSDAR 
CAMPBELL, SAMUEL R. 
Born: October 8, 1762 in Massachusetts
Died: November 8, 1844 
Buried: Hulse Cemetery, Pecatonica, Winnebago County, Illinois. Private Head­ stone 
Spouse: Grace Plum 
Service: Private; Mass. Vt. He served in Capt. John Spoor's Company, Col. John Brown's Regiment for three months; and seven days with Col. John Ashley's Regiment, under Lt. Moses Hubbard by order of Gen. John Fellows; and six days service with Capt. James Campbell. 
Marker: His grave was marked by Rockford Chapter DAR on May 26, 1908. 
Sources: DAR, HR, PI, W 
CANNADY, JOHN 
Born: March 14,1763 in King George's County, Virginia 
Died: December 15, 1836 
Buried: Cannady Home, Montgomery County, Illinois 
Spouse: Mary Shearer 
Residences: From Virginia, he moved to Kentucky, and from there to Montgomery County, Illinois. 
Service: Private; Virginia. He entered the service in Bedford County, Virginia, in September, 1781 in Capt. Charles Callaway's Company, Col. James Calla­way's Regiment; was at the siege of Yorktown and served three months. He enlisted in July,1782 with Capt. Abraham Kirkpatrick, Col. Christian Fehiger, serving six months. He was transferred to Capt. Charles Yarborough 's and Capt. Benjamin Dade's troops and was furloughed home when he became ill. He did garrison duty under Capt. Nathaniel Irish at New London, Camp­bell County, and was discharged in 1783. 
Pension: He was pensioned. Mary Rl654. Mary, widow of John, was living in Hillsboro, Montgomery County when her pension claim was rejected (Act July 7,1838) "For proof of service and marriage." 
Sources: PI , PENSION, W 
CAPPS, DEMPSEY 
Born: September 7,1760 in North Carolina 
Died: March 16, 1839 
Buried: Markley Cemetery, Fairview, Fulton County, Illinois 
Spouse: Sarah (Pool) Overman; born November 8, 1756; died May 18, 1840 Children; Elij ah 
Service: Private; North Carolina. He enlisted May 20, 1777from North Carolina in Capt. Reading Blount's Company, Col. Gideon Lamb's Regiment . He was discharged September, 1782. 
Pension: Sarah W22735(NC) 
Sources: PI, PENSION, County Record 
CARPENTER, BENJAMIN 
Born: About 1754 in Virginia 
Buried: Schuyler County, Illinois 
Service: Private; Virginia Continental troops. He enlisted May, 1776, and was pre sent at Cornwallis' surrender. 
Pension: S32156 (Va.) Illinois pension roll, October 19, 1833, Schuyler County, age 79 
Sources: PENSION, W 
CARR, HENRY 
Born: 1758 in Virginia (Prince William County) 
Died: 1822 
Buried: At New Design, Monroe County, Illinois 
Spouse: Elizabeth Alexander 
Service: Private; Virginia 
Sources: PI 
CARR (Karr), JOSEPH 
Born: March 21,1752 in Virginia 
Died: March 6,1817 
Buried: Turkey Hill Cemetery, on his farm, Freeburg Township, St. Clair County, Illinois 
Spouse: Priscilla Mary 
Residences: He came to Illinois in 1793, settling in Freeburg, St. Clair County 
Service: Corporal; Virginia 
Sources: NSDAR, PI, W 
CARRIGAN, JOHN 
Buried: Crooked Creek, east of Carlyle, Clinton County, Illinois 
Residences: He came to Illinois from Georgia and settled on Crooked Creek, six miles east of Carlyle, Clinton County. 
Service: Soldier; Georgia 
Sources: W, Clinton County History 
CARVER, CHRISTIAN 
Born: 1759, in Northampton County, Pennsylvania 
Died: March 14, 1836 
Buried: Carver Cemetery, Clear Lake Township, Sangamon County, Illinois 
Spouses: (1) Magdalina Ziegler (2) Mary Siegler 
Service: 	Private, North Carolina Continental Troops. He entered the service in Surry County, North Carolina, serving three months from August 1777 in Capt. Henry Smith's Company; and for three months in November 1777 in Capt. John Crouse's Company. 
Pension: Mary W27516 (NC); BLWT 73542-160-'55; Pension Roll, April 23, 1833, Sangamon County, age 74. 
Marker: His name is on a bronze plaque in the south mall of the Old State Capitol, Springfield, placed by Springfield Chapters DAR and SAH, October 19, 1911. 
Sources: DAR, NSDAH, PI, PENSION, W 
CASSADY, WILLIAM 
Buried: Rochester Township, Sangamnon County, Illinois 
Marker: His name is on a bronze plaque in the south mall of the Old State Capitol, Springfield, placed by Springfield Chapters DAR and SAH, October 19, 1911. "There were present at the unveiling ceremony, descendants of more than half the soldiers whose names are engraved upon the tablet; aged men and women came from long distances to attend the exercises given in honor of their Revolutionary ancestors ." 
Sources: DAR, NSDAR,W 
CAVALEY, A. W. 
Born: Virginia 
Died: Schuyler County, Illinois 
Service: Soldier; Virginia. He served as an agent for James Stewart's Artillery of Virginia. 
Pension: He was believed to be pensioned. 
Sources: W 
CESIRE, ANTONIE 
Born: Lachine, Canada
Died: 1779 Cahokia, St. Clair County, Illinois Children; Joseph 
Residences: He was listed in the early Census records of St. Clair County. 
Service: A Frenchman who served with Col. George Rogers Clark. Antoine was said to be the most important citizen in Cahokia in 1778. His son, Joseph, served with him. 
Sources: W 
CESIRE, JOSEPH 
Born: Lachine, Canada 
Died: In Cahokia, St. Clair County, Illinois 
Residences: He was listed in the early Census records of St. Clair County. 
Service: The son of Antonie Cesire, served with his father under Col. George Rogers Clark. Joseph was a Justice in 1781. 
Sources: W 
CHAFFIN (Chafin), ELIAS 
Born: 1760, probably in South Carolina 
Died: After 1834 
Buried: Clinton County, Illinois 
Residences: He came to Illinois before 1825, settling in Sugar Creek precinct, Clinton County. He served on the grand jury in 1825. 
Service: Private; South Carolina Militia 
Pension: S32172(SC); Pension Roll, April 21, 1834, Clinton County, age 71. 
Sources: PENSION, W 
CHANDLER, DANIEL 
Born: About 1761, probably in South Carolina 
Died: After 1833 
Buried: Jefferson County, Illinois 
Service: Private; South Carolina Continental troops. He entered the service in the Ninety-sixth District, South Carolina in February, 1776 under Capt. Jarret Smith, serving four months; in May, 1777, he served with Capt. James Lisle and Col. Jonas Beard for two months, and with Capt. Frederick Lipham for one month. In 1778 and 1779 he served with Capt. James Lisle and Col. John Lisle for over four months. In June, 1780, he served with Capt. William Smith and Col. John Thomas; was in the battles of Cedar Springs, and Musgrove Mill. He was in continuous service until the close of the war. He was made Lieutenant under Capt. Jermiah Williams and Col. John Hammond. 
Pension: S32175(SC); Pension Roll, October 22, 1833 
Sources: PENSION, W 
CHANDLER, JOSEPH 
Born: September10,1753inVermont  
Died: November 7, 1844, age 91, at the home of his son, Hiram 
Buried: Otterville-Noble Cemetery, Grafton, Jersey County, Illinois. Private Head­ stone 
Married: November 26,1779 
Spouse: Patience Mary Andrews, died January 15, 1829 at Marietta, Ohio 
Children: Hiram, Sally 
Service: Private; Connecticut. He served in the Battle of Bennington with his father, who was killed. He enlisted for seven months in Capt. John Sedwick's Company, Col. Benjamin Hinman's Regiment; enlisted in 1776 for one year in Capt. Theodore Woodbridge's Company, Col. Samuel Elmore's Regiment. He was discharged in April, 1777. 
Pension: S35206 (Conn.). He applied for a pension while a resident of Wooster, Washington County, Ohio. 
Marker: His grave has been marked with a tablet. 
Sources: DAR, HS, PI, PENSION 
CHANDOIN, JOHN 
Born: 1759 in Virginia 
Died: Williamson County, which was formed from Franklin in 1839. 
Spouse: Sarah 
Service: Private; Virginia Continental Troops 
Pension: Illinois Pension Roll, April 16, 1833, Franklin County, age 72. Sarah W22745 (Va.) 
Sources: PENSION, W 
CHAPIN , SAMUEL 
Born: September 24, 1760 in Vermont 
Died: February 13, 1842 
Buried: Oquawka Cemetery, near Oquawka, Henderson County, Illinois 
Spouse: Susannah Walbridge 
Residences: From Vermont he removed to New York, then to Pennsylvania, and from there to Indiana, and about 1839 to Knox County, Illinois. 
Service: Private; Vermont. He served from Vermont in 1777 in Capt. Parmelee Allen's Company; from June 16 to July 10, 1778 in Capt. Samuel Robinson's Company, and Col. Samuel Herrick' s Regiment. He served in Capt. Joseph Safford 's Company, Col. Ebenezer Walbridge' s Regiment from August 2, to August 8, 1781. 
Pension: S16706 (Vt.); Pension Census of June 1, 1840, Knox County, age 80, residing with Custavus Volvridge. 
Marker: In Oquawka Cemetery, Henderson County, placed by Mildred Warner Washington Chapter DAR, Monmouth, Warren County, on November 23, 1917.   
Sources: DAR, PI, PENSION, W 
CHAPMAN, DANIEL 
Born: July 25, 1763, Westchester County, New York 
Died: February 8, 1841 
Buried: Chapman Cemetery, Creal Springs, Johnson County, Illinois. Flat marble headstone 
Spouse: Lucretia Finch 
Service: Sergeant; New York. He enlisted in 1775 for nine months under Capt. Richard Sackett and Col. John Thomas; in the spring of 1776 he enlisted for nine months. In 1777 he acted as Scout and was appointed Second Sergeant under Col. Frederick Weisenfeldt, serving one year. In August 1781, they marched south to meet Lord Cornwallis in Virginia. 
Pension: Pension roll, February 15, 1833, age 77; Pension Census, June 1, 1840, Johnson County, age80; Lucretia W23794 (NY) 
Marker: Descendants and Daniel Chapman Chapter DAR, Vienna, marked the grave. 
Sources: DAR, HR, PI, PENSION, W 
CHARLES, ELUAH (Elisha) 
Born: December 17, 1750 in Maryland 
Died: 1830 
Buried: Moore Cemetery, St. Clair County, Illinois 
Spouse: Elizabeth (Isabella or Ibby) Moore 
Service: Patriotic; Service North Carolina. He moved to North Carolina in 1777 where he enlisted and served throughout the war as to General Greene's Army. 
Sources: NSDAR, PI 
CHARLEVILLE, FRANCIS 
Buried: Near Kaskaskia, Randolph County, Illinois 
Service: Captain; Virginia. Frenchman, appointed Captain by Col. George Rogers Clark. 
Pension: R13147(Va.) Half Pay 
Marker: His name is on a bronze marker on the grounds of Sparta High School, Sparta, placed in 1934 by Sparta Chapter DAR. 
Sources: DAR, NSDAR, PENSION 
CHARLEVILLE, JEAN BAPTISTE 
Buried: Near Kaskaskia, Randolph County, Illinois 
Residences: He lived in Kaskaskia after the close of the war and was the head of a family. 
Service: Served as an officer appointed by Colonel Todd with Col. George Rogers Clark. 
Marker: His name is on a bronze marker on the grounds of Sparta High School, placed by Fort Chartres Chapter DAR, Sparta, in 1934. 
Sources: DAR, NSDAR, W 
CHASE, PARKER 
Born: August 22, 1765 
Died: July 22, 1851 
Buried: DuPage County. Illinois 
Spouses: (1) Sarah Evans	 (2) Mary Hayes 
Service: Sergeant; New Hampshire*, Enlisted April 20, 1775, serving as a "Min­ute Ma n" in Capt. Thomas Noyes' Company; thirteen weeks in Capt. Ezra Lunts' Company, with Col. Moses Little's Seventeenth Regiment; again in Capt. Robert Dodge's Company, Col. Ebenezer Travis' Regiment; with Capt. Jacob Powers and Capt. Stephen Jenkins, with Col. Jacob Gerrish, from Suffolkand Essex Counties, Massachusetts. 
*The service given is Massachusetts from Mrs. Walker. 
Pension: Pension Census, June 1, 1840, DuPage County, age 77, residing with S. Dodge, head of family. Polly W903 (NH) BLWT 16271-160-55 
Marker: Grave located by Downers Grove Chapter DAH. His name is on a bronze tablet in the DuPage County Court House at Wheaton, Illinois. 
Sources: DAH, PI , PENSION, W 
CHERRY, GEORGE 
Born: 1761, probably South Carolina 
Died: March 2, 1843 
Buried: Oakdale Cemetery, Oakdale, Washington County, Illinois 
Spouse: Margaret Kirkpatrick 
Service: Private; South Carolina Militia. He served in Capt. Side's South Caro­ lina Volunteers 
Marker: His name appears on a bronze marker at Oakdale Cemetery, placed by Fort Chartres Chapter DAH, October 13, 1935. 
Sources: DAH, HH 
CHESHIRE (Chesier, Cheshier), JAMES 
Born: 1749Prince William County, Virginia 
Died: "Very aged," Fayette County 
Buried: Wilberton Township, Fayette County, Illinois 
Residences: He first settled in Gallatin County. 
Service: Soldier; Virginia. He served in the Virginia troops under Capt. William Farrow, Capt. Luke Cannon, Capt. John Hedges, Capt. Samuel Love, and Col. Lee. He was in the battles of Brandywine, Gates' defeat, Williamsburg, Cowpens and Yorktown. 
Pension: H1910 (Va). Residence Shawneetown, Gallatin County, Illinois, when pension claim suspended under Act of June 7, 1832, "Not on rolls-proof by witnesses. " 
Sources: DAH, PENSION, W 
CHILDERS (Childress), JOHN 
Born: 1765 Warren County, North Carolina 
Died: After 1841 
Buried: Howard Cemetery, Allison Township, Lawrence County, Illinois 
Spouse: Patience 
Service: Private; North Carolina Continental Troops. He enlisted in 1779 for three months with Capt. Howland Blanton and Col. Sewel; again for three months with the same officers. In 1781 he served with Capt. William Johnston. 
Pension: S30928(NC); Pension Roll, April 11, 1833, Lawrence County, age 71 
Marker: His name is on a bronze tablet in the Lawrenceville Court House, placed by Toussaint du Bois Chapter DAR in 1921. 
Sources: DAR, HR, PI, PENSION, W 
CHILDRESS, JOHN 
Born: July 22, 1766, probably North Carolina 
Died: November 8, 1837 
Buried: Old city cemetery, Carmi, White County, Illinois. Government Headstone (dedicated Sept. 21, 1964) 
Service: Private; North Carolina. He served in Col. William Moore's North Carolina Regiment. 
Pension: Pension Roll, White County, June1, 1840, age69 
Marker: His name is on a bronze plaque on a granite monument in city park bearing the names of Revolutionary War soldiers buried in White County. The monument was dedicated by Wabash Chapter DAR, Carmi, in October 1936. 
Sources: DAR, NSDAR, PENSION, W 
CHOATE, GREENBERRY 
Born: 1751 in Virginia 
Died: 1842 
Buried: Eagle Creek, near Equality, Gallatin County, Illinois 
Residences: He first settled in Johnson County but removed to Gallatin County. 
Service: Private; North Carolina; Virginia. He served one month in 1779 under Capt. William Cocke and Col. Andrew Christie in the North Carolina Troops. He enlisted in July, 1780 for four months under Capt. James Lawrence and Col. Andrew Christie; one month with Capt. Ezekiel Smith and Col. Thomas Clark in 1781. 
Pension: S32176 (NC); Pension roll Gallatin County, October 22, 1833, age 83 
Sources: PENSION , W 
CHRISTIAN, DANIEL, SR. 
Born: 1762 in Pennsylvania 
Died: December 26, 1847 
Buried: Oak Hill Cemetery, Mt. Carroll, Carroll County, Illinois 
Spouse: Elizabeth Nikirk 
Residences: He removed to Maryland and from there to Mt. Carroll, Carroll County. 
Service: Private; Pennsylvania. He enlisted at Reading in September, 1776, serv­ing two months under Capt. George Willis; two months with Capt. Kit; seven months in 1780 under Capt. John Spohn and Col. Thomas Butler. 
Pension: S8201 (Penn); Pension Census of June 1, 1840, age 85, residing with George W. Christian, head of family. 
Sources: HR, PI, PENSION, W 
CLAPP, ADAM 
Buried: St. John's Cemetery, near Jonesboro, Union County, Illinois 
Residences: He came to Union County (now Alexander) in 1809, settling on Sandy Creek. He served on the grand jury in Union County in 1818.  
Service: Soldier; North Carolina 
Pension: S30937 (NC) 
Sources: DAR, HR, PENSION, W 
CLARK. HAZEL (HarzilJa) 
Born: 1750 in Pennsylvania 
Died: September 23, 1840 
Buried: Old Sackett Cemetery, Salisbury Township, Sangamon County, Illinois 
Married; 1773 
Spouse: Nancy Wells 
Residences: Came to Sangamon County in 1821 and settled in Salisbury Town­ ship. 
Service: Private; Pennsylvania Militia. He served in Washington County, Penn­sylvania. His wife, Nancy, endured peculiar hardships during the war, being confined in a fort where for two weeks she subsisted on parched corn and water. 
Marker: His name is on a bronze plaque in the south mall of the Old State Capitol, Springfield, placed by Springfield Chapters DAR and SAH, October 19, 1911. 
Sources: DAR, PI, W 
CLARK, GEORGE W. 
Died: After 1840 
Buried: Bovee Cemetery, Cisne, Wayne County, Illinois 
Residences: He removed from Virginia to Kentucky and from there to Gallatin County, and later to Wayne County. 
Service: Virginia Troops 
Pension: Pension Census, Wayne County, June 1, 1840, age 84 
Sources: HR, PENSION, W 
CLARK. JAMES 
Born: April 18, 1755, in Rowan County. North Carolina 
Died: August 25, 1834 
Buried: Wayne or White County, Illinois 
Residences: He removed to Kentucky in 1801, and in 1818 arrived in Wayne County. 
Service: Captain and First Lieutenant; S.C. He enlisted in South Carolina August 1, 1776, serving one year under Capt. John Gowens; four months with Capt. John Earle; from October, 1777 to July, 1778, as First Lieutenant; from July, 1780 to June, 1781, for nine months, under Capts, William Wood, John Nesbit, Samuel Earle, Henry Wood and James McIllhaney, with Col. Thomas. He again served for three months from June, 1782, and was made Captain; and from August, 1782 for one month, making seven enlistments. He was wounded in the thigh at Hiamassee and Blackstock's. 
Pension: S32181 (SC) 
Marker: The name of James Clark appears on a monument placed by Wabash Chapter DAR, Carmi in 1936, honoring soldiers buried in White County. 
Sources: DAR, PENSION, W 
CLARK (Clarke), JOHN 
Born: 1765in Lancaster County, Pennsylvania 
Died ; September 13, 1844 
Buried: Probably Greene County, Illinois 
Service: Private; Virginia Continental Troops. He served from 1778 to the close of war, enlisting three times with Capt. Timothy Downing, Capt. Samuel Teeters and Col. Matthew Williamson and Col. William Crawford. He was in battle with Indians at the time Col. Crawford was captured. 
Pension: Pension Roll, Aug. 22, 1833, age 69; Pension Census of June 1, 1840, Greene County, age 75, residing with Absalom Clark, head of family.  
Sources: PENSION, W 
CLARKSON, CONSTANTINE 
Born: December18, 1762 inVirginia 
Died: March 1836 or December 21,1836 
Buried: Gillham Cemetery, Winchester, Scott County, Illinois. 
Private Headstone 
Spouse: Rhoda Johnson 
Residences: Removed to Morgan County, Illinois. Scott County was formed from Morgan in 1839. 
Service: Private; Virginia Continental Troops Pension: S32180(Va); Morgan County Pension Roll, August 10, 1833, age 71. 
Marker: His name appears on a plaque on the Morgan County Court House in Jacksonville, placed by the Reverend James Caldwell Chapter DAR in 1914. 
Sources: DAR, PI, PENSION, W 
CLAY, ELIJAH 
Born ; About 1759 
Died: After 1834 
Buried: Probably Edgar County, Illinois 
Service: Private; Virginia Militia Troops. He enlisted from Virginia in 1780; was in the battle of Guilford Court House. 
Pension: S32178(Va); Pension Roll, March 4,1834, Edgar County, age 75 
Sources: PENSION, W 
CLAY, TIMOTHY 
Born: November 1, 1761 
Died: August 6, 1831 
Buried: Simsbury Cemetery, Galesburg, Knox County, Illinois 
Spouse: Rhoda Lawson 
Service: Private; New Hampshire; Vermont. Enlisted July 13, 1779, for six months at Chester, Vermont; mustered in New Hampshire; served September 4, 1781 to December 21, 178l. 
Sources: NSDAR, PI 
CLENDENIN, JOHN 
Born: 1759/60 in Virginia 
Died: About 1836 
Buried: Family Cemetery, northeast of Chester, Randolph County, Illinois 
Spouse: Mary Sympson 
Residences: Removed to Green County, Kentucky, and then to Randolph County in 1808. He was listed in Marys River in the 1810 Census and in Randolph County when the 1820 and 1830 censuses were taken. 
Service: Virginia Militia. He served under Capt. Michael Cresap, Jr. 
Sources: CR, NSDAR, W 
CLIFFORD, MICHAEL 
Born: 1759 in New Jersey 
Died: 1835 
Buried: Sangamon County, Illinois 
Spouse: Elizabeth Service
Private: North Carolina Continental Troops. He enlisted in 1775, serving to the close of the war. He was in Capt. John Johnson's Company, Col. Francis Locke's Regiment; was in the battle of Pedee River, and the expedi­tion against the Cherokees in Tennessee. 
Pension: Clifford, Michael (NC); Elizabeth W25416; BLWT 26975-160-55; Pension Roll, December 12,1833, Sangamon County, age 74. 
Marker: His name is on a marker in the south mall, Old State Capitol, Spring­field, placed by Springfield Chapters DAR and SAR, October 19, 1911. 
Sources: DAR, NSDAR , PENSION, W 
CLINE (Kline), JONAS 
Born: ]une 21,1760, in Rochester, New York 
Died: ]uly 20,1840 
Buried: Private cemetery near Fairview, Fulton County, Illinois 
Married; February 8, 1787 
Spouse: Catherine Roos; born November 15, 1766; died after October 1843 
Children; Margaret, Elizabeth, Caty, Rachel, Benjamin, Sarah, and Mary 
Residences: Residence in 1780 was Shawangunk, Ulster County, New York; in 1805 he removed to Wayne County, Pennsylvania; to Beaver County, a total of eleven years in Pennsylvania, 
Service: Private; New York, Enlisted at Shawangunk, Ulster County, New York and served sixteen months from 1780 to 1782 in Capt. Robinson's Company, Col. Johnson's Regiment. 
Pension: He was granted a pension in 1832 while living in Richland County, Ohio; Pension Census, ]une 1, 1840, of Fulton County, age 75. Caty R2049 (NY); Caty, widow of ]onas, resident of Lewistown, Fulton County when pension claim rejected (Act July 7, 1838) "Not six months service proved," 
Sources: NSDAR, PI, PENSION, W 
COCHRAN (Coughran), JOHN 
Died: ]January 29, 1853 
Buried: Cochrane Cemetery, Ash Grove Township, Windson, Shelby County, Illinois 
Sources: HR 
COLLINS, EBENEZER 
Born ; 1750 in New York 
Died: 1840 in Lockport 
Buried: Probably Homer Township, Will County, Illinois 
Service: Corporal; Massachusetts; New Hampshire. He enlisted with Capt. Solo­mon Wadsworth in the Third Company, Fifth Regiment, called the Van Veghten Regiment. He enlisted [uly 10, 1775, and served as a Corporal in Capt, John Parkers' Company, Col. Timothy Bedel's Regiment of New Hamp­shire Rangers. He was discharged December 18, 1775, having served five months, 9 days, age 29. His residence was Hillsboro, Hopinton County. Occupation husbandsman. He drew a pension to time of his death. 
Pension: S42135; Mass. Pension List, Will County, Illinois, June 1, 1840, residing with]ohn V. Singer. 
Sources: PENSION, W 
COLLINS, HENRY 
Born: April11, 1764 in Massachusetts 
Died: April 10, 1847, aged 84 years 
Buried: Mt, Rest Cemetery, Wadsworth, Lake County, Illinois. Private Headstone 
Spouses: (1) Minerva Bacon  (2) Ruth Fifield 
Service: Private; Massachusetts. He enlisted March 3, 1781 in Southboro for three years under Capt. Isaac Newton. 
Pension: S12584 (Mass) 
Sources: HR, NSDAR, PI, PENSION, W 
COLLINS, WILLIAM 
Born: October 9, 1760 at Guilford, Connecticut 
Died: April 19, 1849 
Buried: Glenwood Cemetery, Collinsville, Madison County, Illinois. Private Head­stone 
Spouse: Esther Morris 
Residences: After the war he lived 40 years at Litchfield, Connecticut, and then moved to Madison County, Illinois. 
Service: Private; Connecticut. Enlisted April or May, 1777 at Guilford, Connec­ticut, in Col. Return J. Meigs' Connecticut Regiment, under Captains Humph­rey Barker, Mansfield and Potter. Service eight months. He was in the battles of Valentine's Hill and Delong’s Hill. In 1779 he served as a waiter to his uncle, Brigadier-General Augustus Collins and was stationed at Strat­ford and New Haven . 
Pension: S32184 Connecticut. He applied for a pension August 12, 1840 at Collins­ville, Madison County. 
Sources: HR, PI , PENSION 
COLLINSWORTH. JOHN 
Born: 1763in Virginia 
Died: September 6, 1838 
Buried: Probably St. Clair County, Illinois 
Residences: After the war he removed to Claiborne County, Tennessee, and from there he came to St. Clair County, Illinois. 
Service: Private, Marine; Virginia Continental Troops 
Pension: S30965 (Va); Transfer from Eastern Tennessee September 4, 1833; Pen­sion Roll, March 20, 1833, St. Clair County, age 70 
Sources: PENSION, W 
COLLOM (Collum), JONATHAN 
Born: December 10,1760, in Montgomery County, Pennsylvania 
Died: February 20, 1842 
Buried: Lumbrick Cemetery, east of Charleston, Coles County, Illinois. Private headstone 
Spouse: Martha 
Children: William 
Residences: After the war he removed to Washington County, Tennessee. His son, William, accompanied him to Coles County. 
Service: Drummer and Militiaman; Pennsylvania. In 1778 he was drafted for three months under Capt. Marpole and Col. Dowling. In 1779 he was drafted to serve in New Jersey with Capt. Dowling and Col. George Smith, serving as a drummer. He then served as a Minute Man. When Cornwallis marched through Virginia, he again enlisted, but became ill and was not present at the surrender. 
Pension: R2184; Pennsylvania 
Marker: Marker placed on Route 16, two miles east of Charleston by Sally Lincoln Chapter DAR, Charleston in 1932/33. 
Sources: DAR, HR, PI, PENSION, W 
COMBS, WILLIAM 
Born: Probably Virginia 
Buried: Probably Perry County 
Spouse: Sarah Ann 
Service: Soldier; Virginia. He served in the Virginia troops. 
Pension: Sarah Ann, his widow, applied for pensionafter1836. R2187 Sarah Ann (Va) Residence, Paris, Edgar County, when pension claim suspended (Act July 4, 1836) "Not six months service." 
Marker: His name appears in a list of Militia in the Court House at Chester, Randolph County. 
Sources: HS, PENSION 
CONREY, JOHN 
Born: April 22, 1760 
Died: September 12, 1834 .
Buried: Wynn Cemetery, near Chrisman, Edgar County, Illinois 
Spouse: Sarah Calvin 
Residences: Coming to Illinois, he settled in Edgar County at a place called Bloomfield Ledge. 
Service: Private; New York Continental Troops. He enlisted in 1776 for six months, 1777 for six months, and 1778 for six months. He was in Sackett' s Company, Graham, New York Militia, and in the Battle of White Plains. 
Pension: S32188 (NY); Pension Roll, August 14, 1833, age 74 
Sources: DAR, NSDAR, PI, PENSION, W 
CONWAY, JESSE 
Born: About 1760 probably in Virginia 
Died: About 1839, aged 79 years 
Buried: Greene County, Illinois 
Spouse: Margaret 
Service: Private; Virginia Continental Troops. He enlisted at Reed Island in 1777 for eighteen months and again in 1779 for sixteen months under Capt. Wil­liam Buchanan, Capt. Isaac Riddle, with Col. Boon and Col. Abraham Bowman . 
Pension: Madison County Pension Roll, August 22, 1833, age 73. Margaret WI0674 (Va) 
Sources: PENSION , W 
COOK, JOHN 
Born: December 25, 1761 in Hanover, Morris County, New Jersey 
Died: October 24, 1837 
Buried: Oakwood Cemetery, Joliet, Will County, Illinois 
Service: Soldier; New Jersey. He enlisted August, 1776, serving two years in the Companies of Capts. David Bates, Obadiah Kitchell, Elijah Squire, Benjamin Corey, William Ely, John Scud der, Levi Gardiner, Harrison Baldwin, Lewis Brant and David Lyon with Cols. Benoni Hathaway, Ellis Cook, Sylvanus Seeley and Moses Jacques in the New Jersey troops. 
Sources: HR, W 
COOLEY, JABEZ 
Born: 1729 at Springfield, Massachusetts 
Died: About 1800 at Springfield, Illinois 
Buried: Sangamon County, Illinois 
Spouse: Abigail Hancock 
Service: Private; Massachusetts. He served in 1775 in Capt. Gideon Burt's Com­pany of Col. Tornothy Danielson's Regiment. 
Sources: PI 
COOMBS, WILLIAM 
Born: About 1757 
Died: March 8, 1840 
Buried: Edgar County, Illinois 
Spouse: Mill Cloud 
Service: Orderly Sergeant; Virginia 
Sources: PI 
COOPER, JONATHAN E. 
Born: 1758 in Maryland 
Died: September 10, 1845 
Buried: Falkner Cemetery, Jerseyville, Jersey County, Illinois 
Spouse: Eleanor 
Residences: After the war he removed to Kentucky and then came to Illinois in 1835, settling four miles southwest of Jerseyville. 
Service: Private; Pennsylvania. He served in Pennsylvania as a drummer. 
Pension: Eleanor W6714 (Penn). He was pensioned while living in Kentucky. 
Marker: Marker placed in Falkner Cemetery, near Jerseyville, July 14, 1935 by Ninian Edwards Chapter DAR, Alton, Madison County. 
Sources: DAR, HR, PI, PENSION, W 
COPELAND, WILLIAM 
Born: About 1754 
Died: 1822 
Buried: About three miles southwest of Vienna, Johnson County, Illinois 
Service: Corporal; Virginia. He served in the Virginia Continental Line in Regi­ments commanded by Col. Daniel Morgan for three years, 1776-1779. 
Pension: BLW #1696 for 200 acres issued for services .
Sources: PI, PENSION 
CORDER, LEWIS 
Born: 1760 
Died: 1833 
Buried: Crab Orchard Cemetery, Marion, Williamson County, Illinois 
Spouse: Mary Garner 
Service: Private; North Carolina 
Sources: DAR, HR, PI 
CORNELISON (Carnelison, Conelison), JOHN 
Born: About 1758, probably in North Carolina 
Died: After 1840 
Buried: Madison County, Illinois 
Residences: Removed to Fayette County, Kentucky, from there to Greene County, Illinois, and then to Madison County. 
Service: Private; North Carolina. He enlisted in June 1778 and served in Com­panies commanded by Capt. John Armstrong and Capt. Matthew Ramsey in Regiments of Col. Robert Mebane, Col. Archibald Lytle, and Col. John McLean. He enlisted for four years in Companies of Capt. Smith, Capt. Adolph Hedrick, Capt. Francis Cole, Capt. John Childs, and Capt. Jennings. He was in the battle of Stono. 
Pension: S35209 (NC). He filed for a pension, which was allowed, while living in Kentucky. Pension Census, June 1, 1840, Madison County, age 82, residing with W. C. Johns, head of family. 
Marker: His name is on a bronze tablet at the Madison County Court House, Edwardsville, placed by Ninian Edwards Chapter DAR, Alton. 
Sources: DAR, PENSION 
CORRIGAN (Corgan), PATRICK 
Born: 1760 in Ireland 
Died: 1838 
Buried: Union County, Illinois 
Spouse: Polly Singleton 
Service: Private; Pennsylvania. He served in Capt. James Mercer's Company, 5th Battalion, Lancaster County, Pennsylvania Militia. 
Sources: DAR, PI 
COTTER, WILLIAM 
Born: 1762 
Died: After 1789 
Buried: Near Patterson, Greene County, Illinois 
Spouse: Catherine Vance 
Residences: Resident of Princess Anne County, Virginia 
Service: Private; Virginia 
Pension: He was granted land for his service 
Sources: HS, PI , PENSION 
COTTINGHAM, GEORGE 
Born: 1760 in Maryland 
Died: 1860, age 100 years 
Buried: Charleston, Coles County, Illinois 
Residences: Following the war, he removed to Kentucky, and in 1836 came to Coles County, Illinois. He was a shoemaker by trade and he is said to have made boots for General Washington. 
Service: Soldier; Maryland 
Sources: W, County History 
COTION (Cotten), JOHN 
Born: 1753 in South Carolina 
Died: June 7, 1835 
Buried: Adams County, Illinois 
Service: Private and Sergeant; South Carolina Continental troops. He enlisted at Camden under Capt. William McClintock, and Col. Thomas Sumpter. He was Sergeant of his Company, was wounded in the shoulder and discharged at Augusta, Georgia, three months after the close of the war. 
Pension: S35845 (SC); Pension roll, September 5, 1818 (dropped) 
Sources: PENSION, W 
COUCH, MELLINGTON (Wellington) 
Buried: Preston Presbyterian Cemetery, Preston (near Sparta), Randolph County, Illinois 
Residences: He lived in Marion County, Illinois, before settling in Randolph County 
Service: Virginia, probably. He was in the battle resulting in the surrender of Cornwallis. 
Marker: His name is on a bronze marker on the grounds of Sparta High School at Sparta, placed by Fort Chartres Chapter DAR, Sparta, in 1934. 
Sources: DAR, HR, NSDAR, W 
COUCHRAN (Coughran), JOSEPH 
Born: January 16, 1761, in Virginia 
Died: March 19, 1845 
Buried: Vermilion County, Illinois 
Spouse: Prudence 
Service: Soldier; Virginia. He enlisted in June, 1781 in Hampshire County, Vir­ginia, with Capt. James Anderson, Capt. Alexander Dick, Capt. Isaac Parson, and Col. Edwards, serving for six months. 
Pension: He applied for a pension in Vermilion County in 1834, Prudence W1712 (Va). In the 1840 Pension Census, he was living in Vermilion County, age 74 with William Fields. 
Sources: PENSION. W 
COULTER, ROBERTSTUART 
Born: 1760 in Winsboro, North Carolina 
Died: 1821 in Madison County, Illinois 
Spouse: Margaret Fleming 
Service: Drummer, Private; South Carolina. He served in the Revolutionary War as a Drummer at the age of 14, and was a Private at the age of 16. 
Sources: DAR, PI 
COVELL, HENRY 
Born: About 1747 in Connecticut 
Died: After September 1832 
Buried: Adams County, Illinois 
Spouse: Polly Reed 
Residences: Following the war he removed to New York City, and in 1832 came to Adams County, Illinois. 
Service: Private; Connecticut; Massachusetts; New York. He enlisted at Danbury, Connecticut, as a "Minute Man" in April, 1775, serving until December with Capt. Noble Benedict and Col. David Waterbury. He enlisted in 1781 for one year and five months under Capt. Solomon Woodworth and Col. Marinus Willett of New York; marching from Fort Plain to German Flats, where on September 7, 1781 his Company was killed or captured by Indians. Covell, with four other soldiers, were taken to Fort Niagara and turned over to the British. He was confined until December, 1782, when he was sent to Boston and was discharged. 
Pension: S30954 (Conn; Mass); Pension Roll, July 24, 1832, Adams County, age 87 
Sources: PI, PENSION, W 
COY, CHRISTOPHER 
Born: After 1753 in Maryland 
Died: October 12, 1839 
Buried: Melton or Spring Hill Cemetery, Bridgeport, Lawrence County, Illinois 
Spouse: Elizabeth 
Residences: He removed to Kentucky and from there to Lawrence County. 
Service: Private; Maryland. He enlisted in 1779 under Capt. Henry Gaither and Col. William Smallwood, serving to the close of the war. He was in the siege of Yorktown. 
Pension: Elizabeth W9798 (Md) 
Marker: His name appears on a bronze tablet in the Lawrenceville Court House placed by Toussaint du Bois Chapter DAR in 1921. 
Sources: DAR, HR, PI, PENSION, W 
CRABTREE, JAMES 
Born: About 1760 
Died: After 1833
Buried: Washington County, Illinois 
Service: Private; Virginia Continental troops 
Pension: S32195 (Va); Pension Roll, Washington County, December 17, 1833, age73 
Sources: PENSION, W 
CRABTREE, JOHN 
Born: May 3, 1763 in Randolph County, North Carolina 
Died: October 24, 1835 
Buried: Clear Springs Cemetery, Montgomery County, Illinois 
Spouse: Mida 
Residences: He was one of the early settlers in what was known as "Street Settlement" about four miles from Hillsboro.
Service: Soldier; North Carolina. He entered the service in 1780 under Capt. Edward Williams; he again enlisted under Capt. John Knight. 
Pension: R2419 (NC). Residence, Hillsboro, Montgomery County, when pension claim suspended (Act June 7, 1832) "For further details of service, period, length, grade, localities and officers. " 
Sources: PI, PENSION, W 
CRAIG, THOMAS, SR. 
Born: October, 1762 in Granville County, North Carolina 
Died: After 1833 
Buried: Craig Cemetery, Hurricane Township, Montgomery County, Illinois 
Residences: He settled in East Fork Township, Montgomery County. 
Service: Soldier; North Carolina. He enlisted in 1781, serving in Capt. Smith's Company, Col. McKissick's Regiment. He re-enlisted in Lincoln County in the Indian spy service, under Capt. Brown Stimson and Capt. John Sevier. 
Pension: S30971 (NC); Pension Roll , February 28, 1833, Montgomery County, age 71 
Sources: PENSION, W 
CRAIG, WILUAM 
Buried: Swango Cemetery, Symmes Township. Edgar County, Illinois 
Service: Private; Virginia. He served in Capt. Uriah Springer's Company, Col. John Gibson's Regiment in 1780. He was discharged at Fort Pitt. 
Sources: HS 
CRANE,NOAH 
Born: About 1745 
Died: After January 6, 1836 
Buried: Armstrong Cemetery, Allendale, Wabash County, Illinois 
Spouse: Phebe 
Service: Patriotic 
Service: New Jersey 
Pension: R2448 (NJ). Residence Armstrong, Wabash County when pension claim rejected (Act June 7,1832) "Not six months service." 
Sources: NSDAR, PI, PENSION 
CRAW, REUBEN 
Buried: Craw Cemetery, Sadorus, Champaign County, Illinois 
Service: Private; Connecticut Continental 
Pension: S38643(Conn. Cont.) 
Sources: HR, PENSION 
CREED,COLBAY 
Born: May 4,1758 in Orange County. Virginia 
Buried: Cass County, Illinois 
Residences: Upon coming to Illinois he first settled in Morgan County. 
Service: Soldier; North Carolina. He enlisted in Surrey County. North Carolina, in Capt. Giddings" Company of the Militia. 
Pension: S32194(NC)
Sources: PENSION, W 
CROSE, PHILIP 
Born: 1757 in Hampshire County, Virginia 
Died: July 4, 1848 
Buried: Miller Cemetery (abandoned), Heyworth, McLean County, Illinois 
Spouse: Priscilla Beck 
Children; Solomon, Michael, John, Henry, Philip, Jr., William, Barbara Hand, Mary Alkire, Jane Miles, Rebecca Brock 
Residences: After the war he moved to Illinois, settling in Shawneetown, Gallatin County. He removed to Indiana, where he applied for and was granted a Pension; he returned to Illinois in 1836, settling in Randolph Township, McLean County. 
Service: Private; Virginia; North Carolina. He enlisted from Hampshire County, Virginia, and served in Captain Daniel Richardson's Company for six months in 1780, and for four months in 1781. He was in the battle of Guilford Court House. 
Pension: S32193 (Va) 
Marker: A bronze Revolutionary marker was placed on the Crose family lot in Pennell Cemetery, two miles north of Towanda, Money Creek Township, on June 14, 1956 by Letitia Green Stevenson Chapter DAR, Bloomington. His name is also on the Soldiers monument in Miller Park, Bloomington. 
Sources: DAR, HR, PENSION. W 
CROSS, ZACHARIAH 
Born: March 25,1761 in Baltimore County, Maryland 
Died: February 27,1838 
Buried: Burnt Prairie Cemetery, Burnt Prairie, White County, Illinois. Govern­ment Headstone 
Spouse: Easter Hetty Johnston 
Service: Private; North Carolina. He enlisted in North Carolina in Capt. William Hick's Company, Col. Isaac Shelby's Regiment; in 1779 for two months in Capt. Maxwell's Company with Col. Shelby. He again enlisted in 1781 and 1782. 
Pension: Easter (Ester) R2519; Esther (deceased), widow of Zachariah, residence Carmi, White County, when pension claim rejected (Act July 7, 1838) "Not a widow at date of the act-died before August 23, 1842." 
Marker: His name is on a monument placed by Wabash Chapter DAR, Carmi, in 1936. The government marker was placed by the Chapter and descendants in 1931 and rededicated September 21, 1964. 
Sources: DAR, HR, HS, NSDAR, PI, PENSION 
CROW, WILLIAM 
Born: 1758 in Rockingham County, Virginia 
Died: January 25, 1854 
Buried: Private cemetery in Limestone Township, near Pottstown, Peoria County, Illinois 
Service: Private; Virginia 
Pension: S32196 (Va). He was listed in the 1840 Pension Census, Peoria County, age 81, living with James Crow. 
Sources: NSDAR, PENSION, W 
CROWDER, PHIUP 
Born: April 7, 1760 near Petersburg, Virginia 
Died: February, 1844 
Buried: In Crowder family cemetery, west of Springfield, Sangamon County, Illinois 
Spouses: (1) Susan Parish
	 (2) Rachel Saunders
	 (3) Sally Chandler 
Service: Sergeant; Virginia. Philip served for an elder brother with a family who was drafted. He was present at the surrender of Cornwallis. 
Pension: S30974 (Va); Pension Roll, Sangamon County, April 23, 1833, age 73. Pension Census, Sangamon County, June 1,1840, age 80 
Marker: His name is on a bronze plaque in the south mall, Old State Capitol, Springfield, Sangamon County, placed by Springfield Chapters DAR and SAR, October 19, 1911. A bronze marker was placed on the grave by Springfield Chapter, October 17, 1970. 
Sources: DAR, NSDAR, PI, PENSION, W 
CRUTCHER, HENRY 
Born: About 1740 
Died: 1807 
Buried: Near Kaskaskia, Randolph County, Illinois 
Spouses; (1) Susanna 
	 (2) Martha Beazley 
Service: Quartermaster; Virginia. He served with Col. George Rogers Clark and rendered service by purchasing treasury notes to aid in prosecuting the war. 
Marker: His name is on a bronze marker on the grounds of Sparta High School, Sparta, placed by Fort Chartres Chapter DAR, in 1934. 
Sources: DAR, NSDAR, PI, W 
CUMMINGS, JOSIAH 
Born: Connecticut 
Buried: Mississippi Township, Jersey County, Illinois 
Spouse: Mrs. Gillis 
Service: Soldier; Vermont. He was at the Battle of Bennington, Vermont, and served in General Wayne's campaign against the Indians. He participated in the defeat of the Army of General Arthur St. CLair. 
Sources: HS 
CURRY, JAMES 
Died: Randolph County, Illinois 
Residences: His residence was on Nine-Mile Creek, Randolph County 
Service: Soldier; Virginia. He served under Col. George Rogers Clark and was chosen to undertake any hazardous service. He was believed to have been killed by the Indians. 
Marker: His name is on a bronze marker on the grounds of Sparta High School, Sparta, Randolph County, placed by Fort Chartres Chapter DAR in 1934. 
Sources: NSDAR, W 
CURRY, NICHOLAS 
Died: September 16, 1848 
Buried: McAleb Cemetery. north of Bluffs, Scott County, Illinois. Private Head­stone 
Residences: He lived in Lincoln County, Tennessee, arrived in Illinois in 1832 settling in Coles County. He then moved to Scott County. 
Service: Soldier; South Carolina 
Sources: W 
CUTRIGHT (Cartwright), PETER 
Born: 1759 in Hampshire County, Virginia 
Died: Probably 1841 
Buried: DeWitt County. Illinois 
Spouse: Christiana Carbin 
Residences: He resided for a time in Macon County, Illinois, where he applied for a pension, which was granted in 1833. He lived in Sangamon County in 1834 and in DeWitt County in 1840. 
Service: Soldier; Virginia Militia. He enlisted September 1, 1780 and served for six months with Capt. Daniel Riteson and Capt. Robert Cravens, in Col. Robert Stevens Regiment. 
Pension: S32164 (Va), Pension Roll, Sangamon County, Illinois, May 3, 1834, age 74; Pension Roll, DeWitt County, June 1, 1840, age 81, residing with Samuel Cartwright, head of family. His last pension payment was September 4, 1841. Christiana R1759 
Sources: DAR, PI, PENSION, W 


DAGLEY, THOMAS 
Born: About 1755 
Died: After 1812 
Buried: Union Ridge Cemetery, Norris City, White County, Illinois. Government Headstone 
Service: Patriotic Service;  North Carolina. He served as "Baggage Master" under General Washington and sold provisions to the Army. 
Marker: His grave was marked many years ago by Samuel Elder Chapter DAR, at Eldorado. His name is on a monument placed by Wabash Chapter DAR, Carmi, in 1936. 
Sources: DAR, HR, NSDAR, PI, W 
DAMERON (Damron), JOHN
Born:  January 12, 1757 
Died: April 15, 1835 
Buried: Drake Cemetery, Creal Springs, Williamson County, Illinois. Government Headstone 
Spouses: (1) Anna Ladd
	 (2) Cynthia Thompson 
Service: Soldier; Virginia 
Sources: HR, PI 
DANIEL, ARCHIBALD
Born: About 1764 in Wilmington, Bladen County, North Carolina 
Died: 1844 
Buried: Williamson County, Illinois 
Residences: Coming to Illinois, he settled in Franklin County, but removed to Gallatin County, and from there to Williamson County.
Service: Soldier; North Carolina
Pension: S32204 (NC); Pension Roll, Franklin County, March 21,1833, age 69 
Sources: PENSION, W 
DANIS, JEROME 
Died: Near Kaskaskia, Randolph County, Illinois 
Service: He was a Frenchman who served under Col. George Rogers Clark 
Marker: His name is on a bronze marker on the grounds of Sparta High School placed in 1934by Fort Chartres Chapter DAR, Sparta. 
Sources: DAH, NSDAH, W 
DANIS (Daney), JOSEPH 
Died: Near Kaskaskia, Randolph County, Illinois 
Service: He was a Frenchman who served under Col. George Rogers Clark 
Marker: His name is on a bronze marker on the grounds of Sparta High School placed in 1934by Fort Chartres Chapter DAR, Sparta. 
Sources: DAR, NSDAR , W 
DANIS, MICHAEL
Died: Near Kaskaskia, Randolph County, Illinois 
Service: He was a Frenchman who served under Col. George Rogers Clark 
Marker: His name is on a bronze marker on the grounds of Sparta High School placed in 1934by Fort Ch artres Chapter DAR, Sparta . 
Sources: DAR, NSDAH, W 
Carmi, in 1936. 
Sources: DAR, HR, NSDAR, PI, W 
DAVIS, AQUILLA 
Born: St. Mary's County, Maryland 
Died : August 15, 1831 in Elkhart, Logan County, Illinois 
Buried: Wolf Creek Cemetery, near Sherman, Sangamon County 
Residences: He moved with his parents at an early age to Farquier County, Virginia. He came to Illinois in 1820, settling near Elkhart. removed to Fancy Creek Township, then back to Elkhart. 
Service: Private; Virginia Continental Troops. He enlisted March 19, 1781 under Lt. Robert Craddock and Lt. Luke Cannon, in Col. Thomas Posey's Regiment. 
Pension: S35882 (Va); Illinois Pension Roll, Sangamon County, June 29, 1824. 
Marker: His name is on a bronze plaque in the south mall, Old State Capitol, Springfield, placed October 19, 1911 by Springfield Chapters DAR and SAR. 
Sources: DAR, PENSION, W 
DAVIS, HEZEKIAH 
Born: 1760 
Died: 1834 
Buried: Davis-Allen family cemetery, Murphysboro Township, Jackson County, 
Illinois Service: Private; South Carolina Continental Troops; Infantry and Cavalry 
Pension : Pension Roll, October 21,1833, Jackson County, age 73. 
Sources: PENSION 
DAVIS, HEZEKIAH 
Born: South Carolina Died: 1820 
Buried: White County, Illinois 
Residences: He settled in Jackson County, Illinois, in 1811, but removed to Burnt Prairie, White County. Service: 
Soldier; South Carolina. He served from South Carolina and was wounded. 
Pension: He was pensioned. 
Marker: His name appears on a monument at Carmi, White County, placed by Wabash Chapter DAR in 1936. 
Sources: DAR, W 
DAWSON, JOHN 
Born: July 28, 1750 in Stafford County, Virginia 
Died: December 7, 1839 
Buried: Jacksonville Cemetery (East), Jacksonville, Morgan County, Illinois. Government Headstone 
Spouse: Susan Taylor 
Service: Private; Virginia Militia. He enlisted in September, 1775 in Stafford County, Virginia, under Capt. George Williams; from April 1776 for three months with Capt. George Burrows; for one year with Capt. John Mountjoy; four weeks with Capt. John James; from June, 1781 for four months with Capt. George Burrows; in the fall of 1781 for four months with Col. Joseph Phillips. 
Pension: S32206 (Va), Pension Roll for Morgan County, May 22, 1834, age 83 Marker: His name is on a plaque in front of the Morgan County Court House placed by the Reverend James Caldwell Chapter DAR, Jacksonville in 1914. 
Sources: DAR, HR, PI, PENSION, W 
DAY, DANIEL R.
Born: January, 1763 in Keene, New Hampshire 
Died: 1838 
Buried: Daysville Cemetery, Oregon, Ogle County, Illinois. Private Headstone 
Service: Soldier; New Hampshire. He enlisted April 4, 1780, serving until under Lieutenant Benjamin Ellis and Col. Henry Dearborn.
Sources: HR, W 
DAY, EDWARD D., SR. 
Born: 1760 in Charlotte County, Virginia 
Died: April 11, 1837 
Buried: DeWitt Cemetery (old), DeWitt, DeWitt County, Illinois 
Spouse: Ursula Sublette 
Residences: Edward Day was the grandfather of Honorable W. H. Herndon, a law partner of Abraham Lincoln.
Service: Private; Virginia Militia. He enlisted from Charlotte County, Virginia in Capt. Charles M. Collier's Company, Col. David Morgan's Regiment when sixteen years of age, serving five months; in Capt. William Price's Company, Col. Thomas M. Randolph's Regiment, three months; in Capt. Gideon Spen­cer's Company, Col. Randolph's Regiment, two months. 
Pension: S32200 (Va); Illinois Pension Roll, Sangamon County, Illinois, February 15, 1834, age 73
Marker: His grave was marked by the DeWitt Clinton Chapter DAR, Clinton, October 3, 1969. 
Sources: DAR, HR, NSDAR, PI, PENSION, W 
DECK, MICHAEL 
Born: February 7, 1759 in Rockingham County, Virginia 
Died: April 3, 1843 
Buried: Deck Cemetery, Marine, Madison County, Illinois. Private Headstone 
Married: April 25, 1790 
Spouse: Susan (Susannah) Monger; born April 10, 1759 
Children: Thirteen 
Service: Private; Virginia. He enlisted May 5, 1778 under Capt. Robert Craven; in 1781 under Capt. Michael Coker. He was at the siege of Yorktown. 
Pension: Madison County, Illinois, Pension Roll, April 6, 1833, age 73. He was listed in the 1840 Pension Census, Madison County, age 82. Susannah W22935 (Va). 
Marker: His name is on a bronze tablet at Edwardsville Court House placed by Ninian Edwards Chapter DAR, September 16, 1912. 
Sources: DAR, HR, PI, PENSION, W 
DeHAVEN, SAMUEL C. 
Died: March 23, 1858 
Buried: Rock Cemetary, Sadorus, Champaign County, Illinois
Sources: HR
DENISON (Dennison), WILLIAM, SR.
Born: July 15, 1767 probably in Connecticut Died: November 30, 1838 
Died: November 30, 1838 
Buried: Dennison Cemetery, Bridgeport, Lawrence County, Illinois. Private Headstone 
Service: Soldier; Connecticut. Revolutionary War soldier from Connecticut. He served at Yorktown under Lafayette. 
Marker: He is named on a bronze tablet at the Lawrenceville Court House, placed by Toussaint du Bois Chapter DAR in 1921. 
Sources: DAR, HR, W
Sources: HR 
DENISON, WILLIAM, SR. 
Died: March 23, 1858
Buried: New Boston Cemetery (old section), New Boston, Mercer County, Illinois 
Spouse: Rachel John 
Service: Private; Pennsylvania. He served in the Westmoreland County, Pennsyl­vania Militia. 
Marker: His grave has been marked by the William Dennison Chapter DAR, Aledoand the Mildred Warner Washington Chapter DAR, Monmouth, October 6, 1916. 
Sources: DAR, HR, NSDAH, PI 
DENNY (Denney), CHARLES
Born: December 25,1759 in Pauldmgstown, Dutchess County, New York
Died: August 6, 1839 Buried: St. Johns Evangelical Lutheran Cemetery, or Prairie Memorial Cemetery, Mokena, Will County, Illinois 
Spouse: Lucinda Allen 
Service: Private; New York. He enlisted in 1777 and served nine months under Captains Noah Wheeler and Seth Wheeler, Colonel Roswell Hopkins, in the New York troops. 
Pension: S32213 (NY) 
Sources: HR, NSDAR, PI, PENSION, W 
DEVOIR (DeVoir), LUKE 
Born: Probably New Jersey 
Died: April, 1827 
Buried: Pope County, Illinois 
Service: Private; New Jersey. He served in Capt. Peter Dickinson's Company, Third Battalion, New Jersey troops. 
Pension: S35240(NJ); Illinois pension roll, Pope County, June 6, 1820 
Sources: DAR, PENSION, W
DIAMOND, JOHN 
Born: February 15, 1750 in County Derry, Ireland 
Died: After 1833 
Buried: Diamond Cemetery, near Greenville, Bond County, Illinois 
Children: Daughter, Mary Inman 
Residences: He resided in Georgia for twenty-five years after the war, then in the Cherokee Nation, and thirteen years residence in Tennessee before moving to Fayette County, Illinois. He died in Bond County. 
Service: Private; South Carolina. He enlisted in Camden District, South Carolina before the Briar Creek Battle; served seven weeks under Major Frank. Ross, of Col. Nears Mounted Regiment Was in Cherokee Indian engagement in Georgia; six months in Capt. Patton's Company. 
Pension: S30991 (SC). He applied for a pension September 4, 1832, age 82, Fayette County, Illinois; Illinois Pension Roll. Fayette County, April 16, 1833, age 82 
Sources: HR, NSDAR, PENSION, W 
DICKERSON, KINZER
Born: 1757 in Maryland 
Died: 1837 
Buried: Near Danville, Vermilion County, Illinois 
Service: Soldier; Pennsylvania. He enlisted in 1778 in the Pennsylvania Line of troops, serving one month under Col. Daniel Broadhead; in 1779 for two months in the Company of Captain Uriah Springer in General Lachlen MeIntosh's Regiment. He again enlisted in 1782 for six weeks with Capt. John Crawford and Capt. J. Bean; one month with Capt. George Jenkins in Wheeling , Virginia. 
Pension: B2933 (Penn) Residence, Danville, Vermilion County, when pension claim rejected (Act June 7,1832) "Not six months service." 
Marker: His name appears on a plaque on a Minute Man and drinking fountain in front of the Post Office at Danville, placed by Governor Bradford Chapter DAB, Danville, Vermilion County.
Sources: DAR, NSDAR, PENSION, W
DICKEY, WILLIAM 
Born: May 6, 1764 
Died: June 26, 1832 
Buried: Friends Creek Cemetery, northeast of Argenta, Macon County, Illinois 
Spouse: Mary Stephenson 
Residences: He carne to Illinois in 1829 and settled in Macon County . 
Service: Gunner: Virginia. He enlisted under Capt. William Waters in the First 
Artillery Regiment, commanded by Col. Charles Harrison, serving for three years . Marker: His grave was marked by Stephen Decatur Chapter DAR on June 6, 1912 
Sources: DAR, NSDAR, PI, W 
DINGMAN, JAMES 
Born: 1758 in Northampton County, Pennsylvania Died: September 3, 1836 Buried: Family Cemetery, east of Riverton, Sangamon County, Illinois. Private Headstone 
Service: Soldier;  Pennsylvania. He enlisted in 1778 in Northampton County, Pennsylvania in Captain John Van Etten's Fourth Company, Col. Jacob Stroud's Regiment, Sixth Battalion. 
Marker: His name is on bronze plaque in the Old State Capitol, Springfield, placed October 19, 1911 by Springfield Chapters DAR and SAR. His grave has been marked by Sergeant Caleb Hopkins Chapter DAR, Springfield. 
Sources: DAR, HR, NSDAR, W 
DODGE, JOHN 
Born: Probably Connecticut 
Died: Before 1800 Buried: Probably near Kaskaskia, Randolph County, Illinois 
Service: John Dodge was a trader at Sandusky, Ohio, before the Revolutionary War. He favored the cause of the Colonists and was arrested by the British who took him to Detroit and from there to Quebec. He escaped in 1779. In 1779 he was recommended by Washington to Congress as a man who would be useful in the West. He was appointed Indian Agent in Virginia. Corning to Kaskaskia, Illinois, he rendered aid to General George Rogers Clark. 
Marker: His name is on a bronze marker on the grounds of Sparta High School, Sparta, placed by Fort Chartres Chapter DAR in 1934. 
Sources: DAR, NSDAR, W 
DODSON, THOMAS 
Born: About 1728 
Died: June 2, 1780 (?) 
Buried: Combs Cemetery, Plainview, Macoupon County, Illinois. Private Head­stone 
Spouse: Esther 
Service: Patriotic Service; Maryland 
Sources: HR, PI 
DOLAHIDE, FRANCIS 
Born: 1750 Caswell County, North Carolina 
Died: August 30,1837 
Buried: On a farm in Section 5, Carmi Town shi p, White County, Illinois 
Residences: He located in Hamilton County, Illinois. 
Service: Private; North Carolina Continental troops. He enlisted in 1776 from Caswell County, North Carolina, for several periods of service; in 1781 he re-enlisted, serving to the close of the war. He served under Captains William Morrow, Small, Christopher Taylor, and Samuel Sexton in Regiments com¬manded by Col. Archibald Lytle and Major Dugan. He was in the battles of Eutaw Springs and Yorktown. He served six years. 
Pension: S32220 (NC); Pension Roll, November 15,1832, Hamilton County, age 84. 
Marker: His name is on a marker in Carmi, placed by Wabash Chapter DAR ill 1936 and dedicated September 21, 1964. 
Sources: DAR, PENSION, W
DOLLAR, WILLIAM
Born: 1743 
Died: September 6, 1838 
Buried: Shields Chapel Cemetery, west of Canton, Fulton County, Illinois 
Married: About 1780 in Virginia 
Spouse: Ruth Beezley; died February 19, 1842 
Service: Private; Virginia. He enlisted October 15, 1776, serving until December 
7, 1779 with Captains Alexander Morgan, Marquis Chalmers and Thomas Collet, in Col. Christian Fehigers Regiment. 
Pension: Pension Roll, January 16, 1824, Fulton County; transferred from Indiana Ruth W22965 (Va)
Marker: His grave has been marked by Farmington Chapter DAR 
Sources: DAR, PI, PENSION, W
DONAWAY, CHARLES 
Buried: Probably Gallatin County, Illinois 
Service: Soldier; Virginia Pension: R3010 (Va). Residence Shawneetown, Gallatin County, Illinois, when pension claim suspended (Act June 7,1832) " not on rolls; no proof of service." 
Sources: DAR, PENSION 
DORTCH, ABEL 
Born: About 1759, Mecklenburg County, Virginia 
Died: About 1835 Buried: Franklin County, Illinois 
Residences: After the war he removed to Tennessee, and from there to Franklin County , Illinois. Service: Private; Virginia Continental Troops 
Pension: S32221 (Va): Illinois Pension Roll, Franklin County, September 25, 1833, age 73 
Sources: PENSION, W 
DOSHER (Dosier), ADAM
Born: October 24, 1759 in Virginia 
Died: October 6, 1841 
Buried: Sumpter Cemetery, Carmi Township, White County, Illinois. Private Headstone 
Service: Soldier; Virginia. He served in the Continental army for seven years under General Washington. 
Sources: Letter and Tazewell County History 
DOTY, EPHRAIM 
Born: About 1760 in New Jersey 
Died : August , 1831 
Buried: Doty farm cemetery, Oraville, Jackson County, Illinois. Private Headstone 
Residences: He came to Illinois in 1837. 
Service: Private. He fought in the battle of Bunker Hill, was at the siege of Yorktown, and was in the raid upon tea ships at Boston Harbor. 
Sources: HR, also newspaper article 
DOUGHTON, WILLIAM 
Born: 1760 Died: December 1, 1833 
Buried: Wabash County, Illinois 
Residences: He came to Illinois in 1820 from West Virginia, settling in Wabash County. 
Service: Private; Virginia Line of troops Pension: S32217 (Va), Illinois Pension Roll, Wabash County, October 23, 1833, age 72 
Sources: PENSION, W 
DORTCH, ABEL 
Born: About 1759, Mecklenburg County, Virginia 
Died: About 1835 
Buried: Franklin County, Illinois 
Residences: After the war he removed to Tennessee, and from there to Franklin County, Illinois. 
Service: Private; Virginia Continental Troops 
Pension: S32221 (Va); Illinois Pension Roll, Franklin County, September 25, 1833, age 73 
Sources: PENSION, W
DOSHER (Dosier), ADAM
Born: October 24, 1759 in Virginia 
Died: October 6, 1841 
Buried: Sumpter Cemetery, Carmi Township, White County, Illinois. Private Head stone 
Service: Soldier; Virginia. He served in the Continental army for seven years under General Washington. 
Sources: Letter and Tazewell County History
DOTY, EPHRAIM 
Born: About 1760 in New Jersey 
Died: August, 1831 
Buried: Doty farm cemetery, Oraville, Jackson County, Illinois. Private Headstone 
Residences: He came to Illinois in 1837. 
Service: Private. He fought in the battle of Bunker Hill, was at the siege of  Yorktown, and was in the raid upon tea ships at Boston Harbor. 
Sources: HR, also newspaper article
DOUGHTON, WILLIAM 
Born: 1760 Died: December 1, 1833 
Buried: Wabash County, Illinois 
Residences: He came to Illinois in 1820 from West Virginia, settling in Wabash County. 
Service: Private; Virginia Line of troops 
Pension: S32217 (Va), Illinois Pension Roll, Wabash County, October 23, 1833, age 72 
Sources: PENSION, W 
DOWDEN, CLEMENTIUS
Born: January 11, 1762 Prince George County, Maryland 
Died: After September 1836 
Buried: Elmwood Cemetery, Peoria, Peoria County, Illinois. Private Headstone (shows May 1835) 
Spouse: Martha Wells 
Service: Sergeant: Maryland: Pennsylvania. He served both in Maryland and Pennsylvania with service under Captain Bell and Colonel Vanland. 
Pension: S30995 (Md:Pa) 
Marker: His grave in Elmwood Cemetery has been marked by Peoria Chapter DAR. 
Sources: DAR, HS, NSDAR, PI, PENSION 
DOWNING, JOHN 
Born: After 1762 
Died: December 18, 1838 
Buried: Bower Templeman Cemetery, Logan County, Illinois 
Spouse: Hannah Frakes 
Children: Robert, John, Josiah, Sarah, Hannah, James, Nathan, William, Mary 
Residences: He removed from Pennsylvania to Ross County, Ohio, about 1800. Service: Private; Pennsylvania. He served in Capt. James. Scott's Eighth Com­pany, Third Battalion, Washington County, Pennsylvania Militia. Marker: His name appears on a plaque on the Logan County Court House at Lincoln, placed by Abraham Lincoln Chapter DAR, June 27, 1975. 
Sources: DAR, PI , also family records 
DOYLE, JOHN 
Buried: Randolph County, Illinois 
Residences: He settled near Kaskaskia and taught in one of the earliest schools in Randolph County. He was a French scholar. He was listed in the 1810 Census of Kaskaskia. 
Service: He was a soldier with Col. George Rogers Clark. 
Marker: His name is on a bronze marker on the grounds of Sparta High School placed by Fort Chartres Chapter DAR, in 1934. 
Sources: CR, DAR, NSDAR, W 
DOZIER (Dauge, Doscher, Dosher), PETER
Born: November 2, 1762 in Virginia 
Died: August 6, 1838 
Buried: Medsker Cemetery, Martinsville, Clark County, Illinois 
Spouse: Alley Pritchett 
Service: Private; Virginia Militia 
Pension: Pension Roll October 3, 1832, Clark County, age 72; Alley W22951: BLWT-75004-160-55 (Va) 
Marker: His grave is marked. 
Sources: PI, PENSION, W 
DU BOIS, TOUSSAINT 
Born: About 1753 in Vincennes, Indiana 
Died: March 11, 1816, Clay County, Illinois 
Buried: He drowned while crossing the Little Wabash River and his body was not recovered. 
Residences: In 1774 he settled in Allison Township, Lawrence County, Illinois. Father Pierre Gibault was sent to Vincennes by Col. George Rogers Clark to use his influence on the French inhabitants. Toussaint du Bois, with his father and many others, took the Oath of Allegiance in the little French church. du Bois was selected to confer with Washington regarding supplies. 
Service: His service was with Col. George Rogers Clark. Marker: His name is on a bronze tablet at the Lawrenceville Court House placed by Toussaint du Bois DAR Chapter in 1921. 
Sources: DAR, W, du Bois Indiana County History 
DUBUQUE, JEAN BAPTISTE 
Born: Montreal, Canada 
Buried: St. Clair County, Illinois 
Residences: He was elected Justice in St. Clair County and served several terms. 
Service: He served under Col. George Rogers Clark. After the close of the war he was made Commandant. 
Sources: W 
DUDLEY, JOHN 
Born: February 25, 1758 at Saybrook, Connecticut 
Died: January 2,1846 at home of his son John in Will County 
Buried: DuPage County, Illinois 
Spouse: Lydia S. Booth 
Children: John (son) 
Residences: He came west from Claremont, New Hampshire, settling in Craw­ford County. From there he moved to DuPage County. He served as a Justice. Service: Private; New Hampshire. He enlisted in 1777 in Capt. Samuel Ashley's Company, serving one month and two days. 
Pension: S32225 (NH); The 1840 Pension Census of DuPage County gives his age 82. 
Marker: His name is on a bronze tablet in the DuPage County Court House at Wheaton, Illinois. 
Sources: DAR, NSDAR, PI, PENSION, W, National Genealogical Society Quar­terly, December 1936, p. 108. 
DUFF, JOHN 
Born: Virginia 
Died: 1805; he was killed on Ripple Island 
Buried: Near the old salt springs, Gallatin County, Illinois 
Service: Soldier; Virginia. He gave assistance to Col. George Rogers Clark. 
Sources: W 
DUNCAN, JOHN
Born: About 1762 
Died: 1842 
Buried: On his farm, Clinton County, Illinois 
Spouse: Lydia Residences: After the war, he removed to Kentucky and from there to Illinois, settling in the southwestern part of Clinton County. 
Service: Private; Virginia Continental troops: Kentucky Pension: Pension Roll, September 25, 1833, age 71, Franklin County. Lydia R3126 Her residence was Marion, Williamson County when pension claim rejected (Act July 7,1838) "Not six months service." 
Sources: DAR, HS, PENSION, W 
DUNHAM, WILLIAM 
Died: November 30, 1833 
Buried: Union Cemetery, DeWitt County, Illinois 
Sources: From DeWitt Clinton Chapter, Clinton, Illinois 
DUNLAP, WILLIAM
Born: 1760 in Laurens County, South Carolina 
Died: July 2, 1835 
Buried: Near Villas, Crawford County, Illinois 
Spouse: Nancy 
Residences: In 1818 he removed to Robinson Township, Crawford County Service: Private; South Carolina Militia. He enlisted in March, 1780 for six months with Capt. Joseph Pearson and Col. Casey. In 1781, three months with Captain Duval and Capt. Kenner Hudson; in 1782 under Capt. Kenner Hudson. 
Pension: Illinois Pension Roll, Crawford County, February 10, 1834, age 73. R3139 (SC) Nancy, widow of William, residence Robinson, Crawford County when pension rejected (Act July 7,1838) "For further proof." 
Sources: PENSION, W 
DUNN, CHARLES
Born: Probably Virginia Died: After 1830 Buried: Pope County, Illinois 
Residences: He was recorded in the 1830 Census of  Pope County. 
Service: Soldier; Virginia. He enlisted November 30, 1778 in the Company of Lt. Col. John Cropper, continuing in the service after the close of the war in Pope County, Illinois. 
Sources: CR. W 
DUSENBERRY, JOHN 
Born: About 1752 
Died: September 26, 1833, age 81 
Buried: Old French cemetery, no longer in existence, Peoria County, Illinois 
Service: First Lieutenant; Captain: New York. He enlisted in January, 1776, and served in Col. Rudolphus Hietzman's Third New York Regiment. He was wounded in the battle of White Plains but served from December, 1777 to March, 1778. He served under Col. David Van Schaik, Col. Philip Van Court­land, and Gen. Samuel Parsons, and for two years under Lafayette. 
Sources: W, New York in the Revolution 
DUVALL (Duvoll), WILLIAM
Buried: Rose Hill Cemetery, Cook County, Illinois; Lot #194, Block 5-7, Section C 
Sources: HR 
EDERLIN (Edelin, Edlin), CLEMENT 
Born: Probably Maryland 
Died: May 25, 1839 
Buried: Moore Cemetery, on a farm, Indian Creek Township, White County, Illinois 
Spouse: Lucy Ammons 
Residences: After the war he removed to Kentucky, and then to White County, Illinois. 
Service: Sergeant; Maryland. He served in 1776 in the First Regiment, Maryland, with Capt. John Haskins and Col. William Smallwood; also in Col. John Stone's Regiment. 
Pension: White County, Illinois, Pension Roll, May 2, 1819, transfer from Ken­tucky; W20606Lucy Ammons, former widow; BLWT 101431-160-55 
Marker: The name of Clement Edelin is on a monument placed by Wabash Chapter DAR, Carmi, in 1936, honoring soldiers buried in White County. A government headstone with his name was dedicated by the Chapter and State Officers on September 21, 1964 in the old city cemetery. 
Sources: DAR, PENSION, W 
EDGAR, JOHN 
Born : About 1733 in Ireland 
Died: December 9, 1830 
Buried: Kaskaskia, Randolph County, Illinois 
Married: “November 22, 1823 at Williamsburg, Illinois, Gen. John Edgar, of New Jersey, an officer of the Revolution, age 90 years, married to Eliza Stevens, aged 14 years." (Virginia Magazine of History and Biography, Vol. 67, July 1959, page 322) 
Residences: He settled in Kaskaskia in 1784. During the administration of Gov­ernor Arthur St. Clair, he was elected to the Legislature which convened at Chillicothe, Ohio. He was a man of great wealth and prominence. 
Service: Captain; Pennsylvania. He was in the British Navy; was a resident of Detroit at the outbreak of the Revolutionary War. He espoused the American cause, was seized, imprisoned in Quebec, escaping near Montreal, he found his way to the American lines. He entered the service and was made Captain in the Navy. After the war he was appointed Major General of the Illinois Militia, and in1790 was made a Judge of the Common Pleas Court. 
Pension: S30395(Pa) Marker: His name is on a bronze marker on the grounds of Sparta High School, Sparta, placed by Fort Chartres Chapter DAR in 1934. 
Sources: DAR, NSDAR , PENSION, W 
EDMINSTON, JOHN Born: 1762 
Died: February 7,1852 
Buried: Johnson (Tate) Cemetery, Grandview Town ship, near Kansas, Edgar County, Illinois 
Service: Soldier; New York Militia 
Marker: His grave has a DAR marker. Sources: DAR, HR 
EDMONSON, JOHN Born: February13,1762near Rockbridge,Virginia 
Died: After 1845 in Fulton County, Illinois 
Service: Private; South Carolina. He enlisted in 1778 from South Carolina in Capt. Benjamin Tutt's Company, Col. Leroy Hammond's Regiment; 1781 in Capt. Robert Macksfield's Company, Col. Andrew Pickins' and Col. Robert Anderson's Regiment. 
Pension: S32229 (SC). He applied for a pension in Fulton County in December 1835, affidavit submitted March 6, 1843, age 82, address Ellisville. 
Sources: DAR, PENSION 
EDWARDS, HENRY 
Buried: Drury farm, east of Jacksonville, Morgan County, Illinois. The tombstone no longer exists. 
Residences: He arrived in Morgan County in the 1820's. He owned 160 acres, part of which is now in the Drury farm, on which he built a cabin. 
Service: Soldier; Carolinas. He served under General Anthony Wayne. It was said that when Mr. Edwards enlisted, he led a cow along, as he was very fond of milk. In one of the campaigns, rations were short in both quantity and quality and the soldiers butchered Edwards' cow and distributed the beef. 
Sources: Jacksonville Daily Journal 
EDWARDS, JOSEPH 
Born: Probably Virginia 
Died: Union County, Illinois 
Residences: He came to Union County in 1829. 
Service: Soldier; Virginia. He enlisted in 1776 from Virginia for nine months under Col. Adam Slencar. He was discharged at Martinsburg, Virginia. 
Pension: Residence, Jonesboro, Union County, Illinois, when pension claim sus­pended (Act June 7,1832) "For further proof of service." When he applied for a pension, he stated in his deposition that his property consisted of one bed worth $3, one axe worth $2, one plow worth $3, and one hoe valued at $1, making a total of $9. 
Sources: PENSION, W 
ELLIS , JOHN 
Born: March 9, 1735 in Frederick County, Virginia Died: May 29, 1834 Buried: Union County, Illinois 
Service: Soldier; Virginia. He enlisted in Greenbrier County, Virginia, serving as an Indian spy from 1773 to 1783 by appointment of General Andrew Lewis. He served at Ellis' Fort under Capt. John Cook. 
Pension: S32234 (Va); Residence, Frankfort, Franklin County, Illinois, when pen­sion claim was suspended (Act June 7, 1832) "for further proof and details of service." Later allowed to heirs. 
Sources: PENSION, W 
ELLSWORTH, JOHN 
Born : In New York 
Died : Aged 84 Buried: St. Clair County, Illinois Children: William, resident of McLean County in 1859. 
Service: Soldier; New York. He enlisted in February, 1776 in the Company of 
Capt. John B. Allen in Col. Frederick Weisenfels Fourth New York Regiment. He was in the battles of Bennington, Moses Kill, and at the surrender of Burgoyne. He was reported to have served seven years. 
Pension: R33-.\3 (NY) ; Residence, Kaskaskia, Randolph County when pension claim rejected (ActJune7, 1832)" Desertion." 
Sources: PENSION, McLean County History by Chapman Brothers 
EMMETT, JOHN 
Born: October 22, 1759 
Died: 1847, Gallatin County, Illinois 
Spouses: (1) Margery 
    (2) Margaret B. 
Service: Private; Maryland. He enlisted July 25, 1776 under Lt , Col. Thomas Hughes in the 30th Battalion, Maryland troops. 
Pension : Margaret B. W8692 (Md): Gallatin County, Illinois Pension Census of June 1, 1840, John Emmett, aged 81, was residing with James Barker, head of family. 
Sources: HS, PI, PENSION 
ESSARY, JOHN 
Born: July 5, 1744 in Delaware County, Pennsylvania 
Died: November 17, 1828 Buried: Clark County, Illinois 
Married: June 17,1776, Delaware County, Pennsylvania 
Spouse: Sarah Hester Clark; born January 5, 1758; died October 16, 1818 
Children: Jonathon Service: Private; Virginia. He enlisted September 16, 1776 in Capt. Thomas 
Paxton's Rangers and was discharged November 13, 1776; served as Lieutenant August 7, 1798 in the Delaware County Militia; served thirty-six days from October 21, 1782 to November 25, 1782. He served in Ca pt. James Samuels' Company of Jefferson County Militia under Col. George Rogers Clark on an expedition against the Indians. 
Sources: DAR , PI 
EVANS, JOHN 
Buried : Putnam County, Illinois 
Service: Soldier; Pennsylvania 
Pension: S42714 (Pa), Residence Hennepin, Putnam County, Illinois, when pen­sion claim was rejected (Act June 7, 1832) "for further proof and details of service." 
Sources: PENSION, W 
EVANS, JOSEPH 
Died: September 4, 1832 
Buried: Kirkland Cemetery, Sorento,  Bond County, Illinois; Government Headstone 
Residences: He came to Illinois in 1818, settling in Seminary Township, Fayette County. 
Service: Private; Virginia Continental troops. He served in the Seventh Virginia Regiment. 
Pension: S35289(Va), Pension Roll, March 5,1819, dropped under Act of May 1, 1820; returned to Pension Roll , August 15, 1826. 
Sources: HR, PENSION, W 
EVELAND, FREDERICK 
Born: About 1760 in Pennsylvania 
Died: 1838 at Waynesville, DeWitt County 
Buried: Fremont Cemetery, Andrew Brock Farm, Sec. 10 Funks Grove, McLean County, Illinois Spouse: __ Brock 
Service: Pennsylvania Continental troops 
Sources: DAR, NSDAR 


FARLEY, FRANCIS, SR. 
Born: 1726 in Chesterfield County, Virginia 
Died: 1829 in Shawneetown, Gallatin County, Illinois 
Spouse: Nancy Blankenship 
Service: Private and Patriotic Service; Virginia. He served in Clement's Company of the Virginia Militia and marched with Capt. Adam Clement from Bedford County to the assistance of General Green in South Carolina on May 1, 1781. It is family tradition that he carried mail between Washington's Army and the Virginia Militia Command. 
Sources: PI, "family history" 
FARRIS,JOHN 
Died: Probably Franklin County, Illinois 
Service: Soldier; Virginia 
Pension: R3455 (Va): 
Residence Frankfort, Franklin County, Illinois, when pen­sion claim was rejected (Act June 7,1832) "not on rolls, no proof of service." 
Sources: PENSION 
FEAR (Fears), EDMOND 
Born: 1764 
Died: 1844 in Casey County, Kentucky Buried: White County, Illinois; Government Headstone (old city cemetery, Carmi) 
Service: Private; Virginia. He served in Major Rose's Virginia troops. 
Pension: R3474 (Va); Residence, Carmi, White County, Illinois, when pension claim rejected, Act, June 7,1832, "not six months service" 
Marker: His name appears with those marked by the Wabash Chapter DAR, Carmi, September 21, 1964. 
Sources: DAR, NSDAR, PENSION, W 
FEE, JOHN 
Died: Probably Adams County, Illinois 
Residences: He came to Adams County, Illinois, before 1832. 
Service: Soldier; Pennsylvania. He served from Washington County, Pennsylvania, remaining in the army after the close of the war. 
Sources: W 
FELLOWS, WILLIS 
Born: October 5, 1758 
Died: January 18, 1840 
Buried: Kitchell Cemetery, Crawford County, Illinois 
Spouse: Sarah Hart 
Residences: He removed to Indiana, and from there to Crawford County, Illinois. Service: Private; Massachusetts Continental troops. He served in Capt. Samuel Taylor's Company, Col. Nicholas Dike's Regiment in 1776; in 1777 with Capt. Lawrence Kemp's Company, Col. Leeward's Regiment; with Capt. Benjamin Phillips and Col. Elisha Porter. He also served in 1778, 1779, and 1780. 
Pension: S35303 Conn: Mass. Pension Roll, June 28, 1826, Crawford County, transferred from Indiana. 
Sources: PI, PENSION, W 
FERGES, JOHN E. 
Died: 1842 
Buried: Herrin City Cemetery, Herrin, Williamson County, Illinois 
Service: Private; Virginia. He served in the First Regiment of the Virginia Militia 
Sources: DAR, HR 
FERGUSON, LEWIS , SR. 
Born: 1760 in Virginia 
Died: 1842 Buried: Smote Cemetery, Menard County, Illinois 
Service: Lieutenant; Virginia. He enlisted in 1778 in Culpepper County, Virginia, 
and served until 1780 under Capt. Garland Burrly and Col. Francis Taylor. Pension: S32241(Va). The 1840 Pension Roll shows him living in Menard County, age 80. 
Sources: PENSION, W 
FESLER, JOHN 
Born: About 1745 in Germany 
Died: 1841 Buried: Mason County, Illinois 
Spouse: Eve Moyer 
Service: Private; Pennsylvania. He served in Capt. Emanuel Carpenter's 7th 
Company, 10th Battalion, Lancaster County Militia in 1777; in Capt. Rudolph Statler's Company, 5th Battalion, Lancaster County Militia in 1781. Sources: DAR, PI 
FIELD, LEWIS 
Born: July 4,1763 at Culpepper County, Virginia 
Died: 1845 Buried: Golconda, Pope County, Illinois 
Spouse: Hannah Lewis 
Service: Private; Virginia. He was a Scout under Col. George Rogers Clark and 
was taken prisoner. 
Pension: S30413(Va) 
Sources: DAR, PI, PENSION 
FIELD, LUTHER 
Born: 1762 
Died: 1844 
Buried: Baptist Church Cemetery, Richland County, Illinois 
Service: Soldier; Rhode Island and Vermont. On roll of draft, August 1, 1778, ofCapt. Joseph Sprague's Company, Smithfield, Rhode Island; Col. Brown's Regiment, Rhode Island Militia. He also served in Vermont. 
Sources: NSDAR 
FILES, JOHN, JR. 
Born: March 31, 1760 in England 
Died: October 1, 1834 
Buried: Burnt Prairie Township, White County, Illinois 
Spouse: Mary Foils
Residences: He came to Illinois in 1816, settling in Wayne County. He also lived in Edwards County, and was a resident of White County in 1818, 1820, and 1830.
Service: Private; South Carolina. He served in Col. Picken's South Carolina troops. 
Pension: S32245 (SC) 
Marker: A flat marble Government marker was dedicated by Wabash Chapter DAR , Carmi on September 21,1964. 
Sources: CR, DAR, NSDAR, PI, PENSION, W 
FINLEY, DAVID 
Born: March 9, 1761 in Belfast, Ireland 
Died: September 3, 1838 
Buried: South Henderson Cemetery, Gladstone, Henderson County, Illinois; Private Headstone 
Spouses: (1) Martha
    (2) Jane Ritchie 
Residences: In 1818 he removed to Warren County, Illinois, from Clarke County, Indiana. 
Service: Private; Pennsylvania. He enlisted in Capt. Samuel Miller's Company, Col. Aeneas Mackey's Eighth Regiment, Pennsylvania troops. He was in the battles of Brandywine and Germantown. 
Pension: Pension Roll, September 21, 1818, Warren County, Illinois, transferred from Indiana. Jennette or Janet W25577 (Pa) 
Marker: His grave was marked October 6, 1916 by Mildred Warner Washington 
Chapter DAR, Monmouth. A new marker was dedicated by Daniel McMillan Chapter DAR, Stronghurst in 1961. 
Sources: DAR , HR , NSDAR, PI, PENSION, W 
FINN, PETER 
Born: July 2, 1751 in Baltimore County, Maryland 
Died: After 1840 
Buried: McClelland Cemetery, Harvey's Point, Marion County, Illinois 
Residences: After the war he removed to Kentucky where he applied for a pen­sion. He came to Marion County, Illinois, in 1837.
Service: Soldier; Maryland and North Carolina . He enlisted in 1778 and served nine months in Capt. John Murray's Company, Col. Archibald Thompson's Maryland troops. In 1779 he enlisted in North Carolina in Capt. Valentine Sevier's Company, with Col. Benjarnin Carter and Col. Charles Robertson. In 1780 he served as Sergeant in Capt. Valentine Sevier's Company, Col. John Sevier's Regiment, North Carolina troops. 
Pension: S32244 Md:NC. 1840 Pension Census, Marion County, age 90, residing with William Layson 
Sources: PENSION, W 	
FISK, ROBERT 
Born: 1758 
Died : April 19, 1824 
Buried: Golconda, Sangamon County, Illinois .
Spouse: Elizabeth Jones 
Residences: Widow, Elizabeth (Gilliland) pensioned Ripley County, Indiana, March 4, 1831 at $120 per year. 
Service: Sergeant; Massachusetts. Probably enlisted April 19, 1775 at Lexington, Massachusetts, his residence, as a "Minute Man" in Regiment commanded by Col. John Greaton. He later enlisted for the entire war; was Sergeant in Capt. Joshua Walker' s Comp311Y, Col. David Green's Regiment. He served eight years and one month. 
Pension: Pension Roll, Pope County, Illinois, June 8, 1819, certificate #15257 issued October 12, 1819 to Robert Fisk, Belgrade, Pope County Pension Roll, Sangamon County, Illinois. October 12, 1819; suspended Act May 1, 1820. Mass: Elizabeth W9438 
Marker: His name is on a bronze plaque in the south mall , Old State Capitol, Springfield, placed by Springfield Chapters DAR and SAR, October 19, 1911. Sources: DAB, NSDAR, PI, PENSION, W 
FITZGERALD, GEORGE 
Buried: Gatlin Cemetery, Crouch Township, Hamilton County, Illinois Service: 
Soldier; Virginia. He served in Col. John Gibson' s detachment in the western department. He was discharged May 24, 1780. 
Pension: He was pensioned. 
Sources: W 
FITZPATRICK, EDWARD (Patrick, Edward F.) 
Born: 1760 in Ireland Died: November 21, 1834 in Mclean County, Illinois 
Buried: Hammers-Dammon Cemetery, EI Paso, Woodford County, Illinois; Private 
Head stone 
Married: February 13, 1798 in Iredell County, North Carolina 
Spouse: Polly McLeod 
Residences: He moved from Rowan County, North Carolina to Tennessee, and from there to McLean County, Illinois, in1832(now Woodford). 
Service: Private; North Carolina. He served as a Private in Capt. John Arm­strong 's Company, North Carolina troops. 
Pension: Illinois Pension Roll, McLean County, December 12, 1833, age 74. Patrick, Edward (or Edmund Fitzpatrick); N.C. Polly F. W1988. Her applica­tion, October 20, 1849, Woodford County, age 78, a resident of Tazewell County. 
Marker: The grave is marked by a tablet placed by Peoria Chapter DAR. 
Sources: DAR, HR, PENSION , W 
FLATT, JOHN 
Buried: Probably Greene County, Illinois 
Service: Soldier; Pennsylvania 
Pension: R3602 (Penn); 
Residence Carrollton, Greene County, Illinois, when pension claim rejected (Act of June 7, 1832)" not six months service." 
Sources: PENSION, W 
FLETCHER, ISAAC 
Born: October 26, 1763 in Westford, Massachusetts 
Died: February, 1838 
Buried: Tazewell County, Illinois 
Spouse: Ruth Pierce 
Service: Private; Massachusetts. Substituted for his brother, Levi, who was ill. Isaac Fletcher was wounded in service and honorably discharged in 1782. 
Pension: S38701 (Mass) Sources: DAR, PI, PENSION 
FLORA (Florey, Flory), JOHN 
Born : April 17, 1760 
Died: July 18, 1850 Buried: Flora Cemetery, Long Creek, Macon County, Illinois 
Spouse: Mary Ott 
Service: Private; Pennsylvania. He served in the First Pennsylvania Regiment and was listed on the roll of Capt. David Hanis (or Hanes), May 12, 1776. 
Marker: His grave was marked by Stephen Decatur Chapter DAR, Decatur May 2, 1971. 
Sources: DAR, HR 
FOSTER, ABNER 
Born : About 1759 Died: After 1840 
Buried: Gallatin County, Illinois  
Spouse: Betsy 
Service: Soldier; Massachusetts. He enlisted August 15, 1777 in Capt. Benjamin Adams' Company, Col. Jonathan Johnson's Regiment and served four months. 
Pension: Illinois Pension Census, June 1, 1840, Callatin County, age 81. Betsy W23050 (Mass) 
Sources: PENSION, W 
FOSTER, NATHANIEL 
Born: June 23, 1761 in New Jersey 
Died: September 14, 1837 
Buried: Sangamon County, Illinois 
Spouse: Nancy Mauzy 
Service: Private; Virginia. He enlisted August, 1779 and served under Capt. Hugh Brent, of Col. Garrard's Regiment; enlisted October 1, 1780, served five months under Capt. John Brent, Col. Williams and Col. Lucas. 
Pension: S32253 (Va) 
Sources: DAR, PI , PENSION 
FOWLER, WILLIAM 
Born: South Carolina 
Died: Randolph County, Illinois 
Spouse: Hannah 
Residences: He came to Illinois in 1816, locating in the Harmon Settlement. In 1825 he was living in Mary Township, Randolph County. Service: Private; South Carolina Militia 
Pension: Hannah R3704 (SC); Illinois Pension Roll, Randolph County, May 7, 1834, age 69 
Marker: His name is on a bronze marker on the grounds of Sparta High School, Sparta, placed by Fort Chartres Chapter DAR, June 24, 1934. 
Sources: DAR, NSDAR , PENSION, W 
FOX, REUBEN 
Born: November 18, 1762 near New London, Connecticut 
Buried: Wabash County, Illinois Service: 
Soldier; Connecticut. He enlisted in May, 1779, serving one year in Capt. William Latham's Company, Col. William Ledyard's Regiment; served three months in 1781. 
Pension: S32255 (Conn); Residence Mount Carmel, Wabash County, when pension claim was suspended (Act June 7, 1832) "for further proof'." 
Sources: HS, PENSION 
FRAZIER, JOHN
Born: Virginia 
Buried: Abandoned cemetery, Danville, Vermilion County, Illinois 
Service: Soldier; Virginia. He enlisted near the home of Laurence Washington and served during the entire war. He was present at the surrender of Corn­wallis. 
Marker: His name is on a plaque in front of the Post Office at Danville placed by Governor Bradford Chapter DAR. 
Sources: DAR, W 
FRENCH, JOHN 
Born: November 2, 1760 in Rockingham County, Virginia 
Died : February 17, 1844 
Buried: Old City of McCraw Cemetery, Clinton, DeWitt County, Illinois 
Spouse: Dolly 
Service: Private; Virginia Continental troops Pension: 
Residence Decatur, Macon County, Illinois, when pension claim suspended (Act June 7, 1832) "certificate withheld until someone of responsibility applies for it." Dolly R3795 (Va) 
Sources: DAR, PENSION 
FRIATT, ROBERT 
Born: 1758 
Died: 1841 
Buried: Dutch Ridge Cemetery, Pomona Township, Jackson County, Illinois; Government He adstone Residences: He settled in Dutch Ridge, Jackson County before 1812. Listed Union County census for 1818 and 1820; Jackson County in 1830. 
Service: Private; Virginia Continental troops. He served in Capt. Baldween's Company. 
Pension: S32558 (Va); Pension Roll, October 2, 1833, Jackson County, age 74 
Sources: CR, PENSION, W, Jackson County Genealogical Society 
FRICK, JACOB 
Born: About 1747 in Pennsylvania 
Died : After 1833 
Buried: Jonesboro Precinct, Union County, Illinois 
Spouse: Elizabeth Residences: He came to Illinois in 1823, settling in Jonesboro, Union County Service: Private; Pennsylvania and North Carolina. He enlisted in 1775 in Lower Milford Township, Bucks County, Pennsylvania, in Capt. Harry Huber' s Company; he also served under General Griffith Rutherford in North Carolina. 
Pension: Pension Roll, Union County, Illinois, July 16, 1833, age 83; Elizabeth W23063 (NC) 
Sources: DAR, PENSION,W 
FRIZZELL, EARL 
Born: Massachusetts 
Died: Probably Henderson County, Illinois 
Service: Soldier; Massachusetts Pension: R3811 (Mass); Resident of Oquawka, Henderson County, Illinois, when 
pension claim was suspended (Act June 7, 1832) "for proof of service by the Mass. Rolls -identification by witnesses." 
Sources: DAR, PENSION 
FROST , JOSEPH 
Born: About 1753 
Died: After 1840 
Buried: Bethel Cemetery, Coles County, Illinois 
Spouse: Anna Residences: He settled in Charleston , Coles County, Illinois, in 1831. 
Service: He served in the Virginia line of troops. 
Pension: Illinois Pension Census, Coles County, June 1, 1840, age 87; Anna W23066 (Va) 
Sources: PENSION , W 
FRUIT, JAMES SAMUEL 
Born: November 13, 1762 in Orange County, North Carolina 
Died: October 9, 1834 
Buried: Tunbridge Cemetery, Kenney, DeWitt County, Illinois 
Spouses: (1) __ Clark
		 (2) Mary Reeder 
Children: (first marriage) John, Thomas Clark, Edmund, Alexander, Enoch, Elizabeth, Sibylla, James, Martha 
Residences: He moved from North Carolina to Christian County, Kentucky, in the 1800's; to DeWitt County, Illinois, about 1828. 
Service: Soldier; Pennsylvania 
Marker: His grave has a DAR marker. 
Sources: DAR, PI, Letter 
FRY, ROBERT (see also Friatt, Robert) 
Born: About 1765 
Died: Jackson County, Illinois 
Service: Soldier; Virginia. He served in Capt. Bembridge Goodwin's Company . 
Pension: Pension Census June1,1840, Jackson County, age 85 
Sources: W 

GALBRAITH (Calbreth, Galbreath), WILLIAM 
Born: October, 1757 
Died: May 3, 1839 
Buried: Mitchell Cemetery, near Ashmore, Coles County, Illinois; Private Head­stone 
Spouse: Phebe Forman 
Service: Pvt; Sgt;  Pa; Va. Served in Shannon's Company, Pennsylvania Regiment 
Pension: Galbreath, William, Pa; Va; Phebe W23093 (PennVa) 
Sources: HR, NSDAR, PI, PENSION 
GANNON, WILLIAM, SR. 
Born: February 1, 1758 in Ireland 
Died: After 1840 
Buried: Baptist Cemetery, Elbridge Township, Edgar County, Illinois 
Spouse: Betty Tramble 
Service: Private; North Carolina Continental troops. He enlisted in 1780 from North Carolina and served in the battles of Camden, Guilford Court House, Eutaw Springs, and Hughanne, where he was wounded. 
Pension: S32259 (NC); Pension Roll, Edgar County, July 3, 1833, age 75; Pension Roll, June 1, 1840, Edgar County, age 95, residing with Henry Penington, head of family. 
Sources: PI, PENSION, W 
GARDNER, JONATHAN, JR. 
Born: 1746 in Connecticut 
Died: After 1833 
Buried: Coles County, Illinois 
Service: Soldier; Connecticut: New Jersey 
Pension: S17429 (Conn); Pensioned in 1833 while resident of Geauga County, Ohio; removed to Coles County, Illinois. 
Sources: NSDAR, PENSION 
GARDNER, SAMUEL 
Buried: Possibly Crawford County, Illinois 
Residences: His residence is listed in various sources as Franklin County, John­son County and Crawford County. 
Service: Private; New York Continental. He served in the Fifth Regiment with Col. Lewis Du Bois; and in the Third Regiment Dutchess County Militia under Col. Albert Pawling. 
Pension: S35336 (NY); Pension Roll, July 20,1826, Crawford County, Illinois 
Sources: PENSION, W 
GARRETSON, JAMES 
Buried: Moredock, Monroe County, Illinois 
Residences: Following the war he returned to Virginia then came to Illinois in 1781, settling near Waterloo, Monroe County. 
Service: He served under Col. George Rogers Clark. Sources: W 
GARRISON, ABRAHAM 
Buried: Probably Pike County, Illinois 
Service: Soldier; Virginia Pension: R3927 (Va); 
Residence: Atlas, Pike County, Illinois, when pension claim rejected (Act June 7, 1832) "frontier service-not under military organization or authority." 
Sources: PENSION 
GARRISON, JAMES 
Born: 1747 near Fredericktown, Pennsylvania 
Died: October 13, 1840 Buried: Patterson Cemetery, Patterson Township, Greene County, Illinois 
Service: Private; Infantry and Cavalry North Carolina Continental. He enlisted in 1775 in Wilkes County, North Carolina, for three months in Capt. John Hamlin's Company, Col. Benjamin Cleveland's Regiment; in 1781 in Capt. Alexander Gordon's Company, Col. Joseph McDowell's Regiment. He was in the battles of Cross Creek, the surrender of Ninety-six, and Eutaw Springs. 
Pension: S32260(NC); Pension Roll, June 6,1833, Greene County, age 87; Pension Census of June 1, 1840, Greene County, age 99, residing with Edward Flatt, head of family. 
Sources: HR, PENSION, W 
GARY, WILLIAM 
Died: 1817 
Buried: Wheaton Cemetery, DuPage County, Illinois 
Marker: Marker placed by Perrin-Wheaton Chapter DAR, May 3, 1930 marking grave of  "William Gray" (sic). 
Sources: DAR, NSDAR 
GASTON, GEORGE 
Buried: Bovee Cemetery, Cisne, Wayne County, Illinois 
Children: Probably a brother of James Gaston, buried in Bovee Cemetery, one of nine sons of John Gaston who served in the war. 
Sources: HR 
GASTON, JAMES 
Born: April 15, 1747in Lancaster County, South Carolina (July 24,1761) Walker 
Died: March 6, 1840 
Buried: Bovee Cemetery, Cisne, Wayne County, Illinois (first burial in Bovee Cemetery) 
Spouse: Catherine Creighton; died January 5, 1848 
Residences: After the war he removed to Indiana, and then to Wayne County, Illinois. 
Service: Private; South Carolina. He enlisted in 1778 in Capt. John Marshall's Company; was taken prisoner and confined in Camden jail; enlisted in Capt. William Ingram's Company with Capt. Nisbet , and served until May, 1781. He was in the battle of Hanging Rock. He was one of nine sons of John Gaston to serve in the war. 
Pension: Catherine W23082 (SC) 
Sources: HR, PENSION, W 
GASTON, WILLIAM 
Born: July 10,1757 in South Carolina 
Died: January 12, 1838 
Buried: Covenanter Cemetery near Walnut Hill, Centralia, Marion County, Illinois 
Spouse: Mary McClure Residences: He came to Illinois in 1814, settling at "Devil's Bake Oven." He was said to be a great singer. 
Service: Private; South Carolina. He enlisted five times; first in 1775; in 1776 two periods, fourteen months; in 1780 one year. In 1781 he served one year in Companies of Capt. Thomas Marshall, Capt. John McClure and Capt.  
Pension: S32260(NC);Pension Roll, June 6,1833, Greene County, age 87; Pension Census of June 1, 1840, Greene County, age 99, residing with Edward Flatt, head of family. 
Sources: HR, PENSION, W 
GARY, WILLIAM 
Died: 1817 
Buried: Wheaton Cemetery, DuPage County, Illinois 
Marker: Marker placed by Perrin-Wheaton Chapter DAR, May 3, 1930 marking grave of "William Gray" (sic). 
Sources:  DAR, NSDAR 
GASTON, GEORGE 
Buried: Bovee Cemetery, Cisne, Wayne County, Illinois 
Children: Probably a brother of James Gaston, buried in Bovee Cemetery, one of nine sons of John Gaston who served in the war. 
Sources: HR 
GASTON, JAMES 
Born: April 15, 1747 in Lancaster County, South Carolina (July 24, 1761) Walker 
Died: March 6, 1840 
Buried: Bovee Cemetery, Cisne, Wayne County, Illinois (first burial in Bovee Cemetery) 
Spouse: Catherine Creighton; died January 5, 1848 
Residences: After the war he removed to Indiana, and then to Wayne County, Illinois. 
Service: Private; South Carolina. He enlisted in 1778 in Capt. John Marshall's Company; was taken prisoner and confined in Camden jail; enlisted in Capt. William Ingram's Company with Capt. Nisbet, and served until May, 1781. He was in the battle of Hanging Rock. He was one of nine sons of John Gaston to serve in the war. 
Pension: Catherine W23082 (SC) 
Sources: HR, PENSION, W 
GASTON, WILLIAM 
Born: July 10, 1757 in South Carolina 
Died: January 12, 1838 
Buried: Covenanter Cemetery near Walnut Hill, Centralia, Marion County, Illinois 
Spouse: Mary McClure 
Residences: He came to Illinois in 1814, settling at "Devil's Bake Oven." He was said to be a great singer. 
Service: Private; South Carolina . He enlisted five times; first in 1775; in 1776 two periods, fourteen months; in 1780 one year. In 1781 he served one year in Companies of Capt. Thomas Marshall, Capt. John McClure and Capt. 
John Steele. He was in the battles of King's Mountain, Hooks' Defeat, and Hanging Rock. 
Pension: S32265 (SC); Marion County, Illinois Pension Roll, October 22, 1833, age 75 
Marker: His monument is inscribed: "Served under Washington. " A marker was placed by Isaac Hull Chapter DAR, Salem, in June 19"l. 
Sources: DAR, HR, NSDAR, PI, PENSION, W 
GAYLORD, LEMUEL 
Born: February 14, 1765 in Bristol County, Connecticut Died: November 17, 1854 Buried: Cumberland Cemetery, Wenona, Marshall County, Illinois 
Spouse: Sylvia Murray 
Service: Teamster; Connecticut: Virginia. He served as a Drummer Boy in George Washington' s Virginia Militia and as an Ensign in Col. Roger Enos ' Com­pany, Connecticut.  Lemuel Gaylord's father was killed in the Wyoming Massacre in July, 1778; his mother, Kathryn Gaylord, was a Revolutionary heroine for whom a mon­ument was erected, and for whom the Bristol Chapter DAR in Connecticut is named. 
Sources: PI; W 
GEAR, HEZEKIAH 
Born: April 26, 1761 
Died: August 4, 1822 
Buried: City Cemetery, Galena, Jo Daviess County, Illinois, (his body was re­moved from Pittsfield, Massachusetts, in 1837 by son Hezekiah of Galena, Illinois) 60th NSDAR Report 
Spouse: Sarah Gilbert 
Children: Son, Hezekiah, soldier in Blackhawk War, died 1882, buried same cemetery. 
Service: Private; Connecticut. He served in Capt. David Beebe's Company, in Col. Roger Enos' Regiment. 
Sources: DAR, NSDAR, PI 
GIBBS, (Truman), TRUEMAN 
Born: About 1750 probably at Litchfi eld, Connecticut Died : About 1830 Buried: Spring Grove Cemetery, Bridgeport, Lawrence County, Illinois Service: Trumpeter: Connecticut. He enlisted in 1776 in the Company of Capt. Moses Seymour and Major Elisha Sheldon in the Connecticut Light Horse, serving as trumpeter. 
Marker: His name is on a bronze tablet at the Lawrenceville Court House placed by Toussaint du Bois Chapter DAR in 1921. 
Sources: DAR, HR, PI, W 
GIBSON, NICHOLAS 
Born: 1762 in Virginia 
Died: June 5, 1858 at home of son-in-law 
Buried: Wesley Memorial Cemetery, Passport, Richland County, Illinois, (on border between Richland and Clay Counties); Government Headstone 
Married: About 1795 
Spouse: Lydia, died March 14, 1857, buried Wesley Memorial Cemetery 
Children: James, Jacob, Prudence(Lough), Elizabeth Squires), John W., Temper­ance (Byrne), Robert, Mary (Collins), George, William 
Residences: Nicholas Gibson settled in Denver Township, Richland County. His son, Jacob, remained behind on the old homestead, and served as a surgeon in the Civil War. Elizabeth, Robert, and Mary died in Braxton County, West Virginia. The remaining children and their families are buried at Passport. 
Service: Private; Virginia Militia. He enlisted in 1781 at the age of 16; served two years in Capt. Skinner's Company; honorably discharged in 1783. 
Pension: S8559 (Va) 
Marker: A flat marble government marker was placed by Olney Jubilee Chapter DAR and the Department of Army in September 1964. 
Sources: DAR, PI, PENSION 
GILBERT, ASAHEL 
Born: December 15, 1760 in Hebron, Connecticut 
Died: November 23, 1852 
Buried: Hope Cemetery, Galesburg, Knox County, Illinois 
Spouse: Anna Goodrich, born 1760; died 1827 
Children: Charles W. 
Residences: He came to Galesburg, Knox County, Illinois, in 1847. 
Service: Trumpeter; Connecticut. He enlisted May 1, 1778, serving as trumpeter in Capt. Israel Seymour's Company, Second Brigade, Col. Elijah Sheldon's Regiment. He was discharged in 1780. 
Marker: His grave was marked with a bronze marker by Rebecca Parke Chapter DAR, Galesburg in 1902. 
Sources: DAR, PI, W 
GILBRAITH, JAMES 
Buried: First burial at Kaskaskia; second burial at Fort Gage, Randolph County, Illinois 
Residences: He was living in Kaskaskia in 1810. 
Service: Soldier; Virginia 
Marker: His name is on a bronze marker on the grounds of Sparta High School placed by Fort Chartres Chapter DAR, Sparta, in 1934. 
Sources: DAR, HR, HS, NSDAR 
GILL, BENAJAH (Benaijah) 
Born: About 1762 
Died: September 23, 1834 
Buried: Drake Cemetery, Creal Springs, Williamson County, Illinois 
Spouse: Mary 
Service: Private; New Jersey and Delaware Continental troops. He served in Capt. Hardy's Company. 
Pension: Pension Roll, July 16,1833, Williamson County, age 71. Mary W416 (Del) Sources: HR, PENSION, W 
GILL, BENJAMIN 
Died: After 1840 
Buried: Williamson County, Illinois 
Service: Soldier from North Carolina. He was shot through the ear and thought to be dead. 
Pension: The 1840 Pension Roll lists him in Williamson County, age 80. 
Sources: HR, PENSION, W 
GILL, ROBERT 
Died: 1853 
Buried: Oblong Cemetery, Oblong, Crawford County, Illinois
Service: Private; South Carolina. He served in Rumsey's Company, Smith's Carolina Regiment. 
Sources: HR 
GILL, THOMAS 
Born: August 27, 1755 
Died: August 30, 1838 Buried: Oak Grove Cemetery, Palestine, Crawford County, Illinois; Private Head­stone 
Spouse: Hannah Griswell 
Residences: He lived for a time in Franklin County, but removed to Crawford County, settling four miles northwest of Palestine. 
Service: Private and Captain; South Carolina Continentals. He served as a Cap­tain in General Sumpter's Company. He was wounded at Savannah in 1779 and at Eutaw Springs in 1781. 
Pension: S31061 (SC); Illinois Pension Roll, Pike County, May 7, 1834, age 78 
Marker: His grave has a bronze marker placed by James Halstead, Senior, Chap­ter DAR, Robinson, June 14, 1924. 
Sources: DAR, HR, PI, PENSION, W 
GILLESPIE, WILLIAM 
Born: 1759 in North Carolina 
Died : August 25, 1844 
Buried: Winson Cemetery (on farm near school), Chesterfield, Macoupin County, Illinois 
Service: Private; South Carolina 
Pension: S32267 (SC) 
Sources: HR, NSDAR, PI, PENSION 
GILLHAM, ISAAC 
Born: November 10,1757 in Augusta County, Virginia 
Died : September 16, 1845 
Buried: Wanda Cemetery, South Roxanna, Madison County, Illinois 
Spouse: Jane Kirkpatrick 
Residences: He moved from Virginia to South Carolina in 1763, and from there to Madison County, Illinois. 
Service: Private; South Carolina Continental troops. He enlisted in December, 1777 in Camden District, South Carolina, in Capt. Robert Macupfee's Com­pany, Col. Thomas Neel's Regiment; on March 29, 1778 under Lt. Thomas Gillham (probably his father). He was wounded. He served from May, 1780 to August 18, 1780 in Capt. Jacob Barnett's Company, Col. Thomas Neel's Regiment; and from February 15 to May 1, 1781. He served as a Scout during the winter and spring of 1781 and 1782 in Capt. Jacob Barnett's Company, Col. William Bratton and Major John Hartshorn's Regiment. He was in the battles of Rocky Mount and Fishing Creek. 
Pension: S32270 (SC); Madison County, Illinols, Pension Roll, August 22, 1833, age 74 
Marker: His name is on the bronze tablet at the Madison County Court House, Edwardsville, placed by Ninian Edward Chapter DAR, Alton, September 16, 1912. His grave was marked October I, 1966 by the Chapter and Mrs. Royal Helgevold. 
Sources: DAR, NSDAR , PI, PENSION, W 
GILLHAM, JAMES 
Born: 1753Augusta County, Virginia 
Died: 1813 
Buried: Wanda Cemetery, near Roxanna, Madison County, Illinois; Government 
Headstone Married: 1776 
Spouse: Anne Barnett, sister of Captain Barnett; died 1819 Service: South Carolina. He served in Col. Bratton's and Col. Hill's Regiment under General Sumpter; in Col. Neal's Regiment "York"; and other periods of service during the war. He was one of the seven sons of Thomas Gillham who served with their father. 
Marker: His name is on a bronze tablet at the Madison County Court House, Edwardsville, placed on September 16, 1912 by Ninian Edwards Chapter DAR, Alton. Mrs. Royal O. Helgevold, through General Henry Dearborn Chapter, Chicago, placed a bronze marker on the grave in September, 1961. 
Sources: DAR, NSDAR, W 
GILLHAM, JOHN 
Born: 1752 in Augusta County, Virginia 
Died: March, 1832 
Buried: Wanda Cemetery, South Roxanna, Madison County, Illinois 
Married: In South Carolina 
Spouse: Sarah Clark 
Residences: He came to Illinois in 1802, settling on the west bank of the Cahokia Creek. 
Service: Corporal; South Carolina Continental troops. He enlisted March 23, 1776 from Camden District, Pendleton County, South Carolina, in the Sixth Regi­ment and was discharged June 1, 1777. He also served in the Militia under Col. Thomas Brandon. He was one of the seven sons of Thomas Gillham who served with their father. 
Pension: S32269 (SG); Illinois Pension Roll, Madison County, August 22, 1833, age 77 
Marker: His name is on a bronze tablet at the Madison County Court House, Edwardsville, placed September 16, 1912 by Ninian Edwards Chapter DAR, Alton. A flat government marker was placed in June 1960 by Mrs. Royal Helgevold through General Henry Dearborn Chapter DAR. 
Sources: DAR, HR, NSDAR, PI, PENSION, W 
GILLHAM, THOMAS,JR. 
Born: May17,1749inAugustaCounty,Virginia 
Died: November 30, 1828 
Buried: Wanda Cemetery, South Roxanna, Madison County, Illinois 
Spouse: Susannah McDow 
Service: Lieutenant; South Carolina. He served 210 days in Capt. Jacob Barnett's Company, William Hill's Regiment; fourteen days in Capt. James Thompson's Company, William Bratton's Regiment; forty days in the same company under Lt. Dervin. 
Marker: His name is on a bronze tablet at the Madison County Court House placed September 16, 1912 by Ninian Edwards Chapter DAR, Alton. The grave was marked by Mrs. Royal Helgevold through David Kennison Chapter DAR, Chicago, on September 20, 1966. 
Sources: DAR, NSDAR, W 
GILLHAM, WILLIAM 
Born: About 1750 in Virginia 
Died: October 27, 1825 
Buried: East Newbern Cemetery, Jersey County, lllinois 
Spouse: Jane McDaw 
Residences: William Gillham came to Madison County with his brothers but removed to Jersey County. 
Service: Private and Sergeant; South Carolina. He served in Brandon's Regiment before and after the fall of Charleston. Son of Thomas Gillham, Revolution­ary soldier, whose seven sons and two sons-in-law served. 
Marker: His grave was marked on October 27, 1957 by Abraham Lincoln Chapter DAR, Lincoln. 
Sources: DAR, NSDAR, PI, W 
GILMORE, HURIAH (Uriah) 
Born: 1749 in North Carolina Died: After 1840 Buried: Macoupin County, Illinois 
Residences: He first settled in Morgan County, but apparently lived in Pike and Macoupin Counties. 
Service: Private; Virginia Continental troops 
Pension: S32266 (Va), Pension Roll, Pike County, Illinois, July 16, 1833, age 84; Pension Census, Macoupin County, June 1, 1840, age 92, residing with Seburne Gilmore, head of family. 
Sources: PENSION, W 
GINGER, HENRY 
Born: April 4, 1758 in Germany 
Died: February 8, 1842 
Buried: Britton Cemetery, southeast of Vandalia, Fayette County, Illinois 
Spouse: Chauncy Luster 
Residences: He came to the United States when seven years of age. He lived in North Carolina, moved to Tennessee, and from there to Bowling Green, Fayette County, Illinois, in 1825. 
Service: Private; Pennsylvania Continental troops. He enlisted when sixteen years of age in the Pennsylvania troops and was taken prisoner at Charles­ton, South Carolina. 
Pension: S31064 (Penn); Pension Roll, April 11, 1833, Fayette County, age 67; Pension Census of June 1, 1840, Fayette County, age 94, residing with Wil­liam Rodger, head of family. 
Sources: PI, PENSION, W 
GIRADIN, ANTOINE 
Died: 1802 in St. Clair County, Illinois 
Service: Frenchman who served under Col. George Rogers Clark. He was a Justice in Clark's Court, and was elected a Justice of the Court in the Dis­trict of Cahokia in 1779, serving several times. 
Sources: W 
GIRADOT, PIERE 
Died: Before 1783 in Monroe County, Illinois 
Residences: His widow was listed as "head of family" in 1783. 
Service: He was made Commandant of St. Phillippe and served as Justice. 
Sources: W 
GIVENS, ROBERT 
Buried: Monroe County, Illinois Spouse: Martha 
Service: He served in the Virginia troops 
Pension: Martha (deceased), widow of Robert, resident of Monroe County when pension claim rejected (Act July 7, 1838) "Not a widow under the Act -died before August 16, 1842." 
Sources: HS, PENSION 
GLASS, FRANCIS 
Died: June 12, 1830 
Buried: Family Cemetery, west of Golconda, Pope County, Illinois 
Spouse: Agnes Ramsey, died October 24,1840 
Service: He served from North Carolina. 
Sources: Family Record 
GLENN, JOHN 
Buried: Probably buried in Lawrence County, Illinois 
Service: Soldier; Virginia. He served from Virginia in the Revolutionary War, and continued in the service after the close of the war. 
Pension: Residence Lawrenceville, Lawrence County, when pension claim rejected (Act June 7,1832) " No proof of service." 
Marker: His name is on a bronze tablet at the Lawrenceville Court House placed by Toussaint du Bois Chapter DAR in 1921. 
Sources: DAR, PENSION, W 
GLENN, THOMAS 
Died: July 31, 1846 
Buried: Rock Creek Cemetery, DeWitt County, Illinois 
Sources: Family 
GLOVER, BENJAMIN 
Buried: Probably buried in Pope County, Illinois 
Residences: His residence was Pope County in both the 1818 and 1830 censuses. 
Service: Soldier; Virginia. He served in the war from Virginia, and was also in military service after the close of the war. 
Sources: CR, W 
GODIN, MICHAEL 
Buried: Probably Randolph County, Illinois 
Residences: He was living in Kaskaskia in 1810. 
Service: An officer appointed by Col. Todd in the service of Col. George Rogers Clark. 
Sources: CR, W 
GODIN, TURANJEAU 
Buried: Probably buried St. Clair County, Illinois 
Residences: His descendants were living in Cahokia in 1783. 
Service: Frenchman who served under Col. George Rogers Clark. He gave finan­cial aid to the Americans and was a Justice in Clark's Court. He was ap­pointed Captain at Cahokia. 
Sources: W 
GOODNER, CONRAD 
Born: November 29, 1756 Bavaria, Austria 
Died: August 27, 1837 
Buried: Nashville, Washington County, Illinois 
Spouse: Elizabeth Scherer; born 1782 N.C.; died May 12,1839 Nashville, Wash­ington Co. 
Residences: He came to Illinois, settling in St. Clair County, but removed to Washington County. 
Service: Private; North Carolina Continental troops. 
Pension: St. Clair County, Illinois, Pension Roll, July 16, 1833, age 76; Elizabeth R4113 (NC) (deceased), widow of Conrad, resident of St. Clair County when pension claim rejected (Act July7,1838) "Not a widow at date of the Act died before August 23, 1842." 
Sources: PI, PENSION, W 
GORDON, BENJAMIN 
Born: August 30,1763 in Newberry County, South Carolina 
Buried: Van Burensburg area, Montgomery County, Illinois 
Residences: He lived in the Hurricane settlement, Montgomery County 
Service: Soldier; North Carolina. He enlisted in 1780 under General Thomas Sumter, Mecklenburg County, North Carolina. He was sent as wagoner with the wounded from the battle of Guilford Court House to Gen. Nathaniel Greene' s Army. He later served as mountaineer ranger under Gen. John C. Clark of Georgia. He was discharged in 1783. 
Pension: S32274 (NC); Montgomery County, Illinois, Pension Roll, September 24, 1833, age 71 
GORDON, JAMES 
Born: Carlisle, Pennsylvania 
Died: July 9, 1835 
Buried: Jacksonville, Morgan County, Illinois 
Service: Soldier; Pennsylvania 
Sources: DAR 
GORDON, JESSE 
Born: October 3, 1755 in Virginia 
Died: August 27,1850, Jackson County, Illinois Buried : Stearns-Lindsey Cemetery, Pomona Township, Jackson County, Illinois 
Spouse: Nancy 
Service: Orderly Sergeant, First Lieutenant; Ga:NC. He enlisted in 1776 in the North Carolina troops in Capt. William Shepherd's Company, Col. Joseph Williarms Regiment, as Orderly Sergeant; in 1777 he served as First Lt. in Wilkes County, Georgia, in Capt. James Hawkins' Company, Col. John Stewart's Regiment, serving eleven months. In 1778 he served in Capt. John Gunnells' Company, Col. John Dooley's Regiment. He was in the battle at Kettle Creek with Col. Elijah Clarke. He was taken prisoner during the winter of 1780-81, but was paroled. He broke his parole and aided in driving the British out of Augusta . He was captured and held prisoner until 1782, when he escaped, returning to Wilkes County, where he was captured and held for eight months. He continued in service until the end of the war. After the war he fought Indians. 
Pension: Jackson County, Illinois Pension Census, June 1, 1840, age 85; Nancy W13280 (Ga:NC) 
Sources: DAR, PENSION, W, Jackson County Genealogical Society 
GORDON, JOHN 
Born: Sept ember 26,1763 in Wheeling, West Virginia 
Died: February 5, 1839 
Buried: Mt, Carmel, Wabash County, Illinois 
Spouse: Mary Roundtree 
Residences: After the war he moved to Ohio, and then to Indiana, and in 1819 to Lawrence County, Illinois. In 1829 he Was living in Mt. Carmel, Wabash County. 
Service: Private; Virginia 
Pension: He was pensioned. 
Sources: PI , W 
GOSS, JONATHAN 
Born: About 1760 
Died: After 1840 
Buried: Friendsville Cemetery, Friendsville, Wabash County, Illinois;Government Headstone 
Service: Private; Massachusetts, Rhode Island. He served in Capt. John Minot's Company, Col. Josiah Whitney's Regiment from May to July, 1777. 
Pension: S32275 (Mass:RI); Pension Census, June 1, 1840, Wabash County, age 80 
Sources: HR, PENSION, W 
GRAY, DANIEL S. 
Died: 1855 
Buried: Riverside Cemetery, Aurora, Kane County, Illinois; Private Headstone 
Sources: HR 
GRAY, ELIJAH 
Born: March 12,1764, Richmond, Berkshire County, Massachusetts 
Died: August, 1847 
Buried: Marengo, McHenry County, Illinois 
Spouse: Sarah Raymond 
Service: Private;Connecticut. He served in Samuel Canfield's Regiment of Militia 
Sources: DAR, PI 
GRAY, ELLIOT 
Born: September 17, 1755 in Pelham, Massachusetts 
Died : March 1841 
Buried: Deacon Cemetery, Groveland, Tazewell County, Illinois; Private Head­stone 
Spouse: Hannah Crawford Service: Private; Massachusetts. He enlisted at Pelham, Massachusetts, in the Company of Capt. Elijah Dwight. 
Pension: S30447 (Mass); Pension Census, Tazewell County, Illinois, June 1, 1840, age 85, residing with Eri Gray, head of family. 
Marker: His grave has been marked by Peoria Chapter DAR. 
Sources: HR, PI , PENSION , W 
GREENE, GEORGE 
Born: 1755-60, probably in North Carolina 
Died: July 25, 1834 
Buried: Clary's Grove, or Oakland Cemetery, southwest of Petersburg, Menard County, Illinois 
Spouse: Lucy Jones 
Service: Private; North Carolina. He served in Capt. Robert Porter's Company of Tryon County, North Carolina Militia, serving from October 21, 1777 to December 30, 1777. 
Sources: DAR, HS, NSDAR, PI 
GREENE, HENRY 
Died: May 1, 1837 
Buried: Hughes Cemetery, Schuyler County, Illinois 
Service: Soldier; Maryland. He enlisted in 1779 in Col. Thomas Wolford's Second Regiment, Maryland troops. He was discharged at Annapolis. 
Pension: He was pensioned Schuyler County, Illinois, October 22, 1828. 
Sources: PENSION, W 
GREENE (Green), JAMES 
Born: 1758-60, probably North Carolina 
Died: 1821
Buried: Green Cemetery, Mulberry Grove, Bond County, Illinois (also reported near East Newbern, Greenville) 
Spouse: Sarah Hix 
Service: Private; North Carolina Militia 
Marker: His grave was marked by Ninian Edwards Chapter DAR on May 17, 1972. 
Sources: DAR, HR, NSDAR, PI 
GRIFFITHS, WILLIAM 
Buried: Probably Fulton County, Illinois 
Service: Soldier; New York. He served in the Thirteenth Regiment, New York troops in the Company of Capt. Holter Dunham and Col. John McCrea. 
Pension: S29185 (NY); Residence Lewistown, Fulton County, when pension claim was suspended (Act June 7, 1832) "for further proof and explanation." Pen­sion later allowed. 
Sources: PENSION, W 
GRIGG, BURWELL 
Born: 1755 in Virginia 
Died: 1837 
Buried: Hastings Cemetery, Woburn, Bond County, Illinois 
Spouse: Sabra Elam 
Service: Lieutenant; Virginia. He served as a First Lieutenant in the Brunswick County, Virginia, Militia. NOTE: New information indicates that this Burwell Grigg was not a Revo­lutionary War soldier. 
Sources: HR, NSDAR, PI 
GRISWOLD, ADONIJAH 
Born: June 11, 1758 at New Milford, Connecticut 
Died: September 1, 1841 
Buried: North Cemetery, White Hall, Greene County, Illinois; Private Headstone and Government Headstone 
Spouse: Mary Barton 
Service: Patriotic 
Service: Vermont Militia. He entered the service in Vermont under Capt. Joshua Barnum and Major Gideon Brownson, Vermont Militia, serving as a scout until 1778. He was taken prisoner and held in Quebec until 1781. 
Pension: Pension Roll, November 15, 1833, Greene County, age 74; Pension Cen­sus, Greene County, June 1, 1840, age 88; Mary R4347. (Mary Griswold, widow of above, resident Carrollton, Greene County when pension claim was suspended (Act of July 7, 1838) "No proof of service ­husband improperly pensioned." 
Sources: HR, HS, PI, PENSION, W 
GULICK (Gullick), JOHN, JR. 
Born: About 1743 
Died: After 1815 
Buried: Gulick Cemetery, near Highland, Madison County, Illinois 
Spouse: Rebecca Davidson 
Service: Patriotic Service; North Carolina. He was a Patriot who furnished sun­dries to the Militia. 
Sources: NSDAR, PI 
GUM, JACOB 
Born: August 15, 1764/5, Augusta County, Virginia 
Died: April 15, 1847 
Buried:. Private Cemetery (Gum), Henderson, Knox County, Illinois 
Spouse: Rhoda Bell, born Germany; died March 11, 1852 
Children: John B., Zephaniah, Jessie, James Residences: He was a pioneer of Knox County, Illinois, settling at Henderson Grove May 1, 1828. First minister in the County (Baptist). 
Service: Private; Virginia. He served in Mikel Hubkle's (Michael Humble's) Kaintuck Virginia Militia under George Rogers Clark, from July 18 to August 21,1780; on the payroll of Jefferson County Militia under Capt. John Vertrose in 1782. Marker: His grave was marked by Rebecca Parke Chapter DAR, Galesburg, and Colonel Jonathan Latimer Chapter DAR, Abingdon, on October 31, 1937. 
Sources: DAR, HR, NSDAR, PI 
GUNDY, JACOB 
Born: October 13, 1765 in Pennsylvania 
Died: September 23, 1845 
Buried: Gundy Cemetery near Bismark, Vermilion County, Illinois 
Spouse: Katherine Maury Children: Joseph 
Residences: After the war he removed to Ohio, and from there to Vermilion County in 1830. 
Service: Private; Pennsylvania. He enlisted in April, 1779 in the Lancaster County, Pennsylvania Militia, under Capt. Sebastian Wolf and Quartermaster General Robert Patton, serving as a teamster. 
Pension: S32284 (Penn); Pension Census, Vermilion County, June 1, 1840, age 73, living with Joseph Gundy. Marker: His name is on a plaque above a drinking fountain in front of the Post Office in Danville, placed by Governor Bradford Chapter DAR. 
Sources: DAR, HR, NSDAR, PI, PENSION, W 

HADDEN, ELISHA 
Buried: Probably Coles County, Illinois 
Service: Soldier; North Carolina. He served in the battle of King's Mountain and was wounded in a battle with the Cherokee Indians. 
Pension: R4412 (NC:Va); Residence Coles County when pension claim rejected (Act June 7,1832) "frontier service not under military organization." 
Sources: PENSION, W 
HAGGARD, DAVID
Born: February 4,1764 in Albemarle County, Virginia 
Died: April 15, 1843 
Buried: Evergreen Cemetery, Bloomington, McLean County, Illinois; Private Head stone Married: 1779 
Spouse: Nancy Dawson 
Children: Elizabeth Lander, Martha Routt-Newton, Sally Lander, Melvina Rucker, Cynthia Babbitt, Mahala McCaughan, Louesa Blakely, Louisa Thompson, Dawson, John 
Residences: He removed to Clark County, Kentucky, in 1792; to Trigg County, Kentucky, in 1823, and to Bloomington, McLean County, Illinois, in 1836. 
Service: Private; Virginia. He served in the Virginia line of troops and was at the siege of Yorktown. He was present at the surrender of Lord Cornwallis October 19, 1781. David Haggard had a twin brother Bartlett, not a soldier, who would go to David's camp and exchange clothes with his twin. David would then go home for a visit and "no one would be the wiser." 
Pension: R4429 (V a): Residence Bloomington, McLean County, when pension claim was rejected (Act June 7, 1832) " not six months service." 
Marker: His grave has been marked by Letitia Green Stevenson Chapter DAR, Bloomington. His name is on a bronze tablet of evolutionary War Soldiers from McLean County, Miller Park, Bloomington. 
Sources: DAR, HR, PI, PENSION, W 
HAGGARD, JAMES 
Born: 1759inAlbemarleCounty,Virginia 
Died: August 2, 1843 
Buried: Gardner Township, Sangamon County, Illinois; Private Headstone 
Spouse: Elizabeth Gentry 
Service: Private; Virginia Continental troops.  He enlisted from Albemarle County, Virginia, in 1780 and 1781 in Capt. John Henderson's Company, Col. John Lindsey's Regiment. 
Pension: S31109 (Va); Illinois Pension Roll, Sangamon County, May 2, 1833, age 77 
Marker: His name is on a bronze plaque in the south mall, Old State Capitol, Springfield, placed by Springfield Chapters DAR and SAR, October 19, 1911. Sources: DAR, NSDAR, PI, PENSION, W 
HAIGHT, JOHN 
Born: August 18, 1742/3 in Rye, New York
Died: July15, 1836 in Fishkill, New York 
Buried: Restland Cemetery, Mendota, LaSalle County, Illinois 
Spouse: Miriam Swim 
Service: Captain: New York. He served in Col. Henry Luddtngtori 's Seventh New York Militia . 
Sources: NSDAR, PI 
HAILE, WILLIAM 
Died: 1832 
Buried: Probably buried in Sangamon County, Illinois 
Service: Soldier; Virginia. He enlisted in Richmond County, Virginia, and continued in the service after the close of the war. He was killed by Indians. 
Sources: W 
HAINLINE, GEORGE, SR. 
Buried: Fulton County, Illinois 
Residences: He removed to Illinois, settling in Fulton County. 
Service: Soldier; North Carolina or South Carolina. Descendants report that he was in the battle of King' s Mountain. 
Sources: W 
HALE (Hail), SIMEON 
Born: 1758 
Died: 1842 
Buried: Union Ridge Cemetery, Harolds Prairie Township. White County, Illinois 
Service: Sergeant; Virginia troops Pension: R4435 (Va) (Hail or Hale, Simon). He applied for a pension from White County. 
Residence, Carmi, White County when pension claim rejected (Act June 7, 1832) "For specification and proof of service." 
Marker: His name is on a monument placed by Wabash Chapter DAR, Carmi, in 1936, honoring soldiers buried in White County. A Government Headstone bearing his name in the old cemetery, Carmi, was dedicated by Wabash Chapter on September 21, 1964. 
Sources: DAR. NSDAR, PENSION 
HALL, EDWARD H. 
Born: June 28, 1760 in Georgia 
Died: June 28, 1838 
Buried: Miller Memorial Cemetery, Spring Valley, Bureau County, Ill.; Private Headstone 
Spouse: Rachel Barnes; born September 10, 1759; died September 10, 1878 
Service: Soldier; North Carolina.Heservedin1780 from North Carolina in Capt. Fletcher's Company, Col. Malmeday's Regiment; was taken prisoner and later escaped. 
Pension: S32294 (NC) 
Marker: A bronze marker was placed on the grave by Princeton-Illinois Chapter DAR, Princeton, and the Veterans Committee in July 1960. 
Sources: DAR, HR , NSDAR , PI , PENSION 
HALL, WILLIAM 
Born 1762 near Lancaster, Pennsylvania 
Died: May 13, 1846 
Buried: Hall Cemetery, Collinsville, Madison County, Illinois; Government Head­stone 
Spouse: __Holland 
Residences: He lived in North Carolina and Tennessee, and in 1815 moved to Madison County, Illinois, settling near Collinsville. 
Service: Sergeant; NC and SC. He enlisted April, 1779 at Long Cane, South Carolina, serving for his uncle, William. He marched to Savannah, Georgia. He joined Gen. Benjamin Lincoln at St. Mary's in the Company of Capt. James McCall. He was made Sergeant in Capt. William Alexander's Com­pany; served in the Companies of Capt. Gilbert Falls and Capt. James Duckworth. He aided in the defense of Charleston; entered Capt. John Pitt's Company and was detailed to transfer provisions to Gen. Horatio Gates until the battle of Camden, August, 1780. He was in the battles of Ramsour Mills, Guilford Court House, and the battle of Eutaw Springs, from which engagement he delivered seventy-five prisoners to Gen. Francis Locke. 
Pension: S31089 (NC:SC); Madison County. Illinois Pension Roll, April 9, 1833, age71; Madison County Pension Census, June1, 1840, age 88 
Marker: His name is on a bronze tablet in the Madison County Court House at Edwardsville, placed by Ninian Edwards Chapter DAR, Alton, September 16, 1912. 
Sources: DAR, HR, PI, PENSION, W 
HALLENBECK, MATHIAS 
Born: 1752 
Died: At Erie, Whiteside County, Illinois 
Service: Soldier; New York. He served in the Eighth Regiment, Albany County, New York Militia. 
Sources: HS 
HAMILTON, THOMAS 
Born: December 24,1762 in New Jersey 
Died: February 14, 1841 
Buried: Wolf Creek Cemetery, Saline County, Illinois 
Service: Private; Spy: North Carolina. He enlisted in August, 1780 in North Carolina in Capt. Arthur Forbus' Company, Col. John Paisley's Regiment; three months in Capt. James Wilson's Company, Col. William R. Davie's Regiment; from December 1,1780, for five months in Capt. David Gillaspie's Company, Col. Paisley's Regiment; from March, 1782, for two months under Col. Edward Gwin and Col. William Washington; and a few days with Lt . George Parkes. 
Pension: S6982 (NC); Pension Census, June 1, 1840, Gallatin County, Illinois, age 84 Marker: His name is on a marker on the grounds of the Saline County Court Hou se, Harrisburg, pla ced by Michael Hillegas Chapter DAR in 1931. 
Sources: DAR, NSDAR, PI, PENSION, W 
HANCOCK,BENNETT 
Born: 1756, probably Virginia 
Died: April 7, 1833 Buried: Gallatin County, Illinois 
Service: Soldier; Virginia. He served in the Virginia troops under Col. Christian Fehiger, for which he was granted 100 acres of bounty land. 
Pension: He was pensioned; BLWT-1629-100 issued June 8, 1829(Va) 
Sources: PENSION, W 
HANEY, FRANCIS 
Born: April 19, 1754 in Prince Edward County, Virginia 
Died: After 1840 Buried: Jefferson County, Illinois 
Residences: He first settled in Morgan County, Illinois, but removed to Jefferson County. 
Service: Soldier; Virginia. He enlisted in Botetourt County, Virginia, serving from June to December, 1776 in Capt. Daniel Smith's Company; from September, 1778, for seven months in Capt. George Adams Company; three months in Col. Isaac Shelby's Regiment; nine months with Captains George Maxfield, Robert Caldwell, and John Martin. Pension: S32292 (Va ); Pension Census of Jefferson County, Illinois, June 1, 1840, age 86, residing with William Hick s, head of family. 
Sources: PENSION , W 
HANSON, JOHN 
Born: About 1761 in Virginia 
Died: July 25, 1835 
Buried: Wayne County, Illinois Residences: He lived in Indiana before moving to Wayne County, Illinois. 
Service: Soldier; Virginia: North Carolina. He enlisted in 1778 for two months in Capt. Evans Shelby's Company; in 1781 for ten months in Capt. John  Mcllhaneys Company, Col. Hammond's Regiment, North Carolina troops. 
Pension: He was pensioned. 
Sources: W 
HAPONSTALL (HaptonstaII), ABRAHAM 
Born: April 6, 1761 in Orange County, New York 
Died: October 22, 1847 
Buried: RussellCemetery,Knoxville,Knox County,Illinois;GovernmentHeadstone 
Married: 1784 
Spouse: Rachel Price, died June 16, 1834 
Children: William, Abraham, Margaret, Elizabeth, Sarah, Charity, Samuel, Mary, Charles, Jacob, Rachel. Residences: After the war he removed to Ohio, and from there to Knox County, Illinois. Service: Private; New York. He served for three months in 1775 in Capt. Thomas Moffatt's Company, Col. Nathaniel Woodhull's New York Militia; three months in 1776 in Capt. Seth Marvin's Company; and six months in 1777 in Capt. Francis Smith's Company. He served under General Anthony Wayne at Woodskills, New York. Pension: S32302(NY) 
Marker: His grave was marked by the Rebecca Parke Chapter DAR, Galesburg on October 21, 1956. 
Sources: DAR, NSDAR, PI, PENSION, W 
HARDESTY, HEZEKIAH 
Born: September 2, 1763, Maryland 
Died: About 1846 
Buried: Orio Cemetery, near Allendale, Wabash County, Illinois 
Spouse: Sarah Griffin (Susannah Griffing), born December 15, 1764 
Children: Mary, Richard, Sarah, Rachel, Lydia, Daniel, Samuel, Susannah, jean, Elizabeth 
Residences: He first sett led in Lawrence County, resided for a time in Fulton County, returning to Lawrence County. He died in Wabash County. 
Service: Private; Pennsylvania. He served first in 1778 from Washington County, Pennsylvania, as a substitute for john Hardesty. He had six periods of service from 1778 through May, 1782, under Ensign Charles Goodwin, Captains David Owen, Joseph Cross, Ruble, Joseph Bean, and William Crawford, with Col. William McFarlan and General Lochlan McIntosh. 
Pension: 532296 (Penn); Pension Roll, July 16,1833, age 71; Illinois Pension Cen­sus, Fulton County, June 1, 1840, age 77, residing with Richison Spencer, head of family. 
Marker: The grave was marked by Mount Carmel Chapter DAR on November 5, 1926. 
Sources: DAR, HR, NSDAR, PI, PENSION, W 
HARGRAVE, JOHN 
Born: November 23, 1755 in South Carolina 
Died: September, 1834 
Buried: Union County, Illinois 
Spouse: Kathryn McNeal 
Residences: He came to Union County, Illinois, in 1809. 
Service: Private; South Carolina Continental troops. He enlisted in 1776 for two and one-half months in Capt. Dennis Haukins' Company, Col. Daniel Horry's Regiment; in 1780 for ten months in Capt. Thomas Hemphill's Company, Col. Francis Locke's Regiment; in 1781 for six months in Capt. Francis Boykin's Company, Col. Charles Middleton's Regiment. He was in the battles of Ram sour's Mill and Eutaw Springs. Pension: S32297(SC); Illinois Pension Roll, Union County, July 16, 1833, age 78 
Sources: PI, PENSION, W 
HARKNESS, JAMES 
Born: June 15,1759 
Died: August 18, 1836 
Buried: Harkness Cemetery, Elmwood, Peoria County, Illinois; Government Headstone
Spouse: Elizabeth Edson 
Service: Private; Massachusetts. He enlisted at age 15 as a "Minute Man" from Pelham, Massachusetts, in Capt. Candless' Company, Col. Benjamin Wood­bridge's Regiment. He served as Corporal and Sergeant in Capt. Joseph Perkins' Company, Col. Nathaniel Wade's Regiment in 1778. 
Pension: Elizabeth W21243 (Mass) Marker: His grave has been marked by Peoria Chapter DAR. A bronze marker was placed by Twenty-first Star Chapter DAR and DAR descendants on July 6, 1974. 
Sources: DAR, PI, PENSION, W 
HARLIN, WILLIAM E. 
Died: March 30, 1826 
Buried: Dickeyville Cemetery, Sims, Wayne County, Illinois; Private Headstone Sources: HR 
HARMON, JEHIAL (Jehiel) 
Born: October 5, 1762 in Suffield, Connecticut 
Died: March 3, 1845 
Buried: Greenwood Cemetery, Rockford, Winnebago County, Illinois 
Spouse: Elizabeth West 
Service: Private; Connecticut. He served during the closing six months of the war, taking the place of an older brother who was ill. 
Pension: He was pensioned. Betsey E. W25759 (Conn) BLWT 26615-160-55 
Marker: A bronze marker was placed on his grave by Rockford Chapter DAR on June 14, 1902. 
Sources: DAR, HR, PI, PENSION, W 
HARRELL, JOEL 
Born: 1748 in Bertie County, North Carolina 
Died: June 30, 1846 
Buried: Enfield Cemetery, Enfield, White County, Illinois; Government Headstone 
Spouses: (1) Polly Foster 
	 (2) Betsey Shoulders 
	 (3) Arcadia Smith 
Service: Private; NC:Va. He enlisted in Martin County, North Carolina, serving three months in Capt. May 's Company; moved to Virginia and served two months in 1781 under Major Lockard; he was at the siege of Yorktown. 
Pension: R4635 (NC:Va); Residence, Carmi, White County, Illinois, when pension claim rejected "not six months service ." 
Marker: A Government Headstone placed by Wabash Chapter DAR, Carmi, in about 1934 was dedicated on September 21, 1964. His name is on a monument in city park placed by Wabash Chapter DAR in 1936. 
Sources: DAR, HR, HS, PI, PENSION 
HARRINGTON. JOHN 
Born: About 1750 Sandisfield, Massachusetts 
Buried: Hancock Cemetery, Nauvoo, Hancock County, Illinois Service: Soldier; Massachusetts. He served in the Massachusetts line under Capt . Thomas Francis, Capt. Heyward, and Capt. Bailey, Col. Benjamin Tupper and Col. Ebenezer Sprout. Pension: S4640 (Mass). In 1819 on Ohio Pension Roll, transfer from New York, age 69; drew pension 1833-1834 in Portage County, Ohio. 
Sources: PENSION 
HARRINGTON, JOHN 
Born: February 8, 1764 in Poughkeepsie, New York 
Died: After 1841 
Buried: Victoria , Knox County, Illinois 
Service: Soldier; New York. He enlisted in Capt. Peter Magee's Company, Col. Henry Livingston's Regiment in May, serving until November in the New York line of troops. 
Pension: He applied for a pension in 1841; residence Knoxville, Knox County, Illinois, when pension claim was rejected (Act June 7, 1832) " not having six months service." 
Sources: PENSION, W 
HARRIS, WOOTEN 
Born: March,1759in Virginia 
Died: February 11, 1840 
Buried: Scribner Cemetery, Fillmore Township, later removed to Fillmore Ceme­tery, Montgomery County, Illinois Spouse: Frances Adams Residences: He lived in the Hurricane Settlement of Montgomery County. 
Service : Private; Virginia Continental troops. He enlisted in 1777 in Brunswick County, Virginia, Capt. Elliot's Company, Militia, serving ninety days; he enlisted in Capt. William Peterson's Company, Col. Charles Harrison's Regi­ment, serving until the close of the war. 
Pension: Illinois Pension Roll, Montgomery County, December 1, 1832, age 76. Frances W23186(Va) 
Marker: His grave has a DAR marker. 
Sources: DAR, PI, PENSION, W 
HARRISS (Harris), WILLIAM
 Died : Probably Vermilion County, llIinois 
Service: Soldier; Pennsylvania 
Pension: Residence Danville, Vermilion County, when pension claim was rejected (Act June 7, 1832) " Not six months service." He was awarded Bounty Land. 
Marker: His name is on a plaque on the drinking fountain in front of the Post Office at Danville, placed by Governor Bradford Chapter DAR. 
Sources: DAR, PENSION,W 
HARRISON, ANTHONY A. 
Born: March 18, 1763 in Westmoreland County, Virginia 
Died: 1842 
Buried: Madison County, Illinois 
Service: Private; Virginia. He enlisted in February, 1781 in Greenville County, Virginia, serving five months in Capt. John Lucas' Company; six weeks in Capt. Benjamin Newson's Company; he enlisted in his brother's Company, Capt. Joseph Harrison, Col. Alexander Dick's Regiment. He was in the battle of Petersburg. . 
Pension: S32303 (Va): Madison County, Illinois Pension Roll, January 9, 1834, age 71; Madison County Pension Census, June 1,1840, age 77, residing with William L. Harrison, head of family. 
Marker: His name is on a bronze marker at the Madison County Court House, Edwardsville, erected by Ninian Edwards Chapter DAR, Alton, September 16, 1912. 
Sources: DAR, PENSION, W 
HARRISON, EZEKIEL 
Born: October 6, 1752 
Died: April 17, 1836 Buried: Harrison Cemetery, near Salisbury, Sangamon County, Illinois 
Spouse: Sarah Bryan 
Residences: He came to Illinois in 1822 with his wife, three sons and one daughter, settling in Cartwright Township, Sangarnon County. Harrison­burg, Virginia, was named for his father Thomas Harrison, founder. 
Service: Private; Virginia Continental troops. He was a soldier in the Virginia line of troops and was wounded at the battle of Point Pleasant. 
Pension: Illinois Pension Roll, Sangamon County, May 2, 1833, age 83; Sarah W23211 (Va) 
Marker: His name is on a bronze plaque in the south mall, Old State Capitol, Springfield, placed by Springfield Chapters DAR and SAR on October 19, 1911. The grave was marked by Springfield Chapter October 9, 1969. 
Sources: DAR, PI, PENSION, W 
HARRISON, JONATHAN 
Born: About 1740 in Maryland 
Died: August, 1824, age 85 
Buried: George Doty Farm Cemetery, Oraville, Jackson County, Illinois; Private Headstone Residences: He removed to Illinois in 1823. Sources: HR, Newspaper clipping HARROLSON (Harolson), PAUL Born: Probably in South Carolina Buried: Randolph County, Illinois 
Spouse: Mary Residences: He came to Illinois in 1802, settling on the west side of the Kaskaskia River, near the mouth of Camp Creek. In 1809 he was a Justice of the Peace and from 1803 to 1809 was Commissioner and County Clerk. 
Service: Private; South Carolina Militia Pension: Illinois Pension Roll, September 18, 1833, Randolph County, age 73; Mary W1763 (NC) BLWT-l60-55 
Marker: His name is on a bronze marker on the grounds of Sparta High School, Sparta, placed by Fort Chartres Chapter DAR in 1934. 
Sources: DAR, NSDAR, PENSION, W 
HART, JOHN 
Born: In Virginia 
Died: November 19, 1833 Buried: Paradise Township, Coles County, Illinois 
Residences: He came to Illinois in 1826 from Hardin County, Kentucky, first to Wayne County and then to Paradise township, Coles County. 
Service: Private Virginia Continental troops. He served with George Rogers Clark in 1776 before coming to Illinois, and was in several battles with Indians. 
Pension: S32300(Va); Illinois Pension Roll, Coles County, January 6, 1834, age 75 
Sources: PENSION, W 
HART, THOMAS 
 Born: March 3,1757 in Farmington, Connecticut Died: January 14,1847 Buried: Belvidere Cemetery, Belvidere, Boone County, Illinois 
Spouses: (1) Ruth Payne 
	 (2) Mary? (pension) 
Service: Private; Connecticut. Heenlisted May 1, 1775 and served in Connecticut troops with Capt. John Sedwick, Capt. B. Beebe, Capt. John Pierce and Col. Benjamin Hinman. He was at the burning of Danbury, Connecticut. Pension: Pensioner New York Agency Certificate 20585. On September 29, 1846 he stated that he had resided in Belvidere, Boone County, Illinois, for 3 years; previously resided at De Ruyter, Madison County, N.Y.; Mary W17982 (Conn). 
Marker: His grave has been marked by Rockford Chapter DAR. 
Sources: DAR, NSDAR, PI, PENSION, W 
HARWICK, JACOB 
Born: 1752 in Pennsylvania Died: November 22, 1833 
Buried: Johnson Cemetery, Vienna, Johnson County, Illinois; Private Headstone 
Spouse: Catherine Service: Private; North Carolina. He served three months in 1781inCapt. Thomas Hewitt's Company, Col. Joseph Phillips' Regiment; one year in Capt. Charles Gordon's Company, North Carolina troops. 
Pension: S32298(NC); Johnson County, Illinois Pension Roll, July16,1833, age 81 
Marker: His grave has been marked by Daniel Chapman Chapter DAR, Vienna. His name is on a bronze marker at  the Johnson County Court House, Vienna, placed by the Chapter in 1918. 
Sources: DAR, HR, NSDAR, PI, PENSION, W 
HASSELL, BENJAMIN 
Buried: Probably Jefferson County, Illinois Spouse: Mary 
Service: Soldier; North Carolina 
Pension: R4728(NC).Mary, widow of Benjamin, residence Jefferson County when pension claim suspended (Act July 7, 1838) "Not six months service proved ." 
Sources: DAR, PENSION 
HATCH,EDE 
Born: September 20,1761 
Died: December 6, 1848 
Buried: Pettis Cemetery, Pike County, Illinois 
Spouse: Eunice Chapman 
Service: Private; Drummer: Connecticut 
Sources: HR, PI 
HATCH, MASON 
Buried: Probably buried in Ogle County, Illinois 
Spouse: Mittie 
Residences: His residence was Oregon, Ogle County 
Service: Soldier; NH:VA. He served in the Vermont troops in Capt. Jotham White's Company from June to December, 1781. 
Pension: Mittie R4738 (NH:Va) (deceased) 
Residence, Oregon, Ogle County, when pension claim suspended (Act July 7, 1838) "Not six months service proved." 
Sources: HS, PENSION 
HAWTHORNE, JOSEPH 
Born: July 10, 1756 
Died: October 30, 1849 
Buried: Enfield Cemetery, Enfield, White County, Illinois; Government Headstone 
Spouse: Frances Ellis 
Service: Private; South Carolina. He served in Parrot's Company, Taylor's South Carolina troops. 
Pension: Illinois Pension Census, White County, June 1, 1840, age 84, residing with William Hawthorn; Frances R4774 (SC) 
Marker: His grave was marked by Wabash Chapter DAR in about 1934 and rededicated in September 1964. His name is on a monument placed in city park by the Chapter in 1936, honoring soldiers buried in White County. 
Sources: DAR, HR, NSDAR, PI, PENSION, W 
HAWTHORNE, ROBERT 
Born: March 5,1753 in Monaghe County, Ireland 
Died: July 7, 1834 
Buried: Enfield Cemetery, Enfield, White County, Illinois; Government Headstone 
Spouse: Mary (widow of Robert McCleary, also a Revolutionary soldier) 
Service: Private; South Carolina. He served eleven and one-half months with Capt. Robert Hancock's Company, under Col. Richardson and Col. John Wise. He was in the battle of Stono Ferry. Pension: Mary W23326 (SC) 
Marker: His grave was marked by Wabash Chapter DAR in about 1934 and re­dedicated in September 1964. His name is on a monument placed in city park by the Chapter in 1936, honoring soldiers buried in White County. 
Sources: DAR, HR , NSDAR, PENSION, W 
HAYS (Hayes), WILLIAM 
Died: February 2. 1852 
Buried: Clements Cemetery, Urbana, Champaign County, Illinois; Private Head­stone 
Service: Private; Massachusetts. He served in Capt. David Cowdin's Company, Col. Benjamin Ruggles Woodbridge's Regiment; served as Corporal same Company from May to August, 1775. 
Sources: HR, W 
HEMPINSTALL, ABRAHAM 
Born: 1740 
Died: After 1783 
Buried: Russell Cemetery, Knox County, Illinois 
Spouse: Mary Wilson Service: Ensign: Virginia. He served in the Company of Capt. Thomas Moffatt. 
Sources: HR, PI 
HENDERSON, WILSON 
Born: October, 1758, probably Chester County, South Carolina 
Died: 1847 Buried: Gallatin County, Illinois 
Spouse: Sarah Frost 
Service: Private; South Carolina Pension: Residence Equality, Gallatin County, Illinois, when pension claim was suspended (Act June 7,1832) "for proof of service."
 Sources: PI , PENSION, W 
HENDRICKSON, JOHN 
Died: 1856 
Buried: Union Grove Cemetery, Pittsburgh, Williamson County, Ill . 
Service: Private; North Carolina. He served as a Guard at Washington's funeral. 
Sources: HR 
HENDRYX (Hendrix), NATHANIEL 
Born: About 1755 
Died: After 1840 
Buried: Wabash County, Illinois 
Residences: He moved to Vermilion County, Illinois, and then to Wabash County. 
Service: Private; New York. He served in the Seventeenth Regiment, Albany County Militia. 
Pension: S32309 (NY); Pension Census, Wabash County, Illinois, June 1, 1840, age 85. 
Sources: PENSION, W 
HENLINE, GEORGE, SR. 
Born: Probably Virginia 
Died: 1850 
Buried: Gilbert Cemetery near Armington, Tazewell County, Illinois. No head­stone remains; buried near son whose grave is marked. 
Residences: He came to Hittles' ·Grove, Tazewell County in 1828. 
Service: He fought in the Battle of Blue Licks, Kentucky, August 19, 1782. 
Sources: DAR 
HEREFORD (Hoffard), MALACHI 
Buried: Dodds Cemetery, Eldorado, Saline County, Illinois 
Service: Private; North Carolina: Rhode Island 
Pension: S32323 (RI) 
Marker: His name is on a marker on the Court House lawn in Harrisburg, placed by Michael Hillegas Chapter DAR in 1931. 
Sources: DAR, HR, PENSION 
HERRINGTON, DANIEL 
Born: January 4, 1756 in Pennsylvania 
Died: August 13, 1836 
Buried: Vermilion County, Illinois 
Spouse: Mary McCrea 
Service: Private; Pennsylvania: Maryland. He enlisted in September, 1776 in Capt. Jacob Treck's Company, Col. Michael Swope's York County, Pennsyl­vania, Regiment; he enlisted in Maryland in Capt. Daniel Shaw's Company, Col. Edward Cockey's Regiment, serving a total of ten months. 
Pension: Illinois Pension Roll, Vermilion County, April 23, 1833, age 75; Mary W7729 (Md:Pa) 
Marker: His name is on a plaque on a drinking fountain in front of the Post Office in Danville, placed by Governor Bradford Chapter DAR. 
Sources: DAR, PI, PENSION, W 
HESTER, FERREL (Fare!) 
Died: After 1833 
Buried: Edgar County, Illinois 
Service: Private; Maryland Continental troops: North Carolina. He enlisted in 1776 in Maryland and in 1780 in North Carolina. He was in the battles of Camden and Cowan' s Ford.
Pension: S323I5 (Md:NC); Pension Roll, Edgar County, August 22, 1833, age 83 
Sources: PENSION, W 
HEWES, WILLIAM 
Born: March 22, 1761 in Attleboro, Massachusetts 
Died: June 3, 1855 
Buried: Crete Cemetery, Crete, Will County, Illinois 
Spouse: Abigail Woodcock 
Service: Private; New Hampshire. He enlisted in June, 1780 serving five months in the Company of Captains Caleb Robinson and Nehemiah Houghton, Col. George Reid's Regiment of the New Hampshire troops. 
Pension: S32311 BLWT 7052-160-55 (NH) 
Marker: Sauk Trail Chapter DAR placed a bronze marker on grave in old Crete Cemetery on July 4, 1959. 
Sources: DAR, HR, NSDAR, PI, PENSION, W 
HEWITT, JOHN 
Born: January 25,1761 in Brunswick County, Virginia 
Died: June 6, 1848 
Buried: Reported: Providence Cemetery, East Carrolton, Greene County, Illinois; White Hall Cemetery, White Hall, Greene County, Illinois 
Spouse: Elizabeth Ratliff 
Service: Sergeant; North Carolina Continental troops. He enlisted in August, 1778 in North Carolina; re-enlisted seven times, serving until 1781, under Captains: James Williams, Ballard Smith, William McFarlane, James Moore, John Henderson, Nathan Coodye, Richard White, John Fuller, and Col. Francis Malmady. He was in the battles of Guilford Court House and Ram­sour's Mill. He also served as Sergeant of the Eighth Virginia Regiment. 
Pension: Pension Roll, June 6, 1833, Greene County, age 73; Pension Census of June 1, 1840, Greene County, age 79. Elizabeth W3682 (NC) 
Sources: HR, PI, PENSION, W 
HICKLIN, JONATHAN 
Born: January 22, 1756 in Virginia
 Died: February 14, 1858, age 102 
Buried: Donner Cemetery, near Owaneco, Christian County, Illinois; Government Head stone 
Residences: In 1813 he removed to Indiana, then to Wabash Township, Clark County, Illinois, and from there to Christian County. 
Service: Private; Kentucky. He served as a Spy in the Indian wars in Kentucky. He also served in Company C of the Kentucky Militia. 
Marker: His grave was marked by Peter Meyer Chapter DAR, Assumption in 1965. 
Sources: DAR, HR, NSDAR, W 
HICKS, DAVID 
Born: In Virginia 
Died: December 14, 1834 
Buried: Hicks Cemetery, or White Hall Cemetery, White Hall, Greene County, Illinois; Government Headstone 
Service: Sergeant; Virginia. He had service both in the Virginia Militia and in Crockett's Regiment in 1777. He came to Kaskaskia with Col. George Rogers Clark . 
Sources: HR, HS 
HICKS, JAMES 
Died: 1825
Buried: George Doty Farm Cemetery, Oraville, Jackson County, Illinois 
Sources: HR 
HIGHSMITH, THOMAS 
Born: About 1758, probably in Halifax County, North Carolina 
Died: 1820-1830 
Buried: On farm near Flat Rock, Crawford County, Illinois 
Spouse: Sarah 
Children: Daniel, Elizabeth, Nancy, Mary (Polly), James, Martha (Patsey), Richard M .. LUcinda , Nathaniel, John, William, Hopestill, Sarah Residences: In 1782 Surry County, North Carolina; in 1787 to Duplin County, North Carolina; 1788 Wilkes County, Georgia; 1795 Henry County, Kentucky; about 1799 to Warren County, Kentucky; late 1811 to Fort Allison , Illinois (now Russellville, Lawrence County); 1816 to Henry County, Kentucky; and 1820 to Illinois.
Service: Private; North Carolina
Sources: "Highsmiths of America" by Annette Paris Highsmith 
HILL, REUBEN 
Born: January 19, 1765 in Connecticut 
Died: January 31, 1858 
Buried: Wauconda Protestant Cemetery, Wauconda, Lake County, Illinois; Private Headstone 
Spouse: Patience Service: Private; New York 
Pension: He was pensioned. 
Sources: NSDAR, PI 
HlLMAN (Hillman), THOMAS 
Born: August 22,1757 in Salem County, New Jersey 
Died: May 5, 1835 Bedford, Indiana 
Buried: His remains were removed to Richardson Hill Cemetery, Dahlgren Town­ship, Hamilton County, Illinois; Government Headstone 
Service: Private; New Jersey Militia and Continental line. He served in Capt. Smith's Company, New Jersey troops. Pension: S32320 (NJ). He applied for pension while living in Henry County, Indiana. 
Sources : HR , PEN SION 
HILTEBRAND (Hilterbrand), JOHN 
Died: Randolph County, Illinois 
Residences: He came to Illinois in 1780 and settled on the east side of the Kas­kaskia River, near the mouth of Nine-Mile Creek. 
Service: Soldier under Col. George Rogers Clark 
Marker: His name is on a bronze marker on the grounds of Sparta High School at Sparta, placed by Fort Chartres Chapter DAR, 1934. 
Sources: DAR, NSDAR , W 
HILTON, ANDREW 
Born: 1757 in Charles County, Maryland 
Died: Monroe County, Illinois 
Service: Private; Maryland Continental troops. He served for three months in Capt. Charles Mills Company, Col. Hawkins, Regiment; six months in Capt. Henry Bowman's Company, Col. Hawkins' Regiment. 
Pension: S32316 (Md), Illinois Pension Roll, Monroe County, October 23, 1833, age 77 
Sources: PENSION, W 
HIX, DAVID 
Died: Randolph County, Illinois 
Residences: He came to Illinois in 1780 and settled on the east side of the Kaskaskia River, near the mouth of Nine-Mile Creek. 
Service: Soldier; Virginia. Soldier under Col. George Rogers Clark. 
Marker: His name is on a bronze marker on the grounds of Sparta High School at Sparta, placed by Fort Chartres Chapter DAR, 1934. 
Sources: DAR, NSDAR, W 
HOAR, DAVID 
Died: Randolph County, Illinois 
Service: Soldier; Massachusetts. He remained in the service after the close of the war. 
Marker: His name is on a bronze marker on the grounds of Sparta High School, Sparta, placed by Fort Chartres chapter DAR, in 1934. 
Sources: DAR, NSDAR, W 
HOBART, JONAS 
Born: November 15, 1744 in New Hampshire 
Died: December 15, 1833 
Buried: Foster Cemetery, McDonough County, Illinois (near Fulton County line) 
Spouse: Betty Kemp 
Residences: Coming to Illinois, he first lived in Schuyler County but removed to McDonough County. 
Service: Private; New Hampshire. He enlisted March 17, 1777, following the death of his brother Isaac at Bunker Hill. He served as Corporal in the Fourth Company, First Regiment, New Hampshire troops. He was wounded in the battle of Ticonderoga. He was discharged January 1, 1781. 
Pension: S35416 (NH); Pension Roll, July 5, 1812, April 24, 1816, April 26, 1819; transferred from Vermont 
Sources: NSDAR, PI, PENSION, W 
HODGE, FRANCIS 
Born: About 1760 
Died: About 1836 
Buried: Frankeberger Cemetery, south of Ellsworth, McLean County, Illinois 
Spouse: Nancy Walker 
Children: William Residences: He moved to Tennessee in 1812, and to Blooming Grove, McLean County, Illinois, with his son, William, in 1824. , 
Service: Private; New Jersey . He served in the New Jersey ''Continental Army in the Artillery, commanded by Capt. Charles Harrison. 
Marker: His grave was marked by Letitia Green Stevenson Chapter DAR, Bloom­ington on May 30, 1963. 
Sources: DAR, HR, NSDAR, W 
HOLLIDAY, JOB 
Died: Probably LaSalle County, Illinois 
Service: Soldier; Massachusetts 
Pension: R5148 (Mass); Residence Ottawa, LaSalle County, Illinois, when pension claim was suspended (Act June 7, 1832) "for proof by witnesses, as the Massachusetts Rolls are silent." 
Sources: PENSION 
HOOD, WILLIAM 
Died: January 18,1827 
Buried: Hood Cemetery at the junction of the Skillet Fork and Little Wabash Rivers, near Carmi, White County , Illinois. 
Service: Private; Virginia Continental troops. He served as Ensign in Capt. James Calderwood's Company, Col. Daniel Morgan's Regiment from May 31, 1777 to November 30, 1778. He also served in Col. John Nevill 's Virginia Regiment. 
Pension: S35414(Va); Illinois Pension Roll, White County, January 14, 1826 
Marker: A flat government marker was placed in the old cemetery at Carmi on September 21, 1964 by Wabash Chapter DAR, Carmi. His name is on a mon­ument in city park, placed by the Chapter in 1936 honoring soldiers buried in White County. 
Sources: DAR, PENSION, W 
HOOKER, JOHN 
Died: Probably Franklin County, Illinois  
Service: Soldier; North Carolina. He served from Granville County, North Carolina. 
Pension: R20384 (NC, SC?, VA?); Pension claim rejected when resid ent of Frank­fort, Franklin County, Illinois (Act June 7, 1832) "only 4 months service." 
Sources: PENSION, W 
HOPKINS, REUBEN 
Born: June 1,1748, Dutchess County, New York 
Died: August 12, 1822 
Buried: Lusk Cemetery, Edwardsville, Madison County, Illinois 
Spouse: Hannah Eliot 
Residences: Was elected to the Senate and lower House in Orange County, New York. Service: First Lieutenant: New York. He enlisted in New York; was at the battle of Bunker Hill. 
Sources: NSD AR, PI 
HOPKINS, WILLIAM 
Died: May 11, 1860 
Buried: Kellar Cemetery, Lovington, Moultrie County, Illinois 
Sources: HR 
HORNBACK (Hornbeck), ABRAHAM 
Born: October 21, 1761 in Hampshire County, Virginia 
Died: January 25, 1833 
Buried: Hornback Cemetery, near Petersburg, Menard County, Illinois 
Spouses: (1) Eliza Trumbo, married in Bourbon County, Kentucky 
	 (2) Mrs. Betty Mappin Bracken, married September, 1817; buried in Hornback Cemetery 
Service: Private; Virginia. He enlisted at 15 years of age in Capt. Abel Westfall's Company, Eighth Virginia Regiment, commanded by Col. Abraham Bowman. 
Marker: A bronze table was placed on his grave by Pierre Menard Chapter DAR, Petersburg, on September 13, 1936. 
Sources: DAR, HR, NSDA R, PI 
HORNER, NICHOLAS 
Born: March 26,1732 in England 
Died: January 15, 1814 
Buried: Near Lebanon, St. Clair County, Illinois 
Spouse: Barbara Residences: He was living in Maryland in 1790, but removed to Lebanon, St. Clair County, Illinois, about 1814. 
Service: Soldier; Pennsylvania. He enlisted in Pennsylvania, serving as a Ranger on the frontier; served in the Fourth Pennsylvania Battalion from January 3, 1776 to January 3, 1777 under Col. Anthony Wayne. 
Sources: PI, W 
HOTCHKISS, AMBROSE 
Born : February 16, 1762 
Died: April 10, 1841 Buried: Johnson Cemetery, Grandview Township, Edgar County, Illinois 
Spouses: Sallie Barnard 
	 Lucretia
Service: Private; Connecticut. He served in Capt. Theophilus Munson's Company, Col. Zebulon Butler's Regiment from January 1, 1780 to December 12, 178l. 
Pension: Lucretia Hotchkiss was listed in the Edgar County Pension Census, June 1, 1840, age 7l. 
Sources: HS, PI, PENSION 
HOUGHAM, MOSES 
Born: About 1744 in Maryland 
Died: 1845 (age 101) 
Buried: Scogin Cemetery, southwest of Bloomington, McLean County, Illinois; Private Headstone Spouse: Catherine Pitts 
Children: William, John, Isaac, Runyon, Moses, Jr., Catharine Ellington. 
Residences: He came to McLean County around 1830. 
Service: Patriotic Service: Maryland:Virginia. He served in Capt. Michael Cre­sops  First Company, Maryland Rifles 114 days in 1775. He was mustered out of the service at Pittsburg, Pennsylvania. 
Marker: His grave was marked by Letitia Green Stevenson Chapter DAR, Bloom­ingtou in 1928. His name is on the bronze plaque of the Soldiers monument in Miller Park, Bloomington. 
Sources: DAR, HR, NSDAR, PI, W 
HOUGHTON, AARON 
Born: April 15, 1761 in Hopewell, Hunterdon County, New Jersey 
Died: October 8, 1835
Buried: Rock Creek Cemetery, near Petersburg, Menard County, Illinois 
Spouse: Elizabeth Sexton 
Residences: He removed to Kentucky, and from there to Sangamon County, Illinois in 1834, then to Menard County  
Service: Private; New Jersey. He enlisted in June, 1776, serving until April, 1777 in his father's Company, Capt. Joab Houghton, Col. James Johnson's Regi­ment; one month in Lt. William Parks' Company, Col. Joab Houghton's Regiment . 
Pension: S32330 (NJ) 
Marker. A bronze marker was placed on his grave by Pierre Menard Chapter DAR, Petersburg on July 15,1934. 
Sources: DAR, HR, NSDAR, PI, PENSION, W 
HOUGHTON, JOAB 
Born  July 10, 1725 
Died: October 17, 1798 
Buried: Rock Creek Cemetery, near Petersburg, Menard County, Illinois 
Spouse: Catherine Runyon 
Service: Lieutenant Colonel: New Jersey. He served as Captain and Lieutenant Colonel in the New Jersey Militia 1776-1780. 
Sources: NSDAR, PI 
HOUSE, ELIAS 
Died: May 20, 1826, very aged 
Buried: Union County, Illinois 
Service: Private; North Carolina 
Pension: S24216 (NC); Pension Roll, Union County, Illinois, September 27, 1824, transfer from North Carolina
Sources: PENSION, W 
HOWARD, JOHN 
Born: January 25, 1760 in Southampton County, Virginia
Died: 1838 
Buried: Probably Fulton County, Illinois 
Children: William A. 
Service: Private; Virginia. He served from Brunswick County, Virginia, in Capt. John Macklin s Company, Col. Frederick Macklin's Regiment; also in Capt. William Young's Company, Col. Thomas Brannon's Regiment. He was in the battle of Eutaw Springs.  He was a Baptist minister. 
Pension: R5280 (Va). He applied for a pension at Lewistown in 1834. His resi­dence was Lewistown, Fulton County, Illinois, when pension claim rejected (Act June 7, 1832) " as not having six months service." 
Sources: PENSION, W 
HOWELL, LEWIS 
Born: 1755 
Died: April 27, 1833 
Buried: Saline County, Illinois 
Spouses: (1) Mary Ann Kirk
	 (2) Mrs. Leona Sisk (after the death of Lewis Howell, she married a Barnett) 
Residences: He removed from Virginia to Kentucky and from there to Saline County, Illinois. 
Service: Private; Virginia 
Pension: Pension Roll, February 15, 1820, Gallatin County, transferred from Kentucky. Leona W9719 (Va) 
Marker: His name is on a marker on the Court House lawn in Harrisburg, placed in 1931 by Michael Hillegas Chapter DAR. 
Sources: DAR, NSDAR, PI, PENSION, W 
HUBBARD, PETER 
Born: July 31,1756 in South Carolina 
Died: August 26, 1844 
Buried: Neely (or Hubbard) Cemetery, Mulberry Grove, Bond County, Illinois Spouse: Mary Residences: He lived for a time in Tennessee and from there moved to Bond County, Illinois.
Service: Lieutenant: South Carolina. He served for three years in the Company of Capt. Samuel Wise and Capt. John Carraway Smith, Col. William Thomp­son's Regiment. He was in the battle of Sullivan Island. 
Pension: S31154(SC);Pension Census, Bond County, Illinois, June 1, 1840, age 84 
Sources: HR, NSDAR , PI, PENSION, W 
HUDSON, SAMUEL 
Buried: Probably Madison County, Illinois 
Service: Soldier; New Jersey. He enlisted from Middlesex County, New Jersey. 
Pension: Residence Edwardsville, Madison County, Illinois, when pension claim suspended "for proof of service." (Act June7,1832) 
Sources: PENSION, W 
HUGHES (Hughs), HENRY 
Buried: Wayne County or Lawrence County, Illinois 
Spouse: Keziah 
Residences: He came to Fairfield, Wayne County, Illinois. 
Service: Soldier; Virginia 
Pension: R5453 Keziah, widow of Henry; Kasia R5353. Residence Lawrence County, Illinois, when pension claim rejected (Act July 7, 1838) "For proof of service and specifications." 
Sources: HS, PENSION 
HULS, JAMES 
Born: 1761 in Virginia 
Died: 1834 Buried: Vermilion County, Illinois 
Spouse: Martha 
Service: Private; Virginia. He enlisted March 18, 1778' in Capt. John Stith's Company, Col. John Neville's Fourth Virginia Regiment, serving for one year. 
Pension: Revolutionary War soldier, Vermilion County, Illinois, Pension Roll, September 28, 1833, age 72 Martha R5367 (Va) widow of James, resident of Danville, Vermilion County when pension claim rejected (Act July 7, 1838) "Twelve months service admitted -- proof of marriage deficient." Martha Hulse on Pension Census June 1, 1840, Vermilion County, age 78, living with John Johnson. 
Marker: His name is on a marker on the drinking fountain in front of the Post Office at Danville, placed by Governor Bradford Chapter DAR. 
Sources: DAR, PENSION, W 
HUMPHREY, GEORGE 
Born: 1764 
Died: April 10, 1840 
Buried: Edwards County, Illinois 
Spouses: (1) Martha Gerard 
	 (2) Mary Rose __ married February 25, 1838 
Residences: He removed from Indiana to Clay County, Illinois, and from there to Edwards County. 
Service: Private; Virginia. He served in the Virginia line for eighteen months; enlisted in the fall of 1781, serving until the close of the war in Capt. Hughes' Company, Col. White's First Regiment. 
Pension: Mary W369 (Va). Allowed pension while residing in Indiana. 
Sources: HS, PI, PENSION 
HUNSAKER, ABRAHAM 
Born: April 25, 1764 
Died: January 13, 1841 
Buried: Jonesboro Cemetery, Jonesboro, Union County, Illinois 
Married: 1784 
Spouse: Mary Snider 
Sources: HR 
HUNTER, JOHN 
Buried: Wabash County, Illinois 
Service: Soldier; Virginia. He served in Capt. James Gray's Second Company, Virginia troops in 1778. Pension: R5404(Va); Residence Mt. Carmel, Wabash County, Illinois, when pen­sion claim rejected (Act June 7, 1832) "As having had only five months service." 
Sources: PENSION, W 
HUPP, PHILIP 
Died: February 26, 1865? 
Buried: Piasa Memorial Cemetery, Piasa, Macoupin County, Illinois; Private Headstone 
Sources: HR 
HURLEY, JAMES 
Born: April 1, 1759 
Died: January 11, 1850 
Buried: Camp Ground, DeWitt County, Illinois 
Sources: DAR 
HURST, WILLIAM 
Born: 1755 in Berkeley County, Virginia 
Died: December 7, 1836 
Buried: Mount Carmel Cemetery, Edgar County, Illinois; Private Headstone 
Residences: After the war he removed to Kentucky, and from there to Indiana, and in 1836 to Edgar County, Illinois. 
Service: Private; Pennsylvania. He enlisted in July, 1780 in Westmoreland County, Pennsylvania, in Capt. William Campbell's Company, Col. Archibald Lough­rey's Regiment. On his way to join George Roger Clark's expedition, he was captured by Indians, was held prisoner in Detroit until May, 1781, was taken to Canada and exchanged. He arrived in New York about Christmas, 1781. 
Pension: S31155(Penn) 
Sources: HR, PENSION,W 
JACKSON, JOSEPH 
Born: 1760 in Bucks County, Pennsylvania 
Died: October 11, 1844 
Buried: Jacksonville Cemetery, Jacksonville, Morgan County, Illinois 
Spouse: Margaret Lyster Residences: He removed to Sumner County, Tennessee, and from there to Mor­gan County, Illinois.
Service: Private; North Carolina: South Carolina Continental troops 
Pension: Illinois Pension Roll, Morgan County, July 16, 1833, age 73. Margaret W370 (NC) 
Marker: His name is on a plaque in front of the Morgan County Court House, Jacksonville, placed by Reverend James Caldwell Chapter, DAR, in 1914. 
Sources: DAR, HR, PI, PENSION, W, Morgan County Genealogical Society 

JACKSON, JOSEPH 
Born: 1758 
Died: September 24, 1842 
Buried: Jacksonville Cemetery, Jacksonville, Morgan County, Illinois 
Spouse: Sarah Kirkman 
Service: Patriotic
Service: North Carolina 
Sources: HR, PI , Morgan County Genealogical Society 
JACKSON, SAMUEL 
Born: About 1754 
Died: 1845 
Buried: Franklin Cemetery, Franklin, Morgan County, Illinois 
Service: Soldier; South Carolina: Virginia. He served in Capt. William Blakeney's Company, Harlees Battalion, South Carolina troops. 
Pension: Morgan County, Illinois, Pension Census of June1, 1840, age 86 
Marker: His name is on a plaque in front of the Morgan County Court House at Jacksonville, placed by the Reverend James Caldwell Chapter DAR in 1914. 
Sources: DAR, NSDAR, PENSION, W 
JAGGERS, NATHAN 
Born: October 16, 1759 in Craven County, South Carolina
Died: December 19, 1839 
Buried: Old Bean's Cemetery (abandoned), White County, Illinois 
Service: Private; South Carolina: Virginia. He enlisted in October, 1775 for three months in Capt. Edmund Strange's Company; in 1779 for three months in Capt. John Nixon's Company, Col. John Winns Regiment; in 1780 for a year with Captains Thomas Taylor, George Hastin, Shaw and Kirkwood, in Col. Edward Lacy's Regiment Virginia troops. In 1782 he served for two months in Capt. Neeley's Company in the Regiment of Col. Edward Lacey and Col. McDonald , South Carolina troops. 
Pension: S32339 (SC: VA); Pension Census June I, 1840, White County, lists Nathan "Gaggers," age 81, residing with Anna Driggers. 
Marker: A flat marble government marker was dedicated by Wabash Chapter DAR, on September 21, 1964 in the Old Cemetery, Carmi. His name is on a monument in city park placed by the Chapter in 1936, honoring soldiers buried in White County. 
Sources: DAR,NSDAR , PI, PENSION, W 
JAMES, THOMAS 
Died: November 2, 1833 
Buried: Rochester, Sangamon County, Illinois
Service: Soldier; Pennsylvania 
Sources: W 
JAMES, WILLIAM 
Born: April 14, 1759 in Culpepper, Virginia 
Died: March 5, 1836/7 
Buried: James Cemetery (or Old Asher), near Paris, Edgar County. Illinois 
Married: October 25, 1786 
Spouse: Elizabeth Wells 
Residences: He lived in Montgomery County, Virginia; Jessamin County, Ken­tucky; Scott County, Kentucky; to Illinois in 1829, settling near Paris, Edgar County. 
Service: Soldier; Virginia. He served in Capt. Love's Company Montgomery County, Virginia, in 1781 and marched into North Carolina in the battle of Shallow Ford of the Yadkin River, two months service; he served with the same Company on the Clinch River. 
Pension: R5550 (Va); Resident of Paris, Edgar County, Illinois, when pension claim rejected (Act June 7, 1832) " as having only four months service as a Scout. " 
Sources: HS, NSDAR, PI , PENSION, W 
JANIS, NICHOLAS 
Buried: Randolph County, Illinois 
Residences: He was a resident of Kaskaskia, Randolph County after the close of the war. 
Service: Rank: Captain
Marker: His name is on a bronze marker on the grounds of Sparta High School, Sparta, placed by Fort Chartres Chapter DAR in 1934. 
Sources: DAR, NSDAR, W 
JENKINS, JOB 
Born: Probably in Virginia 
Died: January 1832 
Buried: Morgan County, Illinois 
Spouse: Elizabeth 
Residences: He removed from Virginia to western Tennessee, and from there to Morgan County, Illinois.
Service: Private; Virginia Continental troops. He served in Company No. 3 in the Eleventh and Fifteenth Regiments, commanded by Col. Daniel Morgan; with Capt. Peter Bruin and Capt. William Johnston in 1777 and 1778.
Pension: Illinois Pension Roll, June 3, 1819, transferred from West Tennessee. Elizabeth W26806 (Va) widow of Job, residence Pike County when pension claim suspended (Act July 7, 1838) "For proof of marriage." 
Marker: His name is on a plaque on the front of the Morgan County Courthouse placed in 1914 by the Reverend James Caldwell Chapter DAR. 
Sources: DAR, PENSION, W 
JENKINS, JOHN 
Born: About 1753 in Hardy County, Virginia 
Died: After 1840 
Buried: Copeland Wade Cemetery, Windsor, Shelby County, Illinois 
Service: Private; Virginia. He served from Hardy County, Virginia, in the Light Dragoons. 
Pension: S16425 (Va), Pension Census, Shelby County, Illinois, June 1, 1840, age 87, living with Van C. Dawson, head of family 
Marker: His grave has been marked by DAR. 
Sources: DAR, HR, NSDAR, PENSION, W 
JOHNSON, ANDREW 
Buried: Probably Franklin County, Illinois 
Service: Soldier; North Carolina
Pension: Residence Frankfort, Franklin County, Illinois, when pension claim was suspended (Act June 7,1832) " no proof of service." R5599 (NC) 
Sources: DAR, PENSION 
JOHNSON, ARTHUR 
Born: August 7, 1757 in Brunswick County, Virginia
Died: October 16, 1839 
Buried: Johnson Cemetery (or Seven Mile Prairie Cemetery), northeast of Enfield, White County, Illinois. His body was moved from Bandana Handkerchief, Wayne County; Private Head stone. 
Spouse: Lucy Harmon 
Residences: After the war he removed to Kentucky, and from there to Gibson County, Indiana. He then moved to White County, Illinois. 
Service: Private; Sergeant; Virginia, He enlisted in 1775 in Capt. Richard Mead's Company, Second Regiment, for one year. He served as Corporal in Capt. James Knox' s Company, Col. Abraham Bowman's Eighth Regiment from May, 1776 to April 30, 1777; enlisted January 2, 1777 for three years. He was in Capt. Thomas Berry's Company, Eighth Regiment, serving as Sergeant; in Capt. Abraham Kirkpatrick's Company, same Regiment; in Capt. William Crogan's Company, Fourth Regiment, and in May, 1779 in Capt. Leonard Cooper's Company, Col. John Nevill's Fourth Regiment. 
Pension: He was pensioned in 1818. Lucy WI0152 (Va). 
Marker: A monument has been erected by descendants, upon which is inscribed his military record. His name is on a marker at Carmi, placed by Wabash Chapter DAR, September 21, 1964, He is also named on the monument in city cemetery placed by the Chapter in 1936, honoring soldiers buried in White County. 
Sources: DAR, HR, NSDAR, PI, PENSION, W 
JOHNSON, BENJAMIN 
Born: 1758, Orange County, Virginia 
Died: After 1840 
Buried: Glenwood Cemetery, Collinsville, Madison County, Illinois 
Service: Soldier; Virginia
Pension: Residence was Frankfort, Perry County, Illinois, when pension claim was rejected (Act June 7, 1832) "Having only five months service." Pension Census, Madison County, June 1,1840, age 82
Marker: His name is on a bronze tablet in the Court House at Edwardsville, Madison County, placed by Ninian Edwards Chapter DAR, September 16, 1912. 
Sources: DAR, HR, PENSION, W 
JOHNSON, CHARLES 
Born: September 2, 1757 in North Carolina
Died: October 21, 1821 
Buried: Sugg Cemetery, Pocahantas, Bond County, Illinois; Government Head­stone 
Spouse: Polly Huston
Service: Private; North Carolina. He served in the North Carolina Militia and was in the battles of Cowpens and Guilford Court House. He had two horses shot from under him. 
Sources: HS, NSDAR, PI 
JOHNSON, HUCH 
Born: August 12, 1750
Died: February 4, 1835, age 85 
Buried: Near Trenton, Clinton County, Illinois 
Spouse: Winea Flanegan 
Residences: After the war he removed to Kentucky; came to Illinois in 1812, moved to Missouri, returning to Illinois, he settled near Trenton, Clinton County.
Service: Soldier; North Carolina 
Sources: PI, W 
JOHNSON, JOHN 
Buried: Probably Franklin County, Illinois 
Spouse: Caty
Service: Soldier; New York
Pension: Residence Frankfort, Franklin County, Illinois, when pension claim suspended (Act June 7, 1832) "no proof of service." Caty R5604(NY) 
Sources: PENSION 
JOHNSON, JOHN 
Buried: Little Springs Church Cemetery, Hamilton County, Illinois
Service: Soldier; North Carolina. He served with General Marion from Bladen County, North Carolina. 
Sources: Family Record 
JOHNSON, MOSES 
Born: About 1740
Died: After 1840 
Buried: Old Lost Cemetery, Clay County, Illinois
Service: Soldier; Virginia. He enlisted in 1777 in Capt. Alexander Morgan's Second Company, Col. Alexander Spotswood's Second Regiment.
Pensionr: S36024 (Va); Illinois pension census, Clay County, June 1, 1840, age 100
Marker: A bronze marker was placed on the grave by Vinsans Trace Chapter DAR, Flora, April 7,1973. 
Sources: DAR, PENSION, W 
JOHNSTON, JAMES 
Born: June 9, 1752 in Pennsylvania
Died: September 8, 1826 
Buried: Jackman Cemetery, St. Francisville, Lawrence County, Illinois; Government Headstone 
Spouse: Elizabeth Lindsey 
Children: There were fourteen children. 
Residences: James Johnston, together with nine other families, came from Pennsylvania, floating flat boats down the Ohio from Pittsburgh, then up the Wabash to Vincennes. He moved across the Wabash to Lawrence County, Illinois in 1815. 
Service: Lieutenant Colonel: Pennsylvania. He served three years with the Cum­berland County Militia. He also served with Colonel George Rogers Clark. 
Sources: Reported by Francis Vigo Chapter DAR, Indiana 
JOLLY, BOURLAND (Boling) 
Born: 1766 in Dinwiddie County, Virginia 
Died: 1844 
Buried: Franklin Cemetery, Franklin, Morgan County, Illinois; Government Head­stone 
Married: Chatham County, North Carolina 
Spouse: Hannah Passmore 
Residences: He removed to North Carolina after the war, and from there to Morgan County, Illinois, where he was a resident in 1842. 
Service: Private; Virginia. He enlisted in Dinwiddie County, Virginia, in 1781 and was at the siege of Yorktown. 
Pension: R5686 (Va); Residence was Jacksonville, Morgan County, Illinois, when pension claim was rejected (Act June 7, 1832) as "having less than six months service." 
Marker: His name is on a plaque in front of the Morgan County Court House at Jacksonville, placed by the Reverend James Caldwell Chapter DAH in 1914. 
Sources: DAH, HR, NSDAR, PI, PENSION, W 
JONES, JASPER
Born: Before 1759
Died: After 1780 
Buried: Browning Hill Cemetery, Buckner, Franklin County, Illinois 
Spouse: Lucy Clark 
Service: Corporal: Connecticut 
Sources: HH, PI  
JONES, JOSEPH 
Born: About 1753 in Maryland
Died: August 26, 1826 
Buried: St. Clair County, Illinois
Service: Private; Maryland: New Jersey. He enlisted May 30, 1778 for three years in Pulaski's Loyal Legion. He also served as a substitute from Anne Arundel County, Maryland.
Pension: S36667 (NJ); Pension Roll, St. Clair County, 1822, age 69 
Sources: PENSION, W 
JONES, MOSES 
Born: September 1762
Died: About 1852 
Buried: Williamson County, Illinois 
Spouse: Elizabeth 
Residences: He came to Illinois in 1819, settling in Franklin County, which became Williamson County in 1839.
Service: Private; North Carolina. He enlisted at Gates County, North Carolina, and served in Capt. Benjamin Bailey's Company, Tenth Regiment, North Carolina troops from September 10, 1782 until March 1, 1783.
Pension: S32347 (NC); Pension Claim rejected by Act of Congress May 29, 1830 "not nine months service." Franklin County, Illinois Pension Census, June 1, 1840, age 80 
Sources: PI, PENSION, W 
JONES, PETER 
Born: 1752
Died: 1829 
Buried: Union Cemetery, Waynesville,.DeWitt County, Illinois 
Spouse: Rebecca Scott
Service: Private; Pennsylvania 
Sources: PI 
JONES, SAMUEL
Died: February 24, 1853 
Buried: Paschal Farm, near Markham, Morgan County, Illinois; Private Headstone
Service: Soldier; Virginia
Marker: His name is on a plaque in front of the Morgan County Court House, Jacksonville, placed by the Reverend James Caldwell Chapter DAR, in 1914. 
Sources: DAR, HR, W 
JONES, STEPHEN
Born: About 1763 in New Jersey
Died: After 1840 
Buried: Quincy, Adams County, Illinois 
Residences: In 1837 he was a resident of Caldwell County, Missouri, age 74; resided also in Indiana before settling in Quincy, Adams County, Illinois.
Service: Soldier; New Jersey. He enlisted at Neward, Essex County, New Jersey in Capt. Cornelius Williams' Company, Second Regiment.
Pension: S15903 (NJ); Pension Census, June 1, 1840, Adams County, Quincy City, 2nd Ward, aged 77, residing with Moses Jones, head of family. 
Sources: PENSION, W 
JONES, WILLIAM
Born: March 13, 1744 in Dansbury, Bucks County, Pennsylvania
Died: Sangamon County, Illinois 
Residences: He was living in Sangamon County in 1834
Service: Soldier; New Jersey. He enlisted in Sussex County, New Jersey, for one year in Capt. John B. Scott's Company; he served five months in 1775 in 
Capt. John Seward's Company, Col. Ephraim Martin's Regiment. He served as "waiting man " for Gen. Putnam. He was in the battle of White Plains.
Pension: R5758 (NJ); Residence Springfield, Sangamon County, Illinois, when pension claim suspended "for proof of service." 
Sources: PENSION, W 
JORDAN, JAMES 
Born: November 15, 1755 near Carlisle, Pennsylvania
Died: July 9, 1835 
Buried: Massey farm two miles west of Jacksonville, Morgan County, Illinois 
Spouse: Mary McElwayne 
Children: Hannah (Hart) 
Residences: In 1802 he moved to Livingston County, Kentucky, where he lived for three years; to St. Clair County, Illinois for eight years; to Pike County, Missouri for three years; to St. Louis County, Missouri for three years; returned to Pike County for about three years; moved to Morgan County, Illinois. 
Service: Private; South Carolina. Enlisted Camden District, South Carolina, in November, 1779 in Capt. John Moffett's Company, under Colonels Neil, Bratton, Cleveland and Shelby. He was in the battles of Rooky Mount, Hanging Rock, Eutaw Springs, serving until October, 1781. 
Pension: S32346 (SC). His pension was executed from Morgan County, Illinois, commencing March 4, 1831.
Marker: His name is on a plaque in front of the Morgan County Court House, Jacksonville, placed by the Reverend James Caldwell Chapter DAR in 1914. 
Sources: DAR, HR, NSDAR, PI, PENSION, W 
JUDY, JACOB
Born: 1740 in Basel, Switzerland
Died: 1807 at his Mill 
Buried: Palmier Cemetery, north of Columbia, Monroe County, Illinois 
Spouses: (1) Maria in Switzerland; 2 children born there; Maria died by 1777 
    (2) Elizabeth Sprater, married in 1778 in Ann Arundel County, Mary­land; 2 children 
Residences: He came from Switzerland to Ann Arundel County, Maryland in 1777; removed to Fort Pitt, Pennsylvania, where he served; received a land grant of 5,662 acres in Kentucky, on which he lived from 1786 to 1788; removed to Kaskaskia, Illinois, and lived there from 1788 to 1792. He moved to Monroe County, then a part of St. Clair County, and operated Judy's Mill south of Columbia. 
Service: Patriotic
Service: Pennsylvania: Maryland. Gunsmith; paid substitute to enlist from Frederick County, Maryland in 1778.
Pension: He received a land grant.
Sources: DAR, PI, Family Records 
JUSTUS, MOSES
Born: 1755 in Maryland
Died: After 1840 Buried: Old Ridgeville Cemetery, Schuyler County, Illinois 
Spouse: Mariah. born August1755;diedOctober1841 Children: James, Sarah, Mary, Thomas Jefferson, George Washington, John, Rebecca, Hannah Residences: Old Cheraw District, South Carolina; Blount and Knox Counties, Tennessee; Crawford County, Indiana; settled in Schuyler County, Illinois before 1830. 
Service: Private; North Carolina Continental troops. He enlisted in Mecklenburg County, North Carolina, serving as a " Minute Man" in Capt. John Fifer's Company, July, 1775; served in June, 1779, and February, 1781 under Capt. Samuel Patton, Capt. Caleb Fifer and Capt. Tinnon. He was in the battles of  Stono and Wetzell's Mills.
Pension: S32351 NC:Va; Illinois Pension Roll, Schuyler County, August 14, 1833, age 79; Illinois Pension Census, Schuyler County, June 1,1840, age 85 
Sources: PENSION, W 

KARR, JOHN C.
Born: 1759 in New Jersey
Died: December 16, 1840
Buried: Heyworth Cemetery, Heyworth, McLean County, Illinois; Private Headstone
Spouse: Marcy Lee
Children: Walter, Eleanor Crivling, Joseph, John, Jr., Dr. Thomas, Nancy Noble, 
Rebecca Noble, Mercy Buck, Jacob 
Residences: He settled in Buckles' Grove, near LeRoy, McLean County, Illinois in 1839.
Service: Captain: New Jersey. He served as Captain in the Second Battalion, Somerset County, New Jersey line of troops. Pension: 531184 (NJ); Illinois Census of Pensioners, McLean County, Illinois, June I, 1840, age 81, residing with Tharosa Karl, head of family. 
Marker: Capt. Karl requested that he be buried with the honors of war and that his monument be inscribed "Sacred to the Memory of John Karr, a Soldier of the Revolution of 1776." A bronze marker was placed on his grave by Letitia Creen Stevenson Chapter DAR in 1928. His name is on the bronze tablet in the Soldiers Monument at Miller Park, Bloomington. 
Sources: DAR, HR, PI, PENSION, W 
KEEN, PETER 
Died: 1840
Buried: Friendsville, Wabash County, Illinois
Residences: He removed from New Jersey to Ohio, and in 1814 to Wabash County, Illinois, where he was one of the original proprietors of Palmyra. He later moved to Friendsville.
Service: Soldier; New Jersey Sources: W 
KEHR, DAVID
Born: July 27, 1763 near Philadelphia, Pennsylvania
Died: After 1839
Buried: Near Griggsville, Pike County, Illinois
Residences: He removed to Ohio, and from there to Pike County, Illinois
Service: Indian Spy; Pennsylvania. He enlisted in Northumberland County, Pennsylvania in April, 1780, serving until August 14, 1780 in Capt. Thomas Gaskin's Company, Col. James Hunter's Regiment. He was captured while serving as an Indian Spy, held at Niagara until July, 1783, when he was release. 
Pension: S32359 (Penn); Pension Census, Pike County, June I, 1840, age 78, residing with Jane Watson
Sources: PENSION, W 
KELL ER, ISAAC
Died: Coles County, Illinois
Residences: He came to Coles County, Illinois in 1820.
Service: Sergeant; Virginia. He served from Virginia under Col. George Rogers Clark. 
Sources: W 
KELLY (Kelley), HENRY 
Born: 1742 Rutherford County, North Carolina 
Died: June 13, 1832 
Buried: Kelley Cemetery, Curran, Sangamon County, Illinois; Private Headstone 
Spouse: Mary Whiteside born 1756; died 1840 at Bolivar, Polk County, Missouri 
Children: John, Elisha, George W., William, Elijah, Eleanor, Sally, Mary 
Residences: He came to Illinois about 1819 from Rutherford County, North Carolina .
Service: Corporal: North Carolina. He served in Capt. Augustin Spain's Company, Second Regiment, and in Col. Samuel Jarvis' First Regiment, North Carolina troops.
Marker: His grave was marked by Springfield Chapter DAR. His name is on a bronze plaque in the south mall, Old State Capitol, Springfield, placed October 19, 1911 by Springfield Chapters DAR and SAR 
Sources: DAR, PI, Family Record and County Histories 
KELLY, JOHN 
Died: March 13, 1859
Buried: Tyler Cemetery, Keens, Wayne County, Illinois; Private Headstone 
Sources: DAR, HR 
KENDRICK, WILLIAM 
Born: 1758 in Virginia
Died: December 29, 1835
Buried: Near Mt, Sterling, Brown County, Illinois
Spouse: Fanny Mitchell; died Jul y 31, 184 
Service: Private; Virginia. He enlisted in 1779 in Capt. Thomas Arrnisteds Company, Col. John Green's Regiment, serving eighteen months. He was in the battles of Guilford Court Honse, Camden, Ninety-six, and Eutaw Springs. Pension: Fanny W26743 (Va); Widow, Fanny, applied for pension February 24, 1838 
Sources: PI. PENSION, W 
KENEIPP (Kneip), CHRISTIAN
Died: 1825
Buried: Moffett Cemetery, Luken Township, Lawrence County, Illinois
Service: Hessian Soldier
Sources: Family Tradition 
KENNEY, DANIEL
Died: August 9, 1824
Buried: Crawford County, Illinois
Service: Private; Virginia Continental Troops Pension: S37127 (Va), Pension Roll, Crawford County, November 15, 1820
Sources: PENSION, W 
KENNISON, DAVID
Born: November 10, 1736 in New Hampshire 
Died: February 24, 1852
Buried: Now Lincoln Park, Chicago, Cook County, Illinois 
Residences: His family moved from New Hampshire to Maine while David was very young. 
Service: Private; Patriotic
Service: Mass; Maine. David Kennison was the last survivor of the "Boston Tea Party." He enlisted at the outbreak of the war and was in the battles of Bunker Hill, West Point, White Plains, Long Island, Fort Montgomery, Staten Island, Delaware and Philadelphia, and was present at the surrender of Cornwallis. He was also reported to have served from New York, Delaware, and Pennsylvania. 
Pension: He was pensioned. 
Marker: His grave is marked with a granite boulder, erected by the Chicago Chapter, Daughters of the American Revolution and the Sons of the American Revolution. The monument was unveiled on December 19, 1903. A boulder commemorating David Kennison, the last surviving member of the Boston Tea Party was dedicated on June 15, 1935 by Aaron Miner Chapter DAR and Doctor Bodo Otto Society, C.A.R., in Lincoln Park, Chicago. 
Sources: DAR, PI, W 
KIDD, ROBERT
Died: 1849
Buried: Renault Township, Monroe County, Illinois
Residences: He settled in Renault Township, Monroe County in 1781.
Service: He took part in the capture of Fort Cage under Col. Ceorge Rogers Clark. He is listed in non-commissioned officers and soldiers of Illinois Regiment and the Western Army entitled to Bounty Land.
Sources: W, Bounty Land Record 
KILLEBREW, LAWRENCE
Born: May 10, 1763 at Tarbury Town, Edgecomb County, North Carolina
Died: April 4, 1835
Buried: Morgan County, Illinois Married: October 4, 1783 Spouse. Elizabeth Bullock, born 1766
Children: Six children 
Residences: He came to Illinois from North Carolina prior to 1805 and settled in Morgan County.
Service: Private; North Carolina Continental. He enlisted at the age of 15 in September, 1778 at Tarbury Town in the Continental Army, serving fourteen months in the Line Infantry, and seven months in the Cavalry. From September, 1778 to December 24, 1778, he served in Capt. James Scarborough's Company, Col. Session and Col. Turner's Regiment; through July 25, 1779 in Capt. Criss Scofield's Company. He served in Capt. Armstrong's Company in Col. Hamilton's Regiment until his separation from service in June, 1780. 
Pension: Illinois Pension Roll, Jackson County, July 30, 1834, age 70. Elizabeth W24816 (NC). Elizabeth, widow, pensioned March 4, 1836, resident of Morgan County, Illinois. 
Marker: His name is on a plaque in front of the Morgan County Court House at Jacksonville, placed by the Reverend James Caldwell Chapter DAR in 1914. 
Sources: DAR, PENSION, W, Family Records 
KILLION, JACOB
Born: March 16, 1755 in Orange County, North Carolina
Died: Sept ember 19, 1838
Buried: Lebanon Cemetery, Petersburg, Menard County, Illinois
Spouse: Jennie Killion; died February 18, 1828 in Scott County, Indiana
Residences: After the war, he lived for a time in Virginia.
Service: Private; North Carolina. He served from May 25, 1781 to May 25, 1782 in the Tenth North Carolina Regiment under Col. Abraham, Major Donohoe, and General Greene. Pension: S32362 (NC)
Marker: His grave was marked by Pierre Menard Chapter DAR, Petersburg.
Sources: DAR, HR, NSDAH, PI, PENSION 
KIMES, HENRY 
Born: 1759 in Chester County, Pennsylvania 
Died: August 22, 1833 
Buried: Zion Cemetery, Lincoln, Logan County, Illinois; Government Headstone 
Married: About 1780 
Spouse: Hannah Rudolph
Service: Private; Pennsylvania. He served in Capt. Edward Vernon's Company in the Chester County Militia in 1780; and in Capt. James Denning's Company in 1781 and 1782.
Marker: His grave has been marked by Abraham Lincoln Chapter DAR. His name is on a plaque on the Logan County Court House at Lincoln placed by Abraham Lincoln Chapter DAR, June 27, 1975.
Sources: DAR, PI, W 
KINCAID, DAVID
Died: About 1828
Buried: Kincaid Cemetery, near Winchester, Scott County, Illinois
Sources: HR 
KINCAID, SAMUEL 
Died: At age of 103 years
Buried: Greenhill Cemetery, Crawford County, Illinois 
Residences: Coming to Illinois, he located at Montgomery Township, Crawford County. 
Service: Drummer: Massachusetts. He served with his father, Thomas Kincaid, as a drummer boy at the battle of Bunker Hill. He also served in the War of 1812. 
Sources: W 
KINCAID (Kincade), SAMUEL
Died: After 1819
Buried: Kincaid Cemetery, Bridgeport, Lawrence County, Illinois
Residences: He came from New Jersey to Illinois, settling in Lukin Township, Lawrence County in 1819.
Service: Soldier; New Jersey
Marker: His name is on a bronze tablet on the Lawrence County Court House, Lawrenceville, placed by Toussaint du Bois Chapter DAR in 1921.
Sources: DAR, HR, W 
KINCAID, THOMAS
Born: In Ireland
Died: After 1840, age 105
Buried: Greenhill Cemetery, Crawford County, Illinois \
Children: Samuel
Residences: He came to the United States with the British Army, but served in the Continental Line. He lived in New York, moved to Pennsylvania, to Ohio, to Kentucky, to Indiana, and in 1840 to Crawford County, Illinois
Service: Orderly Sergeant; Massachusetts. He was in the battle of Bunker Hill.
Sources: DAR, W 
KING, HUGH
Born: December 17, 1754 in Scot land
Died: After 1846
Buried: Springhill Cemetery, Danville, Vermilion County, Illinois 
Spouse: Mary Montgomery
Service: Private; North Carolina: South Carolina. He enlisted from Mecklenburg County, North Carolina, in 1778, in the Company of Captains John McRea, William Alexander, and Major William Davis in Col. John Moore's Regiment. He enlisted in the South Carolina troops in 1781, serving with Capt. Andrew Alexander and Col. Wade Hampton in Washington's Dragoons. He served seven times for a period of two years and ten months. He was in skirmishes at Charlotte, North Carolina, Strawberry Fields, Quarter House, and Ninety six. 
Pension: S32365(NC:SC)
Marker: His name is on a bronze marker at the fountain in front of the Post Office in Danville, Vermilion County, placed by Governor Bradford Chapter DAR. 
Sources: DAR, PI, PENSION, W 
KING, JOHN 
Born : About 1767 
Died: After 1834 
Buried: Clinton County, Illinois 
Residences: He came to Illinois in 1817, settling in Shoal Creek Precinct, Clinton County, Illinois.
Service: Private; South Carolina Militia Pension: S32366 (SC); Pension Roll, February 25, 1834, Clinton County, age 67
Sources: PENSION, W 
KIRBY, WILLIAM
Born: Albemarle County, Virginia
Died: June 21, 1869
Buried: Clements Cemetery, old section, Urbana, Champaign County, Illinois; 
Private Headstone
Spouse: Elizabeth
Residences: He removed to Harrison County, Kentucky, and from there to Champaign County, Illinois.
Service: Private; Virginia. He served in the Chesterfield Militia. Pension: He was pensioned. Elizabeth W7998: BLWT-I7717-160-55
Sources: HR, PENSION, \V 
KIRK, WILLIAM 
Born: 1754 in Virginia
Died: 1837
Buried: Scottsville Cemetery, Scottsville, Macoupin County, Illinois
Spouse: Agnes
Service: Patriotic Service: Virginia. He was a Gunsmith detailed to make muskets at Harpers Ferry in Col. Baylor's Light Horse Cavalry.
Sources: DAR, HS, PI 
KITCHEN, JAMES
Born: October 25, 1762
Died: November 9, 1849
Buried: Temple Cemetery, T able Grove, Fulton County, Illinois; Private Headstone
Spouse: Jane 
Service: Private; Pennsylvania. He served from Northampton County, Pennsylvania in Capt. Hugh Caston's Third Company, Fifth Battalion, 1781. 
Pension: S31192 (Pa): Jane R5997. 
Residence: Fulton County, Illinois, in Pension Census of June 1, 1840, age 77, residing with Wilham Hall, head of family.
Marker: His grave was marked by Thomas Walters Chapter DAR, Lewiston on July 30, 1972.
Sources: DAR, HR, PENSION, W 
KNIGHT, CHARLES
Born: September 20, 1760, probably in Virginia
Died: January 4, 1833 
Buried: Union Ridge Cemetery, Norris City, White County, Illinois; Private Headstone
Spouse: Martha Bartlett
Service: Private; Virginia
Marker: His name is on a marker placed by Wabash Chapter DAR, Carmi, in 1936, honoring soldiers buried in White County.
Sources: DAR, HR, NSDAR, PI, W 
KNIGHT, JAMES, SR.
Born: August 20, 1750 in South Carolina
Died: February 3, 1838
Buried: On his farm in Elbridge Township, Edgar County, Illinois
Spouse: Margaret Prigmore
Service: Private; Marine: Pennsylvania. He enlisted from Pennsylvania in 1775 and served on the frigate Randolph in 1776. His ship was in several engage­ments and captured three British ships. Pension: Knight, James: Cont: Navy: Penn. Margaret W27510 (Penn); Pension Roll, Edgar County, July 3, 1833, age 83
Sources: PI, PENSION, W 
KNIGHTEN, THOMAS
Born: March 23, 1753 in South Carolina
Died: September 3, 1835
Buried: St. Clair County, Illinois
Spouse: Jane Freeman
Service: Private and Sergeant; South Carolina Continental troops: Virginia Pension: S32368 SC:Va; Illinois Pension Roll, St. Clair County, July 18, 1833, age 81
Sources: PI, PENSION, W 

LACKEY, ADAM
Born: 1759 in Baltimore, Maryland
Died: February 13, 1836
Buried: Lackey Cemetery (or East Pinkstaff), near Pinkstaff, Lawrence County, Illinois; Private Headstone
Spouse: Catherine Residences: He came to Lawrence County, Illinois in 1815.
Service: Private; Virginia Continental troops. He enlisted in 1777and served for seven months in Capt. William Wallace's Company, Col. Isaac Shelby's Regiment; three months in Capt. William Welcher's Company, Col. Evan Shelby's Regiment. He was in the battle of Monks Corner. 
Pension: Pension Roll, Lawrence County, October 22, 1833, age 79. Catherine W8030 (Va) 
Marker: There are tombstones for Adam-and Catherine Lackey. His name is on a bronze tablet at the Lawrence County Court House, Lawrenceville, placed by Toussaint du Bois Chapter DAR in 1921. 
Sources: DAR, HR, PI, PENSION, W 
LAKE, REUBEN
Born: Derby, New Haven County, Connecticut
Died: August 20, 1842
Buried: Peoria County, Illinois
Service: Soldier; Connecticut. He enlisted in February, 1780 and served in Capt. Samuel Sanford's Company, Fifth Regiment of the Connecticut line, com­manded by Col. Isaac Sherman; transferred to the Second Company of Capt. Walker in the Second Regiment commanded by Col. Webb in the Connecticut line. 
Pension: S35515; BLWT 546-100 (Conn); Pensioned in 1818 while a resident of Geauga County, Ohio
Sources: DAR, PENSI ON 
LAMB, JOHN, SR.
Born: March 14, 1760, probably North Carolina
Died: November15, 1842
Buried: On a farm, near Ridgeway, Gallatin County, Illinois
Spouse: Comfort 
Service: Private; North Carolina Continental troops. He served from January, 1782toJauuary, 1783 in Capt. James Mills' Company, Tenth Regiment, North Carolina troops. 
Pension: Pension Roll, April 23, 1833, age 74; Pension Census, June 1, 1840, Gallatin County, Illinois, age 84; R6094 Comfort Widow, Comfort, was resident of Carmi, White County, Illinois, when pen­sion claim suspended (Act July 7, 1838) "not widow at date of act." 
Sources: DAR, PENSION, W 
LAND, MOSES 
Born: About 1764 in Halifax County, Virginia 
Died: October16, 1847 at home of Abel Fike, St. Clair County 
Buried: Reported both Clinton and St. Clair Counties, Illinois 
Spouse: Charity Beshears; she died May 1834 
Children: James, Henry, Joseph, Phillip, Lewis, Aaron, Moses, Nancy (1. Crownover, 2. Fike), Sally (Hendricks) Residences: He was listed in the 1800 Census of Pendelton County, South Carolina. He came to Illinois in 1817, settling in St. Clair County. 
Service: Private; Virginia. He enlisted June 30, 1780 for eighteen months from Orange County, Virginia. He was sixteen years of age. He served in the Company commanded by Captain Anderson in the Regiment of Colonel John Green, in the Virginia Continental line. He was in the battles of Guilford, Camden, Eutaw Springs and the siege of Ninety-six. He was dis charged December 31, 1781. 
Pension: S36037 (Va): Pension roll, May 15, 1820, removed from St. Clair County. Pension Census, June 1, 1840, resident of St. Clair County, Illinois, age 76.
Sources: DAR, PI, PENSION, W, Obituary 
LANGSTON, WILLIAM C. 
Born: April 15, 1762, Granville, North Carolina 
Died: December 7, 1853 
Buried: Langston Cemetery, near Manito, Mason County, Illinois; Private Head­stone
Spouse: Mary 
Residences: He removed from Wayne County, North Carolina to Coles County, Illinois, he also lived in Morgan County, Illinois. 
Service: Private; North Carolina. In April , 1777 he was in a Company commanded by Capt. John Farrer, rendezvoused at Patensburg, Halifax County, N.C. in Major Dixon's Regiment; three months service. 
Pension: R6149 (NC); Resident of Morgan County, Illinois when pension claim rejected (Act June 7, 1832) " only three months service."
Marker: The grave was marked by Peoria Chapter DAR in June, 1938. 
Sources: DAR, NSDAR , PI , PENSION, W 
LANMAN, JAMES
Born: 1752
Died: 1844
Buried: Schuyler County, Illinois 
Service: Orderly Sergeant; South Carolina Continentals: North Carolina. He enlisted in July, 1776 at Charlestown, South Carolina, and served as Orderly Sergeant in the First Regiment; enlisted in March, 1781 in North Carolina under Col. Nathaniel Greene. He was in the battles of Guilford Court House, and Eutaw Springs, where he was wounded. 
Pension: S31812 S.C.; Illinois Pension Roll, Schuyler County, April 22, 1834, age 82 
Sources: PI, PENSION, W 
LANTIS (Lantz), HENRY 
Born : About 1764 
Died: December 22, 1841 
Buried: Reed Cemetery, Logan County, Illinois 
Spouse: Susanna, born about 1759; died December 23, 1834 
Residences: He came from Ohio and settled in the Sugar Creek Settlement. There is a land entry in West Lincoln Township in 1824. 
Marker: His name is on a plaque on the Logan County Court House, Lincoln, placed by Abraham Lincoln Chapter DAR, June 27, 1975 . 
Sources: DAR 
LATHROP, ISAAC 
Born : In New York
Died: Clark County, Illinois
Service: Soldier; NY: Mass. He served from Massachusetts in Capt. Josiah Keith's Company, Col. John Daggett's Regiment for twenty-four days.
Pension: R6181 (NY); Residence Clark County, Illinois, when pension claim was suspended (Act June7,1832)"for proof from Albany records." Sources: PENSION, W 
LAVIGNE, ANTOINE
Buried: In or near Kaskaskia, Randolph County, Illinois
Service: Frenchman who served under Col. George Rogers Clark as a volunteer in Capt. Francis Charleville's Company.
Marker: His name is on a bronze marker on the grounds of Sparta High School, Sparta, Randolph  County, placed by Fort Chartres Chapter DAR in 1934.
Sources: DAR, W 
LAWSON, JOHN
Buried: Randolph County, Illinois
Spouse: Frances
Service: Private; Virginia. He served in the Virginia line of troops.
Pension: Lawson, John, Va.; Frances W376 (Va). He applied for a pension from Randolph County which was not granted as he had not served for six months.
Marker: His name is on a bronze marker on the grounds of Sparta High School, Sparta, Randolph County, placed by Fort Chartres Chapter DAR in 1934.
Sources: DAR, PENSION, W 
LAWSON, RANDOLPH
Born: 1752 in Cumberland County. North Carolin a
Buried: Johnson County or Lawrence County, Illinois Residences: He removed to Kentucky and from there to Johnson County, Illinois. 
Service: Private; North Carolina. He enlisted in 1780 from Cumberland County, North Carolina, in Capt. Cox or Gholston's Company, guarding baggage during the battle of Camden. He enlisted in 1781 during the battle of Guil­ford Court House and guarded baggage. 
Pension: R6205 (NC); Resident of Lawrenceville. Lawrence County when pension claim was rejected (Act June 7, 1832) " Not having six months service." Residence Vienna, Johnson County when pension claim rejected (Act June 7, 1832) " Not six months service." 
Sources: PENSION, W 
LAWSON, WILLIAM
Born: Scott County, Virginia
Buried: Probably Wabash County, Illinois
Service: Soldier; Virginia
Pension: S32374 (V a); Applied for pension in Wabash County, Illinois
Sources: PENSION, W 
LAYTON, THOMAS 
Die: After 1835
Buried: Clark County, Illinois
Service: Soldier; Northumberland County, Pennsylvania Militia
Pension: S32371 (Penn)
Sources: PENSION, W 
LAYTON, WILLIAM
Buried: Probably Clark County, Illinois
Service: Soldier; New York
Pension: R6218 (NY); Residence Clark County, Illinois, when pension claim suspended (Act June 7, 1832) "for proof of service from Albany records." Papers sent to J. M, Robinson December 18, 1835,
Sources: PENSION 
LEE, WILUAM
Born: January 24, 1763
Died: December 5, 1839
Buried: Downs, McLean County, Illinois 
Spouse: Abigail Bryam
Service: Private; New Jersey
Pension: Abigail W20465 (NJ)
Marker: His name is on a bronze plaque in the Soldiers Monument, Miller Park, Bloomington, McLean County, Illinois.
Sources: DAR, NSDAR, PI , PENSION 
LEMEN (Lemon), JAMES
Born: November 20, 1760 in Berkeley County, Virginia
Died: January 8, 1823
Buried: New Design, Monroe County, Illinois
Spouse: Catharine Ogle Residences: He founded the village of New Design, Monroe County, Illinois.
Service: Private; Virginia. He enlisted in 1777, serving two years. He was in the battle of White Plains.
Sources: PI, W 
LEVENS, HENRY, SR. 
Born: March 26, 1740 in Pennsylvania 
Died: February, 1835 
Buried: Monroe County, Illinois 
Residences: He first settled in Morgan County, Illinois, but removed to Monroe County.
Service: Private and Lieutenant: Pennsylvania: Virginia. Service in the Pennsylvania Continental troops and the Virginia line.
Pension: S32375(Va); Illinois Pension Roll, Monroe County, July18, 1833, age 70
Sources: PENSION, W 
LEWIS, JOHN 
Died: Clay County, Illinois 
Service: Soldier; Virginia. He continued in the service after the close of the war. 
Pension: Five men named John Lewis with Virginia service were pensioned. 
Sources: W 
LEWIS, TIMOTHY 
Born: May 24, 1764 in Ashfield , Massachusetts
Died: June 2, 1858
Buried: Belvidere Cemetery, Belvidere, Boone County, Illinois
Spouse: Thankful Bradley
Service: Private; Massachusetts. He enlisted April 10, 1779 in Capt. Eliphalet Densmore's Company as a substitute for his father, Timothy Lewis, and served for six months; in 1780 he served in Capt. Isaac Newton's Company, Col. Hugh Maxwell's Regiment. 
Pension: S13732:BLWT-54236-160-55 (Mass). Pensioner, New York Agency, September 4, 1857 last payment to William Tillinghast as sttorney for pensioner. Pensioner stated he had resided in Belvidere for two years; previously resided at Oconomawac, Wisconsin. 
Sources: l\'SDAR, PI, PENSION, W 
LIGET (Ligit, Liggett), JOHN 
Born: In Virginia 
Died: After 1833 
Buried: Irving area, Montgomery County, Illinois 
Residences: He settled in the Bostick settlement of Montgomery County. 
Service: Private; Virginia Continental troops: Pennsylvania. He enlisted in 1776 in Capt. John Reese's Company; transferred to Capt. David Plunkett's Company, Fourth Regiment, Light Dragoons of the Pennsylvania line of troops. He was taken prisoner in 1778 but escaped and rejoined the Army under Washington, serving until the close of the war. He was in the battles of White Plains, Trenton, Princeton, Brandywine, and Germantown. 
Pension: S31816 (Va); Pension Roll, Montgomery County, Illinois, February 28; 1833, age 71
Sources: DAR, PENSION, W 
LINTON, JOHN 
Born: 1750 Prince William County, Virginia 
Died: December 4, 1836 Springfield, Washington County, Kentucky 
Buried: Oak Grove Cemetery, Crawford County, Illinois 
Spouse: Ann Mason, died 1832, Springfield, Kentucky
Service: Captain: Virginia
Sources: DAR, PI 
LIPE, LEONARD
Born: 1763 in South Carolina
Died: After April 12, 1849
Buried: Tamaroa Township, Perry County, Illinois
Service: Private; North Carolina
Pension: S32381 Lipe, Leonard: (NC); Illinois Pension Roll, Perry County, Jan­uary 6, 1834, age 71; Pension Census, Perry County, June 1, 1840, age 77
Sources: PI, PENSION, W 
LIPSEY (Lipsie, Lipse), JOHN
Born: 1732
Died: June, 1835, age 103
Buried: Belknap Cemetery, Carthage, Hancock County, Illinois
Spouse: Catharine
Service: Private; Virginia: Pennsylvania
Pension: R6375 (Penn); John Lipse (deceased), Carthage, Hancock County (Act June 7, 1832); "He being deceased, claim to pension to be made by widow or children."
Marker: His name is on a tablet at the Carthage Court House placed by Shad­rach Bond Chapter DAR in 1910.
Sources: DAR, HR, NSDAR, PI, PENSION, W 
 UTTLE, JAMES
Buried: Bond County, Illinois 
Service: Private; New York. In his deposition before the Circuit Court in 1826 he stated that he enlisted in 1777 in Capt. Andrew Moodie's Company, Col. John Lamb's Regiment of Artillery and that he continued in service until 1781 when he was honorably discharged. 
Sources: HS 
LIVELY, JOHN
Born: Probably South Carolina
Died: 1826
Buried: Central, Randolph County, Illinois Residences: He came to Randolph County, Illinois in 1805.
Service: Soldier; South Carolina. He also served in the War of 1812.
Marker: His name is on a bronze tablet on the grounds of Sparta High School, Sparta, placed by Fort Chartres Chapter DAR in 1934.
Sources: DAR, NSDAR, W 
LOCK, JAMES
Born: August 24, 1761 in Berkeley County, Virginia
Died: Probably in Wayne County, Illinois
Service: Soldier; Virginia. He enlisted in 1779 for three months in Capt. Samuel McCutchen's Company, Col. William Bone's Regiment; in 1781 for three months in Capt. John McCormick's Company, Col. William Darke's Regiment.
Pension: S31220 (Va)
Sources: PENSION, W 
LOCKRIDGE, JOHN
Born: About 1761 in Augusta County, Virginia
Died: 1848, age 87
Buried: Chatham Cemetery, Chatham, Sangamon County, Illinois Residences: He came to Illinois in 1835 with four sons and four daughters, settling in Ball Town ship, Sangamon County.
Service: Soldier; Virginia. He enlisted early in the war and was in the battles of Guilford Court House and Cowpens.
Pension: S31218 (Va); Pension Roll, Sangamon County, June 1, 1840, age 79, residing with William A. Lockridge
Marker: His name is on a bronze plaque in the south mall, Old State Capitol, Springfield, Sangamon County, placed by Springfield Chapters DAR and SAR October 19, 1911.
Sources: DAR, HR, NSDAR, PENSION, W 
LOGUE, THOMAS
Buried: Probably Monroe County, Illinois
Service: Soldier; Pennsylvania
Pension: R6417 (Penn); Residence Waterloo, Monroe County, Illinois, when pen­sion claim suspended (Act June 7, 1832) "for further explanations."
Sources: PENSION, W 
LONG, JAMES 
Born: Virginia 
Died: 1828 
Buried: Smith Grove Cemetery, Greenville, Bond County, Illinois 
Service: Private; Virginia Militia 
Marker: His grave is marked. 
Sources: DAR, HR, NSDAR, W 
LONG, JOHN
Born: 1732 in Granville, North Carolina
Died: February 10, 1839
Buried: New Douglas Cemetery, New Douglas, Madison County, Illinois
Spouse: Frances Estes, married at Caswell County, North Carolina Residences: John Long owned large tracts of land in Madison County and oper­ated a hotel.
Service: Private; North Carolina Continental. He enlisted March 1, 1781, serving three months in Capt. James Pearce's Company; August 1, 1781 for three months in Capt. Hargron Searsays Company, Col. Thomas Taylor's Regi­ment. He was in the battle of Guilford Court House.
Pension: Long, John: NC: R6429; Madison County, Illinois Pension Roll, April 23, 1833, age 71; Frances R6426 (NC), Frances, widow of John, resident St. Clair County when pension claim rejected (Act July 7, 1838) "Not a widow at date of the Act."
Marker: His name is on a bronze marker in the Madison County Court House at Edwardsville, placed by Ninian Edwards Chapter DAR, September 16, 1912.
Sources: DAR, PENSION, W 
LONG, WILLIAM 
Born: August 9,1756 near Mt. Vernon, Virginia
Died: Probably Jefferson County, Illinois Residences: HelivedinMt.Vernon,Jefferson County,Illinois
Service: Soldier;Virginia. He was in the battles of Brandywine and Germantown. He was said to have served as one of Washington's body guards.
Sources: W 
LOOKER, OTHNIEL 
Born: October 6,1757 in Morris County, New Jersey 
Died: July 23,1845 
Buried: Kitchell Cemetery, Palestine, Crawford County, Illinois; Private Head­stone 
Spouse: Pamela Clark Residences: He removed to Hamilton County, Ohio, and from there to Crawford County, Illinois. He served as Governor of Ohio. 
Service: Private; New Jersey. He enlisted in May 1776 and served for seven months in the Company of Capt. Obadiah Kitchell and Capt. David Bates, in the Regiment of Col. Oliver Spencer and Col. Ephraim Martin; from 1777 to 1782, eighteen months in the Company of Capt. Kitchell and Capt. Jonas Ward, Col. Matthias Ogden's Regiment. He was in the battles of Long Island and White Plains. 
Pension: S32386 (NJ) 
Sources: PI, PENSION, W 
LORTON,ROBERT
Born: February 15, 1747 in Charlotte, Virginia
Died: May 16, 1833
Buried: White Hall Cemetery, White Hall, Greene County, Illinois; Government Headstone 
Spouse: Tabitha Ganaway Residences: After the war he removed to Kentucky, then to Illinois, locating in Bond County, and from there to Greene County. He was the founder of Lorton Prairie, near White Hall. 
Service: Sergeant; Virginia. He enlisted in 1776 from Charlotte, Virginia, serving two years in Capt. John Morton's Company, in the Fourth Virginia Regi­ment, commanded by Col. Adam Stephen. He was in the battles of Trenton, Germantown and Brandywine. He served six months in Capt. John Holcomb's Company. 
Pension: Pension Roll, October 12, 1819, Greene County. Tabitha, R6454 (Va) (deceased), widow of Robert, resident of Carrollton, Greene County, when pension claim suspended (Act July 4,1836) "For further proof and identity." 
Marker: His grave was marked by descendants and the Doctor Silas Hamilton Chapter DAR, Jerseyville. 
Sources: DAR, HR, NSDAR, PI, PENSION, W 
LUCAS, ABRAM (Abraham)
Born: About 1761 in Morris County, New Jersey
Died: July 2, 1841
Buried: Steenbergen Cemetery, southwest of Mt. Pulaski, Logan County, Illinois
Spouse: Marcy (Martha) Kelsey, died August 1835
Service: Indian Spy: Pennsylvania. He enlisted in Capt. Brinton's Company, Col. Lachlen Mclntoshs Regiment, serving four months on the frontier of Pennsylvania. He served as a spy in an expedition against the Indians. 
Pension: S6503 (Penn). He applied for a pension in 1836 from Springfield, Sanga­mon County which was not granted as he had served less than six months.
Marker: His name is on a plaque on the Logan County Court House, placed by Abraham Lincoln Chapter DAR June 27, 1975. 
Sources: DAR, PENSION, W, Sangamon County Genealogical Society 
LUKE, THOMAS
Buried: Probably Lawrence County, Illinois
Service: Soldier; Pennsylvania 
Pension: R6516 (Penn); Residence Lawrenceville, Lawrence County, Illinois, when pension claim was rejected (Act June 7,1832) "not having six months service."
Marker: Hi s name is on a bronze tablet in the Lawrenceville Court House, placed by Toussaint du Bois Chapter DAR , 1921.
Sources: DAR, PENSION, W 
LUMPKINS, JOHN G.
Born: About 1756 probably in Pittsylvania County, Virginia
Died: After 1840
Buried: Williamson County, Illinois
Service: Soldier; Virginia
Pension: Pension Census, Williamson County, June 1, 1840, age 84
Sources: PENSION, W 
LUNSFORD, GEORGE
Born: June 8, 1762 in Virginia
Died: December 8, 1808 
Buried: New Columbia Cemetery, Columbia, Monroe County, Illinois; Government Headstone
Spouse: Mary Ann Judy 
Service: Private; Virginia. Soldier with Col. George Rogers Clark who captured Kaskaskia and Prairie du Rocher in 1778; enlisted January 20, 1780 and was discharged February 18, 1783. 
Sources: DAR, HR, PI , W 
LUSK, JAMES 
Born: August 15, 1754 in South Carolina
Died: September 27, 1803
Buried: Golconda, Pope County, Illinois 
Spouses: (1) Letitia Thomas 
	 (2) Sarah McElwayne
Service: Major: South Carolina. He served under General Thomas Sumter.
Sources: NSDAR, PI 
LUTTRELL, MICHAEL 
Born: October, 1751 in Fauquier County, Virginia 
Died: December 19, 1844 
Buried: Haddon Cemetery, near Iuka, Marion County, Illinois Residences: He removed to Illinois, settling near Salem, Marion County, later in Iuka Township.
Service: Private; South Carolina; Virginia. He served in 1781 in Capt. George Shelton's Company, Col. William Churchill's Regiment; the same year in Capt. Thomas Helm's Company, with Major John Chun in the Virginia troops.
Pension: S32021SC; Illinois Pension Roll, Marion County, October 22, 1833, age 75; Pension Roll, April 12, 1834, age 82; Pension Census, June 1, 1840, Marion County, age 89, residing with Elizabeth Shelter, head of family.
Sources: PENSION, W 
LYERLE (Lierly), CHRISTOPHER
Born: About 1760 in North Carolina
Died: September 30, 1832
Buried: St. John's Cemetery, Jonesboro, Union County, Illinois
Spouse: (3) Barbara
Service: Private; 'North Carolina. He enlisted in 1780 at the age of 16 in Capt. Archibald Lytle's Company, Col. John G. McRaes Regiment, North Carolina troops, serving for eighteen months.
Pension: He was pensioned.
Sources: DAR, HR, PI, W, Roster of N.C. Soldiers in the American Revolution 
LYERLY (Lierly), ZACHARIAH 
Born: June 2, 1755 in Culpepper County, Virginia 
Died: March 15, 1847 
Buried: Lierly Cemetery, Kellerville, Adams County, Illinois 
Spouse: _ _ Harkey 
Residences: He first settled at Dutch Ridge, Jackson County, Illinois. 
Service: Private; North Carolina Continental troops. He enlisted for three months in 1777, in Rowan County, North Carolina, in Capt. George H. Berger's Company; in 1778 for five months in Capt. Thomas Cook's Company; for six months in Capt. Richard Grimes' Company, Col. Robert Rutherford's Regi­ment.  He was in the battles of Reedy Fork and Guilford Court House. 
Pension: S32389 (NC); Pension Roll, August 14, 1833, Jackson County, age 78; Pension Census, June 1,1840, Jackson County, Illinois, age 85 Sources : PI, PENSION, W 
LYNN, DAVID 
Born: 1764 in Connecticut
Died: After 1840
Buried: South Henderson Cemetery, near Gladstone, Henderson County, IIIinois 
Residences: In about 1832 he removed to Warren County, Illinois. Henderson County was formed from Warren County in 1841.
Service: Private; Connecticut. He enlisted in 1780 in the Company of Capt. Marvin Lord, and Lt. William Lynn (his brother), in Col. Herman Swift's Regiment .
Pension: S32388 (Conn); Pension Census, Warren County, Illinois, June 1, 1840, age 76, residing with Samuel Lynn.
Marker: A new marker was placed on the grave by the Daniel McMillan Chapter DAR, Stronghurst, Henderson County, in 1961.
Sources: DAR, PENSION, W 
MADDOX , BENJAMIN 
Died: September 15, 1846
Buried: Bethel Ridge Cemetery, near Carlinville, Macoupin County, Illinois: Private Headstone
Sources: HR 

MAHON, THOMAS 
Born: About 1750, near Richmond, Virginia
Died:1840
Buried: Mahon-Stephens Cemetery, near St. Peter, Fayette County, Illinois 
Married: February 18, 1795at Amherst County, Lexington Parish, Virginia
Spouse: Elizabeth (Susan) Johnson
Children: William Pope, Susan Goodman, Delaware, Pliant, John Johnson, Elizabeth Duncan
Service: Private; Virginia
Sources: PI, Family Information 
MAILLET, PAULETTE 
Born: 1753 at Mackinac, Michigan 
Died: 1805 
Buried: Peoria County, Illinois 
Residences: He was an Indian trader, and reported to have founded Peoria in 1778. He was killed in a quarrel with a Frenchman in 1805. 
Service: Hearing of the defeat of Thomas Brady at St. Joseph, Michigan, in 1777, to revenge the killing of men by the British and Indians, he headed a force which marched to St. Joseph and captured the Fort. 
Sources: W 
MAKEMSON, THOMAS 
Born: 1753in Pennsylvania
Died: After July 19, 1828
Buried: Near Oakwood, Vermilion County, Illinois
Spouse: Jane Lindsey
Service: Private; Pennsylvania. He enlisted with Capt. William Brown, commander of the Floating Battery, Putnam Station, twelve miles below Philadelphia. He served three years.
Pension: R20191 (Penn)
Marker: His name is on the fountain in front of the Post Office in Danville, Vermilion County, placed by Governor Bradford Chapter DAR.
Sources: DAR, PI, PENSION, W 
MALLORY, SAMUEL
Born: May 1, 1765 in New York
Died: After 1840
Buried: Fulton County, Illinois
Residences: Hourden, Connecticut; Winstead, Litchfield County, Connecticut; Bristol, Ontario County, New York; Fulton County, Illinois.
Service: Sergeant: Connecticut. He served from July to December, 1780 in Capt. Samuel Comstock's Company, Eighth Regiment, Connecticut; drafted Septem­ber 1781, and served one month in Capt. Mansfield's Company, of New Haven, Connecticut.
Pension: R6848 (Conn); Residence Lewistown, Fulton County, Illinois, when pension claim rejected (Act June7,1832)" as not having six months service." Residence Fulton County, Pension Census, June 1, 1840, age 70-80.
Sources: DAR, PENSION, W 
MANLEY, DAVID
Born: Easton, Massachusetts
Buried: Russell Cemetery, between Knoxville and Gilson, Knox County, Illinois
Children: George
Service: Private; Massachusetts. He enlisted from Easton, Massachusetts, in August 1778 in Capt. Samuel White's Company, Col. Thomas Carpenter's Regiment; discharged in September 1778. 
Marker: His grave was marked by Rebecca Parke Chapter DAR, Galesburg on October 21,1956.
Sources: DAR, HR, NSDAR, W 
MANN, ABEL 
Born: About 1759
Buried: Williamson County, Illinois
Residences: He settled in Franklin County, Illinois. Williamson County was formed from Franklin in 1839.
Service: Private; Virginia Continental troops 
Pension: S32396 (Va); Pension Roll, Franklin County, Illinois, August 22, 1833, age 74 
Sources: PENSION, W 
MARTIN, ISAAC
Born: About 1736, Middlesex County, New Jersey
Died: After 1790, Woodbridge, Middlesex County, New Jersey
Buried: Old Griffith Cemetery, Brownstone, Fayette County, Illinois
Spouse: (2) Phoebe (Webb) Harland
Service: Private; Middlesex County, New Jersey Militia
Sources: DAR, NSDAR, PI 
MARTIN, JOHN 
Buried: Sunset Cemetery, Adams County, Illinois 
Service: Soldier; Virginia 
Pension: R6954 (Va); Residence Quincy, Adams County, Illinois when his pension claim was rejected (Act June 7, 1832) "not having six months service."
Sources: PENSION, W 
MASSIE, THOMAS
Born: December26, 1759 in Albemarle County, Virginia Died: August 19, 1835
Buried: Old Salem Cemetery (or Morgan Cemetery), Curran, Sangamon County, Illinois
Spouses: (1) Fannie Hudson
	 (2) Rebecca Collyer 
Service: Private; Spy: Virginia Continental troops. He enlisted from Albemarle County, Virginia .
Pension: S31235 (Va): Illinois Pension List, Sangamon County, April 23, 1834, age 74 
Marker: His name is on a bronze plaque in the south mall, Old State Capitol, Springfield, placed by Springfield Chapters DAR and SAR, October 19, 1911. His grave was marked by Springfield Chapter DAR on January 12, 1973. 
Sources: DAR, NSDAR, PI, PENSION, W 
MATHER, ELIHU
Born: Windsor, Connecticut
Died: September, 1831
Buried: Near Edwardsville, Madison County, Illinois
Spouse: Polly
Service: Sergeant: Connecticut. He enlisted from Windsor, Connecticut, in Capt. 
Daniel Allin's Company, Col. Samuel Wyllys' Third Regiment; was made Sergeant January 1, 1781 in Col. Zebulon Butler's Fourth Regiment.
Pension: Madison County, Illinois, Pension Roll, July 23, 1821; R4360; former widow, Polly Crovenor or Grosvenor
Marker: His name is on a bronze tablet at the Madison County Court House, Edwardsville, placed by Ninian Edwards Chapter DAR, September16, 1912.
Sources: DAR, PENSION, W 
MATTESON, THOMAS 
Born: 1756 in West Greenwich, Newport County, Rhode Island
Died: After 1840
Buried: DuPage County, Illinois
Residences: After the war he removed to Ashtabula County, Ohio, and from there to DuPage County, Illinois.
Service: Soldier; Rhode Island. He enlisted June 8, 1776 with Lt. George Tennant and Col. Nathan Brown, Third Regiment, Rhode Island.
Pension: R7037 (RI); Residence, Naperville, DuPage County, Illinois when pen­sion claim rejected (Act June7, 1832)" no proof of service." Pension Census, DuPage County, June 1, 1840,  age 86, residing with Stephen Mattison, head of family.
Sources: PENSION, W 
MAULDING, AMBROSE 
Born: August 1, 1735 in Virginia
Died: August 25, 1833
Buried: Near the Ten-Mile Baptist Church, McLeansboro, Hamilton County, Illinois
Service: Soldier; Virginia
Marker: His monument bears the inscription: "Immortal may their memory be who fought and died for Liberty."
Sources: HR, W 
MAXEY, B. N. 
Buried: Pleasant Grove Cemetery. Mt.Vernon, Jefferson County, Illinois 
Service: Soldier; Tennessee. Capt. Scoovey 's Third Tennessee Infantry. He was also in the War of 1812. 
Sources: HR 
MAXEY (Maxcy), JOEL
Born: 1762 in Rockingham County, Virginia
Died: December 27, 1844
Buried: Old Salem Cemetery, north of Riddle Hill, Sangamon County, Illinois; Government Headstone
Spouses: (1) Susan Hill
	 (2) Betsey Ann Howard 
Residences: After the war he removed to Kentucky, and from there to Sangamon County, Illinois 
Service: Private; Virginia. He enlisted in the Virginia line and was in the battle of Guilford Court House. 
Pension: Illinois pension roll, Sangamon County, August 14, 1833, age 72; Pen­sion Roll, Sangamon County, June 1, 1840, age 78; W5331; Betsy Ann (Va) BLWT-26968-160-55 
Marker: His name is on a bronze plaque in the south mall, Old State Capitol, Springfield, placed by Springfield Chapters DAR and SAR, October 19, 1911. The grave was marked by Springfield Chapter on January 31, 1975. 
Sources: DAR, HR, PI, PENSION, W 
MAXWELL, JOHN A.
Born: April 25, 1765 in Kent County, Maryland
Died: August 19, 1857
Buried: Woodlawn Cemetery (or Rhoades), south of Bloomington, McLean County, Illinois
Spouse: Jane Brazelton 
Service: Soldier; North Carolina; 1775-1783
Marker: The grave was marked on May 4, 1974 by DeWitt Clinton Chapter DAR, Clinton, and Letitia Green Stevenson Chapter DAR, Bloomington.
Sources: DAB, PI 
MAYBERRY, FREDERICK 
Buried: Big Hill Cemetery, Hamilton County, Illinois
Service: Soldier; Virginia
Pension: R6557 (Va); Residence Hamilton County, Illinois, when pension claim rejected (Act June 7, 1832) "not six months service."
Sources: PENSION, W 
MAYFIELD, JOHN
Born: Probably Warren County, Halifax District, North Carolina
Buried: Probably Macoupin County, Illinois
Service: Soldier; North Carolina. He served from Warren County, Halifax Dis­trict, North Carolina . 
Pension: He applied for a pension in Macoupin County, Illinois.
Sources: W 
McADAMS, JOSEPH
Born: 1753-55 in York County, Pennsylvania
Died: September 22, 1840 near Dudleyville, Illinois 
Buried: Camp Ground Cemetery, six miles south of Creenville, Bond County, Illinois 
Spouses: (1) Jemima Justice, married December 14, 1781, Orange County, North Ca rolina
	 (2) Sarah Bradford, married November 17,1788, Orange County, N.C. 
Children: 	First marriage: Samuel
Second marriage: John, Hannah, Joseph, Jemima, Thomas Bradford, Sarah Bradford, Jesse, H. William,
	James, Robert, Sloss 
Residences: Joseph McAdams lived in York County, Pennsylvania; Orange County, North Carolina; Sumner County, Tennessee; Logan County, Kentucky; Mont­gomery and Bond Counties, Illinois 
Service: Private; North Carolina. He enlisted for three months from Hawfield, Orange County, North Carolina, in April or May 1779 in Capt. John Carring­ton's Company, Colonel Martin Armstrong's Regiment. He marched from Hillsborough to Georgia, joining General Benjamin Lincoln at Stono, where he was in battle in June. In 1780 he enlisted in the light cavalry commanded by Captains Nathaniel Christman, Hodge, and Quinn in the Regiment com­manded by Colonel William O'Neale and Colonel Robert Mabin. He was a pilot for Colonel Lee in the battle of Holt's Race Paths. He was in the battle of Hillsboro and was in command of the party which killed Major Moore, the Tory. He was discharged in 1782. 
Pension: S33081 (NC); Pension Census, June 1, 1840, Bond County, Illinois, age 87, residing with James McAdams, head of household. 
Sources: HR, NSDAR, PI, PENSION, W, and Descend ant 
McADAMS, WILLIAM
Born: 1760 in York County, Pennsylvania
Died: September 4, 1843
Buried: Probably buried at Jarvis, Madison County, Illinois
Spouse: Mary Hendricks
Service: Private; North Carolina Continental troops. He enlisted in the spring of 1779 at Hawsfield, Orange County, North Carolina, and served three months in Capt. John Carrington's Company, Col. Martin Armstrong's Regiment; for two years from 1780 to1782 in Capt. William Douglass' and Capt. Nathaniel Christman's Company, Col. William O'Neale's Regiment. 
Pension: S33083 (NC); Madison County, Illinois pension roll, August 22, 1833, age 74; Madison County Pension Census, June 1, 1840, age 83 residing with Aaron Rule , head of family. 
Marker: His name is on a bronze tablet in the Madison County Court House at Edwardsville, placed by Ninian Edwards Chapter DAR, September 16, 1912. 
Sources: DAR, PI, PENSION, W 
McALLISTER (McCallister), EDWARD
Born: March 4, 1758
Died: May 30, 1833 
Buried: On a farm in Emma Township, White County, Illinois 
Spouse: __ DeHart 
Service: Private; Virginia 
Marker: His name is on a flat government marker in the Old Cemetery, Carmi, dedicated by Wabash Chapter DAR on September 21, 1964. His name is also on a monument in the city park placed by the Chapter in 1936, honoring soldiers buried in White County. 
Sources: DAR, NSDAR, PI 
McCLINTOCK, SAMUEL
Born: 1763in Augusta County, Virginia 
Died: After 1840 
Buried: Tazewell County, Illinois 
Service: Soldier; Virginia. He served three times in 1781 in Companies com­manded by Capt. James Trimble, Capt. William Smith and Capt. William Kincaid, in Regiments of Col. Sampson Matthews, Col. William Boyer, and Col. Samuel Vance. He was at the siege of Yorktown. 
Pension: R6624 (Va); He was a resident of Tazewell County in 1840; Residence Tremont, 'Tazewell County when pension claim rejected (Act June 7, 1832) "For further proof. " 
Sources: PENSION\V 
McCLURE, SAMUEL
Born: May 16, 1748 in Augusta County, Virginia
Died: December 18, 1845
Buried: Forsyth Cemetery, Marshall, Clark County, Illinois
Spouse: Jane Hamilton
Service: Private; Virginia Continental. He was a soldier before the Revolutionary War, serving in 1774, 1775 and 1781. He was in the Companies of Capt. George Matthews, Capt. William Anderson, Capt. Thomas Smith, and Capt. Zaccheus Johnson, in Regiments commanded by Col. William Boyer and Col. Abraham Smith. 
Pension: S33079 (Va); Pension roll, August 14, 1833, age 86; Pension Census, Clark County, Illinois, June 1, 1840, age 90
Marker: His grave has been marked.
Sources: DAR, HR, PI, PENSION, W 
McCLURE, THOMAS
Born: July 15, 1765 in Rockingham County, Virginia
Died: January 3, 1847
Buried: Stout's Grove Cemetery, near Danvers, McLean County, Illinois; Private 
Headstone
Residences: In 1781 he removed to Kentucky and in 1827 to Stout's Grove, McLean County, Illinois.
Service: Soldier; Kentucky. He aided in fighting the Indians. His brother Robert was killed by Indians. On one Indian foray, McClure spotted a raccoon in a tree. He shot it, but the falling animal lodged in the tree. McClure removed his ammunition, climbed the tree and brought down the raccoon. He caught up with his companions who had covered three miles, missed his bullet pouch and returned for it. It was late at night when he rejoined the march, having walked fifty miles that day. The long march caused a constriction of the tendons of one leg which lamed him for life. 
Sources: DAR, W 
McCLINTOCK, SAMUEL
Born: 1763 in Augusta County, Virginia 
Died: After 1840 
Buried: Tazewell County, Illinois 
Service: Soldier; Virginia. He served three times in 1781 in Companies com­manded by Capt. James Trimble, Capt. William Smith and Capt. William Kincaid, in Regiments of Col. Sampson Matthews, Col. Willi am Boyer, and Col. Samuel Vance. He was at the siege of Yorktown. 
Pension: R6624 (Va); He was a resident of Tazewell County in 1840; Residence Tremont, Tazewell County when pension claim rejected (Act June 7, 1832) “For further proof." 
Sources: PENSION, V 
McCLURE, SAMUEL
Born: May 16, 1748 in Augusta County, Virginia
Died: December 18, 1845
Buried: Forsyth Cemetery, Marshall, Clark County, Illinois
Spouse: Jane Hamilton
Service: Private; Virginia Continental. He was a soldier before the Revolutionary ]]
War, serving in 1774, 1775 and 1781. He was in the Companies of Capt. George Matthews, Capt. William Anderson, Capt. Thomas Smith, and Capt. Zaccheus Johnson, in Regiments commanded by Col. William Boyer and Col. Abraham Smith . 
Pension: S33079 (Va); Pension roll, August 14, 1833, age 86; Pension Census, Clark County, Illinois, June 1, 1840, age 90
Marker: His grave has been marked.
Sources: DAB, HR, PI, PENSION, W 
McCLURE, THOMAS
Born: July15,1765 in Rockingham County, Virginia
Died: January 3, 1847
Buried: Stout's Grove Cemetery, near Danvers, McLean County, Illinois; Private Headstone
Residences: In 1781 he removed to Kentucky and in 1827 to Stout's Grove, McLean County, Illinois. Service. Soldier; Kentucky. He aided in fighting the Indians. His brother Robert was killed by Indians. On one Indian foray, McClure spotted a raccoon in a tree. He shot it, but the falling animal lodged in the tree. McClure removed his ammunition, climbed the tree and brought down the raccoon. He caught up with his companions who had covered three miles, missed his bullet pouch and returned for it. It was late at night when he rejoined the march, having walked fifty miles that day. The long march caused a constriction of the tendons of one leg which lamed him for life. 
Sources: DAR, W 
McCLURKIN (McClerken), THOMAS, SR.
Born: 1756 in Chester County, Camden District, South Carolina 
Died: March 22, 1845 
Buried: Oakdale Cemetery, Oakdale, Washington County, Illinois 
Spouse: Elizabeth Smith 
Residences: He removed to Kentucky, and from there to Indiana. He then moved to Washington County, Illinois. 
Service: Private; South Carolina. He served in Terners Winn s South Carolina Regiment. 
Pension: Elizabeth W21792 (SC); Washington County Pension Census, June 1, 1840, age 95.
Marker: A bronze marker was placed on his grave by Fort Chartres Chapter, DAR, Sparta, October 13, 1935.
Sources: DAR, HR, PI, PENSION, W 
McCOOL, JOSEPH B.
Born: 1750 in Greenville County, South Carolina
Died: 1825 
Buried: Shawneetown, Gallatin County, Illinois 
Spouse: Mary Thomas, born 1755; died after 1830 at Shawneetown 
Children: William, Thomas, PollyAnn, Abraham 
Service: Soldier; Patriotic
Service: South Carolina. He served from April 10 to May 11, 1779 under the command of Brigadier General Williamson, Major Brandon's detachment; in Capt. James Reed's Company, Col. Wyman's Regiment; in General Thomas Sumter's brigade from July 1, 1780 to August 6, 1781 under different officers, a total of 335 days. 
Sources: DAR, PI 
McCOY, DANIEL
Born: October 15, 1761
Died: February 23, 1836
Buried: West Side Cemetery, Clayton, Adams County, Illinois
Spouse: Agnes Kemper 
Service: Private; Virginia. He served in the Companies of Capt. Lawson and Capt. Francis Cowherd. 
Pension: He was pensioned.
Sources: NSDAR, PI 
McCULLOUGH, WILLIAM R. 
Born: 1756 in Baltimore, Maryland
Died: November 23, 1832 
Buried: Family cemetery (abandoned), Dry Grove Township, McLean County, Illinois
Spouse: Mary 
Children: Peter, Margaret Wyatt, Alexander 
Residences: He came to McLean County, Illinois, about 1830 from Fleming County, Kentucky. 
Service: Corporal: Maryland. He served for two years, 1776-1778, in Capt. Alex­ander Lawson Smith's Company, Col. Moses Rawling's Regiment, Maryland troops. 
Pension: S36085 (Md)
Marker: His name is on a bronze tablet in the Soldiers Monument, Miller Park, Bloomington, placed by Letitia Green Stevenson Chapter DAR.
Sources: DAR, HR, PI, PENSION, W 
McCUMBER, JOHN
Buried: Probably Cass County, Illinois
Spouse: Philadelphia
Residences: After the war he came to Illinois and resided in Cass County.
Service: Soldier; Virginia
Pension: Philadelphia, widow of John, resident Bardstown, Cass County, when pension claim rejected (Act July 7, 1838) "For a new declaration and proof of service."
Sources: HS, PENSION 
McDANIEL, LUANN (McDonnell)
Died: January 11, 1850 
Buried: Mechanicsburg Cemetery, near Dawson, Sangamon County, Illinois; 
Private Headstone
Spouse: Robert McDaniel, born 1757; died July 4, 1826; buried Bracken County, Kentucky. Revolutionary Soldier; Private; Pennsylvania: North Carolina
Children: George, Henry, James, William, Joseph, Jonathan, Robert, Betsey, Sally, Polly.
Service: Nurse: Pennsylvania. She was a nurse at Valley Forge.
Marker: The grave was marked by Sergeant Caleb Hopkins Chapter DAR, Spring­field, June, 1975.
Sources: DAR, HR, Family Information 
McDANIEL, RANDLE
Born: 1755 in Frederick County, Maryland
Died: Hamilton County, Illinois
Residences: From South Carolina he moved to White County, Illinois. Hamilton County was formed from White County in 1821. 
Service: Soldier; South Carolina. He served three months in 1775 in Capt. John Patton's Company, Col. Holt Richardson's Regiment. 
Pension: 532402(SC)
Sources: PENSION, W 
McDILL, JOHN 
Born : 1748/9 
Died: March 15, 1824 
Buried: Old Bethel Cemetery, Sparta, Randolph County, Illinois; Private Headstone
Spouse: Jane Bell Service: Private; South Carolina
Marker: His name is on a bronze marker on the grounds of Sparta High School, Sparta, Randolph County, placed by Fort Chartres Chapter DAR in 1934. The Chapter marked his grave on June 15,1930.
Sources: DAR, HR, NSDAR, PI 
McELYEA, WILUAM 
Born: About 1758
Died: After 1840
Buried: Denning Cemetery, Orient, Franklin County, Illinois Spouse: Frances Service: Private; North Carolina Continental troops. He served in Capt. Alexander Brevard's Tenth Regiment, North Carolina troops until March, 1783.
Pension: Pension Roll, December 12, 1833, age 75; Pension Census, Franklin County, June 1, 1840, age 82; Frances 533084: BLW 47903-160-55
Sources: DAR, HR, PENSION, W 
McEVERS, JAMES 
Born: September 1, 1755 in Scotland 
Died: January 24, 1829 at his residence in Morgan County, Illinois 
Buried: Glasgow Cemetery, Glasgow, Scott County, Illinois 
Spouse: Louisa Howard married in 1778/9 at Claverack, New York; born March 20, 1764; died November 29, 1847 
Children: James, Charles, Elizabeth, Sarah, Nancy, William, John, Elisha, Arch­ibald, Walter, Seneca, Robert 
Residences: His residence in 1790 was Hillsdale, Columbia County, New York; in 1800Clinton County, New York; 1810, Vineyard, Isle of Motte, Vermont; in 1816 he was in Athens County, Ohio. Upon coming to Illinois, he probably first lived in Creene County. 
Service: Private; Massachusetts. He served in Capt. Thomas Prichard's Company of the Third Massachusetts Regiment, commanded by Col. John Greaton. His name was on muster roles dated April, 1782 and May 20, 1782. He was honorably discharged at West Point in August, 1783. 
Pension: Illinois pension roll, June 7, 1819, transfer from Ohio. Louisa W23936. Louisa was granted a pension on July 20, 1840, entitling her to receive $80 per year, commencing March 4, 1831. The amount in arrears $720, plus the first semi-annual allowance of $40 was paid to her son Seneca, for her, by the pension agent at Vandalia in silver half dollars. He carried it home in his saddle bags. Louisa McEvers, Pension Census, June 1, 1840, Scott County, age 78, residing with Seneca McEvers, head of household. 
Marker: Although the exact grave was not located, the family, the American Legion Auxiliary and a DAR chapter placed a memorial marker in the Glas­gow Cemetery in 1935. 
Sources: DAR, HR, PENSION, W, Family History 
McGAHY, DAVID
Died: September, 1851 
Buried: On Netherly farm, Palestine, Crawford County, Illinois
Residences: He was a prominent citizen of Crawford County and a member of the Illinois State Legislature. 
Service: Private; Virginia
Sources: W 
McGHEE, WILLIAM
Born: 1761inLouisa County,Virginia
Died: October 6, 1843
Buried: Diamond Grove Cemetery, Downs, McLean County, Illinois
Spouse: Rebecca (Satterfield) Downs
Children: Mary Tovrea
Residences: He came to McLean County, Illinois in 1828.
Service: Private; NC:Pa, He enlisted from Mecklenburg County, North Carolina and served in Ca pt. Pond's Company, Col. Wade's Regiment; in Capt. Bracken's Company, Col. Lofton's Regiment; in Capt. Smith's Company, Col. William Shepard' s Regiment; in Capt. John Armstrong's Company, Col. Nicholas Lewis' Regiment, North Carolina troops. He was in the battle of Wilmington. 
Pension: S36696: BLWT 1004-100 issued December 6, 1821, residence Ohio; Illinois pension roll, McLean County, Illinois, March 15, 1834, age 72; Pension Census, McLean County, June 1, 1840, age 84 
Marker: His grave was marked by Letitia Green Stevenson Chapter DAR, Bloom­ington in 1928. His name is on the bronze tablet in the Soldiers Monument, Miller Park, Bloomington. 
Sources: DAR, PENSION, W 
McGILL (Magill), ANDREW
Born: June 22, 1758 in Maryland
Died: About 1842
Buried: Village Cemetery, Indian Creek Township, White County, Illinois
Service: Soldier; North Carolina. He was a wagoneer and was at the battle of Cowpens driving one of the baggage wagons.
Pension: R6826 (NC). Residence Shawneetown, Gallatin County, Illinois, when pension claim was rejected (Act June 7,1832) " not military service."
Sources: DAR, PENSION 
McKELVEY, HUGH
Born: 1762 in Ireland
Died: March 13, 1835
Buried: Old Bethel Cemetery, Sparta, Randolph County, Illinois; Private Headstone
Spouse: Sarah Aiken
Service: Private; South Carolina 
Marker: His name is on a bronze marker on the grounds of Sparta High School, Sparta, Randolph County, placed by Fort Chartres Chapter DAR, 1934. His grave was marked by the Chapter June 25, 1931. 
Sources: DAR, HR, PI 
McKINNEY, JOHN 
Buried: Probably Madison County., Illinois 
Service: Soldier; North Carolina 
Pension: R6764 (NC); Residence Edwardsville, Madison County, Illinois, when pension claimsuspended(Act June7,1832)"not under military organization.”
Sources: PENSION 
McMAHON, CON STANTINE 
Buried: Klepinger Cemetery, Landes Township, Crawford County, Illinois 
Service: Soldier; Pennsylvania. He served in Capt. John Brisben’s Company, Third Regiment. He was discharged April 1, 1777. 
Pension: R6784 (Penn); Residence Lawrenceville, Lawrence County, Illinois, when pension claim was suspended (Act June7,1832) "for proof and specifications."
Sources: HR, PENSION, W 
McMILLAN, DANIEL
Born: 1752
Died: August 14, 1838
Buried: South Henderson Cemetery, near Gladstone, Henderson County, Illinois; 
Private Headstone
Spouse: Jane Thompson
Service: Private; South Carolina
Pension: S32397 (SC)
Marker: The grave has been marked by Mildred Warner Washington Chapter DAR, Monmouth and by Daniel McMillan Chapter DAR, Stronghurst.
Sources: DAR, HR, PI, PENSION, W 
McNABB, CHARLES 
Born: In Mary land
Died: February 1, 1822
Buried: Randolph County, Illinois 
Service: Sergeant: Maryland Continental troops. He enlisted January 7, 1778 in Capt. William Beattys Company, Sixth Company, First Maryland Regiment; he also served in the Seventh Company, Third Regiment. 
Pension: S36082 (Md). He was in the Randolph County, Illinois, pension roll for June 24, 1819.
Marker: His name is on a bronze marker on the grounds of Sparta High School, Sparta, Randolph  County, placed by Fort Chartres Chapter DAR in 1934.
Sources: DAR, NSDAR, PENSION, W 
McNABB, WILLIAM 
Born: 1760
Died: August 12, 1835
Buried: Lebanon Cemetery, Petersburg, Menard County, Illinois
Spouse: Mary Crawford
Service: Private; Pennsylvania. He served in the Seventh Battalion, Pennsylvania Militia .
Marker: The grave has been marked by Pierre Menard Chapter DAR, Petersburg.
Sources: DAR, HR, PI 
McNARY, HUGH 
Born: August 21, 1762 probably North Carolina
Died: October 2, 1841 
Buried: Petty Cemetery, Pittsfield, Pike County, Illinois; Government Headstone
Spouse: Elisabeth
Service: Pri vat e: North Cnroliua
Pension: S33067 (NC). His pension application was filed in Morgan County, Illinois. Residence Jacksonville, Morgan County when pension claim rejected (Act June 7, 1832) "For further proof. Papers returned September 19, 1833 to William Thomas." Pension Census, Pike County, June 1, 1840, age 79, with James McNary, head of household. 
Sources: DAR, HR, PI, PENSION, W 
McNEELY, DAVID
Died: In Pope County, Illinois
Spouse: Rebecca 
Service: Private; Virginia Continental troops 
Pension: S35524 (Va); Illinois pension roll, Pope County, April 15, 1827. Rebecca: WI051
Sources: PENSION, W 
McPEETERS (McPeters), DAVID 
Born: January 14, 1756 
Died: March 27, 1846
Buried: Morgan County, Illinois 
Spouse: Susannah Loyd 
Service: Private; North Carolina 
Pension: Susannah W2145(NC) 
Marker: His name is on a plaque in front of the Morgan County Court House, Jacksonville, placed by Reverend James Caldwell Chapter DAR ill 1914.
Sources: DAR, NSDAR, PI, PENSION, W 
McROBERTS, JAMES
Born: 1760 in Glasgow, Scotland
Died: After 1840 
Buried: Monroe County, Illinois
Spouse: Mary
Children: Son Samuel McRoberts, was elected United States Senator from Illinois in 1841.
Residence: He came to Kaskaskia in 1786; in 1797 he was a resident of Maeys­town, Monroe County.
Service: Private; Pennsylvania Continental troops: Virginia. He enlisted when eighteen years of age, serving to the close of the war.
Pension: Illinois pension roll, Monroe County, April 28, 1834, age 71; Pension Census, Monroe County, June 1, 1840, age 78, residing with James McRoberts, head of family. Mary W2225: BLWT-26888-160-55 (Penn. Va)
Sources: PENSION, W 
McWITHEY (McWithy), JAMES
Born: About1760 in New York
Died: After 1830 in Pike County, Illinois 
Service: Private; New York Continentals. He served from Charlotte County, New York in the Seventeenth Regiment.
Pension: S32398 NY:VT; Pension roll, Pike County, Illinois, April 8, 1830, age 70
Sources: PENSION, W 
MEAD, WILLIAM
Died: 1850 
Buried: Old Riverside Cemetery, Moline Town ship, Rock Island County, Illinois 
Service: Soldier; Pennsylvania
Sources: HR 
MEADOWS, JAMES 
Born: About 1755 in Virginia 
Died: May 9, 1838
Buried: Meredian Cemetery, Berwick, Warren County, Illinois 
Spouse: Jane Heryford
Children: While a young man he moved from Virginia to North Carolina. In 1794 he moved to Kentucky, and in 1832 to Illinois, settling in Warren County. 
Service: Private; North Carolina. He enlisted in North Carolina and served until the close of the war. 
Pension: R7082 Jane (NC) widow of James, resident of Monmouth, Warren County, when pension claim rejected (Act July 7, 1838) "For further proof of service and marriage." Jane Meadows pensioned in Mercer County list of June 1, 1840, age 71.
Marker: His grave has been marked by Puritan and Cavalier Chapter DAR, Monmouth .
Sources: DAR, HR, NSDAR, PENSION, W 
MEADOWS, WILLIAM
Born: 1756 in Maryland
Died: October 18, 1831 
Buried: Prairie Township Chapel, Scotland, Edgar County, Illinois; Government Headstone and Private Headstone 
Service: Private; Virginia Militia. He enlisted in 1776, probably in Maryland.
Sources: HR, PI, W 
MEANS, WILLIAM
Born: May 3, 1763
Died: June 11,1848 
Buried: Edgar Cemetery, Paris, Edgar County, Illinois; Private Headstone 
Spouses: (1) Nancy McElroy 
	 (2) Mrs. Susan Seal Chenowith
Residences: After the war he lived in Ohio; moving to Illinois ill 1822, he located in Paris Township, Edgar County.
Service: Private; South Carolina. He enlisted in 1780 in Capt. Robert Ferris' Company and was engaged in Gen. John Green's campaign of the south. 
Pension: Pension roll, November 15, 1833, Edgar County, age 72. Susan W5368: BLWT-28635-160-55 (SC)
Sources: HR, PI, PENSION, W 
MEEK, BAZEL
Born: March7, 1763 in Virginia
Died: January 12, 1844 
Buried: Olio Cemetery, near Eureka, Woodford County, Illinois; Private Headstone
Spouse: Eleanor Roberts
Residences: He came to Illinois in 1832, settling in what is now Woodford County. 
Service: Private; Virginia. He served from August to October, 1777 in Capt. Hugh Stevenson's Company, 
Pension: S15521 (Va); Pension Census, Tazewell County, Illinois, June L 1840, age 79
Marker: His grave was marked by the Peoria Chapter DAR and the Historical Society of Woodford County.
Sources: DAR, HR, NSDAR, PI, PENSION, W 
MEISENHEIMER, PETER
Born: 1755probably in Pennsylvania
Died: After March 20, 1835
Buried: St. John's Cemetery, Jonesboro, Union County, Illinois
Spouse: Magdalena
Residences: He carne to Illinois in 1819, settling in what became known as Meisenheimer Precinct, Union County.
Service: Private; North Carolina Continentals. He enlisted from Cubarras County, North Carolina.
Pension: S32408 (NC); Illinois pension roll, Union County, July 18, 1833, age 79
Sources: HR, PI, PENSION, W 
MELTON, BENJAMIN
Born: 1756 in Albemarle County, Virginia
Died: September 15, 1845
Buried: Melton Cemetery, Bridgeport, Lawrence Co unty, Illinois 
Spouses: (1) Nancy King 
	 (2) Elizabeth Coy Fowler 
Residences: In 1790 he was living in Hillsboro District, North Carolina. He came to Illinois in 1820, settling in Lawrence County, 
Service: 	Private; North Carolina. He served from Caswell County, North Carolina, in 1776 for three months in Capt. Berry Turner's Company, Col. Henry Dixon's Regiment. He was in the battle of Stone's Ferry on the Ashley River. Dragged 1781. Served three months under Capt. William Hardin and Major Sharp; in the battle of Guilford Court House, III 1781 he was drafted and served one year with Capt. Tillman Dixon, Col. Henry Dixon, He was in the battle of Eutaw Springs, skirmished near St. Johns' Island, near Stone's Ferry. 
Pension: Pension Roll, Lawrence County, September 28, 1833, age 67; Pension Census, Lawrence County, June 1, 1840, age 76. Elizabeth W2227 (NC) BLWT 26773-160-55 
Marker: His name is on a bronze tablet at the Lawrence County Court House, Lawrenceville, placed by Tonssaint du Bois Chapter DAR in 1921.
Sources: DAR, HR, PI, PENSION, W 
MELTON, WILLIAM
Born: About 1758 Died : After 1840
Buried: Melton Cemetery, Bridgeport, Lawrence County, Illinois
Service: Private; North Carolina. He served in Capt. Henry Dixon's Company from December 13, 1776 to February 1, 1780 in the First Regiment, North Carolina troops. Pension. S36117 (NC); Pension Census, June 1, 1840, Lawrence County, age 82, residing with Benjamin Melton, head of family.
Marker: His name is on a bronze tablet at the Lawrence County Court House, Lawrenceville, placed by Toussaint du Bois Chapter DAR in 1921.
Sources: DAR, HR, PENSION, W 
MEREDITH, DAVI S
Born: 1750
Died: After May 24, 1825
Buried: Joe Bunk Cemetery, East Lake Drive, Springfield, Sangamon County, Illinois
Spouse: (3) Ann Pritchard
Service: Private; Infantry Militia, Virginia troops
Sources: NSDAR, PI 
MERIFIELD, JOHN 
Buried: Probably Macoupin County, Illinois 
Service: Soldier; Virginia
Pension: R7119 (Va): Resident of Carlinville, Macoupin County, Illinois, when pension claim suspe nded (Act June 7, 1832) "for further proof and specifi­cations."
Sources: PENSION 
MILLER, BENJAMIN
Born: March 18, 1712
Died: 1785
Buried: Brimfield Cemetery, Brimfield, Peoria County, Illinois
Spouse: Prudence Newman
Service: Doctor: New York
Marker: The grave has been marked by Peoria Chapter DAR.
Sources: DAR, HR 
MILLER, CHRISTIAN 
Died: September 14, 1842 
Buried: Oak Hill Cemetery, Buckhart, Sangamon County, Illinois 
Sources: HR 
MILLER, FRANCIS
Born: October 16, 1753 at sea
Died: February 19, 1843
Buried: Oakwood Cemetery, Greenfield, Greene County, Illinois; Government Headstone
Spouse: Jane
Service: Captain: North Carolina. He enlisted in Mecklenburg County, South Carolina in 1775. He was made Captain and served in the Riflemen Rangers with Col. Robert Irwin until 1781. He was in the battles of Hanging Rock and Guilford Court House. 
Pension: Pension roll, May 13, 1834, age 80; Pension Census, Greene County, Illinois, June 1, 1840, age 96, residing with David Miller, head of family. Jane W23984 (NC) 
Sources: DAR, HR, PI, PENSION, W 
MILLER, GORDON
Died: June 8, 1810
Buried: Evergreen Cemetery, Chester, Randolph County, Illinois; Government Headstone
Marker: His name is on a bronze marker on the grounds of Sparta High School, Sparta, Randolph County, placed by Fort Chartres Chapter DAR in 1934.
Sources: DAR, HR 
MILLER, JOHN A. 
Born: Probably in Orange County, Virginia 
Buried: Greene or Madison County, Illinois 
Service: Soldier; Virginia. He enlisted from Orange County, Virginia, serving from April to September, 1778 in Company No. Five, with Col. Thomas Marshall in the Third Regiment. 
Pension: (Va) R7211; Residence Edwardsville, Madison County, Illinois, when pension claim was rejected (Act June 7, 1832) "not under military organiza­tion." He applied for a pension from Greene County. Residence, Carrollton, Greene County when pension claim was rejected (Act June 7, 1832) "Not six months service." 
Sources: PENSION, W 
MILLER, MICHAEL
Died: After 1840
Buried: Monroe County, Illinois
Residences: He came to Illinois in 1800 from Pennsylvania, settling south of the Moore tract in Monroe County, Illinois. 
Service: Private; Virginia Continental troops 
Pension: S33103 (Va); Illinois pension roll, Monroe County, Illinois, July 18, 1833, age 73; Pension Census, Monroe County, June 1,1840, age 86, residing with Jesse Wiswell, head of family. 
Sources: DAR, PENSION, W 
MILLER, PETER 
Buried: Casper Cemetery, Anna, Union County, Illinois 
Service: Soldier; North Carolina and South Carolina. He probably enlisted from Rowan County, North Carolina, but was in several battles in South Carolina.
Sources: HR, W 
MILLINGTON, PETER
Born: In Vermont
Buried: Magippa or Zion Cemetery, Cotton Hill Township, Sangamon County, Illinois 
Residences: After the war he moved to Ohio, and from there to Illinois, settling in Cotton Hill Township, Sangamon County. 
Service: Sergeant and Lieutenant: Vermont. He enlisted in Vermont, accompany­ing Ethan Allen and Benedict Arnold on their expedition to Quebec. He was taken prisoner, and when released enlisted in Capt. William Hutchiri's Com­pany. 
Marker: His name is on a bronze plaque in the south mall, Old State Capitol, Springfield, placed by Springfield Chapters DAR and SAR, October 19, 1911.
Sources: DAR, NSDAR, W 
MILLS, HAYDON 
Buried: Probably Randolph County, Illinois
Residences: He settled east of Kaskaskia, Randolph County, Illinois, above the mouth of Nine Mile Creek.
Service: He was a soldier with Col. George Rogers Clark. 
Marker: His name is on a bronze marker on the grounds of Sparta High School, Sparta, Randolph County, placed by Fort Chartres Chapter DAR in 1934.
Sources: DAR, NSDAR, W 
MILLS, JOHN H. 
Died: 1840
Buried: Bovee Cemetery, Cisne, Wayne County, Illinois
Service: Private; South Carolina 
Pension: Pension Census. Wayne County, Illinois, June L 1840, age 79. 
Sources: HR, PENSION, W 
MINER, AARON 
Born: March 17, 1757
Died: March 29, 1849
Buried: Elk Grove Cemetery, Arlington Heights, Cook County, Illinois
Spouse: Hannah Baldwin
Service: Private; Connecticut Militia. He enlisted in 1775 in Capt. Phineas Porter's Company from Hinman, Connecticut. His service was around Lake George, Lake Champlain, St. John's and Montreal.
Pension: He was pensioned. Hannah R7256 (Conn)
Marker: His grave was marked by General Henry Dearborn Chapter DAR, Chi­cago, July 13, 1931.
Sources: DAR, HS, NSDAR, PI, PENSION 
MINZES (Menzies, Minns), JOSEPH
Born: April 9, 1755
Died: April 14, 1849
Buried: Franklin County, Illinois
Spouse: Margaret Karnatser
Service: Private; North Carolina. He enlisted at Salisbury, North Carolina, in 1781, serving eighteen months in Capt. Edward Yarbrough's Company in the Third Regiment, North Carolina troops. 
Pension: Pension roll, August 28, 1830, transferred from Kentucky; Pension Cen­sus, Gallatin County, Illinois, June 1, 1840, age 85, residing with Daniel Miner, head of family. W2148 (widow not named in pension file) (NC) 
Sources: PI, PENSION, W 
MISNER (Mizner), HENRY
Born: September 22, 1759 in Berks County, Pennsylvania
Died: September 15, 1848
Buried: Millington and Newark Cemetery, Newark, Kendall County, Illinois; Private Headstone
Spouse: Barbara Stacker
Residences: After the war he moved to Indiana, removing to Kendall County, Illinois. Kendall County was formed from LaSalle County in 1841. 
Service: Boatman; Spy; Pennsylvania. He enlisted in a Northumberland County, Pennsylvania, Regiment serving in Capt. Timothy Green's Company for fif­teen months.
Pension: S16482 (Penn); Illinois Pension Census, LaSalle County, June 1, 1840, age 84.
Marker: A monument was erected by Illini Chapter DAR, Ottawa on June 14, 1897.
Sources: DAR, PI, PENSION  
MITCHELL, REV. EDWARD 
Born: August 3, 1760 Cecil County, Maryland 
Died: December 3, 1837 
Buried: Family cemetery, Knobelock Farm, southeast of Belleville, St. Clair County, Illinois; Private Headstone 
Spouse: Nancy Haley 
Residences: He removed with his parents from Maryland to Virginia, settling in Fincastle, Botetourt County. He came to Illinois in 1818, settling at Turkey Hill, St. Clair County. 
Service: Corporal; Quartermaster; Virginia. He enlisted as a Private, was made Corporal, then Captain of the First Virginia Rifles. He was Quartermaster in Col. William Campbell's Regiment. He was in the battles of Guilford Court House and Haw River. 
Pension: W23991 Ann (or Nancy) (Va)
Marker: A bronze tablet was placed on the grave by Belleville Chapter DAR in 1920.
Sources: DAR,HR, PI, PENSION,W 
MITCHELL, JAMES
Born: March 27, 1727 in Cecil County, Maryland
Died: June 11, 1819
Buried: Family cemetery, Knobelock Farm, southeast of Belleville, St. Clair County, Illinois
Spouse: Mrs. Molly (Pryer) Berry
Children: Rev. Edward Mitchell
Residences: He came to St. Clair County, Illinois, in 1818, accompanied by his son Edward.
Service: Quartermaster: Virginia. He served in the Albemarle Barracks and was in the battles of Guilford Court House and Clover Lick, May I, 1780.
Marker: A bronze tablet was placed on the grave by Belleville Chapter DAR in 1920.
Sources: DAR, NSDAR, PI, W 
MITCHELL, REV. SAMUEL 
Born: March 15, 1759 Cecil County, Maryland 
Died: April 25, 1840 
Buried: Near Galena, Jo Daviess County, Illinois 
Spouse: Malinda Cecil 
Residences: He removed with his family from Maryland to Botetourt County, Virginia. He came to Illinois in 1817, settling near Belleville, St. Clair County. It is probable that he was the "Reverend Mitchell" who opened the Consti­tutional Convention in 1818 with prayer. He and his brother Edward built a church in Belleville. He removed to Galena and lived to a great age, preach­ing when he was 80 years of age. 
Service: Private; Virginia. He served from Botetourt County, Virginia.
Pension: Malinda W4030 (Va): Residence Belleville, St. Clair County when pen­sion claim rejected (Act June 7, 1832) "Not six months service."
Sources: PI, PENSION, W 
MOIERS, ELIAS 
Buried: Probably Union County, Illinois 
Residences: He came to Illinois in 1828, settling in Union County.
Service: Soldier; South Carolina. He served for ten months in Capt. William Williams' Company, Col. William Polk's Regiment. He was discharged on the "High Hills" of Santee, South Carolina. 
Pension: Mairs or Moirs, Elias (NC:SC) S2719. Pension claim rejected while a resident of Union County, Illinois, Act of Congress May 29, 1830 "not serving in Regiment on Continental establishment." In his pension application in 1828 he stated that he was wholly disabled and enumerated his possessions as 'one horse, one saddle, bridle and saddle bags. 
Sources: PENSION, W 
MONTGOMERY, JOHN
Born: August 5, 1764 in Virginia
Died: January 26, 1845
Buried: Princeville Cemetery, Princeville, Peoria County, Illinois; Private Head­stone
Spouses: (1) Susanna Porter
	            (2) Elizabeth Harris
Service: Private; Virginia. He was enlisted by his father when thirteen years of 
	                  age in the Virginia line.
Pension: Illinois Pension Census, Peoria County, Illinois, June I, 1840, 
                     age 75, residing with John Hines, head of family.
Marker: Marked by Peoria Chapter DAR November 20, 1973.
Sources: DAR, NSDAR, PI, PENSION, W 
MONTGOMERY, JOHN 
Buried: Probably near Kaskaskia, Randolph County, Illinois
Residences: He returned to Illinois after the war, locating four miles from Kas­kaskia, Randolph County, where he built a small water mill.
Service: Private, serving with Col. George Rogers Clark.
Pension: He was given a tract of land (Bounty Land).
Marker: His name is on a bronze marker on the grounds of Sparta High School, Sparta, Randolph County, placed by Fort Chartres Chapter DAR in 1934.
Sources: DAR, NSDAR, W
MOODY, EDMUND (Edward)
Born: Sept ember 18, 1755 in Albemarle County, Virginia
Died: Sept ember 10, 1839 
Buried: Morgan County, Illinois 
Spouse: (2) Sarah Hamilton 
Residences: After the war he lived in Kentucky, moving from there to Morgan County, Illinois. 
Service: Private; Virginia: Pennsylvania Continentals
Pension: Sarah W25726 (Vu) BLWT 85081-160-55. He was listed on the Illinois pension list for Jackson County, September 25, 1833, age 89.
Marker: His name is on a plaque in front of the Morgan County Court House at Jacksonville, placed by the Reverend James Caldwell Chapter DAR in 1914.
Sources: DAR, NSDAR, PI, PENSION, W 
MOONEY, BRIEN (Brian)
Died: Before 1836
Buried:  Adams County, Illinois
Spouse: Margaret 
Service: Soldier; Virginia: Georgia. He enlisted March 14, 1776 in the Virginia Line, Capt. Talbott's  Company, Col. William Irvine's Regiment. 
Pension: Margaret R7310 (Ca. or Va.). His widow, Margaret, applied for a pen­sion after 1836. Residence Adams County when pension claim suspended (Act July 7, 1838) "For further proof of service and marriage." 
Sources: HS , PENSION 
MOORE, ABEL (Absolom) 
Died: October 27, 1814 
Buried: Parkinson Cemetery (Paddock Woods), St. Jacob, Madison County, Illinois Government Headstone 
Marker: Tablet on gravestone in private cemetery near Alton, Madison County, Illinois, marking grave of Abel Moore and his wife who were massacred by Indians on October 27, 1814; placed by Ninian Edward Chapter DAR on October 27, 1928. 
Sources: DAR, HR 
MOORE, ANDREW
Born: June 9, 1758
Died: April 20, 1845 
Buried: Union Grove Cemetery, Granville, Putnam County, Illinois
Spouse: Elizabeth Shepherd
Service: Patriotic
Service: Pennsylvania. He served with the Pennsylvania Rangers 
Pension: S41902 (Pa); Residence Hennepin, Putnam County, Illinois, when pen­sion claim was suspended (Act June 7, 1832) "For further proof and specifi­cations." 
Sources: HR, PI, PENSION, W 
MOORE, ASA 
Born: About 1765,
Died: After 1834, probably Edgar County, Illinois
Spouse: Elizabeth 
Residences: After the war he moved from Maryland to Pennsylvania and then to Edgar County, Illinois, 
Service: Private; Maryland Militia. He enlisted in 1778 and was in the battle of Stony Point. 
Pension: Pension roll January 23, 1834, Edgar County, age 69. Elizabeth W3030 (Md) 
Sources: PENSION, W 
MOORE, CHARLES 
Born: January 11, 1763 in Hanover County, Virginia
Died: September 19, 1839
Buried: Probably buried in Woodford County, Illinois (separated from McLean 
County in 1841) 
Spouse: Martha Cunningham 
Residences: He came to Illinois in 1823, settling near Buffalo Hart Grove in Sangamon County, but moved to McLean County, now Woodford County. While going to draw his pension the stage upset and he died from the injuries. 
Service: Private; North Carolina. He enlisted from Salisbury District, Rowan County, North Carolina, serving three months in Capt. James Craig's Com­pany, Major Montflorance's Regiment; three months in Capt. Benjamin Smith's Company, Col. Matthew Brandon's Regiment; six months in Capt. Robert Cladsby's Company. He was in the battle of King's Mountain. 
Pension: Illinois pension roll, McLean County, March 15, 1834, age 70. Martha W24005 (NC) 
Sources: NSDAR, PI, PENSION, W 
MOORE, CHARLES
Buried:  Randolph County, Illinois
Marker: His name is on a bronze marker on the grounds of Sparta High School, Sparta, Randolph County, placed by Fort Chartres Chapter DAR in 1934.
Sources: DAR 
MOORE, JAMES 
Born: February 14, 1750 in Maryland
Died: About 1788
Buried: Bellefontaine Cemetery, south of Waterloo, Monroe County, Illinois
Spouse: Katherine Briggs
Residences: He was the leader of a colony which came to Illinois in 1781, settling at Bellefontaine, near Waterloo, Monroe County.
Service: Captain: Virginia. He served with Col. George Rogers Clark and re­ceived his commission as Captain from Governor Patrick Henry.
Sources: NSDA R. PI, W 
MOORE, J. MILTON 
Died: 1844
Buried: Bellefontaine Cemetery, near Waterloo, Monroe County. Illinois
Children: Youngest son of Capt. James Moore.
Service: Ranger: Virginia. He was also in the War of 1812.
Sources: NSDAR 
MOORE, JOHN
Buried: Big Spring Cemetery, Monroe County, Illinois
Service: Lieutenant: Virginia. He also served in the War of 1812.
Sources: NSDAR 
MOORE, JOSIAH 
Buried: Springdale Cemetery, Brimfield, Peoria County, Illinois
Service: Soldier; Massachusetts
Sources: HR 
MOORE, RISDON 
Born: November 20, 1760 in Delaware
Died: June 10, 1828
Buried: Shiloh Cemetery, Shiloh, St. Clair County, Illinois; Private Headstone
Spouses: (1) Scarberry Marshall
	 (2) Anna Dent 
Residences. The Moore family came to the United States from Wales in 1732, settling in Delaware. After the war, Risdon Moore moved to North Carolina, then to Georgia, and then to St. Clair County, Illinois. He was speaker of the Illinois House of Representatives in 1814, and was a member of the first, third and fourth legislatures. He was the great grandfather of Governor Charles S. Deneen. 
Service: Sailor: Delaware. He entered the United States Navy at the age of six­teen, serving from 1775 to 1783. 
Sources: HR, PI, W 
MOORE, THOMAS 
Born: January 24, 1760 in Rockingham County, Virginia 
Died: 1848 
Buried: Moore Cemetery, Carlinville, Macoupin County, Illinois; Private Head­stone 
Residences: He removed to Kentucky and in 1831 to Macoupin County, Illinois. 
Service: Sergeant: Virginia. He served in Capt. Peter May's Company, Col. John Glenn's Regiment, Virginia troops. 
Pension: Pension roll, Macoupin County, January 9, 1834, age 74. Three men named Thomas Moore were pensioned from Virginia. 
Sources: HR, PENSION, W 
MOORE, THOMAS L, 
Died: Probably Clinton County, Illinois 
Service: Sergeant: Virginia. He was a Sergeant in Capt. Uriah Springer's Com­pany with Col. George Rogers Clark. 
Pension: He applied for a pension in Clinton County, Illinois, which apparently was not granted but he did receive a land grant for Revolutionary War ser­vice. BLWT 12348 issued February 23, 1792. Residence Clinton County when application for pension suspended (Act June 7, 1832) "For return of original papers sent September 23, 1847 to Uriah Manly." 
Sources: PENSION, W 
MORGAN, HENRY 
Born: December 7, 1758 in North Carolina 
Died: February 22, 1849 
Buried: White County, Illinois 
Service: Soldier; North Carolina: Virginia. He enlisted March 24, 1779 for five months in Capt. Robert McLane's Company, Col. John Collier's Regiment; in August 1780 he served in Capt. Flower Swift's Company, Col. William Campbell's Regiment, Virginia troops. In 1781 he served eighteen months in Capt. Robert McLanes Company with Major Joel Paisley in the North Caro­lina troops. He was in the battles of Wetzell's Mill, Sandy Creek and Lind­ley's Mill, 
Pension: Susan W3709 (NC:Va). Illinois pension list, June 1, 1840, White County, Illinois, age 87 
Marker: A flat marble government marker was dedicated in the old cemetery, Carmi by Wabash Chapter DAR on September 21, 1964, His name is also on a monument in city park placed by the Chapter in 1936, honoring soldiers buried in White County. 
Sources: DAR, NSDAR, PENSION, W 
MORGAN, JAMES 
Born: April 5, 1748 in Frederick County, Virginia 
Died: March 1, 1840 
Buried: Vermilion County, Illinois 
Spouse: Margaret Joliff 
Residences: He came to Illinois in 1820. 
Service: Private; Virginia: Pennsylvania. Enlisted in August 1778 from Monongalia County, Virginia, for three months in Capt. James Brenton's Company, Col. John Evans' Virginia Regiment. In General Mclntoshs campaign to Ohio County and assisted in erecting Ft. McIntosh and Laurens; in 1779 enlisted from Morgantown, Virginia, for three months in Capt. Samuel Mason's Company, Col. Broadhead's Pennsylvania Regiment. Marched up the Alle­gheny River and destroyed several towns, one of which was Corn Planter Town. In 1780 enlisted as Indian spy under Col. John Evans, Virginia troops. Discharged August 15, 1780. 
Pension: S33138 (Penn:Va). He was allowed pension on March 11, 1835, resident of Vermilion County, Illinois. Certificate 30189 issued September 26, 1835. 
Sources: DAR, PI, PENSION 
MORRELL, JOHN 
Buried: Fayette County, Illinois 
Service: Soldier; Virginia: Pennsylvania 
Pension: H7396 (Va). He applied for a pension from Fayette County, Illinois. Residence Vandalia, Fayette County when claim for pension suspended (Act June7, 1832) "For further proof and more consistent statement." 
Sources: PENSION, W 
MORRIS, TRAVIS 
Born: June 12, 1758 in Richmond County, Virginia 
Died: After 1833 
Buried: Probably Union County, Illinois 
Residence: After coming to Illinois he lived in both Alexander and Union Coun­ties. 
Service: Private; North Carolina: Virginia Continentals. He enlisted in 1777 for three months in Capt. John Hedges' Company, Col. Jesse Eural's Regiment, Virginia troops; six months in North Carolina troops in Companies of Capt. Charles Madden, and Capt. Samuel Hampton, with Major Joseph Winston. 
Pension: S33123 (NC:Va); Illinois pension roll, Union County, October 22, 1833, age 74 
Sources: PENSION, W 
MORRISON, JOSEPH 
Born: November 30, 1759 in Martinsburg, Berkeley County, Virginia
Died: August 25, 1835
Buried: Mt. Morriah Cemetery, Kell, Marion County, Illinois
Spouse: Elizabeth
Service: Private; Virginia. He served nine months in 1776 in the Companies of 
Capt. John Lyle, Capt. Anthony Odel, and Capt. Jacob Linder, in Regiments commanded by Col. John Morrow and Col. William Morgan. He aided in erecting Fort McIntosh, and was at the surrender of Yorktown. 
Pension: S31268 (Va), Illinois pension roll, Marion County, April 9, 1833, age 73 
Sources: DAR, HR, PI, PENSION, W 
MORTON, CHARLES 
Buried: Old City Cemetery, near Charleston, Coles County, Illinois
Marker: Grave marked by Sally Lincoln Chapter DAR, Charleston, Coles County, Illinois, 1932-33.
Sources: DAR 
MORTON, TOMAS 
Born: August 29, 1752in Chester County, Pennsylvania
Died: After March 28, 1835
Buried: Near Oakwood, Vermilion County, Illinois
Spouse: Elizabeth Paul?
Residences: In 1780 he moved to Jefferson County, Kentucky; was in Nelson County, Kentucky in 1786 and Logan County in 1810. On April 16, 1814, he was appointed Associate Judge of Perry County, Indiana. He removed to Vermilion County, Illinois, where he applied for pension November 5, 1832. 
Service: Ensign; Private; Captain: Pennsylvania: Virginia. He enlisted from Pennsylvania in 1775 as Ensign in the Company of Capt. James Elliott; in 1776 in Capt. James Lee's Company, Col. Robert Culbertson's Regiment; in June 1777 as Pvt. under Lt. Anderson. In 1780 and 1782 he served as Capt. in Col. Cox' s Regiment, under General George Rogers Clark with Virginia troops in Kentucky. He was in skirmishes at Stratton Island and with Indians at Chillicothe. He was afterward commissioned in the Nelson County Militia; Lt . Col. July 13, 1790. 
Pension: S32411 (Pa:Va); Illinois pension roll, Vermilion County, April 17, 1833, age 81. 
Sources: PENSION, W 
MOSS, ZEALLY 
Born: March 6, 1775, Loudoun County, Virginia
Died: October 31, 1839
Buried: Springdale Cemetery, Peoria, Peoria County, Illinois; Private Headstone
Spouses: (1) Elizabeth Berry
	 (2) Jennette Glascock 
Service: Captain: Wagoneer: Virginia Continental line. He enlisted in 1777 from Loudoun County, Virginia, and served two years. He re-enlisted in 1780 and served to the close of the war. 
Pension: Jenny W24164 (Cont.Va.), Moss, Jenny, widow of Zeally (Wagonmaster, Va.) on Illinois pension roll October 1,1839.
Marker: The grave has been marked by the Zeally Moss Society, Children of the American Revolution, Peoria.
Sources: DAR, HR, NSDAR, PI, PENSION, W 
MULLINS, JAMES
Born: Before 1744 Died : 1827 
Buried: Mt. Pleasant Cemetery, Prairie Town ship, Crawford County, Illinois
Spouse: Mary Tombs, married May 26, 1764; buried Cassady Cemetery, south of Paris
Service: Private; Virginia 
Pension: Mary, widow of James, resident of Paris, Edgar County when pension claim rejected (Ac tJuly 4, 1836) "For proof and specifications."
Marker: The grave is marked.
Sources: DAR, HR, PENSION 
MULLINS, JAMES
Died: Before 1836
Buried: Cassady Cemetery, Symmes Township, Edgar County, Illinois
Spouse: Mary 
Service: Soldier; Pennsylvania. He served in Capt. John Craig's Company, Col. John Van Etten's Regiment, Northumberland County, Pennsylvania.
Pension: After 1836 Mary, widow, applied for a pension.
Sources: HS, P ENSION 
MURPHY, JOHN
Died: Probably Jefferson County, Illinois "very aged"
Residences: He came to Illinois in 1818.
Service: Soldier; North Carolina
Sources: W 
MURPHY, JOHN
Born: Northern Ireland
Buried: Near Lost Prairie, Perry County, Illinois
Residences: He came to Illinois in 1818 and settled near Lost Prairie, now 
Cutler, Peny County. Murphysboro, Jackson County, is said to be named in his honor.
Service: Soldier; North Carolina. He was in the battle of King's Mountain.
Sources: W 
MURPHY, JOHN 
Born: Probably Burke County, North Carolina 
Died: Anna Township, Union County
Residences: After the war he settled in Cape Girardeau, Missouri, from there he moved to Alexander County, Illinois, and in 1816 to Anna Township, Union County.
Service: Soldier; North Carolina. He first fought on the side of the British, but becoming convinced that the Continentals were in the right, entered the United States Army.
Sources: W 
MURRAY, ALEXANDER 
Born: 1761 
Died: September 7, 1845 
Buried: Log Church Cemetery, Hanover, JoDaviess County, Illinois 
Spouse: Isabella Duguid 
Service: Sergeant: New York. He served in Col. Willitts’ Regiment of Northern New York Volunteer Militia.
Pension: S11121 (NY)
Marker: Grave marked by Apple River Canyon Chapter DAR on June 14, 1974.
Sources: DAR, PI, PENSION 
MURRAY, DANIEL 
Died: August 5, 1820
Buried: Randolph County, Illinois
Spouse: Rachel Horner
Residences: He was living in Kaskaskia, Randolph County, Illinois, with his brother William before the arrival of Col. Clark.
Service: Patriotic
Service: Quartermaster: Illinois. He served under Col. George Rogers Clark in the Kaskaskia Campaign in Illinois.
Marker: His name is on a bronze marker on the grounds of Sparta High School, Sparta, Randolph County, placed by Fort Chartres Chapter DAR in 1934. 
Sources: DAR, NSDAR , PI,  W 
MYERS, WILLIAM 
Died: Probably Clinton County, Illinois
Service: Privateer: Maryland: Pennsylvania
Pension: R7545 (Md:Pa) Sea Service. He applied for a pension from Clinton 
County, Illinois, which was rejected under (act June 7, 1832) "Privateer service." He was granted bounty land for his war service.
Sources: PENSION, W 

NANCE, ZACHARIAH 
Born: May 5, 1760 in Charles City, Virginia
Died: December 22, 1835 
Buried: Farmers' Point Cemetery, Peters burg, Menard County, Illinois; Private Headstone 
Spouses: (1) Jane Wilkins
	 (2) Elizabeth (Morris) Bingley
Residences: His residence was in a part of Sangamon County which became Menard County in 1839.
Service: Private; Virginia Continental. He enlisted in New Kent County, Virginia, in Capt. James Pendleton's Company, Col. Charles Harrison's Regiment. He was in the battles of Monmouth and Stony Point. Pension: S31272 (Cent. Va.); Elizabeth, widow of Lewis Bingley, Virginia, pen­sioned W24325 (Va). (Much family data on file in this claim.)
Marker: Pierre Menard Chapter DAR, Petersburg, placed a bronze marker on his grave on October 23, 1932. His name is on a marker on the Sangamon County Court House, Springfield, placed by Springfield Chapter DAR on October 19, 1911.
Sources: DAR, HR, NSDAR, PI, PENSION, W 
NEER, JACOB 
Buried: Smith Grove Cemetery, Greenville, Bond County, Illinois
Service: Private; New York. He served in the Eighth Albany County Militia under Col. Robert Van Rensselaer.
Marker: The grave is marked.
Sources: HR, W 
NEW, JOHN 
Born: About 1756
Died: July 1849, age 93 
Buried: Livingston County, Illinois
Residences: He is listed in the 1850 Mortality Schedule for Livingston County, Illinois,age93;occupation,"papermaker " and "Revolutioner."
Sources: HR 
NEWELL, NORMAN 
Born: August 28, 1761
Died: April 6, 1850 
Buried: Tazewell County, Illinois 
Spouses: (1) Rosetta 
	 (2) Lucy Frisbee
Service: Private; Connecticut Continental. He enlisted in 1777 in Capt. Ezekiel Curtis' Company, serving eight months. 
Pension: S40199 (Conn); Illinois Pension Census, Tazewell County, June 1, 1840, age79, residing with Thomas Brooks, head of family.
Sources: DAR, HS, PI, PENSION 
NEWTON, JOSEPH 
Born: 1760 in North Carolina
Died: 1842 
Buried: Williamson County, Illinois
Residences: He came to Illinois in about 1815, settling in Pope County; he re­moved to Williamson County.
Service: Soldier; North Carolina. He served as a substitute and was in the battles of Cowpens and Guilford Court House. Pension: Residence Golconda, Pope County when pension claim was rejected (Act June7, 1832) "For  proof of identity with the soldier of the North Caro­line line."
Sources: PENSION, W 
NIXON, GEORGE, SR. 
Born: 1752 
Died: August 5, 1842 
Buried: Glenwood Cemetery, Coal Valley, Rock Island County, Illinois; Govern­ment Headstone 
Spouses: (1) Sarah Seeds 
	 (2) Martha 
Residences: After the war he removed to Ohio, and from there to Rock Island County, August 15, 1841. He is the great grandfather of President Richard M. Nixon. 
Service: Lieutenant; Delaware. He enlisted December 15, 1776 for three months as Ensign in Capt. George Evans Company, Col. Thomas Duffs Regiment, Delaware troops. He was made Lieutenant on October 15, 1777 and served until July 1778 in Capt. David Mckee's Company, Col. Thomas Duffs Regiment. He was in the battles of Princeton and Brandywine. 
Pension: S8919 (Del)
Sources: PI, PENSION, W 
NORTON, JAMES 
Buried: Probably buried in Gallatin County, Illinois
Residences: His residence at the time of the 1850 Federal Census was Gallatin County.
Service: Soldier; Virginia. After the Revolutionary War he continued in service in the Sixth United States Infantry.
Sources: CR, W 
O'FLYING, PATRICK 
Born: 1750 probably New Hampshire
Died: October 7, 1821 
Buried: Morgan County, Illinois 
Residences: After the war he removed to Ohio, and from there to Morgan County, Illinois. 
Service: Quartermaster Sergeant: New Hampshire Continental. He enlisted in April, 1775 at Cambridge, Massachusetts for eight months in Capt. John Moore's Company, Col. John Stark's Regiment; in 1776 as Orderly Sergeant in Capt. John Nesmith's Company, Col. Livingston's Regiment; in 1777 for six monthsin Capt. Daniel Livermore's Company, Col. Alexander Scarnmonds Regiment. He was Quartermaster Sergeant in Capt. Zachariah Bears Com­pany. He served from January to May 1778 and from January to December 1781, as sub-conductor of wagons in Gen. Poor's Brigade. He was in the battles of Bunker Hill, Quebec, Bemis Heights, and with the Indians on the Susquehanna under General Sullivan. He also served in the War of 1812. 
Pension: S35542 (NH Cont) (family information on file); Illinois pension roll, Morgan County, May 6, 1820, transferred from Ohio. 
Marker: His grave has been marked by the Reverend James Caldwell Chapter DAR, Jacksonville. 
Sources: DAR, PI, PENSION, I, V 
OCDEN, STEPHEN D, 
Died: May 1. 1812 
Buried: Tomkins Farm (Ogden Cemetery), near Paris, Edgar County, Illinois 
Service: Soldier: New Jersey. He served from Morris County, New Jersey, in the Eastern Battalion; wounded September 13, 1777, at Second River. 
Pension: R7776 (NJ). He was pensioned in Morgan County, Kentucky. 
Marker: His grave has been marked. 
Sources: DAR, HR, PENSION, W 
OGLE, BENJAMIN 
Buried: Probably St. Clair County, Illinois 
Residences: Illinois State Census, resident St. Clair County 1818-1820; Illinois State Census, resident Madison County, 1820; Illinois State Census, resident Greene County 1830. 
Service: Soldier: Virginia 
Pension: R7778 (Va): Pension claim rejected (Act of June7, 1832) while resident of Belleville, St. Clair County, Illinois, “not under competent military author­ity." 
Sources: CR, PENSION 
OGLE, JOSEPH 
Born: June 17, 1741 in Virginia (Delaware?) 
Died: February 21, 1821 
Buried: Near O 'Fallon, St. Clair County, Illinois 
Spouses: (1) Drusilla Biggs
	 (2) Jemima Meiggs 
Residences: He came to Illinois in 1785 from Wheeling, Virginia, settling in New Design. In 1802 he was a founder of Ridge Prairie, near the present town of O'Fallon. The earliest Methodist Church services were held in his home. Ogle County, Illinois, was named for Joseph Ogle, pioneer politician and Lieutenant of Territorial Militia, 
Service: Captain: Virginia. He commanded a Company of Virginia troops. His commission was signed by Patrick Henry. He commanded the American Forces in the battle of Fort Henry under the direct command of General Washington at Wheeling, Virginia. 
Pension: He received a land grant. 
Marker: A bronze marker was placed on the Ogle County Court House at Oregon by Rochelle Chapter DAR on June 14, 1924 "honoring Captain Joseph Ogle, pioneer, soldier, circuit rider, for whom the county was named." 
Sources: DAR, PI, W 
OLMSTED (Olmstead), JOSEPH 
Born: Probably Connecticut
Died: After 1832 
Buried: Pike County, Illinois 
Service: Soldier: Connecticut. He enlisted from Ridgefield, Connecticut, and served in the Fifth Regiment commanded by Col. Philip Bradley. 
Pension: S11157 (Conn). He applied for pension from Pike County, Illinois, in 1832. Olmstead, Joseph (deceased) residence Pittsfield, Pike County when pension claim rejected (Act June 7, 1832) "Died before the passage of the Act." 
Sources: PENSION, W 
OLNEY, SAMUEL 
Died: 1833 
Buried: Marquiss Cemetery, Monticello, Piatt County, Illinois; Government Head­stone 
Service: Lieutenant: Rhode Island. He served in Capt. Bowen's Rhode Island Militia .
Marker: His grave has a bronze DAR marker placed by Remember Allerton Chapter DAR, Monticello. 
Sources: DAR, NSDAR 
O'NEAL, BRYANT 
Died: April 11, 18__ 
Buried: Scott Cemetery, Jacksonville, Morgan County, Illinois 
Service: Soldier: Pennsylvania. He served in the Army under General Anthony Wayne.
Marker: His grave has been marked by the Reverend James Caldwell Chapter DAR, Jacksonville. 
Sources: DAR, HR 
O'NEILL, CONSTANTINE
Born: 1753
Died: September 16, 1834 
Buried: Woodford County, Illinois 
Spouse: Catharine Shepherd 
Service: Private: Pennsylvania. He enlisted October 2, 1781 from Bethlehem Township in Capt. Thomas Parkeson's Company, Col. Thomas Clarke's Regiment, Pennsylvania troops. Pension: Catharine W5446 (Penn). His widow, Catharine, applied for a pension after 1836. Residence Metamora, Woodford County, when pension claim re­jected (Act July 7, 1838) "Defective proof of marriage." 
Sources: HS , PI , PENSION 
OOLEY, DAVID 
Buried: Probably buried in White County, Illinois 
Service: Soldier: Virginia Pension: R7808 (Va). Residence was Carmi, White County, Illinois, when pen­sion claim rejected (Act of June7, 1832) "not military service." 
Sources: PENSION 
OSBORN, WILLIAM
Born: 1758 
Buried: Brownsville Cemetery, near Murphysboro, Jackson County, Illinois; Pri­vate Headstone
Children: James Ford, Hawkins S., William D., Udosia (daughter) 
Sources: County Record 
OUTHOUSE, PETER
Born: 1757
Died: August 1, 1834 Buried: Clinton County, Illinois 
Spouses: (1) Georetta
	 (2) Mrs. Nancy Duncan 
Residences: He removed to Kentucky, and in 1818 came to Illinois, settling in the southwest part of Clinton County. 
Service: Private: Maryland. He enlisted in Fredericktown, Maryland, in the Seventh Regiment serving in 1780; from October 26, 1780 until November 1783 with Lt. William Lamar and Capt. Lloyd Beall in the Ninth Company. 
Pension: S46400 (Md), Pension Roll, August 13, 1828, Clinton County; transferred from Oldham County, Kentucky. 
Sources: PI, PENSION, W 
OVERSTREET, JOHN
Born: 1760in Virginia
Died: July 7, 1848 at home in Fancy Creek Township, Sangamon County Buried: West Cemetery, Athens, Men ard County, Illinois; Government Headstone Married: 1785 Spouse: Nancy Dabney Residences: After the war he removed to Ohio and from there to Fancy Creek Township, Sangamon County. 
Service: Private: Virginia. He enlisted in the First Virginia Cavalry when fifteen years of age; enlisted in 1777 for three years in the Fourteenth Regiment. He was in the battles of Monmouth, Stony Point, Brandywine, Germantown, and the siege of Yorktown. He endured great hardship at Valley Forge. He was also in the War of 1812. 
Pension: S40231 (Va); Illinois Pension Census, June 1, 1840, Sangamon County, age 80, living with Dabney Overstreet. 
Marker: The grave has been marked by Pierre Menard Chapter DAR, Petersburg. His name is on a bronze plaque in the south mall, Old State Capitol, Spring­field, placed by Springfield Chapters DAR and SAR October 19, 1911. 
Sources: DAR, HR, NSDAR, PI, PENSION, W 
OWEN (Owens), MASON
Born: September 8, 1760 in King's County, Virginia
Died: 1846 
Buried: Montgomery County, Illinois 
Residences: In 1807 he moved to Kentucky, and in 1827 came to Montgomery County, Illinois. Service: Private: Virginia Continental. He served eight months in Capt. Joseph Rogers' Company; ten months in Ca pt. George Strother's Company; five months in Capt. William Bunbury's Company, Col. John Skinner's Regiment. He was in several skirmishes and at the siege of Yorktown. Pension: S32426 (Va): Pension roll, September 25, 1833, age 74; Pension Census, Montgomery County, Illinois, June1, 1840, age 80, living with Mason Owen. Sources: PENSION, W 
OWEN, PETER
Born: 1765 in Lunenburg County, Virginia 
Buried: Pleasant Grove Cemetery, Mt. Vernon, Jefferson County, Illinois 
Service: Soldier: Virginia. He enlisted in May 1781 in Capt. William Ray's Com­pany, Col. Nelson' s Regiment. He also served as a Minute Man. He served in the War of 1812. Sources: HS 
OWSLEY, THOMAS
Born: 1731
Died: November 1, 1796 
Buried: Oak Ridge Cemetery, Springfield, Sangamon County, Illinois. His body was moved from Owsley Plantation in Crab Orchard, Kentucky. 
Service: Private: Virginia
Marker: The grave was marked by Springfield Chapter DAR, October 17, 1970. 
Sources: DAR 


PACE, JOEL, SR
Born: July 28, 1762 in Virginia 
Died: August 31, 1831
Buried: Pace Cemetery, Mt. Vernon, Jefferson County, Illinois 
Spouse: Mary East
Residences: After the war he removed to Kentucky, and from there to Mt. Vernon Township, Jefferson County, Illinois. 
Service: Private; Virginia Pension: Pension claim rejected, Resolution of Congress, May 29, 1830 "on account of amount of property." Mary W21917 (Va) 
Sources: HR, NSDAR, PI, PENSION, W 
PADDOCK, GAIUS (Gains) 
Born: November 2, 1758 in Massachusetts 
Died: August 11, 1831
Buried: Paddock Cemetery, eight miles north of Edwardsville, Madison County, Illinois Spouse: Mary Woods 
Service: Private; Massachusetts Continental. He enlisted January 1 ,1776 in Capt. Isaac Wood's Company, Col. William Larned's Regiment; assisted with the evacuation of New York; was in the battle of Trenton and the skirmish at Frog Neck; in 1779 and 1780 he served under Lt. Joseph Bates, Col. Gamaliel Bradford's Regiment. 
Pension: Polly W26850 (Mass) 
Marker: His name is on a bronze tablet in the Madison County Court House at Edwardsville, placed by Ninian Edwards Chapter DAR, September 16, 1912.  Ninian Edwards Chapter also placed a tablet in the woods at Paddock Cemetery honoring Gaius Paddock and naming his wife and seven daughters, in October 1927. 
Sources: DAR, HR, NSDAR, PI, PENSION, W 
PADFIELD, WILLIAM 
Born: August 25, 1765 in Maryland
Died: August 25, 1840 
Buried: Union Grove Cemetery, south of Summerfield, St. Clair County, Illinois 
Spouses:	(1) Mary Tadlock, married May 10, 1787 in Greene County, Tennessee
	(2) Nancy Jarvis, married November 20,1834, St. Clair County 
Children: 	(10-12) Joseph, Thomas, William, James, Hiram 
Residences: He removed to Christian County, Kentucky, and in 1815 to Summer­field, St. Clair County, Illinois. He served as Justice of the Peace in St. Clair County; built the first grist mill and saw mill. 
Service: Soldier; Maryland. He enlisted and served as a driver of a provision wagon. 
Sources: W, Family Data 
PAGAN, DAVID 
Born: About 1745 
Died: January 16, 1815
Buried: Probably near Kaskaskia, Randolph County, Illinois 
Spouses: (1) Elizabeth Farrell
	 (2) Mrs. Mary (Carter) Harman
Residences: He settled On Nine-Mile Creek, near Kaskaskia, Randolph County, Illinois
Service: Private; Virginia. He served under Col. George Rogers Clark. Pension: Pagan, David, Va: Mary W5499
Marker: His name is on a bronze marker on the grounds of Sparta High School, Sparta, placed by Fort Chartres Chapter DAR in 1934
Sources: DAR, NSDAR, PI, PENSION, W 
PAINTER, JOHN 
Born: About 1756
Died: 1851
Buried: Willson Cemetery, Cambria, Williamson County, Illinois
Service: Soldier; North Carolina: Virginia. He served in the Virginia troops and was paid at Romney. 
Pension: Pension Census, Williamson County, June 1, 1840, age 84
Sources: HR, PENSION 
PAINTER, JOSEPH 
Born: 1744 in New Jersey
Died: After 1840
Buried: Probably at Hutton, Coles County, Illinois
Service: Private; North Carolina Continental. He enlisted six times between 1777 and 1781, serving in the Companies of Capt. William Bateman, Capt. John Turnbull, Capt. James Robinson, and Capt. Gillyfalls, in Regiments com­manded by Col. Bateman, Col. Hugh Brevard, Col. James Armstrong and Col. William Davidson. He was in the battle of Bamsour's Mill and in Indian skirmishes. 
Pension: S32430 (NC); Illinois pension roll, Coles County, January 31, 1834, age 90. He was also listed as a resident of Coles County in the Pension Census of June 1, 1840. 
Sources: PENSION, W 
PALMER, EPHRAIM 
Born: December17, 1760 in Massachusetts
Died: June 30, 1852
Buried: Kishwaukee Cemetery, Rockford, Winnebago County, Illinois 
Spouse: Margaret Force
Service: Private; Connecticut: New York. He enlisted in 1777 when seventeen years of age in Capt. Sylvanus Cobb's Company for one month; from 1778 to 1779 he served one year in Capt. Samuel Lockwood's Company, Col. John Wood's Regiment. He was taken prisoner June 7, 1779 and confined in the Small Pox Hospital in New York; was exchanged in February 1780. He enlisted from Salem, New York, as a substitute in Capt. William Stevens' Company and was assigned the task of guarding the notorious Major Andre. 
Pension: S8940 (Conn:NY) 
Marker: His grave was marked by Rockford Chapter DAR in June 1907. 
Sources: DAR, HR, PI, PENSION, W 
PARKER, JOHN 
Born: September 5, 1758 in Maryland 
Died: May 19, 1836 
Buried: Probably Coles County, Illinois 
Spouse: Sarah White
Service: Private; Virginia Continental. He served in Capt. Buller Claiborne's Company, Col. Alexander Spotswood's Regiment in 1777. Pension: S32435 (Va), Illinois pension roll, Coles County, Illinois, October 22, 1833, age 75. 
Sources: PI , PENSION, W 
PARKS, SAMUEL 
Born: About 1747in Virginia
Died: After 1840
Buried: Old Lost Cemetery, Clay County, Illinois 
Service: Soldier; North Carolina 
Pension: S8937 (NC); Illinois pension census, Clay County, June 1, 1840, age 93 
Marker: The grave was marked by Vinsans Trace Chapter DAR, Flora, April 7, 1973. 
Sources: DAR, PENSION, W 
PARR, MATHIAS 
Born: 1746 in New York 
Died: November 5, 1844 
Buried: White County, Illinois 
Residences: Coming to Illinois, he settled in Fayette County, but removed to White County. 
Service: Private; New York. He served in the Second Regiment, New York troops, commanded by Col. Philip Van Courtland. 
Pension: S46402 (NY); Pension roll, December 1l , 1830; Pension Census, White County, June 1, 1840, age 85, residing with Edmund Thompson, head of family . 
Marker: A flat government marker in the old cemetery at Carmi was dedicated by Wabash Chapter DAR on September 21, 1964. His name is also on a monument in city park placed by the Chapter in 1936, honoring soldiers buried in White County. 
Sources: DAR, NSDAR, PENSION, W 
PATTERSON, ALEXANDER K, 
Born : In New York
Buried: Cozard Farm Cemetery, Elvaston, Hancock County, Illinois
Residences: Patterson, New Jersey was named for a son of Alexander Patterson 
Service: Soldier; New York . He served in Col. John Hathorn's Fourth Regiment, Orange County, New York, Militia
Marker: His name is on a plaque in the Hancock County Court House, Carthage, honoring Revolutionary War soldiers and sailors, placed by Shadrach Bond Chapter DAR, July 10, 1910
Sources: DAR, HR, NSDAR, W 
PATTERSON, JAMES 
Born: July 5, 1758 in Montgomery County, Virginia 
Died: December 3, 1838 
Buried: Patterson Cemetery, Sullivan, Moultrie County, Illinois 
Spouses: (1) Miss Nelson
	 (2) Sarah Davidson
Service: Private; North Carolina. He enlisted in 1775 from Rutherford County, North Carolina, serving three months in Capt. James Wilson's Company, Col. John Rutherford's Regiment; in August 1777 for three months in Capt. Jesse Lytle's Company, Col. John Rutherford's Regiment; in September 1780, three months in Capt. Williams' Company, Col. Arthur Campbell's Regiment; nine months in Capt. Jesse Lytle's Company, Col. John Rutherford's Regiment. He was in the battles of King's Mountain, Cowpens, Guilford Court House, and Yorktown. He was wounded at Cowpens. 
Pension: Patterson, James (NC) Sarah WI086I; BLW 30924-160-55 
Sources: HR, PI, PENSION, W 
PATTERSON, SOLOMON 
Born: March I, 1763 
Died: May 21, 1842 
Buried: Manchester Cemetery, Manchester, Scott County, Illinois 
Spouse: Mary Melick 
Residences: He came to Monroe County, Illinois, from Cumberland County, Pennsylvania, but removed to Scott County, where he died at the residence of his daughter, Mrs. McCracken. 
Service: Private; Pennsylvania. He served from Cumberland County, Pennsyl­vania, in Capt. John McConneIl's First Company, Fourth Battalion, Col. Samuel Culbertson's Regiment. 
Sources: HR, PI, W 
PATTON, THOMAS 
Born: April 25, 1735 in Marlboro, Pennsylvania
Died: After 1833
Buried: Palestine Township, Crawford County, Illinois 
Service: Private; North Carolina Continental troops. He enlisted in 1779 from North Carolina, serving six months in Capt. John Hardy's Company, Col. Joseph McDowell's Regiment; in 1780 for six months in Col. William Camp­bell's Regiment; in 1781 for six months in Capt. William Neal's Company, Col. William Campbell's Regiment. He was in the battles of Ramsour's Mill, King's Mountain, Cowpens, and Guilford Court House. 
Pension: S35303 (NC); Pension roll, January 28, 1826, transferred from Indiana; Pension roll, September 25, 1833, age 98 
Sources: PENSION, W 
PEAKE (Pease), JOHN 
Born: December 28, 1756 in Fairfax County, Virginia 
Died: December 21, 1841 
Buried: Old Salisbury Cemetery, Sangamon County, Illinois 
Spouse: He did not marry. 
Residences: He removed to Salisbury District, Sangamon County, Illinois. 
Service: Soldier; Virginia. He enlisted for six months and served in the Fifth Troop of First Regiment of Light Dragoons of Virginia in Capt. Henry "Light Horse Harry" Lee's Company, serving until April 1777. He enlisted in September 1777 for three months in Capt. Benjamin Harrison 's Company, Thirteenth Regiment, commanded by Major Martin Pickett. 
Pension: S32439(Va). He was pensioned in Kentucky in 1833
Marker: His name is on a bronze plaque in the west mall, Old State Capitol, Springfield, placed by Springfield Chapters DAR and SAR, October 19, 1911. 
Sources: DAR, PENSION, W 
PEARCE, MOSES 
Born: Sept ember 23, 1755 
Died: 1822/25 
Buried: Indian Creek Township, White County, Illinois 
Married: February 12, 1775 
Spouse: Jemima Robinson
Residences: He came to White County in 1816
Service: Patriot: North Carolina. Signed a petition for division of Anson County, North Carolina
Sources: DAR 
PEEBLES, JOHN 
Born: January 31, 1763 
Died: October 6, 1849 
Buried: Peebles Cemetery, Chesterfield, Macoupin County, Illinois; Private Head­stone 
Spouses: 		(1) Wilmath Owen 
(2) Martha Johnsy 
Residences:  He removed to Kentucky and from there to Macoupin County, Illinois. 
Service: Private; South Carolina. He served under Capt. William Nettles' Com­pany in General Francis Marion's South Carolina troops. He was in the battles of Eutaw Springs, Cowpens, the "Truce Lands," and in scouting parties. 
Pension: Residence Carlinville, Macoupin County when pension claim rejected (Act June 7 ,1832) " For proof from the South Carolina records." 
Marker: A bronze marker was placed by Ninian Edwards Chapter DAR, Alton, September 30, 1962. 
Sources: DAR, HR, NSDAR, PI, PENSION, W 
PENNEY (Penny), WILLIAM 
Born: 1751in North Carolina
Died: March 15, 1821
Buried: Richland Cemetery, Pleasant Plains, Sangamon County, Illinois; Private 
Headstone 
Spouse: Elizabeth 
Residences: He removed to Pope County, Illinois, and from there to Richland Creek, Sangamon County
Service: Captain; Civil Service: North Carolina. He served as Captain of a Cavalry Company. 
Pension: He was pensioned.
Marker: His grave was marked by Springfield Chapter DAR, October 13, 1970.  His name is on a bronze plaque in the south mall, Old State Capitol, Spring­field, placed by Springfield Chapters DAR and SAR, October 19, 1911. 
Sources: DAR, HR, NSDAR, PI, W 
PENNINGTON, CHARLES
Born: June 6, 1758
Died: September 5, 1845
Buried: Orterbein Cemetery, Charleston, Coles County, Illinois; Private Headstone 
Spouse: Cassandra Swathlander
Service: Drummer and Wagoneer; New Jersey, Pennsylvania Pension: R8098 (Penn) Residence Coles County when pension claim rejected (Act June 7,1832) "Not military service." 
Sources: HR, NSDAR, PI , PENSION 
PERKINS, RUFUS 
Born: About 1763 at Bridgewater, Massachusetts 
Died: January 20, 1848 (10-30-1848?) 
Buried: Buffalo Grove Cemetery, Polo, Ogle County, Illinois; Government Headstone 
Residences: After the war he lived in New York; coming to Illinois in 1847, he settled at Buffalo Grove, near Polo, Ogle County. The aged veteran made the long journey by stage and steamboat to Chicago, and from there to Buffalo Grove by lumber wagon. 
Service: Private; Massachusetts. He enlisted at Ashfield, Massachusetts, in Capt. Abel Dinsmore's Company, serving three months; six months in the Company of Capt. Canston and Capt. Samuel Hughs; enlisted August 10, 1778 serving until January 1, 1779 in Capt. Enoch Chapin's Company; in Capt. Oliver Shattuck's Company, commanded by Lt. Col. Barnabas Sears, until his dis charge in 1781. 
Marker: A bronze tablet in his memory was placed in the Polo Historical Library by the Historical Society, the Rockford Chapter DAR, Rochelle Chapter DAR, Elder William Brewster Chapter DAR at Freeport, Dixon Chapter DAR, and the Grand Army Post. The tablet was unveiled by Edgar Thomas Clinton, a great, great, grandson of Rufus Perkins. 
Sources: DAR, HR, NSDAR, W 
PERKINS, UTE 
Born: July 15, 1761 
Died: March 11, 1842
Buried: Old Cemetery, near Nauvoo, Hancock County, Illinois 
Spouse: Sally Gant 
Service: Private; North Carolina 
Pension: R8118 (NC); Residence Carthage, Hancock County, Illinois, when pen­sion claim was rejected (Act June 7, 1832) "not having six months service." 
Marker: The grave has been marked by Shadrach Bond Chapter DAR, Carthage. 
Sources: DAR, PI, PENSION, W 
PERRIN, NATHANIEL 
Born: About 1762 
Died: After 1840 
Buried: Pike County, Illinois 
Service: Soldier; Massachusetts. He enlisted January 6, 1779 and served until June 5, 1780 in Capt. Benjamin Frothingham's Company; he again served from October 1780 until January 1781. 
Pension: He applied for a pension while living in Tennessee. He was a resident of Pike County, Illinois, in the Pension Census, June 1, 1840, age 78. 
Sources: PENSION, W 
PHELPS, RUFUS R. 
Born: 1767 in New York 
Died: 1839 at Holcombe 
Buried: Lindenwood Cemetery, Lindenwood, Ogle County, Illinois 
Service: Private; New York. He enlisted from Dutchess County, New York, and was stationed at Fort Herkimer. He was wounded and discharged, receiving a soldier's bounty. 
Marker: His grave was marked by the Rockford Chapter DAR, June 19, 1909. 
Sources: DAR, HR, W 
PHILLIPS, DAVID 
Born: 1755in Orange County, North Carolina
Died: 1826
Buried: On his farm on Richland Creek, north of Belleville, St. Clair County, Illinois
Spouse: Sophia Hoffman, born 1757 
Children: Peggy (Conrad Carr), Katy (Abner Carr), Sally (Henry Stout), Hannah (William Stout), Mary (McCuller), Joe, Samuel, Isaac, Jeremiah, Daniel , William, Benjamin
Residences: After the war he removed to Kentucky and from there to Illinois, settling on Richland Creek, St. Clair County
Service: Private; North Carolina. He served in Capt. Turner's Company. He was discharged in October 1778, having served for 30 months. 
Pension: He received North Carolina Military Warrant #15, dated September 15, 1787, entitling him to 228 acres in Davidson County on the south side of Big Harpeth River. He retained title to this warrant. 
Sources: DAR, W 
PIGGOTI, JAMES 
Born: April 24, 1732 at Norwalk, Connecticut 
Died: February 20, 1799 
Buried: East St. Louis, St. Clair County, Illinois 
Spouse: Frances James
Service: Captain; Pennsylvania. He was a privateer; removed to Westmoreland County, Pennsylvania, where he was made Captain of the Pennsylvania Associators, April 6, 1776; Captain Eighth Pennsylvania, August 9, 1776, serving under General Arthur St. Clair. He was in the battles of Brandywine and Saratoga; resigned October 22, 1777; served subsequently in Clark's Illinois Regiment on the Ohio. He was placed in command of Fort Jefferson, five miles below the mouth of the Ohio River. He came to St. Clair County in 1783 and established a fort west of Columbia, Monroe County. In 1795 he built a ferry between East St. Louis, Illinois and St. Louis, Missouri. 
Sources: PI, W, Letter 
PILLARS (Pillers), JAMES
Born: About 1760 in Virginia
Died: 1833/34
Buried: Randolph County, Illinois
Residences: James and Richard Pillars, brothers, were residents of Fort Massac in 1781, and in 1793 moved to Randolph County, Illinois
Service: Private; Virginia
Marker: His name is on a bronze marker on the grounds of Sparta High School, Sparta, Randolph County, placed by the Fort Chartres Chapter DAR in 1934
Sources: DAR, NSDAR, PI, W 
PILLARS, RICHARD 
Died: Randolph County, Illinois
Residences: Richard and James Pillars, brothers, were residents of Fort Massac in 1781, and in 1793 moved to Randolph County, Illinois
Service: Private; Virginia 
Sources: NSDAR, W 
PINKSTAFF, ANDREW
Born: 1743 in Frederick County, Virginia
Died: September 10, 1841
Buried: East Pinkstaff Cemetery, Pinkstaff, Lawrence County, Illinois; Private 
Headstone 
Spouse: Winnie Owens; died August 9, 1833", age 74
Service: Private; Virginia. He enlisted in Capt. George Berry's Company, serving six months. He also served eighteen months under Col. Daniel Morgan. He was in the battles of Cowpens and Guilford Court House. 
Pension: R8261 (Va); Andrew (deceased) residence of Lawrenceville, Lawrence County, when pension claim rejected (Act June 7, 1832) "For further proof from the Virginia records." 
Marker: His name is on a bronze tablet in the Lawrence County Court House, Lawrenceville, placed by Toussaint du Bois Chapter DAR in 1921
Sources: DAR, HR, PI, PENSION, W 
PINSON, AARON 
Born: 1754/5
Died: After April 12, 1838
Buried: Old Hanks Road Cemetery, Paris, Edgar County, Illinois 
Spouse: Sarah Lycan
Service: Civil
Service: Patriotic Service; North Carolina 
Sources: PI, Letter 
PIPER, ASA
Born: About 1761 in Massachusetts
Died: After 1833
Buried: Crawford County, Illinois
Service: Private; Massachusetts Continental. He enlisted when seventeen years of age in Capt. Phineas Parker's Company from Concord, serving six months from June 5, 1780; he again enlisted in June 1781. Pension: S31302 (Mass); Pension roll, Crawford County, April 11, 1833, age 72 
Sources: PENSION, W 
PIXLEY (Pigsley), JOB
Died: September 11, 1834
Buried: Friendsville Cemetery, Friendsville, Wabash County, Illinois 
Residences: He came to Illinois in 1809, settling in Barney Prairie, Wabash County, later moving to Friendsville. He was listed in Edwards County in the 1820 census. Wabash County was taken from Edwards County in 1824. 
Service: Soldier; Massachusetts Continental: Rhode Island. He enlisted from Dighton County, Massachusetts in 1775 in Capt. Peter Pitts Company, Col. Timothy Walker's Regiment, serving for more than five years with these and other officers. 
Pension: S35562 (Mass: RI) 
Sources: CR, HR, PENSION, W 
PLANT, WILLIAMSON 
Born: 1763 in Louisa County, Virginia
Died: 1830
Buried: Sugg Cemetery, Pocahantas, Bond County, Illinois or Pocahantas Cemetery, Greenville, Bond County 
Spouse: Frances Watts
Service: Private; Virginia. He enlisted in Capt. Richard Clough's Company, Fifth Regiment, Virginia troops. He enlisted in the Militia, serving at various times until the close of the war. Pension: S39016 (Va) Sources: HR, PI, PENSION, W 
PLATT, JONAS 
Born: 1759 in Fairfied County, Connecticut
Died: 1842 Danbury, Connecticut
Buried: Bushnell Cemetery, Bushnell, McDonough County, Illinois
Service: Private; Connecticut. He served in the Fourth Connecticut Militia during the Fishkill campaign in October 1777. 
Sources: NSDAR 
POSEY, THOMAS 
Born: July 9, 1750 in Virginia
Died: March 18, 1818
Buried: Shawneetown, Gallatin County, Illinois; Private Headstone 
Spouses: (1) Martha Matthews 
	 (2) Mary (Alexander) Thornton 
Residences: In 1794 he removed to Kentucky where he was a Senator in 1805­1806. He was Governor of Indiana Territory, serving until it became a state in 1816. He removed to Illinois, settling in Gallatin County. 
Service: 	Lieutenant-Colonel; Virginia. He served as Captain in the Seventh Virginia Regiment, March 20, 1776; Major in Col. Daniel Morgan's Second Virginia Regiment, April 30, 1778; transferred to the Seventh Virginia Septem­ber 14, 1778; Lieutenant-Colonel, September 8, 1782; transferred to First Virginia, January 1, 1783; retired March 10, 1783; Brigadier-General United States Army, February 14, 1793. He was in the battles of Monmouth, Stony Point, and at the siege of Yorktown. He was in the war of 1812. 
Pension: BLWT 1733-450 (no papers) 
Sources: DAR, PI, PENSION, W 
POST, CALEB
Born: In New Jersey
Died: 1843
Buried: North Cemetery, White Hall, Greene County, Illinois
Service: Private; New Jersey Militia. He served under Captain Seely. 
Pension: S32451 (NJ); Pension roll, Greene County, March 6, 1834, age 71
Marker: His name is inscribed on the White Hall Soldiers Monument
Sources: DAR, HR, HS, PENSION, W 
POWELL, LEVIN (EIevcr) H
Born: 1763 in Loudoun County, Virginia
Died: November 28, 1836
Buried: Tazewell County, Illinois 
Spouse: (2) Elizabeth Cohagan
Residences: His Illinois residence was Tremont, Tazewell County
Service: Private; Virginia: South Carolina. He enlisted in 1780in South Carolina and served in Capt. Joseph Hughes' Company, Col. Anthony White and Col. William Washington's Regiment of Continental Dragoons. He was in the battles of Guilford Court House and Eutaw Springs. He was discharged in Richmond in 1783. 
Pension: Elizabeth R8394 (SC) BLWT 1935-100. Elizabeth, widow of Eleven, resi­dence Tremont, Tazewell County when pension claim rejected (Act July 7, 1838) "Married in 1807 -no claim." 
Sources: HS, PI, PENSION 
POWERS, ABNER
Born: December 15, 1760in Richmond, New Hampshire
Died: September 25, 1852 at St. Charles, Illinois
Buried: Canada Corners Cemetery, Lily Lake, Kane County, Illinois 
Spouse: Sabra Porter 
Service: Drummer; Corporal; New Hampshire. He enlisted in 1776, serving until December 1781 in Col. John Stark's Regiment. He enlisted in 1778 for two years in Capt. William Farwell's Seventh Company. He was in the battles of Bennington, Saratoga, Valley Forge, and Yorktown. He is reported to have also served from Vermont, Pennsylvania, and Virginia. 
Marker: The original marble slab on his grave was replaced with a thirty foot monument. Three companies of the Third State Regiment, five hundred members of the Grand Army, together with a large number of citizens, attended the dedication ceremonies. The grave was marked by Elgin Chapter DAR in 1900-190l. 
Sources: DAR, HR, NSDAR, PI, W 
PRICE, LOUIS J. 
Buried: Memorial Section, White Hall City Cemetery, White Hall. Greene County, 
Illinois; Government Headstone
Service: Soldier; New Jersey Militia 
Sources: HR 
PRICKETT, GEORGE
Born: 1757 in Maryland
Died: July 20. 1846
Buried: Woodlawn Cemetery, Edwardsville, Madison County, Illinois 
Spouse: Sarah Anderson
Residences: After the war he lived in one of the Carolinas, moved to Georgia, then to Kentucky and in 1808 to Madison County, Illinois
Service: Private; Virginia
Marker: The grave was marked by Edwardsville Chapter DAR, September 17, 1972
Sources: DAR, PI, W 
PRIME (Primm), JOHN
Born: May 17, 1750 in Stafford County, Virginia
Died: March 12, 1837
Buried: On farm near Belleville, St. Clair County, Illinois 
Spouse: Elizabeth Hansbrough (1761-1832) 
Children: They had seventeen children
Residences: He came to Illinois in 1803, settling near Belleville, St. Clair County
Service: Private; Virginia Continental. He served seven years in the war from Stafford County, Virginia; was present at the surrender of Cornwallis. 
Pension: S32458 (Va): Illinois pension roll, St. Clair County, January 17, 1834, age 84 
Sources: NSDAR, PI, PENSION, W 
PROCTOR, UTILE PAGE 
Born: 1760 in Granville County. Virginia 
Died: November 15, 1852 
Buried: Concord Cemetery, McLeansboro, Hamilton County, Illinois 
Spouses: (1) Sarah Jane Woodruff 
	 (2) Sarah Bates 
Service: Private; Patriotic
Service: Virginia. He enlisted in March 1778 in Capt. Cornelius Riddle's Company, serving to the close of the war. He remained in service until August 1794. 
Pension: Pension Roll, January 9, 1834, Hamilton County, age 73. Sarah W576 BLWT 26622-160-55 (Va) 
Sources: HR, PI , PENSION, W 
PROCTOR, NICHOLAS 
Born : 1755 probably in Virginia 
Died: After 1834 
Buried: Hamilton County, Illinois 
Spouse: Catherine 
Children: Nicholas Proctor is probably a brother of Little Page Proctor. 
Service: Private; Virginia Militia 
Pension: Pension roll, Hamilton County, May 3, 1834, age 78. Catherine W8537 ­BLWT-26634-160-55 (Va) 
Sources: PENSION, W 
PRUITT, MARTIN 
Born: 1748 in Virginia 
Died: 1844 
Buried: Pruitt Cemetery, Bethaldo, Madison County, Illinois 
Residences: He came to Illinois in 1806, settling in Madison County
Service: Private; Sergeant: Virginia Continental. He enlisted in the fall of 1778 for two years in Capt. William Campbell's and Capt. William Edmintori's Company, Col. William Campbell's Regiment. He was in the battle of King's Mountain. Pension: S32455 (Va). Madison County, Illinois pension roll, August 22, 1833, age 85; Madison County, Illinois pension census, June 1, 1840, age 93, residing with Solomon Pruitt, head of family
Marker: His name is on a bronze tablet at the Madison County Court House, Edwardsville, placed by Ninian Edwards Chapter DAR, Alton, September 16, 1912
Sources: DAR, NSDAR, PENSION, W 
PULLIAM (Pullen, Pullin), GEORGE 
Buried: Cumberland Cemetery, Ball Township, Sangamon County, Illinois 
Spouse: Nancy
Service: Soldier; Virginia Continental 
Pension: He received a tract of land in Sangamon County: BLWT 13724-160-55 Nancy W8539 (Va)
Marker: His name is on a bronze plaque in the south mall, Old State Capitol, Springfield, placed by Springfield Chapters DAR, and by SAR, October 19, 1911 
Sources: DAR, PE NSION, W 
PULLIAM, JOHN 
Born: In Botetourt County, Virginia 
Died: 1813 
Buried: Fayetteville, St. Clair County, Illinois 
Residences: After the war he removed to Kentucky and in 1796 to New Design, Monroe County. He later moved to Fayetteville, St. Clair County. 
Service: Soldier; Virginia 
Sources: W 
PULLIAM, THOMAS 
Buried: Probably Wabash County, Illinois 
Residences: He came to Wabash County, Illinois in 1815, settling in the town of Coffee
Service: Soldier; Virginia 
Sources: W 
PURVIANCE, JOHN 
Born: June 19, 1760 at Lancaster County, Pennsylvania 
Died: September 27, 1833 
Buried: Richland Baptist Cemetery, Pleasant Plains, Sangamon County, Illinois; Private Headstone 
Spouses:  (1) Nancy Ferguson; born June 1763, Prince George County, Virginia; died July 11, 1796, Cabarrus 			County, North Carolina; married July 10, 1783, Mecklenburg County, N.C. 
	(2) Elizabeth Lisenby; born December 12, 1761, Wales; died March 23, 1832, Sangamon County, 			Illinois; married December27,1798, Cabarrus County, N.C. 
Children: (first marriage): Matilda (Sam Irwin), Elizabeth (John Plunkett), David Simpson, Margaret (William 				Irwin), Alexander C., John G. 
	  (second marriage): James, Samuel, Nancy (Peter Shepherd), Eliza (Peter Lanterman) 
Residences: The family moved from Pennsylvania to Mecklenburg (later Cabarrus) County, North Carolina in 1765; moved to Jackson County, Tennessee for one year and then to Madison (later Sangamon) County, Illinois in 1820. 
Service: Private; North Carolina; South Carolina. He enlisted in North Carolina in 1779 in Capt . James White's Company, commanded by General Griffith Rutherford, serving three months. He was in the battle of Stony River. In 1780 he served in the Cavalry with Capt. William Penny for three months in battles and skirmishes of Lynch Creek and Rugby Mills. In July, 1781 he enlisted in Capt. William Alexander's Company and served ten months; was at Kegaree River, Friday's Ferry, and Eutaw Springs. In August 1782 he enlisted in Capt. Robert Burns' Company, Col. William Davis' Regiment. He was captured at Fort Orangeburg, South Carolina. 
Pension: S32459 (NC:SC); Illinois pension roll, Sangamon County, October 23, 1833, age 71 
Marker: His name is on a bronze plaque in the south mall, Old State Capitol, Springfield, placed by Springfield Chapters DAR and SAR, October 19, 1911. The grave was marked by Springfield Chapter on May 12, 1970. 
Sources: DAR, HR, NSDAR, PI, PENSION, W, Family Information 
PUTNAM, HOWARD
Born: February 11, 1762
Died: January 24, 1834
Buried: Country Cemetery near Allendale, Wabash County, Illinois 
Spouses: (1) Hannah Greene
	 (2) Caroline Jones
Service: Private; Massachusetts. He served in Capt. Drury's and Capt. David Cook's Company, of Col. Crane's Massachusetts Regiment. 
Pension: S17032 (Mass) 
Sources: NSDAR, PI , PENSION 
PYATT (Piatt), EBENEZER
Born: 1755 in Pennsylvania
Died: 1835
Buried: Jackson County, Illinois 
Spouse: Rebecca Milburn
Residences: After the war he removed to Ohio, and from there to Tennessee, then to Kentucky and in 1814 to Jackson County, Illinois, settling at Dutch Ridge
Service: Sergeant; Pennsylvania Continental: Virginia. He enlisted in Virginia, serving four years. He also served in the Pennsylvania Continental line. 
Pension: S32460 (Pa): Pension roll, October 9, 1833, Jackson County, age 79 
Sources: PI, PENSION, W 

QUEEN, JOHN E. 
Died: June 16, 1849 
Buried: Rowan Cemetery, Makanda, Jackson County, Illinois 
Sources: HR 

RALSTON, WILLIAM
Born: 1759 in Virginia 
Died: 1835 
Buried: Morgan Cemetery, Gardner Township, Sangamon County, Illinois
Residences: After the war he removed to Kentucky, and in 1828 came to Illinois, settling in Gardner Township, Sangamon County.
Service: Soldier; Virginia. He enlisted and served in Virginia and was present at the surrender of Cornwallis. 
Pension: He was pensioned.
Marker: His name is on a bronze plaque in the south mall, Old State Capitol, Springfield, placed by Springfield Chapters DAR and SAR, October 19, 1911. The grave was marked by Springfield Chapters November 24, 1973.
Sources: DAR, NSDAR, W 
RAMSEY, ALEXANDER 
Died: September 8, 1855
Buried: Mt. Erie Cemetery, Mt. Erie, Wayne County, Illinois; Private Headstone
Service: Soldier; Pennsylvania 
Pension: BLWT 1333-200(Pa)
Sources: HR, PENSION 
RAMSEY, ALLEN
Born: June 12, 1764
Died: August 15, 1845
Buried: Orio Cemetery, Friendsville, Wabash County, Illinois; Government Head­stone
Spouses:  (1) Barbara Decker
	(2) Elizabeth Reedy
Service: Private; Pennsylvania Militia 
Pension: 532467 (Pa); Pension Census, Wabash County, June 1,1840, age 76. He received bounty land for his service in the war.
Sources: HR, HS, PI, PENSION 
RANDLE, ISHAM
Born: March 23, 1758 in Brunswick County, Virginia
Died: After April 18, 1838
Buried: Goshen, Madison County, Illinois
Spouse: Frances Jackson 
Service: Private; Virginia; North Carolina Continental. He enlisted in Montgomery County, North Carolina in 1780 for three months in Capt. Abner Crump's Company, Col. Dowy Leadbetter's Regiment. He enlisted in November 1781 in Brunswick County, Virginia, for four months in Capt. Edmund Wilkins' Company. 
Pension: S31313 (NC:Va); Madison County, Illinois pension roll June 19, 1833, age 74 
Marker: His name is on a bronze tablet at the Madison County Court House, Edwardsville, placed by Ninian Edwards Chapter DAR, Alton , September 16, 1912. 
Sources: DAR, PI, PENSION, W 
RANDLE, RICHARD 
Born: August 21, 1752 in Brunswick County, Virginia
Died: April 14, 1842
Buried: Goshen, Madison County, Illinois
Spouse: Polly Rufty
Children:: Richard and Isham Randle were probably brothers.
Service: Private; Virginia Continental: North Carolina. He enlisted in Brunswick 
County, Virginia in 1777 for six weeks in Capt. John Macklin's Company, Col. Charles Harrison's Virginia line; he enlisted in August 1780 for nine months in the Company of Capt. James Allen and Capt. West Harris in the North Carolina troops. 
Pension: S32464 (NC:Va); Madison County, Illinois pension roll, February 10, 1834, age 80; Madison County, Illinois pension census, June 1, 1840, age 88 
Marker: His name is on a bronze tablet at the Madison County Court House, Edwardsville, placed by Ninian Edwards Chapter DAR, Alton, September 16, 1912. 
Sources: DAR, PI, PENSION, W 
RANDLEMAN, MARTIN 
Born: December 25, 1761 in South Carolina
Died: 1846
Buried: St. Clair County, Illinois
Spouses:  (1) Mary Fur 
	  (2) Mary Holcomb 
Residences: He came to Illinois in 1801 and a year later settled in Belleville, St. Clair County.
Service: Private; Wagoneer: North Carolina 
Pension: S31946 (NC); Illinois pension roll, St. Clair County, April 9, 1833, age 72
Sources: PI, PENSION, W 
RATIAN (Rotten), JOHN
Born: 1747
Died: October 11, 1821
Buried: Vaughn Cemetery, Wood River, Madison County, Illinois 
Spouse: Mary Green
Service: Soldier; North Carolina. He served in Capt. Robert Porter's Company, North Carolina troops from Tryon County from November 18, 1777 to December 30, 1777.
Marker: His grave is marked. 
Sources: DAR, HS, NSDAR, PI 
RAWLINGS, NATHAN 
Born: October 1750 in Pennsylvania 
Died: May 10, 1821 
Buried: Spring Hill Cemetery, Bridgeport, Lawrence County, Illinois 
Spouse: Mary Rankin 
Residences: He came to Lawrence County, Illinois in 1816. 
Service: Captain: Pennsylvania. He served as Private in Capt. Andrew Swearingham's Company, Third Battalion, Washington County, Pennsylvania Militia, enlisting May 18, 1782. He was a Ranger on the Frontier in Capt. Eleazer Williamson's Company, Westmoreland County from 1778 to 1783. 
Pension: S4041 (Pa . Agency)
Marker: His name is on a bronze tablet in the Lawrence County Court House, Lawrenceville, placed by Toussaint du Bois Chapter DAR in 1921. 
Sources: DAR, HR, PI, PENSION 
RAYBURN, ROBERT
Born: October 23,1761, Augusta County, Virginia
Died: August 16, 1836 
Buried: Irish Grove Cemetery, Middletown (on line between Logan County and Menard County, Illinois) 
Spouses:  (1) Jane Logan , married 1790, died 1802, had five Children: 
	(2) Mildred (Millie) Brown, died 1823, had one child 
Children: (first marriage) Milton, Joseph H., Francis, David L. , Jane Barnett 
	(second marriage) Panthea Brown Barnett 
Residences: After the war he lived in Adair County, Kentucky. He came to Illi­nois in 1827 and located on the present site of Middletown, building a small log cabin. In two or three years he sent for his children Milton , Joseph and the girls, who met their father at Shawneetown. 
Service: Private; Virginia. He served two and one half months in Capt. William Kincaid's Company, commanded by Lt . Robert Campbell; served from Jan­uary 10 to April 18, 1781 with the following officers: Capt. John Givins, Capt. Sampson Mathews, Col. Thomas Rankins, and Capt. William Boyer. He was engaged in battles near Portsmouth, Va. and Jamestown. 

Pension: S32463 (Va)
Marker: His grave was marked by Pierre Menard Chapter DAR, Petersburg, on September 15, 1935.
Sources: DAR, HR, NSDAR, PI, PENSION, Obituary 
RECTOR, JOHN
Died: May 25, 1805
Buried: Rector Township, Saline County, Illinois 
Service: Sergeant: Virginia. He enlisted April 5, 1778 for three years in Capt. Cocke's Company, in Col. George Rogers Clark's Regiment. John Rector was a Government Surveyor, engaged in surveying the township lines in a part of the Northwest Territory. A band of Black Hawk Indians shot him as he was letting his horse drink in the old Clay Ford, four miles north of Eldor­ado. He rode back to camp, but died within moments. His companions buried him on the spot, under a cedar tree, and then, methodically set down in their records, a surveyor's technical description of the location of the grave. 
Marker: Near Harrisburg there is a monument marking the grave of John Rector. 
Sources: "Saline County, A Century of History" 
REDMAN (Redmon), GEORGE
Born: March 24,1757 in Rowan County, North Carolina
Died:  April 5, 1837
Buried: Cemetery, Shelly Green farm , south of Paris, Edgar County, Illinois
Spouse: Henrietta
Service: Private; Wagoneer: North Carolina: South Carolina. He enlisted from 
Rowan County, North Carolina, serving as a Wagoneer. He was in General John Greene's campaign. 
Pension: S32473 (NC:SC); Pension Roll, Edgar County, August 22, 1833, age 77
Sources: PI , PENSION, W 
REDMAN (Redmon), SAMUEL
Died:  After 1830
Buried: St. Clair County, Illinois 
Residences: He came to St. Clair County from Rockingham County, Virginia. He was listed in the St. Clair County, Illinois Census of 1818, 1820, and 1830.
Service: Soldier; Virginia 
Pension: R8643 (Va): 
Residence Belleville, St. Clair County, when pension claim rejected (Act June 7,1832) "not having six months service."
Sources: CR, PENSION, W 
REED, CHARLES
Buried: Calhoun Cemetery, Calhoun, Richland County, Illinois. (Richland formed 
from Lawrence in 1841)
Service: Soldier; New Jersey 
Pension: R8652; Residence Lawrenceville, Lawrence County, Illinois when pension claim was rejected (Act June 7, 1832) " not having six months service."
Marker: His name is on a bronze tablet at the Lawrence County Court House, Lawrenceville, placed by Toussaint du Bois Chapter DAR in 1921.
Sources: DAR, PENSION, W 
REED (Read), JACOB
Born: About 1765Died:  August 1, 1858, age 93 
Buried: Pleasant Hill Cemetery, Kansas, Edgar County, Illinois; Government Headstone
Service: Soldier; Massachusetts 
Sources: DAR, HR 
REED (Read), MATTHEW
Buried: Princeville Cemetery, Peoria County, Illinois
Marker: The grave has been marked by Peoria Chapter DAR, Peoria. 
Sources: DAR 
REVIS, HARRIS
Born: 1750, Northampton County, North Carolina
Died:  1837
Buried: Wright Cemetery, Montgomery County, Illinois
Residences: He came to Illinois with his brother Henry, who is buried in Madi­son County. Harris Revis settled in Montgomery County; was a County Commissioner. 
Service: Soldier; North Carolina. He enlisted in 1780 from Salisbury, Rowan County, North Carolina under Sergeant Elias Langham. He was stationed at the Magazine, where he remained until the close of the war. 
Sources: W 
REVIS, HENRY
Born: August 11, 1752 in Northampton County, North Carolina
Died:  After 1834
Buried: Near Collinsville, Madison County, Illinois
Residences: He came to Illinois with his brother Harris who settled in Montgomery County.
Service: Soldier; North Carolina. He enlisted in the fall of 1775 from Surry County, North Carolina in Capt. Jacob Free 's Company; he later enlisted in Capt. William Neville's Company, Col. Martin Armstrong's Regiment, serving for one year. 
Pension: S32475 NC; Madison County, Illinois pension roll January 29, 1834, age 80 
Marker: His name is on a bronze tablet in the Madison County Court House, Edwardsville, placed by Ninian Edwards Chapter DAR, Alton , September 16,1912. 
Sources : DAR, PENS10N, W 
RHODES, DANIEL 
Born: 1750
Died:  June 14, 1830
Buried: Ogden Cemetery, five miles south of Paris, Edgar County, Illinois
Spouse: Lydia Rhodes
Service: Minute Man; Massachusetts. He enlisted April 19, 1775 as a "Minute Man" in Capt. Samuel Payson's Company, Col. John Creaton's Regiment for eight days; three months in Col. Joseph Read's Regiment; in September 1776 in a Battalion stationed at Hull. 
Pension: Lydia Rhoads WI3850 (Mass) 
Sources: HR, NSDAR , PI, PENSION, W 
RICHARDS, JOSEPH 
Buried: Near Kaskaskia, Randolph County, Illinois 
Service: Soldier; Virginia. He served under Capt. Francis Charleville and Col. George Rogers Clark. He also rendered service by purchasing Treasury notes to aid in prosecuting the war. 
Marker: His name is on a bronze marker on the grounds of Sparta High School, Sparta, placed by Fort Chartres Chapter DAR in 1934. 
Sources: DAR, NSDAR, W 
RICHARDS, WILLIAM H. Died: January 19, 1842
Buried: Richland Cemetery, Olney, Richland County, Illinois; Private Headstone
Sources: HR 
RICHARDSON, JAMES 
Born: August 25, 1757 in Middlesex County, Virginia
Died:  March 28, 1842
Buried: McCord Cemetery, Irving, Montgomery County, Illinois; Private Head­stone
Spouse: Jemima
Service: Private; North Carolina Continental: Virginia. He enlisted in August 1780 in Capt. Lemuel Smith's Company, Col. Peter Perkins' Regiment, Virginia troops. He also served in Capt. Miner Smith's Company, General Grif­fith Rutherford's command, North Carolina troops. He was in the battles of Brick House and Georgetown. 
Pension: S31323 (NC :Va); Illinois pension roll, Montgomery County, February 28, 1833, age 76; Illinois Pension Census, Montgomery County, June 1, 1840, age 82, residing with James M. Rutledge, head of household 
Sources: DAR, HR , PI, PENSION, W 
RICHARDSON, JOSHUA A. 
Born: December 19, 1762 in Bedford County, Virginia 
Died:  March 14, 1844
Buried: Union Cemetery. Palmyra, Macoupin County, Illinois; Private Headstone 
Spouses:  (1) Mary Snow
	(2) Mary Burnett 
Service: Private; Virginia 
Pension: Mary W3603 (Va); Pension Census, Macoupin County, Illinois, June 1, 1840, age 79
Sources: HR, NSDAR, PI, PENSION, W 
RIGGS, CHARLES
Died: February 24, 1839
Buried: Sand Hill Cemetery, Mt. Carmel, Wabash County, Illinois
Service: Corporal; Maryland. He served in Major Henry Gaither's First Maryland Regiment.
Sources: HR 
RIGGS, HOSEA
Born: 1760 in Virginia
Died:  October 29, 1841
Buried: Cemetery, two miles east of Belleville, St. Clair County, Illinois
Residences: He came to Illinois in 1796, settling in the American Bottom, Mon­roe County; he later moved to St. Clair County and lived two miles east of Belleville, He was an exhorter in the Methodist Church and was the first minister of that denomination in the County. 
Service: Private; Pennsylvania Continental line 
Pension: S32483 (Pa): Illinois pension roll, St. Clair County, February 9, 1834, age 72; Illinois 'Pension Cen sus, St. Clair County, June 1, 1840, age 80
Sources: PENSION, W 
RITCHEY (Richey), JOHN
Born: 1765/6 Amelia County, Virginia
Died: After 1840
Buried: Probably Fulton County, Illinois
Residences: He resided in Virginia, and Maury and Marshall Counties, Tennessee.
Service: Soldier; Virginia. He was drafted when sixteen years of age at Lunnen­burg, Virginia and served three months in Capt. Benjamin Biggs' Company. 
Pension: R8780 (Va). Applied for pension July 24, 1837 at Marshall County, Tennessee, age 71. Pension rejected (Act June 7, 1832) when he was a resi­dent of Fulton County, Illinois, "not having six months service." He was listed in the Pension Census, June 1,1840, Fulton County.
Sources: PENSION, W, 
ROACH, FRANCIS 
Born: 1739 in Fairfax County, Virginia 
Died: 1845, age 106 
Buried: Hamel, Madison County, Illinois
Service: Soldier; North Carolina. He enlisted in April 1776 from Dobbs County, North Carolina in Capt. Joseph Session's Company, in a Regiment com­manded by Col. Richard Caswell and Col. John Bryant; enlisted in 1781 for three months in Capt. John Doughty's Company; in 1782 for two months with Col. George Rogers Clark. After the Revolutionary War, he enlisted in 1786 in Capt. John Doughty's Company, Col. Benjamin Logan's Regiment of the Militia. 
Pension: S32494 (NC); Illinois Pension Census, Madison County, Illinois, June 1, 1840, age 101, residing with David Roach, head of family. 
Marker: His name is on a bronze tablet at the Madison County Court House, Edwardsville, placed by Ninian Edwards Chapter DAR, Alton on October 16, 1912. 
Sources: DAR, PENSION, W 
ROARK (Rourke), WILUAM
Born: June 6, 1760in New Jersey
Died: March 4, 1841
Buried: Old Cottage Grove Cemetery, six miles from Harrisburg, Saline County, Illinois
Residences: Coming to Illinois, he settled in Gallatin County, from which Saline County was formed in 1847. 
Service: Private; Teamster: New Jersey: Pennsylvania. He enlisted four times, serving eight months until 1782 in Companies of Capt. John Fleet, Capt. Mark Thompson, Capt. John Maxfield and Capt. Michael Catt. He served under General George Rogers Clark, was taken prisoner and held in Detroit and Canada. He was paroled in 1783. 
Pension: S32495 (NJ:Pa); Illinois Pension Census, Gallatin County, June I, 1840, age 78, residing with Michael Roark, head of amily . 
Marker: His name is on a marker on the Court House lawn at Harrisburg, Saline County, placed by Michael Hillegas Chapter DAR in 1931. 
Sources: DAR, NSDAR, PI, PENSION, W 
ROBERT, ABRAHAM 
Born: November 11, 1768 in Vermont 
Died: March 10, 1857 
Buried: Twelve Mile Grove Cemetery, near Rockford, Winnebago County, Illinois 
Service: Soldier; New York. He served in the Company of Capt. James Connor. 
Marker: Abraham Roberts name is on a monument in Vermont honoring the state's "Green Mountain Boys" who fought in the Battle of Ticonderoga.
Sources: DAR 
ROBERTS, DAVID
Born: December 17,1760 at Williamstown, Massachusetts 
Died: May 5, 1847 
Buried: Near Pleasant Hills, Pike County, Illinois 
Spouse: (2) Lucy 
Children: David, Jr. 
Residences: At the close of the war he was granted land in Chittenden County, Vermont and resided there until 1813 when he removed to Delaware County, Ohio, and from there, in 1839, to Pike County, Illinois, where he resided with his oldest son, David, Jr. 
Service: Sergeant; Vermont. He enlisted early in 1777 at Pownal, Vermont under five different Captains, as Private, Corporal and Sergeant. He was discharged in 1781. 
Pension: Lucy W2353 (Vt) 
Sources: PI, PENSION, County History-Letter 
ROBERTS, EBENEZER 
Born : In Maine
Died: Probably Rock Island County, Illinois
Residences: He removed to Dearborn County, Indiana, and from there to Illinois in 1840. He was a blacksmith.
Service: Soldier; Connecticut Continental. He was wounded in battle. 
Pension: S18575(Conn)
Sources: PENSION, "Rock Island County History" 
ROBERTS, JOHN 
Buried: Near Kaskaskia, Randolph County, Illinois
Service: Lieutenant; Virginia. He served in Col. George Rogers Clark's Illinois Regiment, 1779-1782, and rendered-service' by purchasing Treasury Notes to aid in prosecuting the war.
Marker: His name is on a bronze marker on the grounds of Sparta High School, Sparta, Randolph County, placed by Fort Chartres Chapter DAR in 1934, 
Source: DAR, NSDAR ,W 
ROBERTS, THOMAS 
Buried: Morgan County, Illinois
Residences: After the war he removed to Tennessee and from there to Morgan 
County, Illinois
Service: Soldier; Virginia, After the war he served in the United States Rangers.
Marker: His grave has a DAR marker.
Sources: DAR, W 
ROBERTSON, JOHN
Born: 1755
Buried: On a farm near Orleans, Morgan County, Illinois
Service: Soldier; Delaware
Marker: His name is on a plaque in front of the Morgan County Court House, Jacksonville, placed by the Reverend James Caldwell Chapter DAR in 1914.
Sources: DAR, HR , W 
ROBERTSON, ZACHARUUI
Born: May 6, 1760 
Died: January 16, 1839
Buried: Rose Cemetery, Bismark, Vermilion County, Illinois
Spouses:  (1) Mary 
	(2) Elizabeth Jones
Residences: After the war he removed to Harrison County, Kentucky, and in 1834 came to Illinois, settling in Newell Township, Vermilion County.
Service: Private; Maryland: Virginia 
Pension: S35632 (Md) BLWT 1055-100
Marker: His name is on a plaque in front of the Post Office, Danville, Vermilion County, placed by Governor Bradford Chapter DAR in 1915.
Sources: DAR, HR, PI, PENSION, W 
ROBINSON, DAVID
Died: 1833
Buried: Waterloo, Monroe County, Illinois
Service: He served in the Revolutionary War and the War of 1812.
Sources: NSDAR 
ROBINSON, JAMES
Born: January 10, 1760 in Pennsylvania
Died: September 3, 1834
Buried: Lawrenceville City Cemetery, Lawrenceville, Lawrence County, Illinois; 
Private Headstone
Spouses:  (1) Martha Boyce 
	(2) Hannah Chapman Allender 
Service: Private; Pennsylvania. He enlisted in 1776 and served in Companies of Capt. Andrew Kilbreth and Capt. James Waugh, commanded by Col. Fred­erick Watts and Col. Anthony Wayne, in the Pennsylvania line. He enlisted in 1778 in Capt. James Lord's Company; in 1779 he served ten months in Capt. John Rowan and Capt. Michael Simpson's Company. He was in the battle of Trenton and with the Six Nations of Indians. 
Pension: R8901 (Pa), Residence Lawrenceville, Lawrence County, when pension claim rejected (Act June 7,1832) "For further proof." 
Marker: His name is on a bronze tablet at the Lawrence County Court House, Lawrenceville, placed by Toussaint du Bois Chapter DAR in 1921.
Sources: DAR, HR, PI, PENSION, W 
ROBINSON, JOHN
Born: In 1750 in South Carolina
Died: July 20, 1835 
Buried: Williamson County, Illinois 
Residences: He lived in Franklin County, Illinois before moving to Williamson County. 
Service: Private; North Carolina Continental. He enlisted in 1776 and served three months in Capt. John Lyles' Company, Col. James Lyles' Regiment; three months in 1777 with the same officers; three months in 1781 in Capt. Jeremiah Williams" Company, Col. Samuel Hammond' s Regiment. He was in several engagements with the Cherokee Indians. 

Pension: Illinois pension roll, Franklin County, April 12, 1834, age 83. Private; NC Continental troops 
Sources: PE NSION, W 
ROBINSON, JOHN 
Born: About 1763, probably North Carolina 
Buried: Macoupin County, Illinois
Residences: He first settled in Morgan County, Illinois, but was in Macoupin County at the time of his death.
Service: Private; Musician: North Carolina. He served from North Carolina in Capt. Thomas Evans' Company of the Tenth Regiment as a musician. He enlisted in 1782 and served eighteen months. 
Pension: Illinois pension roll, Madison County, August 22, 1833, age 70; Private, NC Continental troops. Removed to Macoupin County, Illinois. 
Sources: CR, DAR, PENSION, W 
ROGERS (Rodgers), JOHN 
Buried: Lawrence County, Illinois 
Service: Soldier; Virginia. He remained in the service after the close of the Revolutionary War.
Marker: His name is on a bronze tablet at the Lawrence County Court House, Lawrenceville, placed by Toussaint du Bois Chapter DAR in 1921. 
Sources: DAR, W 
ROGERS, PETER 
Born: July 1, 1753/4 in New London, Connecticut 
Died: November 5, 1849 
Buried: Waterloo, Monroe County, Illinois 
Spouses:	(1) Nancy Greene 
(2) Abigail Darrow 
Residences: He settled at Waterloo, Monroe County. He took an active part in the 1840 campaign, making speeches for his candidate. He was then 87. 
Service: Fife Major; Connecticut. He enlisted in 1775, serving until a short time before the close of the war as a "Fife Major." He was with Capt. William Coit in a cruise on an armed schooner, when they captured a sloop and a schooner. He was in the battles of Germantown and Monmouth. He served in Capt. Caleb Gibbs' Company, in General Washington's Life Guards, com­manded by Col. John Drake. 
Pension: S33582 (Conn); Pension Census, Monroe County, Illinois, June 1, 1840, age 86 
Sources: DAR, PI, PENSION, W 
ROGERS, WILLIAM
Buried: Franklin County, Illinois 
Service: Soldier; Virginia 
Pension: R17514 (Va); BLWT 2369-300 Va Half Pay. Residence, Frankfort, Frank­lin County, Illinois when pension claim rejected (Act June 7, 1832) "not six months service." 
Sources: PENSION, W 
ROPER, GEORGE 
Born: 1765 in North Carolina 
Died: February 28, 1845 
Buried: Walnut Hill Cemetery, Walnut Hill, Marion County, Illinois 
Residences: He evidently lived in Tennessee, moving to Clinton County, Illinois, from there to Jefferson County, and finally to Marion County. 
Service: Private; North Carolina Continental troops. He served from North Caro­lina in Capt. Anthony Sharp's Company, in the Tenth Regiment , and was discharged April 15, 1782. 
Pension: S36261(NC); Pension roll, March 11, 1825, transferred from West Tennes­see. By Resolution of Congress, May 29, 1830, pension claim rejected "on account of amount of property." Pension Census, Jefferson County, Illinois, June 1, 1840, age 78. 
Sources: HR, PI, PENSION, W 
ROSE, RICHARD
Born: 1754
Died: February 14, 1842 
Buried: Mt. Pulaski Cemetery, Augusta, Hancock County, Illinois 
Spouse: Mary 
Service: Private; Virginia Continental troops 
Pension: Pension roll, May 14, 1828, transferred from Indiana and removed to Schuyler County, Illinois; Pension Census, Adams County, Illinois, June 1, 1840, age 81, resident Third Ward, Quincy. Mary Wl1187 (Va) BLWT 75111­160-55 
Marker: A new marble monument was dedicated May 14, 1960 by SAR and Shadrach Bond Chapter DAR. His name is on a tablet in the Hancock County Court House, Carthage, placed by Shadrach Bond Chapter DAR, July 10, 1910. 
Sources: DAR, HR, NSDAR, PI, PENSION, W 
ROSS, REUBEN 
Born: 1756 in Harvard, Maryland 
Died: After 1840
Buried: Rogers Cemetery, near Waverly, Morgan County, Illinois 
Residences: He removed to North Carolina, and in 1829 came to Illinois settling in South Palmyra Township, Section 8, Morgan County. That area is now in Macoupin County. 
Service: Soldier; Maryland Continental. He enlisted July 30, 1776 in Capt. Alex­ander Lawson Smith's Company, in Col. Moses Rawling's Fourth Regiment, commanded by Col. J. C. Hall. 
Pension: S40361 (Md): Morgan County, Illinois Pension Census, June 1, 1840, age 78; Residence Morgan County when pension claim rejected (Act June 7, 1832) "For further description of service." 
Marker: The grave is marked. 
Sources: DAR, PENSION, W 
ROWE, HEZEKIAH
Born: June 17, 1759 in North Carolina
Died: 1835
Buried: Allen Cemetery, near Coffeen, Montgomery County, Illinois; also reported Sugg Cemetery, Bond County
Residences: He settled in Bond County, Illinois; Montgomery County was formed in 1821.
Service: Private; South Carolina Continental troops 
Pension: S32496 (NC:SC); Resident of Bond County, Illinois on pension roll of April 16, 1834, age 74
Marker: The grave was marked by Benjamin Mills Chapter DAR, Greenville.
Sources: DAR, PENSION, W 
ROWELL, DANIEL
Born: About 1760
Died:  After 1840
Buried: Burt Cemetery, west of Paris, Edgar County, Illinois
Service: Private; New Hampshire: Connecticut. He enlisted in 1777 from Connec­ticut in Capt. Jonathan Humphrey's Company, Col. Samuel McClelland's Regiment. 
Pension: S40364 (NH); Pension Census, Edgar County, Illinois, June 1, 1840, age 80, residing with James Wilson, head of family.
Sources: PENSION, W 
ROWLAND, SAMUEL 
Died: April 10, 1836 
Buried: Lawrence County, Illinois 
Sources: Letter from descendant 
ROWLEY, ABIJAH
Born: April 27, 1758 in Massachusetts
Died:  July 23, 1850
Buried: John Barrett's farm cemetery, Ipava, Fulton County, Illinois; Private Headstone
Spouse: Elizabeth Culver
Children: Elizabeth Dickenson, Cloe Cole, Lydia Johnson, Maria Lively, Cornelia Hall, Samuel
Residences: Cattaraugus County, New York; transferred to Prebie County, Ohio; to Fulton County, Illinois. Occupation: blacksmith.
Service: Private; Artificer: Connecticut Continental: Massachusetts. He served in Capt. Osborn 's Company, Col. Jeduthon Baldwin's Regiment in the Massachusetts line from 1777 to 1780. He served in the Tenth Company, Fourth Connecticut Regiment. 
Pension: S32493 (Conn) Transferred from Prebie County, Ohio to Fulton County, Illinois, to live with
Marker: His grave has a DAR marker.
Sources: DAB, HR, NSDAR, PI, PENSION 
ROWLEY, REUBEN C.
Born: About 1752
Died: After 1840
Buried: Pleasant Township Cemetery, Ipava, Fulton County, Illinois
Spouse: Susannah
Service: Soldier; New York: Vermont Continental. He is believed to be the Reuben Rowley who served from Albany, New York in the Fourth Regiment commanded by Col. Kilian Van Rensselear. 
Pension: Resident of Town ship 6N ;5E, Fulton County, Illinois, Pension Census June 1, 1840, age 88. Susannah W2859 (Vt)
Sources: PENSION, W 
ROYAL, THOMAS, JR.
Born: March 27, 1754 in Manchester, England
Died:  August 1834 
Buried: Joe Brunk Cemetery, near Glenarm, Sangamon County, Illinois 
Spouses:  (1) Hannah Cooper
	  (2) Rebecca Matthews 
	  (3) Mrs. Ellen Brunk
Residences: At the close of the war he removed to Ohio, and from there to Ball Township, Sangamon County, Illinois.
Service: Private; Pennsylvania. He served from Philadelphia, Pennsylvania. 
Pension: He was pensioned.
Marker: His name is on a bronze plaque in the south mall, Old State Capitol, Springfield, placed by Springfield Chapters DAR and SAR, October 19, 1911.
Sources: DAR, PI, W 
RUSSELL, PHILIP T.
Born: November 14, 1765 
Died: August 17, 1842 
Buried: Wilson Cemetery, west of Cambria, Williamson County, Illinois 
Married:  (1) January 6, 1792, Pittsylvania County, Virginia 
Spouses:  (1) Elizabeth Stuart, born August 31, 1770; died May 12, 1828 
	  (2) Mary Williams 
Residences: He removed to Tennessee, and in 1817 came to Franklin County, Illinois, living in Williamson County at the time of his death. Williamson County was formed from Franklin County in 1839. 
Service: Wagon Master; Private; Virginia. He served in the Virginia Continental troops and was present when Cornwallis surrendered. 
Pension: Pension roll, Franklin County, December 17,1833, age 69. Mary W2575: BLWT 26883-160-55 (Va)
Sources: HR, NSDAR, PI, PENSION, W 
RUTHERFORD, LARKIN
Buried: North of Belleville, St. Clair County, Illinois
Residences: He came to Illinois in 1800, settling north of Belleville, St. Clair County .
Service: Soldier; Virginia. He was one of Col. George Rogers Clark's soldiers; was at the storming of Fort Sackville in 1779.
Sources: W 
RUTLEDGE, JOHN
Buried: McCord Cemetery, Montgomery County, Illinois
Service: Soldier; Virginia. He enlisted July 12, 1781 from Botetourt County, Virginia under U . H. Waterson.
Sources: HS 
RYAN, JAMES
Born: In Virginia
Died:  After 1840
Buried: Bright Cemetery, Greenup Township, Cumberland County 
Residences: He came to Illinois and settled in Coles County. Cumberland County was created from Coles County in 1843. 
Service: Private; Virginia: North Carolina Continental troops. He enlisted February 28, 1777 in Capt. James Calderwood's Company, in the Eleventh and Fifteenth Virginia troops, commanded by Col. Daniel Morgan. 
Pension: S32501 (NC); Illinois pension roll, Crawford County, August 13, 1833, age 72; Illinois Pension Census, Coles County, June 1, 1840, age 83. 
Sources: PENSION, W 

SADORUS, HENRY, SR.
Died: July 18, 18__ 
Buried: Sadorus Cemetery, Sadorus, Champaign County, Illinois 
Sources: HR 
SAMPSON, ISAAC 
Born: 1760 in Orange County, New York 
Died: August 11, 1838 
Buried: Fulton County, Illinois 
Spouse: Merrifield 
Service: Fifer; Private; New York. He enlisted May 24, 1777 in Capt. Stewart's Company, Col. Lewis Du Bois' Fifth New York Regiment. 
Pension: His pension was transferred from Ohio to Illinois, September 4, 1835. 
Sources: DAR, PI, PENSION 
SAUCIER, JEAN BAPTISTE 
Buried: Cahokia, St. Clair County, Illinois 
Children: Jean Baptiste Saucier, son of the soldier, was one of the first Judges in Cahokia. Francis Saucier was also a son. 
Residences: The home acquired by Jean Baptiste Saucier in Cahokia is believed to have been built about 1737. It is the oldest house in Illinois, and perhaps the oldest in the Midwest. It is constructed of logs, French style, with the logs set upright, as opposed to frontier cabins which had horizontal logs. The house had four rooms, casement windows with glass panes and exterior shutters; the interior walls were plastered. In 1793 Francis Saucier, son of Jean Baptiste, sold the log house to St. Clair County for a court house and jail. It was here that the first court sessions and the first election in the Illinois country were held. The house has been completely restored by the state. 
Service: Military Engineer. He planned Fort de Chartres in 1752 and later moved to Cahokia. 
Sources: W 
SAUNDERS, GEORGE 
Born: About 1748 
Died: May 21, 1820, age 72 
Buried: On a farm at Orleans, Morgan County, Illinois 
Spouse: Elizabeth
Service: Soldier; Virginia Continental troops and remained in service after the war. He also rendered Patriotic Service in South Carolina.
Pension: Elizabeth W5986 (Va)
Sources: PI , PE NSION, W 
SAWINE, SAMUEL 
Buried: Probably Kane County, Illinois
Service:  Soldier; Massachusetts. He served in the Massachusetts troops from September 4 to September 11, 1778 in Capt. John Walter's Company.
Pension: He applied for a pension from Kane County (two pensioned by same name). 
Sources: PENSION, W 
SCARBOROUGH, JOHN 
Born: April 1762 in Virginia 
Died: August 15, 1846 
Buried: Franklin Cemetery (Lot 24), Benton, Franklin County, Illinois 
Residences: After the war he removed to Posey County, Indiana, and from there to Franklin County, Illinois. Franklin County was formed from White County in 1818. 
Service: Private; South Carolina Continentals. He enlisted in 1780, serving until April 4, 1781 in Companies of Capt. Samuel Selden and Capt. James Green; he again enlisted in Capt. John Hughes' Company, Col. Anthony White's Regiment, serving until June 1783,whenhe was discharged. 
Pension: Transferred from Posey County, Indiana to Franklin County, Illinois; Pension Census, Franklin County, Illinois, age 79, June 1, 1840 
Marker: His name is on a monument honoring Revolutionary soldiers buried in White County, placed by Wabash Chapter DAR, Carmi, in 1936. 
Sources: DAR, HR, PENSION, W 
SCHOLL, ABRAHAM 
Born: December 15,1765 in Rowan County, North Carolina Di ed: 1851/2
Buried:  Griggsville Cemetery, Griggsville, Pike County, Illinois; Private Head­stone 
Spouses: (1) Nellie Humble
	 (2) Tabitha Noe
Service:  Private; Virginia. He enlisted in 1781 in Fayette County, Virginia, serv­ing several short terms in Companies of Capt. William Hays, Capt. John Constant, Capt. Charles Hazelrigg, and Capt. John McOowell, in Regiments of Col. John Todd, Col. Benjamin Logan, Col. Daniel Boone and Col. Trotter. He was in the skirmish at Bryant's Station. 
Pension: R9265 (Va), Illinois pension roll, Pike County, residence Atlas, when pension claim rejected (Act June 7, 1832) "not having six months service."
Marker: A marker was placed by Nancy Ross Chapter DAR, On July 4, 1935.
Sources: DAR, HR, NSDAR, PI, PENSION, W 
SCOTT, JOHN 
Born: May 29, 1763 in York County, Pennsylvania
Died:   March (or November) 13, 1847
Buried:  Rock Creek Cemetery, near Waynesville, DeWitt County, Illinois 
Spouses: (1) Ann Cravtin 
(2) Nancy Keith 
Residences: The Scott family moved to Sangamon County, Illinois in 1824, was living in McLean County in 1832, and had moved to DeWitt County by 1840. 
Service: Private; Virginia Continental troops. He enlisted in May 1780 from Washington County, Virginia in Capt. James Dysart's Company, Col. William Gamble's Regiment, serving one year. He was in the battles of King's Moun­tain and Wetzell's Mills. 
Pension: S32509 (Va): Illinois pension roll, McLean County, September 25, 1833, age 71; Illinois Pension Census, DeWitt County, June1, 1840, age 77, resid­ing with John Maxwell, head of family 
Marker: His grave was marked by DeWitt Clinton Chapter DAR and descendants of John Scott on November 8,1970. 
Sources: DAR , HR , NSDAR, PI, PENSION, W 
SCOTT, WILLIAM 
Born: May 17,1745 Augusta County, Virginia
Died: 1828
Buried:  Shiloh Valley, St. Clair County, Illinois
Spouse:  Mary Scott, born 1748 in Augusta County, Virginia
Children:  Elizabeth (Jarvis), James, Samuel, William, Joseph, John, Alexander
Service:  Ensign and Lieutenant, Virginia Militia. He was appointed Ensign in Capt. Baird's Company of Militia on September 13,1781 and was a Lieute­nant on September 11, 1782. He served from Botetourt County, Virginia, formed from Augusta County in 1769. He probably also served from Chester­field County, Virginia. 
Sources: DAR, History of St. Clair County, Illinois 
SCOTT, WILLIAM 
Born: 1755 in Virginia
Died: October 4, 1836
Buried: On Drury farm, near Orleans, Morgan County, Illinois (probably removed to Scott Cemetery, Jacksonville); Government Headstone
Spouse:  Silvey R. 
Service: Private; North Carolina Continental troops and Virginia Militia 
Pension: Illinois pension roll, Morgan County, April 16, 1833, Private, NC Con­tinental troops. Silvey  R., R9314 (Va), Application of widow, Silvey R. Scott was rejected while resident of Jacksonville, Morgan County (Act July 7, 1838) "Six months service allowed -marriage in suspense. " 
Marker: His name is on a plaque in front of the Morgan County Court House, Jacksonville, placed by the Reverend James Caldwell Chapter DAR, in 1914. 
Sources: DAR, HR, NSDAR, PENSION, W 
SCOTT, WILLIAM
Buried:  Probably Union County, Illinois 
Residences: He carne to Illinois, settling in Union County. 
Service: New York. He served in the New York line of troops, enlisting at age 16. 
Sources: W 
SCROGGINS, HUMPHREY 
Born: 1763 in the Carolinas 
Died: July 1845 
Buried: Steenbergen Cemetery, near Mt. Pulaski, Logan County, Illinois 
Spouse: Sarah Kirby 
Service: Private; Virginia. He served in Capt. S. Tarrant's Company with Major George Waller, Col. Abram Penn's Regiment. In 1781 his Regiment was ordered from Henry County, Virginia to the assistance of General Edward Stevens and General Nathaniel Greene at the battle of Guilford Court House. Hewasalsoat the siege of Yorktown. 
Pension: R9325 (Va); Residence, Springfield, Sangamon County, when pension claim was rejected (Act June 7, 1832) "not having six months service. " 
Marker: His name is on a plaque on the Logan County Court House, Lincoln, placed by Abraham Lincoln Chapter DAR, June 27, 1975. 
Sources: DAR, PI, PENSION, W 
SCROGGINS (Scoggins), JONAH
Born: 1763 in Brunswick County, Virginia 
Died: January 25, 1845 
Buried: Carrollton, Greene County, Illinois 
Married: January 1, 1789 in Marion County, South Carolina 
Spouse: Anna Hightower, born June 10, 1764; died October 28, 1850 in Greene County 
Service: Private; Sergeant; North Carolina Continental. He enlisted in 1778 from Burke County, North Carolina in Capt. Robert Temples' Company with Major Charles Pickney. He served in Companies of Capt. Philip Taylor, Capt. Philip Thomas and Capt. John Whitley, with Major William Dennis and Major Robert Rayford. 
Pension: Pension roll, June 6, 1833, Greene County, age 70; Pension Census, Greene County, Illinois, June 1, 1840, age 77, residing with R. H. Scroggins head of family.  Ann W24913 (NC). Ann, widow of Jonah Scroggins was resi­dent of Carrollton, Greene County, when pension claim rejected (Act July 7, 1838)"Not being a widow at date of Act." She was later pensioned. 
Sources: PENSION, W 
SEAGRAYES, JACOB
Born: About 1763 
Died: June 7, 1835
Buried: Seagrave Cemetery, Clinton County, Illinois 
Spouse: (2) Alinair 
Residences: After the war he removed to Tennessee and from there to Clinton County, Illinois. 
Service: Private; North Carolina Continental. He enlisted in 1778 from Granville County, North Carolina, serving two and one-half years in Capt. Joseph Rhodes' Company, Col. Dixon's Regiment. He was in the battle of Eutaw Springs. 
Pension: S39067 (NC); Pension roll, Clinton County, June 18, 1822, transferred from West Tennessee. He was also listed in the 1830 Census for Clinton County. 
Sources: CR, PI, PENSION, W 
SELLERS, HOWELL 
Born: March 1762in Charlotte County, North Carolina 
Died: After October 17, 1832 
Buried: Pike County, Illinois 
Spouse: Margaret Conner 
Service: Private; South Carolina: North Carolina. He served in the North Carolina troops, and was in the battles of Stono, Brier Creek, and the siege of Savan­nah . 
Pension: S31357(SC) 
Sources: PI, PENSION, W 
SEXTON, SAMUEL 
Buried: Probably Henderson County, Illinois 
Service: Soldier; South Carolina
Pension: R9400 (SC); Residence, Oquawka, Henderson County, Illinois when pension claim rejected (Act June 7, 1832) "no proof of service-no correct statement ." 
Sources: PENSION 
SEYBOK, ROBERT 
Died: After 1840
Buried:  Madison County, Illinois Residences: He carne to Illinois in 1783, and with other settlers, took refuge in Kaskaskia, Randolph County, from the Indians. He sold his property in Randolph County in 1798. He was a resident of Madison County in the censuses of 1818, 1820, 1830, and 1840.
Service: Soldier; Virginia. He served with Col. George Rogers Clark. He rendered service by purchasing Treasury Notes to aid in prosecuting the war. 
Marker: His name is on a bronze marker on the grounds of Sparta High School, Sparta, Randolph County, placed by Fort Chartres Chapter DAR in 1934. 
Sources: CR, DAR, NSDAR, W 
SEYMOUR, JARRETT
Buried:  Providence Cemetery, near Franklin, Morgan County, Illinois
Marker:  His name is on a plaque in front of the Morgan County Court House, Jacksonville , placed by the Reverend James Caldwell Chapter DAR, 1914. 
Sources: DAR, NSDAR, W 
SHARP, WILLIAM
Born:  1762 in Maryland
Died:   After 1840
Buried:  Randolph County, Illinois
Spouse:  Elizabeth
Service:  Corporal: Maryland: North Carolina. He enlisted May 29, 1778 in the 
Fifth Regiment, Maryland; was made Corporal in October 178l.
Pension: Illinois Pension Census, Randolph County, June 1, 1840, age 78. Eliza­beth R9430 (Md:NC)
Marker:  His name is on a bronze marker on the grounds of Sparta High School, Sparta, Randolph County, placed by Fort Chartres Chapter DAR, in 1934.
Sources: DAR, NSDAR, PENSION, W 
SHAW, SAMUEL
Born:  1756 in Ireland
Died:   July 1, 1833
Buried:  Adams County, Illinois
Service:  Private; Pennsylvania Continental. He enlisted from Cumberland County, 
Pennsylvania; in 1776 for two months in Capt. John Clarke's Company, Col. Frederick Watts' Regiment; in 1777for four months in Capt. David Mitchell's Company; in 1778 for three months in Capt. William Blaine's Company, Col. Samuel Lyon's Regiment. He was in the battles of White Marsh and Gulf Mills. 
Pension: S31359 (Penn); Pension roll, Adams County, February 9, 1833, age 78 
Sources: PENSION, W 
SHEPHERD, CHARLES
Born: About 1754 in Pennsylvania
Died: After 1840
Buried: Quincy, Adams County, Illinois 
Residences: After the war he removed to New York and after 1818 to Illinois, settling in the Third Ward,  Quincy, Adams County.
Service: Soldier; Pennsylvania. He served in the Fourth Regiment, Pennsylvania Artillery from February 1777 to November 3,1783. 
Pension: S43128 (Penn); Pension Census, Adams County, Illinois, June 1, 1840, age 86, residing in Quincy City, 2nd Ward, with Benjamin Ramsey, head of family’
Sources: PENSION, W 
SHERLEY, THOMAS
Born: About 1759
Died: After 1833
Buried: Jackson County, Illinois 
Service: Private; South Carolina Continental troops 
Pension: S17083(SC); Pension roll, Jackson County, October 22,1833, age 74
Sources: PENSION, County Records 
SHIPMAN, DAVID
Born: Probably Virginia Died: After 1840 
Buried: Antioc Cemetery, Tazewell County, Illinois 
Residences: After the war he removed to Fayette County, Illinois, and from there to Tazewell County. 
Service: Private; Virginia Continental. He enlisted in 1780 from Virginia in Capt. Robert Craven's Rifle Company. He served on an alarm toward Blue Ridge, and also with his wagon, hauling for the army.
Pension: S32518 (Va); Pension Roll, Fayette County, Illinois dated April 17, 1834, age 68. Residence Tremont, Tazewell County when pension claim rejected (Act June 7, 1832) "Not six months service." Pension Census, June 1, 1840, resident of Tazewell County. 
Sources: PENSION, W 
SHORT, JOSHUA 
Born: 1752 in Virginia 
Died: May 7, 1842 
Buried: Concord Cemetery (Government Cemetery), Petersburg, Menard County, Illinois; Government Headstone
Spouse:  Parthena 
Residences: At the close of the war he came to Sangamon County, Illinois. He was one of the aged men who rode in a canoe mounted on wheels and rigged as a ship in the procession at the Whig gathering in 1840.
Service:  Private; Virginia. He enlisted in 1776, was in the Sixth Virginia Regi­ment, serving until 1778. 
Pension: S36301 (Va), Pension Census, Menard County, Illinois, June 1, 1840, age 90, residing with James Short, head of family.
Marker: His grave is marked.
Sources: DAR, HR, NSDAR, PI, PENSION, W 
SHOTO, ANTHONY D.
Born: March 6, 1754 in Madrid, Spain
Buried: Franklin County, Illinois 
Residences: He accompanied his father to New Orleans and to Illinois after the war. 
Service: Soldier; South Carolina. He enlisted in May 1780 in Capt. John Land and Capt. Middleton Isbel's Company, Col. Burnett's Regiment, South Carolina troops. He was in the battles of Rocky Mount, the siege of Ninety­ Six, Camden, and Eutaw Springs, serving two years. 
Pension: R9536 (SC); Residence Frankfort, Franklin County, Illinois when pen­sion claim (Act June 7, 1832) was suspended "for proof from South Carolina records." 
Sources: HS, PENSION 
SHULL (Sholl, Scholl), PETER 
Born: January 11, 1761 in Pennsylvania 
Died: November 22, 1834 in Henderson, Kentucky 
Buried: Burnt Prairie Cemetery, Burnt Prairie, White County, Illinois 
Spouse: Anna Dorothea 
Residences: After the war he lived in Ohio County, Kentucky; removed to White County, Illinois, where he lived with a daughter.
Service: Private; Pennsylvania. He enlisted in 1778 from Northampton County, Pennsylvania, in Capt. Adam Stohler's First Company, Third Battalion.
Col. Michael Pabst's Regiment of the Militia.
Pension: Anna Dorothea W9289 (Penn)
Marker:  His name is on a monument honoring Revolutionary War soldiers buried in White County, placed by Wabash Chapter DAR, Carmi in 1936. A marker was also dedicated in the old cemetery at Carmi by Wabash Chapter DAR on September 21, 1964. 
Sources: DAR, HR, NSDAR, PI, PENSION, W 
SIMMONS, STEPHEN
Born:  November 7, 1765 at Ashford, Connecticut
Died:   February 24, 1835
Buried:  Probably Wabash County, Illinois Residences: He came to Illinois about 1817, settling in Wabash County.
Service:  Soldier; Connecticut
Pension: S32521 (Conn)
Sources: PENSION, Family Information 
SIMPKINS, JOHN G.
Born:  1756, probably in New York
Died:   July 22, 1843 
Buried: Williamson County, Illinois 
Spouse: Margaret 
Residences: After the war he removed to Franklin County, Illinois, moved to Vigo County, Indiana, back to Franklin County. Williamson County was formed from Franklin County in 1839. 
Service: Private; New York Continental. He enlisted in New York in 1777, serv­ing until 1781 in Capt. John Rudolph's Company, Col. Henry Lee's Regiment in the Continental Dragoons. He was discharged in South Carolina. 
Pension: Pension roll, January 10, 1827, Franklin County. Margaret R9592 (NY), widow of John G. Simpkins, residence, Marion, Williamson County when pension claim rejected (Act July 7, 1838) "Barred under Act of April 30, 1844." 
Sources: PENSION, W 
SIMS (Simms), AUGUSTINE (Augustus) 
Born: May 27, 1763 in Virginia 
Died: February 10, 1851 
Buried: Rogers Cemetery, south of Waverly, Morgan County, Illinois; Private Head stone 
Spouse: Nancy Farmer 
Service: Private; Virginia Continent al. He enlisted in 1781 from Henry County, Virginia, age 19, for three months in Capt. George Hartson's Company, Col. Abraham Penn and Col. St. George Tucker's Regiment; in July 1781 he en­listed for three months in Capt. Hayman Crite's Company, Col. Richardson's Regiment. 
Pension: S32520 (Va); Illinois pension roll, Morgan County, March 26, 1833, age 71 
Marker: His name is on a plaque in front of the Morgan County Court House, Jacksonville, placed by the Reverend James Caldwell Chapter DAR in 1914. A Revolutionary Soldier marker was purchased by descendant s and given through Eulalona Chapter DAR, Klamath Falls, Oregon. The grave, lost for many years, was located by a member of Springfield Chapter. 
Sources: DAB, HR, NSDAR, PI, PENSION, W 
SIMPSON, WILLIAM 
Born: October14,1755i nPrince William County, Virginia 
Died: March 21,1839 
Buried: Simpson Cemetery, Barnhill Town ship, Wayne County, Illinois 
Spouse: Elizabeth Cheshier 
Service: Private; Virginia 
Pension: Elizabeth R9596 (Va) (deceased), widow of William, residence was Carmi, White County, when her pension claim was rejected (Act July 7, 1838) "Not a widow at date of the act - died before August 23, 1842." 
Sources: HR, PI, PENSION, W 
SIX (Saxe), JOHN 
Died: 1848
Buried: Versaille Township, Brown County, Illinois 
Residences: After the war he removed to Pennsylvania, then to Tennessee, and from there to Kentucky. He came to Illinois in 1825, lived in Pike County and Scott County, but died at the home of his son in Brown County. 
Service: Soldier; Virginia. He enlisted in the Virginia troops when sixteen years of age, taking the place of his stepfather. He was present at the surrender of Cornwallis. 
Pension: S36758 (Va) 
Sources: PENSION, W 
SKINNER, ELI
Born:  July 30, 1760
Died:   Jul y 2, 1851
Buried:  Elk Grove Cemetery, Arlington Heights, Cook County, Illinois 
Spouses: (1) Lucinda Nims
     (2) Eleanor
Service:  Private; Massachusetts: Vermont. He was a flier in Capt. Well's Com­pany, Col. Asa Whitcomb's Regiment; was in Capt. Nehemiah Lovell's Company, Bennington, Vermont.
Pension: S31366 (Mass)
Marker: His grave was marked by General Henry Dearborn Chapter DAR, Chicago, August 25, 1931.
Sources: DAR, PI, PENSION 
SWAN, THOMAS
Born: About 1761 in Pennsylvania
Died: After 1840
Buried: Wayne County, Illinois Residences: After the war he came to Illinois, settling in McLean County, but removed to Wayne County after 1833.
Service: Private; North Carolina Continental troops
Pension: S32522 (NC); Pension roll, McLean County, Illinois, December 12, 1833, age 73; Pension Census, Wayne County, Illinois, June 1, 1840, age 79
Sources: PENSION, W 
SLOCUMB, JOHN CHARLES
Buried: Slocum Cemetery, Concord, White County, Illinois Residences: Removed from Georgia in 1783 to Kentucky, and some years later to Concord, White County.
Sources: DAR 
SMART, LABAN 
Born: November 9, 1758, Franklin County, North Carolina
Died:  November 28, 1840
Buried: Pin Oak Township, Madison County, Illinois
Spouse: Susannah Simmons
Service: Private; North Carolina. He enlisted early in 1780 and served for three months in Capt. William Brickle's Company, with Col. Allen, Col. Sessions, and Col. Kinyon; in 1781 for three months in Capt. Daniel Jones' Company, Col. William Linton's Regiment. 
Pension: S31375 (NC); Madison County, Illinois pension roll, September 25, 1830, age 76
Marker: His name is on a bronze tablet in the Madison County Court House, Edwardsville, placed by Ninian Edwards Chapter DAR, September 16, 1912.
Sources: DAR, PI, PENSION, W 
SMITH, AARON 
Born: April 5, 1765
Died: February 1, 1841
Buried: White Hall City Cemetery, White Hall, Greene County, Illinois; Govern­ment Headstone
Spouse: Agnes 
Residences: He removed to Tennessee and from there to Greene County, Illinois.
Service: Private; North Carolina Continental troops. He enlisted in 1781 from North Carolina, serving thirty-four days in Companies of Capt. Elijah Moore and Capt. Michael Randolph, commanded by Col. Archibald Lytle and Col. Henry Lee. He was wounded in the battle of Eutaw Springs. 
Pension: Agnes R9681 (NC). In 1818 residence was Anderson County, Tennessee; Pension roll, January 13, 1819, transferred from East Tennessee where he had resided in Washington County. 
Marker: His name is on the Soldiers Monument at White Hall. 
Sources: HR, HS, PENSION, W 
SMITH, BENAJMIN 
Born: 1760 in Hoboth, Bristol County, Massachusetts
Died: 1841
Buried: Wabash County, Illinois, Residences: He came from Alleghany County, New York to Edwards County, Illinois in 1816, settling in Lancaster Precinct. Wabash County was formed from Edwards in 1824.
Service: Soldier; New York: Rhode Island
Pension: S32527 (RI). He was a resident of Wabash County in 1832.
Sources: PENSION, W 
SMITH, EUJAH 
Buried: Randolph County, Illinois 
Residences: Returning to Illinois after the war, he settled east of Kaskaskia, above the mouth of Nine Mile Creek.
Service: He was a soldier who served with Col. George Rogers Clark.
Marker: His name is on a bronze marker on the grounds of Sparta High School, Sparta, Randolph County, placed by Fort Chartres Chapter DAR in 1934.
Sources: DAR, NSDAR, W 
SMITH, ELIJAH
Born: August 15, 1755 in New Jersey
Died: March 11, 1835
Buried: East Cemetery, Jacksonville, Morgan County, Illinois; Private Headstone 
Spouses:  (1) Elizabeth L. Litten
	(2) Lucretia Jones Residences: His residence in 1776 was Hunterdon, New Jersey; in 1778, Washing­ton  
     County, Virginia; in 1832, Rutherford County, Tennessee; 1835, to Morgan County, Illinois.
Service:  Sergeant: New Jersey: Virginia
Pension: Lucretia WI0504 (NJ:Va) BLWT 101700-160-55; Lucretia Jones, former widow of Elijah Smith
Marker:  His name is on a plaque in front of the Morgan County Court House, Jacksonville, placed by the Reverend James Caldwell Chapter DAR in 1914.
Sources: DAR, HR, NSDAR, PI, PENSION, W 
SMITH, HENRY
Buried:  Randolph County, Illinois 
Residences: Returning to Illinois, he settled east of Kaskaskia, Randolph County, above the mouth of Nine Mile Creek.
Service: He was a soldier who served with Col. George Rogers Clark.
Marker: His name is on a bronze marker on the grounds of Sparta High School, Sparta, Randolph County, placed by Fort Chartres Chapter DAR, in 1934.
Sources: DAR, NSDAR, W 
SOMMERS, STEPHEN 
Born: 1762 in Connecticut 
Buried: Wabash County, Illinois 
Service: Private; Connecticut. He enlisted July 3, 1781 in the First Regiment, commanded by Col. John Durkee.
Pension: S32521 (Conn)
Sources: PENSION, W 
SORNBERGER, GEORGE 
Born: June 15, 1759 in New York (or Holland) 
Died: September 27, 1841 
Buried: Victoria Cemetery, Victoria, Knox County, Illinois; Government Headstone 
Spouses:  (1) Margaret Manson
(2) Catherine Woolcott
Children: Diantha
Service: Private; New York. He served from Dutchess County, in Fourth New York Militia commanded by Col. Buswell Hopkins.
Marker: A marker was placed by George Sornberger Chapter DAR, Victoria.
Sources: DAR, NSDAR, PI, W 
SOWERS, JOHN
Born: March 10, 1760 in Rowan County, North Carolina
Died: August 22, 1834
Buried: Old Wetaug Cemetery, Wetaug, Pulaski County, Illinois (near the Union County line)
Spouse:  Agnes Owens
Children: John, Jr., Fluanna (1) Reader; (2) Short Residences: He came to Adams County, Illinois in the early 1800's but removed to Union County.
Service: Private; North Carolina. He enlisted in July 1776 at the age of sixteen, serving one month; in 1781served six months in Capt. John Lop's Company.
Pension: S32532 (NC)
Marker: The grave was marked by DAR and C.A.R. descendants and Egyptian Chapter DAR. Cairo, on November 2, 1963.
Sources: DAR, NSDAR, PI, PENSION, W 
SPEARS, GEORGE
Born: August 11, 1764 in Virginia
Died: April 16, 1838
Buried: Greenwood Cemetery, Tallula, Menard County, Illinois
Service: Private; Pennsylvania. He enlisted September 9, 1778 in Capt. James Bower's Company, Col. Harmon's Regiment, Pennsylvania troops. He was a Lieutenant in the Seventh Company, Col. Allan's Kentucky Regiment in the War of 1812. 
Marker: His grave was marked by Pierre Menard Chapter DAR, Petersburg, in 1938-39.
Sources: DAR, HR, HS, NSDAR 
SPENCER, WILLIAM
Born: 1757 in Pennsylvania
Died: December 3, 1841
Buried: Klan Farm Cemetery, Lawrenceville, Lawrence County, Illinois 
Spouse: Susanna 
Service: Soldier; Pennsylvania
Marker: His name is on a bronze tablet on the Lawrence County Court House, Lawrenceville, placed by Toussaint du Bois Chapter DAR in 1921.
Sources: DAR, HR, PI, W 
SPENNEY, BENJAMIN
Died: February 23, 1853
Buried: Martinsville Cemetery, Martinsville, Clark County, Illinois
Sources: HR 
STALLINGS, ABRAM
Buried: Falling Springs Cemetery, St. Clair County, Illinois
Service: Soldier; Virginia. He served in Capt. William Reddick's Company, of the Virginia troops.
Sources: HS 
STAMM, GEORGE
Born: In Maryland
Buried: Kaskaskia , Randolph County, Illinois
Service: Private; Musician: Maryland Continental. He enlisted in May 1780 from Fredericktown, Maryland in the Company of Capt. John Smith and Capt . Christopher Orendorff, Col. John Eccleston's Sixth Maryland Regiment.
Pension: S36329 (Md). He was listed on the Randolph County pension roll of June 14, 1820.
Marker: His name is on a bronze marker on the grounds of Sparta High School, Sparta, Randolph County, placed by Fort Chartres Chapter DAR in 1934.
Sources: DAR, NSDAR, PENSION, W 
STEELE, JOHN
Born: 1737 in Virginia
Died: September 11, 1820 
Buried: Steeleville Cemetery, Steeleville, Randolph County, Illinois
Spouse:  Mary Residences: After the war he removed to Tennessee and in 1789 came to Ran­dolph County, Illinois. He was the founder of Steeleville.
Service: Captain; Virginia
Marker: His name is on a bronze marker on the grounds of Sparta High School, Sparta, Randolph County, placed by Fort Chartres Chapter DAR in 1934.
Sources: DAR, NSDAR, PI, W 
STEWART, ALEXANDER
Born: About 1750 in New Jersey
Died: After 1840
Buried: Wabash County, Illinois
Service: Soldier; New Jersey. He served in both the Continental Army and the Militia.
Pension: S36330 (NJ); Illinois pension roll, Wabash County, March 31, 1818, transferred from New Jersey. Pension Census, Wabash County, June 1, 1840, age 90.
Sources: DAR, PENSION, W 
STEWART, CHARLES
Died: 1833
Buried: Stewart family cemetery, Girard, Macoupin County, Illinois; Gov ernment Headstone
Sources: HR 
STEWART, WILLIAM 
Born: January 10, 1763 in Mecklenburg County, North Carolina 
Died: November 19, 1856 
Buried: Old Cemetery, Carmi, White County, Illinois; Private Headstone 
Spouse: Mary Newel 
Residences: He moved to Livingston County, Kentucky; in 1806 moved to Crit­tendon County, Kentucky, and from there to Carmi, White County, Illinois.
Service: Private; North Carolina. He enlisted in 1780 in Col. John Patton's North Carolina Regiment; and in Col. Armstrong's Regiment. He was wounded at the battle of Camden.
Pension: S14580: (NC) BLWT 13892-160-55
Marker: A government marker was dedicated by Wabash Chapter DAR, Carmi, September 21, 1964. His name is also on a monument in city park placed by the Chapter in 1936 honoring soldiers buried in White County.
Sources: DAR, HR, NSDAR, PI, PENSION, W 
STILES, RICHARD
Born: Probably Massachusetts 
Died: Brown County, Illinois 
Service: Soldier; Massachusetts. He enlisted November I, 1777 in Capt. John Burnham's Company, Col. Michael Jackson's Regiment, serving until Jan­uary 27,1778. 
Pension: His residence was Brown County when pension application was rejected (Act June 7, 1832) "Not having six months service ."
Sources: PENSION, W 
STILLWELL, JOHN 
Born: Probably in Virginia
Buried: Bellmont, Wabash County, Illinois Residences: He came to Illinois in 1820, settling in the town of Bellmont, Wa­bash County.
Service: Soldier; Virginia
Sources: W 
STOKES, EDMOND (Edward)
Buried: Oakland Cemetery, Meredosia, Morgan County, Illinois
Service: Soldier; Virginia: North Carolina
Pension: R3265 (NC); Residence Jacksonville, Morgan County, when pension claim suspended (Act June 7, 1832) "for further proof and specifications."
Marker: His name is on a plaque in front of the Morgan County Court House, Jacksonville, placed by the Reverend James Caldwell Chapter DAR, in 1914.
Sources: PENSION, W 
STOPPLEBEAN, JACOB
Born: In New York
Died: 1845
Buried: Hull Cemetery, Randolph County, Illinois
Spouse: Ann 
Residences: Coming to Illinois, he settled in Randolph County where he obtained land. He was considered to be very eccentric, always wearing his hat in the house. He claimed to be two years younger than General Washington. 
Service: Soldier; New York. He enlisted in the Albany County Militia, Eighth Regiment, commanded by Col. Robert Van Rensselaer; re-enlisted in the Levies under Col. Marinus Willett. 
Pension: W16739 (NY)
Marker: His name is on a bronze marker on the Sparta High School grounds, placed by Fort Chartres Chapter DAR, Sparta, Randolph County in 1934.
Sources: DAR, NSDAR, PENSION, W 
STOUT, JESSE 
Born: About 1755
Died: August 25, 1834
Buried: Morgan County, Illinois Spouse: Mary
Service: Private; New Jersey Continental troops
Pension: Illinois pension roll, Morgan County, November 5, 1833, age 80. W25108 (NJ) Mary Stout, widow, pensioned, Illinois, March 4, 1836. Margaret Stout, Pension Census, June 1, 1840, Scott County, age 83, residing with Nathaniel Stout, head of household. 
Sources: PI, PENSION, W 
STRAHAN, DAVID 
Born: March 1, 1755 in North Carolina
Died: 1838
Buried: Baptist Cemetery, reported both Clayton and Paloma, Adams County, Illinois Residences: Coming to Illinois, he reportedly settled in Adams County, but was a resident of Morgan County in 1833.
Service: Private; North Carolina Continental troops
Pension: S32538 (NC); Illinois pension roll. Morgan County, June6, 1833, age 79 
Sources: HR, PENSI ON, W 
STRANGE, JOHN
Born: About 1746 in Westchester County, New York
Died: 1840, age 94
Buried: Russell Cemetery, Knoxville, Knox Cou nty, Illinois
Service: Soldier; New York. He enlisted from Westchester County, New York in 1777 in the Third Regiment, Militia, commanded by Col. Pierre Van Courtland.
Pension: Pension Census, Knox County, Illinois, June1,1840, age 94
Marker: Rebecca Parke Chapter DAR, Galesburg, placed a bronze marker on his grave on October 21, 1956.
Sources: DAR, HR, NSDAR, PENSION, W 
STRAWN, ISAIAH 
Born: October 28, 1758 in Bucks County, Pennsylvania 
Died: August 12, 1843 
Buried: Florid Cemetery, Florid, Putnam County, Illinois 
Married: August 13, 1781 in Somerset County, Pennsylvania 
Spouse: Rachel Reed; died April 1843 
Residences: After the war he located until 1817 at Knox County, Ohio; in 1817 he moved to Putnam County, Illinois.
Service: Teamster; Pennsylvania. He enlisted when nineteen in opposition to his Quaker parents and served in the transportation line. On October 4, 1777 in the battle between Washington and the British under General Gray at Ger­mantown, he rushed into battle, seizing the musket of a fallen comrade and neighbor who was killed. He was wounded in this battle.
Marker: His grave was marked by Christopher Lobinger Chapter DAR in 1938­1939.
Sources: DAR, HR, NSDAR, PI, W 
STRINGFIELD, JOHN 
Born: February 13, 1762 in North Carolina
Died: January 5, 1822
Buried: Kern Cemetery, northeast of Springfield, Sangamon County, Illinois
Spouse:  Sarah Boydston 
Residences: He came to Sangamon County, Illinois in December 1821, but only lived nine days.
Service: Patriotic Service: North Carolina. He was in the battle of King's Moun­tain, October 7, 1780.
Marker: His name is on a bronze plaque in the south mall, Old State Capitol, Springfield, placed October 19, 1911 by Springfield Chapters DAR and SAR.
Sources: DAR, PI, W 
STRONG, DAVID
Born: July 6, 1743
Died: August 19, 1801
Buried: Old Fort Wilkinson (Ville), near Grand Chain, Pulaski County, Illinois
Spouse: Chloe Richmond
Service: Captain: Connecticut. Sergeant in Burrall's Continental Regiment, Jan­uary, 1776; taken prisoner at the Cedars, May 19, 1776, First Lieutenant, Fifth Connecticut, January 1, 1777; transferred to Second Connecticut, Jan­uary 1, 1781; Captain, May 2, 1781; retired January 1, 1783. He continued in the service until 1796, attaining the rank of Lieutenant-Colonel. 
Marker: Egyptian Chapter DAR, Cairo, placed a marker at the Old Fort on May 30, 1936.
Sources: DAR, PI, Heitman's Historical Register 
STRONG, JAMES
Born: 1743 in Ireland
Died: August 22, 1818
Buried: Rushville, Schuyler County, Illinois
Spouse: Margaret Brainard (1754-1816) 
Service: Artillery; Pennsylvania. He was Clerk of Capt. Isaac Coren's Company, Pennsylvania Artillery in 1779, Brigadier General Henry Knox, commander.
Sources: DAR, PI 
STUART, JAMES
Born: December 1,1762 in South Carolina
Died: October 1845
Buried: Fairfield Township, Wayne County, Illinois
Spouse: Susan Residences: After the war he removed to Kentucky, and from there to Wayne County, Illinois.
Service: Soldier; South Carolina. He enlisted in1779 for two months in Capt. H. McClure's Company, Col. E. Lacey's Regiment; 1780, three months in Capt. John McClure's Company; in Capt. John Steele's Company and Capt. Philip Walker's Company, serving five periods in Col. Lacey's Regiment. He was in the battles of Rocky Creek, Hanging Rock, King's Mountain, Fort Granby, siege of Ninety-six, Haddrell's Point, and Eutaw Springs. 
Pension: Susan W8762 (SC); Pension Census, Wayne County, Illinois, June 1, 1840, age 78 
Sources: PENSION, W 
STUFFLEBEAN, JOHN
Died: January 16, 1844
Buried: Randolph County, Illinois
Spouse: Elsie
Service: Soldier; New York: Pennsylvania
Pension: Elsee (Elsie) Rl02831/2 (NY), widow of John, applied for a pension after 1836. Residence Randolph County when pension claim rejected (Act July 7, 1838) "For proof of service and marriage."
Marker: His name is on a bronze marker on the grounds of Sparta High School, placed by Fort Chartres Chapter DAR in 1934.
Sources: DAR, PENSION, W 
STURGESS (Sturgis), AARON
Born: June 20, 1760 in Connecticut
Died: October 23, 1842
Buried: Bureau County, Illinois
Spouse: Sarah Morehouse
Service: Private; Connecticut. He enlisted May 1, 1778 for three years, serving as a musician in Capt. Ozias Marvin's Company, Brigadier General Gold Sellick Sillma's Fourth Regiment.
Pension: S31389 (Conn); Pension Census, Bureau County, Illinois, June 1, 1840, age 80.
Sources: PI, PENSION, W 
SUMMERS, JOSEPH
Born: 1749 in Kent County, Delaware
Buried: Scott County, Illinois 
Residences: He came to Illinois, settling in Morgan County. Scott County was formed from Morgan County in 1839. 
Service: Soldier; North Carolina. He enlisted from Guilford County, North Car­olina, serving for three months in Capt. Thomas Flack's Company, Col. James Martin's Regiment; six months in Capt. Edward Gwynn's Company; three months in Capt. Elliott's Company, Col. Henry Lee's Regiment. 
Pension: S31399 (NC) 
Sources: PENSIO N, W 
SUTTON, WILLIAM
Born: 1764 in Virginia 
Died: After 1833
Buried: Gallatin County, Illinois 
Service: Private; Virginia Continental. In 1781 he served for two months in Capt. John Jackson's Company, Col. Thomas Merriwether's Regiment; six months in Capt. Thomas Eaton's Company, Col. William Darke's Regiment. He was in the battle of Yorktown. 
Pension: Pension roll, Gallatin County, Illinois, April 23, 1833, age 79 
Sources: PENSION, W 
SWAIN, CORNELIUS
Buried: Brown County, Illinois
Service: Soldier; Virginia
Pension: RI0328 (Va). Residence Brown County, Illinois when pension claim was rejected (Act June 7, 1832) "not having proof of service." 
Sources: PENSION 
SWEET, REV. JONATHAN 
Born: April 29, 1761 in Rhode Island 
Died: December 20, 1837 
Buried: Holmes Cemetery (private), southeast of Jacksonville, Morgan County, Illinois
Spouse: Temperance Holdredge, died November 1842 
Residences: During the Revolution, the family lived at Hancock, Berkshire County, 
Massachusetts. Rev. Sweet came to Illinois from Burlington, Otsego County, New York in 1822 and was a resident of Morgan County by 1823. He was a Baptist minister. 
Service: Private; Massachusetts. He enlisted on January 10, 1781 for three years in Capt. John Trotter's Company, Col. Rufus Putman's Fifth Massachusetts Regiment. 
Sources: Family Records 
SWEET, WILBUR
Buried: Beaubien Cemetery, Sweet's Grove on Ogden Avenue, DuPage County, Illinois
Marker: His name is on a bronze tablet in the DuPage County Court House at Wheaton, Illinois.
Sources: DAR 

TANNER, A.
Born: About 1754
Died: After 1840 
Buried: Morgan County, Illinois 
Pension: Pension Census, June 1, 1840, resident of Morgan County, Illinois, age 86 
Sources: DAR, PENSION 
TANNER, SAMUEL
Died: Saline County, Illinois
Service: Soldier; Virginia Pension: RI0390 (Va). Applied for pension from Saline County, Illinois, but re­jected as not having six months service. Residence Frankfort, Franklin County when pension claim rejected (Act June 7, 1832) "Not six months service ." 
Sources: PENSION, W 
TAYLOR, GEORGE 
Born: October 12,1761, Albemarle County, Virginia 
Died : January 12, 1834 
Buried: Schuyler County, Illinois 
Residences: Had residence in Union County, South Carolina, Washington and Adair Counties, Kentucky before coming to Pope County, Illinois. He later moved to Schuyler County. 
Service: Soldier; Virginia: Pennsylvania. He enlisted in 1777 in Col. Daniel Brodhead's Regiment, commanded by General Lachlan Mcintosh. He served in Pennsylvania in 1778and helped guard prisoners at Saratoga . 
Pension: S32548 (Va) 
Sources: PENSION, W 
TAYLOR, JESSE 
Born: In England 
Buried: Near Walpole, Hamilton County, Illinois 
Spouse: Mary (Polly) 
Residences: Coming to Illinois, he settled near Olga, Hamilton County.
Service: Soldier; Virginia Pension: Mary (Polly) W1l598-BLWT 12705-160-55 (Va) 
Sources: PENSION, W 
TAYLOR, JOHN 
Born: 1754 
 Died: 1837
Buried: Phillips Town ship (along Penn-Central RR tracks), Carmi, White County, Illinois 
Children: Surviving children listed in 1850 pension record: William, Elizabeth, Sophia Baker Residences: In 1820 he lived in Humphrey County, Tennessee; moved to White County, Illinois in 1829.
Service: Private; Pennsylvania. He served in Col. Watt's Pennsylvania Regiment. Pension: S24392 (Pa)
Marker: His name is on a flat government marker in the old cemetery, Carmi, placed by Wabash Chapter DAR, September 21, 1964. His name is also on a monument in city park placed by the Chapter in 1936 honoring soldiers buried in White County. 
Sources: DAR, NSDAR, PENSION, W 
TAYLOR, RICHARD 
Born: 1755, Cumberland Co unty, Virginia
Died: After 1833
Buried: Pike County, Illinois 
Residences: He moved to Bedford County, Virginia, then to Smith County, Tennessee, and in 1832 to Pike County, Illinois.
Service: Private; Virginia Continental troops. He enlisted August 4, 1779 from Frederick County, Virginia, serving as an Ensign. Pension: S32549 (Va); Illinois pension roll, Pike County, October 2, 1833, age 78
Sources: PENSION, W 
TAYLOR, WILLIAM 
Died: August 25, 1821
Buried: Woodlawn Cemetery, near Clinton, DeWitt County, Illinois
Sources: DAR 
TEDRICK, MICHAEL 
Born: May 10, 1752 at sea
Died: February 10, 1834
Buried: Clinton County, Illinois
Service: Private; Cavalry North Carolina Militia. He enlisted from Anson County, North Carolina, serving three times with Capt. William Hay, Capt. Solomon Wood and Capt. Robert High, in Col. Francis Malmedy' s Regiment. 
Pension: S32242 (NC). He was on the Pension roll for Clinton County, February 25, 1834. (He died before first payment of pension.)
Sources: PENSION, W 
TEEL, LEVI
Died: Randolph County, Illinois Residences: He came to Randolph County, Illinois, settling on Nine-Mile Creek.
Service: Soldier; Virginia. He was a soldier with Col. George Rogers Clark. He was wounded in a battle with Indians.
Marker: His name is on a bronze marker on the Sparta High School grounds, placed by Fort Chartres Chapter DAR in 1934.
Sources: DAR, NSDAR, W 
TERRY, BARNET
Died: January 23,1851, age 90
Buried: Taylor Cemetery, south of Atwood, Douglas County, Illinois
Sources: DAR 
TERRY, STEPHEN
Buried: Lawrence County, Illinois
Service: Soldier; Virginia. He continued in the service after the close of the war.
Marker: His name is on a bronze tablet at the Lawrence County Court House, Lawrenceville, placed by Toussaint du Bois Chapter DAR in 1921.
Sources: DAR, W 
THADOWEN, JOHN
Born: About 1755
Died: After 1840
Buried: Probably Gallatin County, Illinois 
Pension: Pension Census, Gallatin County, Illinois, June 1, 1840, age 85
Sources: PENSION. W 
THAXTON, WILLIAM
Born: About 1767 in Caswell County , North Carolina
Died: April 5, 1850
Buried: Hickory Grove Cemetery, Wrights, Greene County. Illinois; also reported White Hall Cemetery. White Hall. Greene County; Government Headstone 
Spouse : Sarah Gravitt
Service: Private; Patriotic Service; North Carolina. He enlisted from Caswell County, North Carolina in May 1782 in the Tenth Regiment, Continental line. 
Pension: RI0483 (NC); Residence Carrollton, Greene County, when pension claim rejected (Act June 7, 1832) "No proof of service from the North Carolina records." 
Sources: HR, PI, PENSION, W 
THOMAS, JAMES
Born: 1750 in Maryland 
Died: November 2, 1833
Buried: Concord Cemetery, Menard County, Illinois 
Residences: After the war he lived in Indiana, and Kentucky, before moving to Sangamon County, Illinois. Menard County was formed from Sangamon in 1839. 
Service: Private; Pennsylvania Continental. He enlisted in 1776 and served for six years in Capt. David Hopkins' and Capt. David Plunkett's Companies, Col. Stephen Mayland's Fourth Continental Dragoons. He was wounded in the battles of Germantown and Savannah; was also in the battles of Brandy­wine, Monmouth, and the siege of Yorktown. 
Pension: S22549 (Pa): Illinois pension roll, Sangamon County, January 23, 1821, transferred from Kentucky. 
Sources: PENSION, W 
THOMAS, JOHN, JR. 
Born: May 26, 1751 
Died: July 29, 1819 
Buried: Shiloh Cemetery, Shiloh, St. Clair County, Illinois 
Spouse: Margaret McElwayne 
Residences: He came to Illinois, settling in St. Clair County. He was Treasurer of the Northwest Territory, and was commissioned Illinois State Treasurer on October 9, 1818. 
Service: Colonel: South Carolina. He served with his father, Colonel John Thom­as, Sr. When his father was taken prisoner in 1780, he succeeded him in the command of the Regiment (1780-1782) under General Marion. He was known as the "Hero of Cedar Springs." 
Sources: PI, W 
THOMPSON,ALEXANDER
Born: September 3, 1758 in Cumberland County, Pennsylvania
Died: September 25, 1840 
Buried: Newton Township Cemetery (or Kingsbury), Erie, Whiteside County, Illinois Spouse: Sarah Scroggs Residences: He removed to Indiana, and later to Newton Township, Whiteside County, Illinois. 
Service: Private; Pennsylvania. He enlisted in August 1776 and served four months in Capt. Alexander Laughlin's Company, Col. William Clark's Regiment; he also served in Col. Arthur Buchanan's Regiment. 
Pension: S32555 (Penn) BLWT 31325-160-55; Illinois Pension Census, Whiteside County, June 1, 1840, age 82 
Marker: Morrison Chapter DAR marked the grave of Alexander Thompson on September 13, 1967 "the only Revolutionary War Soldier buried in Whiteside County." 
Sources: DAR, HR, NSDAR, PI, PENSION, W 
THOMPSON, ARCHIBALD 
Born: September 3, 1760 in Abbeville District, South Carolina 
Died: September 5, 1833
Buried: Preston Cemetery, Preston, Randolph County, Illinois Married: 1783 Spouse: Mary McBride
Service: Soldier; South Carolina. He served in the South Carolina State Troops under Brigadier-General Thomas Sumter.
Marker: His name is on a bronze marker on the Sparta High School grounds, Randolph County, placed by Fort Chartres Chapter DAR in 1934. 
Sources: DAR, NSDAR, PI 
THOMPSON, BERNARD 
Died: September 24, 1842 
Buried: Old Morgan City Cemetery, Chapin, Morgan County, Illinois; Private Headstone 
Service: He served in the Revolutionary War and the War of 1812.
Marker: The grave has been marked by the Reverend James Caldwell Chapter DAR, Jacksonville. 
Sources: DAR, HR 
THOMPSON, JAMES 
Buried: Lawrence County, Illinois 
Service: Soldier; Virginia. He continued in service after the close of the Revolu­tionary War. 
Marker: His name is on a bronze tablet at the Lawrence County Court House, Lawrenceville, placed by Toussaint du Bois Chapter DAR in 1912. 
Sources : DAR, W 
THOMPSON, JOHN 
Born: -1758/9 in Botetourt County, Virginia
Died: March 27, 1843 
Buried: Thompson Cemetery, on Doyle Farm, Barrow; or White Hall, Greene Count y, Illinois; Government Headstone 
Spouse: Winney Brickey 
Service: Private; Virginia Continental troops. He enlisted in 1781for three months in Capt. Henry Pawling's Company, Col. William McClenahan's Regiment; in 1781 for six months in Capt. David May's Company, Col. Thomas Fleming's Regiment. He was serving under General Greene in the siege of Yorktown and surrender of Cornwallis. 
Pension: Pension list, Greene County, Illinois, January 9, 1834, age 75
Sources: PI, PENSION, W 
THOMPSON, JOHN 
Died: After 1833 
Buried: White County, Illinois 
Service: Soldier; Virginia 
Marker: He was pensioned in 1833 while residing in Indiana. 
Marker: His name is on a monument placed by Wabash Chapter DAR, Carmi, in 1936. 
Sources: DAR, W 
THOMPSON, THOMAS 
Born: January 19, 1829 
Buried: Wabash County, Illinois 
Service: Private; Virginia Continental troops 
Pension: Wabash County, Illinois pension roll, October 25, 1822 
Sources: PENSION, W 
THORNHILL, HENRY 
Born: 1757 in Virginia 
Died: 1835 
Buried: Goshen (not now listed as town), Madison County, Illinois 
Service: Private; Virginia Continental troops. He enlisted in Rockingham County, Virginia in Capt. Robert Craven's Company and served six months; in Capt. Daniel Ragan's Tenth Virginia Regiment for three months. He was discharged Yorktown, five days before the surrender of Lord Cornwallis. 
Pension: S32557 (Va); Madison County, Illinois pension roll, September 25, 1833, age76 
Marker: His name is on a bronze tablet at the Madison County Court House, Edwardsville, placed by Ninian Edwards Chapter DAR, September 16, 1912. 
Sources: DAR, PI, PENSION, W 
TINDALL, THOMAS 
Born: 1761 
Died: 1832 
Buried: Lusk Cemetery, Edwardsville, Madison County, Illinois 
Service: As a boy he helped feed the Army. 
Sources: NSDAR 
TIPSOWARD (Tipsonard, Tipsward), GRIFFIN
Born: In 1755 in Pennsylvania
Died: After 1835
Buried: Hutton Township, Coles County, Illinois 
Residences: After the war he resided in Kentucky, coming to Illinois in 1810. He was reported to have lived in Effingham and Fayette Counties. He was listed in the 1830 Census for Fayette County. Effingham County was created from Fayette County in 1831. He was in Coles County in 1835. 
Service: Soldier; North Carolina. He enlisted in 1775 from Rowan County, North Carolina in General Griffith Rutherford's Brigade with Col. McKatty, Major Horn and Capt. William Grimes. He was in the battle of Eutaw Springs under General Nathaniel Greene; the battle of King's Mountain under Col. Isaac Shelby; the battle of Charleston under Col. McKatty and Capt. John McGuire. 
Pension: RI0617 (NC); Residence, Coles County, Illinois in 1835. Residence Charleston, Coles County when pension claim rejected (Act June 7, 1832) "No proof of service - statement inconsistent." 
Sources: CR, PENSION, W 
TODD, BENJAMIN 
Buried: Ables Cemetery, Fayette County, Illinois; also reported: Ishell Cemetery, Hurricane Township, Montgomery County apparently on the Fayette-Mont­gomery County line . 
Residences: After the war he came to Illinois settling in Fayette County. 
Service: Sergeant; Maryland. He enlisted in December 1777, serving in the Fourth Maryland Regiment. 
Sources: W 
TOLIDAY, JOHN 
Born: In October 1763 near Poughkeepsie, New York 
Died: 1849 
Buried: Oak Grove Cemetery, Le Roy, Illinois 
Children: Lydia Clarke, William, Fleming, John, Jr. 
Residences: He came to Illinois in 1830 and settled in the "Old Town Timber'; near Le Roy, McLean County. 
Service: Private; New York. He served in Capt. Samuel Bowman's Company of the New York Rangers for four months; in Capt. James Harrison's Company, Col. Lewis Du Bois' Regiment for six months. 
Pension: RI0629 (NY): Residence Bloomington, McLean County when pension claim rejected (Act June 7, 1832) " Not on the Tons -no proof of service." 
Marker: His grave was marked by Letitia Green Stevenson Chapter DAR, Bloom­ington, in 1927. His name is on the bronze tablet in the Soldiers Monument, Miller Park, Bloomington. 
Sources: DAR, NSDAR, PENSION, W 
TONG, WILLIAM
Born: August 9, 1756 in Prince George County, Maryland 
Died: February 8, 1848 
Buried: Mt. Vernon, Jefferson County, Illinois 
Spouses: (1) Eleanor Ford 
	 (2) Elizabeth Thomas 
Service: Private; Maryland. He enlisted in 1776, serving one year in Capt. Resin Beals Company, Col. William Smallwood's Regiment. He enlisted in March 1777 in Capt. Thomas Dent's Company, Col. Luke Marbury's Regiment and was in several battles. 
Pension: In pension census of June 1, 1840, Callatin County, Illinois; listed as "Mr. Tong, residing with Thomas Tong," age 90. Elizabeth W1333 (Md) BLWf 26749-160-55 
Sources: HS, PI, PENSION 
TOULOUSE, JOSEPH
Buried: Near Kaskaskia, Randolph County, Illinois 
Service: He was a Frenchman who served in Capt. Francis Charleville's Com­pany with Col. Ceorge Rogers Clark.
Marker: His name is on a bronze marker on the Sparta High School grounds, Randolph County, placed by Fort Chartres Chapter DAR in 1934.
Sources: DAR, NSDAR, W 
TRACY, EBENEZER
Born: November 5, 1762 in Sharon, Connecticut
Died: September 18, 1835 
Buried: Ottawa, LaSalle County, Illinois 
Spouse: Electa Howard
Service: Private; Massachusetts. He enlisted from Lennox, Massachusetts and served in Capt. Hunt's Company, Col. Jackson's Regiment. Pension: S42536 (Mass). He applied for pension at age of 55 while resident of New York which was allowed for three years service.
Sources: PI , PENSION 
TROTTIER, FRANCIS
Died: Before 1783
Buried: Holy Family Church Cemetery, Cahokia, St. Clair County, Illinois
Service: Soldier; Illinois. He was a Frenchman who served under Col. George Rogers Clark. He gave financial aid to the Americans. He was made Com­mandant of Cahokia.
Marker: His grave is marked.
Sources: DAR, W 
TROUT, JOHN 
Buried: In White County, Illinois 
Marker: His name is inscribed on a bronze plaque on a granite monument in city park bearing the names of Revolutionary War soldiers buried in White County. The monument was dedicated by Wabash Chapter DAR, Carmi , in October 1936. 
Sources: DAR, County Report 
TURLEY, JAMES
Born: 1761 in Virginia 
Died: June 4, 1836 
Buried: Carlyle Cemetery, Sangamon County (now Logan County), Illinois 
Spouses: (1) Agnes Kirby , married in Virginia 
	 (2) Mrs. Sarah (Hoblett) Lucas, widow of Thomas Lucas Children: There were fourteen children; seven 
	       sons and seven daughters. 
Residences: From Virginia, he removed to Kentucky, carrying their two first born children in baskets, one swung on each side of a steady pack horse. He later settled in what is now Mt. Pulaski Township, Logan County. He was living near Lake Fork timber in the spring of 1820. He was an arbitrator among the Indians, who called him the "Big Chief." 
Service: Private; Virginia. He enlisted in 1777 in Capt. Thomas Pollard's Com­pany, Col. Rumsey's Regiment; he served four weeks in 1781 in Col. Lyon's Regiment. He was in the battle of Germantown. 
Pension: Illinois pension roll, Sangarnon County, March 16, 1833, age 72 
Marker: His name is on a bronze marker at the Old State Capitol, Springfield, placed by Springfield Chapters DAR and SAR, October 19, 1911. His name is on a bronze plaque on the Logan County Court House, Lincoln, placed by Abraham Lincoln Chapter DAR, June 27,1975. 
Sources: DAR, NSDAR, PI, PENSION, W 
TURNER, ANDREW
Born: April 5, 1762 in Rowan County, North Carolina
Died: August 8, 1842 
Buried: Rohrer Cemetery, near Waverly, Morgan County, Illinois (also known as Conlee Cemetery); Government Headstone 
Spouses: (1) Comfort Turner
	 (2) Mary Samples 
Service: Private; North Carolina Continental troops. He served from North Caro­lina from 1775 to 1783 under General Morgan and Colonel Phillips. He was in the battle of Cowpens. 
Pension: Illinois pension roll, Morgan County, Illinois, May 2, 1833, age 72. Mary W25505 (NC) BLWT 31730-160-55 
Marker: A bronze marker was placed October 11, 1959 on the grave in Conlee Cemetery, Morgan County, Illinois by Ninian Edwards Chapter DAR, Alton. His name is on a bronze plaque in front of the Morgan County Court House, Jacksonville, placed by the Reverend James Caldwell Chapter DAR in 1914. 
Sources : DAR, HR, NSDAR, PI, PENSION, W 
TURNER, ARCHIBALD 
Died: June 9, 1855
Buried: Riddle Hill Cemetery, Riddle Hill, Sangamon County, Illinois 
Sources: HR 
TURNER, JABEZ
Born: January 31, 1756 in New Haven, Connecticut
Died: December 12, 1846
Buried: Godfrey Cemetery, Godfrey, Madison County, Illinois; Government Head ­stone 
Spouse: Rebecca Wolcott 
Residences: After the war, he removed to Great Barrington, Massachusetts, and from there to Columbia County, New York. Coming to Illinois, he settled in Madison County. 
Service: Private; Connecticut. He enlisted in May 1775, serving six months in Capt. Samuel Wilmot's Company, Col. Andrew Ward's Regiment; six weeks in 1776 in Capt. Caleb Allen's Company, Col. jabez Thompson's Regiment; in December 1776 in Capt. Peter Johnson's Company; in April 1777 in Capt. Caleb Mix' Company; and in October 1777 in Capt. James Hillhouse's Com­pany.  He was in the expedition to St. John 's and Montreal; was serving when the British threatened New York, and retreated with his Regiment from Long Island; was present when the entrance of the British into New Haven was resisted. 
Pension: S31440 (Conn)
Marker: His grave has been marked. His name is on a bronze tablet in the Madison County Court House, Edwardsville, placed by Ninian Edwards Chapter DAR, Alton, September 16, 1912. 
Sources: DAR, HR, PI, PENSION, W 
TUTTLE, ANDREW
Born: About 1759 in Connecticut 
Died: Probably after 1840 
Buried: Wabash County, Illinois 
Spouse: Betsy; died September 27,1843 
Service: Soldier; Connecticut. He enlisted in September 1781 in Col. Samuel Crawford's Regiment from Milford, Connecticut. 
Pension: Betsey W22464(Conn). Widow Betsy Tuttleon Wabash County, Illinois pension roll March 4, 1836. Pension census, Wabash County, June 1, 1840, age 81.
Sources: PENSION , W 
TUTWILER, JOHN
Born: About 1757 in Virginia
Died: After 1840
Buried: Fairview Cemetery, Kansas, Edgar County, Illinois 
Residences. He came to Illinois and for a time resided in Coles County, but removed to Edgar County.
Service: Soldier; Virginia Pension: S36354 (Va): Illinois Pension Census, Coles County, June 1, 1840, age 83, residing with A. Wiley, head of family.
Sources: HR, PENS ION, W 
TYNER (Tiner, Tinor), JOSHUA
Born: July 21, 1767
Died: December 26, 1838
Buried: Franklin County, Illinois 
Spouse: Winifred Teasby 
Residences: He first came to Jackson County, Illinois, but removed to Franklin County .
Service: Private;Georgia Continental troops Pension: S32561 (Ga); Pension Roll, March 28, 1833, Jackson County, age 67. Re­moved to Franklin County, Illinois.
Sources: PI, PENSION, W 

ULMER, JACOB 
Born: July 3, 1758 in Winchester, Frederick County, Virginia 
Died: After 1835 
Buried: Fulton County, Illinois 
Spouse: Matilda Baumgartner 
Service: Private; South Carolina: Virginia. He served from Orangeberg District, South Carolina in Capt. Benjamin Brown's Company, and in Capt. Joseph Keller's Company. 
Pension: RI0799(Va). Residence Lewistown, Fulton County when pension claim rejected (Act June 7, 1832) "Not six months service." He was a resident of Fulton Cou \nty October 28, 1835, age 77. 
Sources: PI, PENSION, W 
UNDERWOOD, PHINEAS 
Born: 1763 in Vermont 
Died: 1840 
Buried: Walnut Ridge Cemetery, Virginia, Cass County, Illinois; Government Headstone Spouse: Sarah Burtis 
Residences: In 1826 he came to Illinois, settling near Virginia in Cass County. In 1912, through the efforts of the Grand Army Post of Virginia, he was re­moved from an abandoned cemetery to Walnut Ridge Cemetery. 
Service: Private; Vermont. He served in Fletcher's Vermont Militia and ten months in Capt. Josiah Fish's Company. 
Pension: S32562(Vt) 
Sources: HR, PI, PENSION, W 
UNSELL (Unzeli), FREDERICK 
Born: In Pennsylvania 
Died: After 1840 
Buried: Clark County, Illinois 
Spouse: Jane 
Service: Private; Pennsylvania
Pension: Pension roll, Clark County, February 11, 1834, age 70; Pension Census, Clark County,  June 1, 1840, age 90. Jane W22472 (Penn) 
Sources: PENSION, W 

VAN DEVENTER, BARNABAS 
Born : March 24, 1761 
Died: January 4, 1851 
Buried: Jacksonville, Morgan County, Illinois 
Spouse: Elizabeth Siler 
Service: Private: Virginia 
Pension: He was pensioned. 
Sources: NSDAR 
VAUGHN, FREDERICK 
Born: November 26, 1766 in Connecticut 
Died: August 10, 1845 
Buried: Spring Lake Cemetery, Aurora, Kane County, Illinois 
Spouses: (1) Lucy Blodgett 
	 (2) Catherine Cornet 
Service: Private: Connecticut. He enlisted under Lieutenant Colonel Samuel Can­field in the Connecticut Militia. 
Pension: S32565(Conn) Marker: Frederick Vaughn was buried in Root Street Cemetery but Aurora Chap­ter DAR had his remains removed to Spring Lake Cemetery and erected a granite and bronze marker in 1915. 
Sources: DAR, HR, PI, PENSION, W 
VEATCH (Veach, Veech), ELIAS 
Born: May 5, 1759 in Pendleton County, South Carolina 
Died: September 13, 1839 
Buried: Enfield Cemetery, Enfield, White County, Illinois (later moved to western Illinois) 
Spouse: Jean Brown, born in Maryland 
Service: Private: South Carolina. He served in the war from South Carolina. 
Pension: Jane or Jean RI0926 (SC), widow of Elias, resident of Nashville, Wash­ington County when pension claim was rejected (Act July 7, 1838) "Not on the rolls - no proof of service." Marker: His name is on a monument at Carmi, White County, placed by Wa­bash Chapter DAR in 1936. 
Sources: DAR, HR, PI, PENSION, W 
VEATCH, ISAAC 
Buried: Enfield Cemetery, Enfield, White County, Illinois 
Marker: His name is on a monument at Carmi, White County, placed by Wabash Chapter DAR in 1936. 
Sources: DAR 
VERDIN, JAMES 
Born: August 25, 1756, Newberry District, South Carolina 
Died: June 18, 1843 
Buried: Fayette County, Illinois 
Married: August 1784 in South Carolina 
Spouse: Sarah, died September 30, 1845 
Children: (11 children) Levi, James, Hugh Radford, Elizabeth, Charity, Jane, William (living in 1843) 
Residences: After the war he removed to Missouri, later to Fayette County, Illinois. 
Service: Soldier: South Carolina. He enlisted June 1, 1778 for three months in Capt. Francis Boykin's Company, Col. William Thompson's Regiment. He served fifteen months. 
Pension: Illinois Pension Census, June 1, 1840, age 88, residing in Western Divi­sion of Fayette County. Sarah W22485 (SC) widow, on pension roll Fayette County, June 19, 1843. 
Sources: PENSION, W, Letter 
VICK, JOSHUA 
Born: May 20, 1762 
Died: February 25, 1833 
Buried: Union County, Illinois 
Residences: After the war he removed to Tennessee and from there to Union County, Illinois. 
Service: Private: Virginia Continental troops 
Pension: S31452 (Va); Illinois pension roll, Union County, April 23, 1833, age 71 
Sources: PI, PENSION, W 
VICKERS, JAMES 
Died: May 20, 1846 
Buried: In Hamilton County, Illinois (formerly White County) 
Spouse: Elizabeth Service: North Carolina Continental line 
Pension: He applied for a pension in March 1832. Elizabeth RI0939 (NC) 
Marker: His name is on a monument at Carmi, White County, placed by Wa­bash Chapter DAR in 1936.  A flat government marker was placed in the Old Cemetery, Carmi, and dedicated by Wabash Chapter DAR, September 21, 1964. 
Sources: DAR, PENSION 
VINCENT (Vinson), WILLIAM 
Born: 1759 in Hampshire County, Virginia 
Died: 1836 
Buried: Rock Creek Cemetery, Waynesville, DeWitt County, Illinois 
Residences: Coming to Illinois, he settled at Long Point Timber, DeWitt County. 
Service: Soldier: North Carolina Pension: Residence, Bloomington, McLean County, Illinois when pension claim was rejected (Act June 7, 1832) "not having six months service." 
Marker: His grave has been marked by the DeWitt Clinton Chapter DAR. 
Sources: DAR, NSDAR, PENSION, W 
VINCINIER (Vinciner), GEORGE 
Born: About 1761 
Died: After 1833 Buried: Greene County, Illinois 
Service: Private: Kentucky Artillery Continental troops: Virginia. (During the Revolutionary War, Kentucky was always known as the "County of Kentucky, State of Virginia" hence the record shows he had Virginia service.) 
Pension: S32569 (Va); Pension roll, Greene County, June 6,1833, age 72 
Sources: PENSION, W 
WADE, OBADIAH 
Born: October 1761 
Died: 1838 
Buried: Copeland Wade Cemetery, Windsor, Shelby County, Illinois
WADE, OBADIAH 
Born: October 1761 
Died:1838 
Buried: Copeland Wade Cemetery, Windsor, Shelby County, Illinois 
Spouse: Abigail Residences: After the war he removed to Kentucky, where he was pensioned. He came to Illinois, settling in Shelby County.
Service: Private; North Carolina Continental. He enlisted March 1, 1779 in Capt. Liniers Company, Col. Johnson's Regiment, North Carolina troops. He was in the battle of Stono and was discharged July 18, 1779. 
Pension: Illinois pension roll, Shelby County, October 23, 1833, age 73. Residence Shelbyville, Shelby County, when pension claim was rejected (Act June 7, 1832) "Not six months service." 
Sources: DAR, HR, PI, PENSION, W 
WAGGONER, ISAAC 
Born: September 11, 1761 in South Carolina 
Died: August 24, 1838 
Buried: Whitefield Cemetery, Windsor, Shelby County, Illinois; also reported, same Cemetery, Bruce, Moultrie County 
Spouse: Emsey Holyfield
Service: Private; South Carolina Militia Pension: S32578(SC) 
Sources: NSDAR, PI, PENSION 
WALDEN, ELIJAH 
Born: May 15, 1766 
Died: May 16, 1856 
Buried: Walden Cemetery, DeWitt County, Illinois 
Sources: DAR, Letter 
WALKER, BENJAMIN 
Born: October, 1758 
Died: August 5, 1847 
Buried: Walker's Grove Cemetery, near Easton, Menard County, Illinois, His body remains in the above cemetery, but his tombstone was removed to Fullerton Park Cemetery, Havana, Mason County. 
Spouses:  (1) Ann Crawford, died about 1820in Indiana 
	  (2) Mary Criswell
Children: John, James, Robert, Henry, Samuel, Rosana (Hibbard), and Maria (Lynn) 
Residences: From Harrisburg, Virginia, the family moved to West Moreland County, and in about 1794 to North Bend, Ohio. They moved to Dearborn County, Indiana, and on September 29, 1829 he applied for a pension. He moved to Madison County, Indiana, for a short time and in 1840 or 1841 came to Walker's Grove, near Easton, Mason County, to live with his son James. Benjamin Walker was a great wild game hunter and one day while away from home an Indian appeared with a gun on his shoulder. Mrs. Walker, who was preparing game and wild fruit for her children was terrified, but appearing calm, seated the children and motioned for the Indian to be seated at the table. He sensed the presence of someone outside and mo­tioned for the door to be opened. Benjamin Walker entered, and was able to talk with the Indian. Following the meal, he gave his guest a pipe and they smoked together. He offered the Indian a blanket and he slept on the floor. The Indian stayed for several days while they hunted together, then disappeared as mysteriously as he had come. In Dearborn County, Indiana, Benjamin was a hunting companion of Daniel Boone and was a neighbor of General William Henry Harrison. 
Service: Private; Pennsylvania: Maryland. He served at various times in the Pennsylvania troops from June 1, 1776 to March 1779. He is also reported to have served as a Sergeant in the Maryland troops, in Capt. Monroe's Com­pany, Col. Hartley's Regiment. He served for two years. 
Pension: W14088; Pension Census, Menard County, Illinois, June 1, 1840, age 82, residing with James Walker, head of family. 
Marker: The Pierre Menard Chapter DAR, Petersburg, placed a marker at the Fullerton Park Cemetery on October 21, 1972. 
Sources: DAR, PI, PENSION, W 
WALKER, HENRY 
Born: About 1759 in Virginia 
Died: After 1833 
Buried: Fayette County, Illinois 
Residences: He was an early resident of Fayette County, Illinois. He was a Baptist minister and preached in Wheatland Township before the first church was built at Loogootee in 1851. He was a Justice of the Peace in Ramsey Township. 
Service: Private; Virginia Continental 
Pension: S31459 (Va). Pension claim rejected while resident of Fayette County, Illinois, by Resolution of Congress, May 29, 1830 "as not serving in Regiment on Continental Establishment." He was later pensioned for service as Spy in Virginia Continental Service, pension roll dated April 9, 1833, age 74. 
Sources: DAR, PENSION, W 
WARNER, GIDEON
Died: 1856 
Buried: Pleasant Hill Cemetery, Melton Township, DuPage County, Illinois; Private Headstone
Service: Soldier; Massachusetts
Marker: The grave has been marked by Perrin-Wheaton Chapter DAR, Wheaton. His name is on a bronze tablet at the DuPage County Court House, Wheaton. 
Sources: DAR 
WARNER, JOSEPH 
Born: March 15, 1738 in Anne Arundel County, Maryland
Died:  September 5, 1842, age 104 
Buried: Cherry Point Cemetery, Wenona, Marshall County, Illinois; Private Headstone 
Spouse: Ruth Troat 
Residences: In 1802 he removed to Ohio, and in 1838, at the age of 100 came to Illinois, riding the entire distance on horseback. He settled at Cherry Point, Marshall County, but longing for his old home in Ohio, when 102 years of age, he started back, walking across the unbroken prairie. After walking twelve miles, friends gave him shelter and persuaded him to return to Cherry Point. That winter, he started for church on foot in a sleet storm and coming to a creek, crossed by crawling over the icy poles. 
Service: Private; Virginia. He enlisted from Fairfax County, Virginia; served in George Washington's Virginia Militia. He was in the battle of Germantown in 1779. 
Marker: Streator Chapter DAR marked the grave on October 25, 1915. 
Sources: DAR, HR, NSDAR, PI, W 
WATSON, ABNER (Abram) 
Born: May 1754 in Virginia 
Died: May 1847 
Buried: Hopewell Cemetery, near New Windsor, Mercer County, Illinois. He was buried in the timber on his farm, but later removed to the cemetery. Government Headstone.
Residences: After the war he came to Illinois, settling in Shelby County, but re­moved to Mercer County. 
Service: Private; Virginia. He enlisted in August 1781 serving until the end of September in Capt. Robert Stubblefield's Company, Col. William Darke's Regiment. He was in the battle of York, Virginia. He served under Lafayette. 
Pension: R1199 (Va). Residence: Shelbyville, Shelby County, Illinois when pen­sion claim rejected (Act June 7, 1832) "not having six months service." 
Marker: A bronze marker was placed by William Dennison Chapter, Aledo, October 17, 1969. 
Sources: DAR, NSDAR, PENSION, W 
WATSON, MAJOR
Died: March 16, 1848 
Buried: Linn Hebron Cemetery, Hebron, McHenry County, Illinois 
Sources: HR 
WAYNE, BENJAMIN 
Born: About 1752 in Fairfax County, Virginia
Died: About 1832in Edgar County, Illinois 
Spouse: Mary
Service: Private; Virginia. He served in Capt. Thomas Hill's Company, Seventh Virginia Regiment, commanded by Col. Alexander McClennahan, Lt. Col. Holt Richardson, Lt. Col. William Holt. In November 1778 he transferred to the Fifth Regiment, commanded by Col. William Russell. 
Sources: DAR, PI 
WEEDMAN, JEREMIAH 
Born: October 20, 1763 
Died: January 28, 1856 
Buried: Camp Ground Cemetery, near Clinton, DeWitt County, Illinois 
Sources: DAR 
WELLS, CHESTER
Died: June 16, 1860 
Buried: Twelve Mile Grove Cemetery, near Rockford, Winnebago County, Illinois; Private Headstone
Service: Soldier; Connecticut. He served as Captain in Col. Belden's Regiment of Militia from Connecticut. 
Sources: DAR 
WELLS, LEWIS 
Born: 1750 in Greenfield County, South Carolina
Died: August 12, 1846 
Buried: McElvain Cemetery, DuQuoin, Perry County, Illinois 
Spouse: Elizabeth Bates 
Service: Private; South Carolina. He served in Capt. Thomas Brandon's South Carolina troops. On June 30, 1786 he was paid for supplies which he fur­nished during the war. 
Sources: DAR, HR, HS, NSDAR, PI 
WEST, BENJAMIN 
Born: 1743 in Maryland 
Died: Near Belleville, St. Clair County, Illinois, very aged 
Residences: He came to Illinois in 1818, settling near Belleville, St. Clair County. 
Service: Soldier; Virginia. He enlisted from Botetourt County, Virginia and was on the staff of General George Washington. 
Sources: W 
WEST, HEZEKIAH 
Born: November 7, 1763 in Maryland
Died: July 29, 1845 
Buried: West Eden Cemetery, near Cypress, Johnson County, Illinois; Govern­ment Headstone 
Spouse: Priscilla Osborn 
Residences: Hezekiah West was a member of the Illinois State Convention from John son County in 1818. 
Service: Private; South Carolina. He enlisted in Capt. Robert Frost's Company, Col. Richard Winns Regiment, serving four months; in 1782 in Capt. John McCoot's Company of Mounted Rangers, Col. David Hopkins' Regiment, serving for three months. His father served in the same Company and was killed in 1778. 
Pension: S34519 (SC); Pension roll, Johnson County, July 18, 1833, age 77. Pen­sion Census, Johnson County, June 1, 1840, age 76 (sic) 
Marker: His name is on a bronze marker on the lawn of the Johnson County Court House, Vienna, placed by Daniel Chapman Chapter DAR in 1918. The Chapter has also marked his grave. 
Sources: DJ\R, HR, PI, PENSION, W 
WEST, NATHANIEL 
Born:  May 6, 1750 in Baltimore County, Maryland 
Died: After 1840 
Buried: In Old Lost Cemetery. Clay County, Illinois 
Residence: He first settled in Lawrence County, Illinois; removed to Madison County and then to Clay County.
Service: Private; Virginia Continentals. He enlisted in 1778 from Cross Creek, Virginia in Col. William Crawford's Regiment. He served three months, during which time he assisted with the building of Fort McIntosh. He served in 1779 and 1780 in Capt. Matthew Richie's Company, Col. Crawford's Regiment. 
Pension: S32583 (Va); Illinois pension roll, Madison County, July 18,1833, age 83; Illinois Pension Census, Clay County, June 1,1840 , age 90
Marker:   The grave was marked by Vinsans Trace Chapter DAR, April 7, 1973. 
Sources: DAR, PENSION, W 
WEST, ROBERT 
Born: Bertie County, North Carolina 
Buried: Probably Gallatin County, Illinois
Service: Soldier; North Carolina: South Carolina. He served from Bertie County, North Carolina. In 1825 General Lafayette visited Gallatin County and was greeted by the old soldier who had served as his bodyguard. 
Pension: R1l347 (NC:SC). Applied for pension from Gallatin County, Illinois, residence Equality, suspended for "further proof and explanations." 
Sources: PENSION, W 
WHITAKER (Whittaker), ALEXANDER 
Buried: Probably Randolph County, Illinois
Service: Soldier; Maryland: North Carolina Pension: R1l392 (NC). He applied for a pension from Randolph County. His residence was Kaskaskia, Randolph County when pension claim rejected (Act June 7,1832) "For amended declaration."
Marker: His name is on a bronze marker on the Sparta High School grounds, Randolph County, placed by Fort Chartres Chapter DAR in 1934. 
Sources: DAR, NSDAR, PENSION, W 
WHITE, JOHN 
Born: 1739 in Massachusetts
Died: March 23, 1835 
Buried: Tallula Union Cemetery (old section), Tallula , Menard County, Illinois; Private Headstone 
Married: 1770 in Pennsylvania 
Spouse: Elizabeth Gordon 
Residences: During the Revolution, his residence was Canonsburg, Pennsylvania; he moved to Green County, Kentucky in 1790 and to Sangamon County (now Menard) in 1819. 
Service: Private; Pennsylvania. He enlisted in 1776 in Capt. Benjamin Loxley's Company; enlisted January 28,1782 from Canonsburg in Capt. William Fife's Company, Second Battalion of Washington County, Pennsylvania; he again enlisted on June 22, 1782. 
Pension: Sangamon County, Illinois pension roll, October 23, 1832, age 92.
Marker: His grave was marked by Pierre Menard Chapter DAR on September 1, 1940. His name is on a bronze plaque in the south mall, Old State Capitol, Springfield, placed by Springfield Chapters DAR and SAR, October 19, 1911. 
Sources: DAR, HR, HS, NSDAR, PI, PENSION, W 
WHITE, JOHN 
Born: May 7, 1761 in Stafford, Connecticut 
Died: September 2, 1834 
Buried: Wabash County, Illinois 
Spouse: Martha 
Service: Soldier; Connecticut. He enlisted in 1776 for three months; in 1777 for six months; in 1778 for nine months, in Capt. Abner Robinson's Company. 
Pension: Martha W16786 (Conn) 
Sources: PENSION, W 
WHITE, THOMAS 
Born: December 26, 1763 near Denten, Dorchester County (now Caroline), Mary­land
Died:  December 17, 1843 
Buried: White Cemetery, Colchester, McDonough County, Illinois; Government 
Headstone Married: September 11, 1788 
Spouse: Sarah Small, born July 10, 1769 
Residences: In 1806 he moved to Ross County, Ohio, settling on Dry Run Creek; in 1825 he moved to Fayette County, Ohio, and purchased 120 acres of land. He served as County Commissioner from 1812 to 1818. In 1840 he sold the farm and moved to McDonough County. 
Service: Private; Maryland. He enlisted in 1776 in Capt. Benjamin Spyker's Company, Col. John Quinby's Seventh Maryland Regiment and served for one year. He was in the battle of White Plains. 
Marker: The grave was marked by General Macomb Chapter DAR, Macomb, in 1936. 
Sources: DAR, HR, NSDAH, PI 
WHITE, THOMAS 
Buried: Mitchell Farm (or Sugg Cemetery), near Greenville, Bond County, Illinois
Service: Lieutenant: Pennsylvania. He served in Capt. William Armstrong's Com­pany, Col. Thomas Bull's Pennsylvania Battalion of the Flying Camp; was taken prisoner at Fort Washington on November 16, 1776. He escaped on June 27, 1777, and served in Col. William Montgomery's Regiment. 
Marker: His grave has been marked by Benjamin Mills Chapter DAR, Greenville. 
Sources: DAR, HR, NSDAR, W 
WHITEHEAD, ROBERT
Died: At advanced age, near Kaskaskia, Randolph County, Illinois
Service: He was a soldier with Col. George Rogers Clark.
Marker: His name is on a bronze marker on the grounds of Sparta High School, Sparta, Randolph County, placed by Fort Chartres Chapter DAR in 1934. 
Sources: DAR, NSDAR, W 
WHITESIDE, JOHN 
Born: In Tryon County, North Carolina 
Died: At Bellefontaine 
Buried: Near Waterloo, Monroe County, Illinois 
Residences: After the war he removed to Kentucky, and coming to Illinois in 1793, he settled at New Design, Monroe County. He later lived at White­side's Station and at Bellefontaine. Bellefontaine is French for "beautiful fountain," one of the earliest post-Revolutionary settlements in the American Bottom; now part of Waterloo. 
Service: Soldier; North Carolina. He was in the battle of King's Mountain. He was a brother of Capt. William Whiteside. 
Pension: He was pensioned while living in Kentucky. 
Sources: W 
WHITESIDE, WILLIAM, JR. 
Born: 1747 probably Tryon County, North Carolina
Died: 1815 
Buried: Whiteside's Station, Monroe County, Illinois Spouse: Mary Booth Residences: He came to Illinois in 1793 and erected a fort on the road from Cahokia to Kaskaskia, known as "Whiteside's Station."
Service: Private; North Carolina. He was in the battle of King's Mountain. He was a brother of John Whiteside. 
Sources: PI, Reynolds -1887 "Pioneer History of Illinois” pp. 185-190 
WIGGS, WILLIAM 
Born: 1758 in North Carolina
Died: 1835 in Johnson County, Illinois 
Service: Private; North Carolina Continental troops. He enlisted from Wayne County, North Carolina in 1775, serving for thirty-five days in Capt. William Fellows' Company; in 1779 in Capt. John Canada's Company, five months; in 1781 in Capt. Joseph Sessions Company, three months. He was in the battle of Guilford Court House. 
Pension: S32608 (NC); Pension roll, Johnson County, July 18, 1833, age 77
Marker:  His name is on a bronze marker on the Johnson County Court House grounds, Vienna, placed by Daniel Chapman Chapter DAR in 1918. 
Sources: DAR, PENSION, W 
WILCOXON, OTHA LYCURGUS
Died: 1863 
Buried: Brookside Cemetery, near Tonica, LaSalle County, Illinois 
Sources: DAR 
WILDERMAN, GEORGE 
Born: About 1750
Died: 1821 
Buried: Freeburg Township, St. Clair County, Illinois 
Spouse:  Patience Dorsey
Service: Patriotic Service; Maryland. He took the Oath of Fidelity on February 3, 1778 at Baltimore, Maryland. 
Sources: DAR, PI 
WILLARD, WILLIAM 
Born: 1755 in Loudoun County, Virginia
Died: November 9, 1846 
Buried: Atkinson Cemetery (in field, no access), near Colchester, McDonough County, Illinois Spouse: Jane Cook Residences: He first resided in Morgan County, Illinois, but removed to McDon­outh County.
Service: Private; Virginia Continental. He enlisted in July 1778 from Leesburg, Virginia and served seven months in Capt. James Haticans Company, Col. David Shepherd's Regiment, General Martin commanding; in 1781 in Capt. William Douglass' Company . He joined the French Army under Lafayette at Ruffan Ferry. He was at the surrender of Cornwallis. 
Pension: S31489 (Va); Illinois pension list, Morgan County, May 2, 1833; Illinois Pension Census, McDonough County, June I, 1840, age 89.
Marker: His name is on a plaque at the Morgan County Court House, Jackson­ville, placed by the Reverend James Caldwell Chapter DAR in 1914. 
Sources: DAR, NSDAR, PI, PENSION, W 
WILLIAMS, HENRYJ. 
Buried: Hamilton County, Illinois 
Residences: After the war he removed to West Tennessee and from there to Hamilton County, Illinois. 
Service: Soldier; Virginia. He served from Virginia and continued in the service in the United States Infantry after the close of the Revolutionary War. 
Sources: W 
WILLIAMS, ISIAH 
Born: In Vermont
Died: Boone County, Illinois 
Service: Private; New York. He served in the Third Regiment, Dutchess County, New York Militia 
Sources: W 
WILLIAMS, JOSEPH J. 
Born: About 1757, probably in Pennsylvania
Died: After 1834 
Buried: Jackson County, Illinois Residences: He came to Illinois after 1818 settling in Elk Township, Jackson County.
Service: Private and Sergeant; Pennsylvania Continental troops 
Pension: S40702 (Pa), Illinois pension roll, Jackson County, March 24, 1834, age 77 
Sources: PENSION, W 
WILLIAMS, THOMAS 
Buried: Probably Gallatin County, Illinois 
Residences: He came to Illinois while it was a part of the Northwest Territory, and settled in Jefferson County. He was in Gallatin County in 1819.
Service: Soldier; North Carolina. He served in both the Infantry and Cavalry. 
Pension: Illinois pension roll, Gallatin County, October 21, 1819; pension sus­pended March 4, 1820. 
Sources: PENSION, W 
WILLIAMSON, JACOB 
Born: January 5, 1759 in New Jersey
Died: June 1838 
Buried: On Deal farm, Danvers Township, McLean County, Illinois; Private Headstone 
Spouses:  (1) Hannah Ten Broeck 
	  (2) Martha (Baldwin) Suydam 
Children: Margaret, Sarah, Catherine B. (Goodenough), Martha (Lindsey), Jane (Jackson), Ester Ann (Perkins), Josephine (Mitchell), Jacob L. D., Solomon, Chapman 
Residences: He came to Illinois in about 1826, settling at Hittle's Grove Town­ship, Tazewell County and from there he removed to Danvers, McLean County. 
Service: Private; New Jersey Militia 
Pension: S33917(NJ) 
Marker: A bronze marker was placed in Stout's Grove Cemetery, Danvers, by Letitia Green Stevenson Chapter DAR, Bloomington in 1928. 
Sources: DAR, HR, PI, PENSION, W, "Revolutionary Soldiers Buried in McLean County" by Milo Custer 
WILLIAMSON, THOMAS 
Born: 1757 in Hampshire County, Virginia 
Died: After 1840 
Buried: Crab Apple Township, Iroquois County, Illinois 
Residences: After the war he removed to Ohio, and in 1825 to Indiana, and in 1832 came to Illinois, settling in Crab Apple Township, Iroquois County.  
Service: Soldier; Virginia. He served for one year in Capt. John Anderson's Company, Virginia troops . He mad e gun powder for the army . 
Pension: SI6580 (Va): Illinois Pension Census, Iroquois County, June 1, 1840, age 79 
Sources: PENSION, W 
WITT, JESSE 
Born: May 11, 1760 in Virginia 
Died: April 4, 1852 
Buried: Henderson Cemetery, Henderson, Knox County, Illinois; Private Head­stone 
Spouse: Ruth Whitborn May17, 1770 inVirginia; died October 18, 1839 
Children: John, Richard, Jesse, Thenis, Atalia, Ann, Stacey, Jonathan, Hesekiah 
Service: Private; North Carolina. He enlisted in January 1777for three years and served in a Company designated as Capt. George L. Lambert's, Capt. Syrus L. Robert’s, Col. William Davies', Fourteenth Virginia Regiment; transferred in May 1779 to Capt. John Overton's Company, same Regiment and was discharged September 6, 1779. 
Marker: Rebecca Parke Chapter DAR, Galesburg, marked the grave in 1952. 
Sources: DAR, NSDAR, PI 
WOOD, ABRAHAM 
Born: February 1, 1753in Frederick County, Maryland
Died: October 14, 1833 
Buried: Edgar County, Illinois 
Service: Private; North Carolina Continental.  He enlisted in July 1777 from North Carolina and served for six months in the Company of Capt. John Johnson and Capt. James Chapman, in Col. Matthew Lock's Regiment. 
Pension: S32611(NC); Pension roll, EdgarCounty,Illinois,August14,1833, age81 
Sources: PENSION, W 
WOOD, DANIEL 
Born: June 29, 1751 in New York 
Died: October 3, 1843 in New York 
Buried: Woodland Cemetery, Quincy, Adams County, Illinois 
Spouses:  (1) Mary Scofield 
	  (2) Catherine Crouse 
Children: Son, John Wood, born December 20, 1798, Moravia, Cayuga County, New York. Elected Lieutenant Governor of Illinois in 1856. Served as Gov­ernor upon the death of Governor Bissell, March 18, 1860. 
Service: Doctor; Captain; New York. He served as a surgeon in William Mal­com's Additional Regiment from March 1777 to April 1779, New York troops. 
Pension: Catharine WI8382 BLWT 2373 - issued April 28, 1791. 
Sources: HR, PI, PENSION, W 
WOOD, JOHN 
Born: November 29, 1754 in Maryland 
Died: November 11, 1832 
Buried: Newkirk cemetery near Friendsville, Wabash County, Illinois 
Spouses:  (1) Martha Ogle
	  (2) Rachel (Greathouse) Bratton 
Residences: He came from Kentucky in 1809 and made the first permanent settle­ment. He brought apple trees from which originated the "Wood Apple." He built Fort Wood (against Indians). 
Service: Ensign; Maryland. He served as Ensign in the Maryland troops; served under Wagon Master, John Pierce. Appointed Sergeant in Capt. Philip Smith's Company, Col. James Johnson's Battalion, Frederick County, Maryland. 
Pension: Rachel W2311 (Md), Widow, pensioned; BLWT 11171-160-55; Illinois pension roll, Wabash County, March 23, 1833, age 79 (Published 1835) 
Marker: A bronze marker was placed on July 4, 1929 by Mt. Carmel Chapter DAB. 
Sources: DAR, NSDAR, PI, PENSION, W 
WOODS, JOHN 
Born: 1752 in Savannah, Georgia 
Died: October 21, 1831 
Buried: Franklin Cemetery, Franklin, Morgan County, Illinois; Government Head­stone 
Spouse:  Leanah Milton 
Service: Private; Scout; Georgia. He was a Scout and was entrusted with mes­sages from General Marion to General Washington. He served as Paymaster to the First Battalion, Georgia troops, with the rank of Captain. He served throughout the war with his brothers William and Nathaniel. 
Pension: He received a grant of 250 acres of land. 
Marker: His name is on a plaque at the Morgan County Court House, Jackson­ville, placed by the Reverend James Caldwell Chapter DAB in 1914. 
Sources: DAR, HR, NSDAR, PI, PENSION, W 
WOODSIDE, SAMUEL CUNNINGHAM 
Born: 1737 in Camden District, South Carolina
Died:  July 22, 1819 
Buried: Randolph County, Illinois 
Spouse: (2) Jennet (Jane)  
Service: Private; Patriotic
Service: South Carolina. He served from Chester Dis­trict in Capt. Michael Dickson's Company, commanded by General Thomas Sumter. He was in the Snow Campaign, Florida Campaign, Battle of Fish Dam Ford, and Blackstock. He served until the surrender of Cornwallis. 
Pension: Jane B11833 (SC) widow of Samuel, residence Kaskaskia, Randolph County when pension claim rejected (Act July 4, 1836) "For proof of service and marriage." 
Marker: His name is on a bronze marker on the Sparta High School grounds, Randolph County, placed by Fort Chartres Chapter DAR in 1934. 
Sources: DAR, HR, HS, NSDAR, PI, PENSION 
WORKS, ASA 
Born: August 25, 1763 in Leicester, Massachusetts
Died: February 15, 1845 
Buried: Nauvoo, Hancock County, Illinois 
Spouse: Abigail Marks
Service: Private; Massachusetts 
Pension: R1l869 (Mass). Residence, Nauvoo, Hancock County when pension claim rejected (Act June 7, 1832) "Not on the rolls of Bigelow's Regiment - no proof of service."
Marker: The grave is marked. 
Sources: DAR, HR, PI, PENSION, W 
WORTHEN, ROBERT 
Died: November 15, 1854 
Buried: Public Highway Cemetery (East edge of Bluff), Murphysboro, Jackson County, Illinois; Private Headstone 
Sources: HR 
WRIGHT, JAMES, SR. 
Born: About 1753 probably in Virginia
Died: September 5, 1845 
Buried: Franklin Cemetery, Franklin, Morgan County, Illinois; Government Head­sto ne Spouse: Frances Finney
Service: Captain; Virginia Continental line. He served as Second Lieutenant, Eleventh Virginia, July 31, 1776; First Lieutenant March 25, 1777; regiment designated Seventh Virginia, commanded by Col. John Morgan, September 14, 1778; Captain, July 2, 1779; taken prisoner at Charleston May 12, 1780, and was a prisoner on parole to close of war. 
Pension: Morgan County, Illinois pension list, May 2, 1833, age 71; Pension Census, Morgan County, June 1, 1840, age 84
Marker: His name is on a plaque at the Morgan County Court House, Jackson­ville, placed by the Reverend James Caldwell Chapter DAR in 1914. Sources: DAR, HR, NSDAR, PI, PENSION, W 
WRIGHT, JOSEPH 
Born: 1760 in Mecklenburg County, Virginia
Died: About 1835 
Buried: Monroe County, Illinois 
Spouse: Nancy Bryant 
Service: Private; Virginia Continental. He enlisted in March 1780 for three months in Capt. John Thompson 's Company, Col. John Glenn's Regiment; enlisted in 1781 for three months in Capt. Paul Waddletous Company, Col. John Glenn's Regiment. 
Pension: S31503 (Va); Monroe County, Illinois pension roll, May 2, 1833, age 74 
Sources: PI, PENSION, W 

YANCEY, AUSTIN 
Born: August 29, 1752 
Died: After April 5, 1836 
Buried: Fulton County, Illinois 
Spouse: Sarah Garrison 
Service: Private; South Carolina: North Carolina. He served four months in 1776 from Rutherford County, North Carolina in Col. William Graham's Regiment; in 1781 from Ninety-Sixth District, South Carolina in Capt. Holloway's Com­pany, Col. Hinders' Regiment, commanded by General Greene. 
Pension: R1l921 (NC:SC). He applied for a pension from Franklin County, Illinois on October 3, 1833, age 81 years, 1month, 10 days. "Arthur Yancey," residence Lewistown, Fulton County, when pension claim rejected (Act June 7,1832) "Not having six months service," (April 5, 1836).
Sources: PI, PENSION 
YOUNG, SAMUEL 
Born: May 7, 1762 in Cumberland County, Pennsylvania 
Died: February 28, 1841 
Buried: Young Cemetery, Salem Township, Marion County, Illinois 
Residences: After the war he removed to Rowan County, North Carolina; then to Rutherford County; then Spartanburg County, South Carolina; then to Frank­lin County, Georgia; then to Sumner County, Tennessee; then to Logan County, Kentucky; then to Indiana; then to Gallatin County, Illinois, and finally to Marion County, August 1813, the first settler in that County.
Service: Private; Virginia and Pennsylvania. He enlisted May 7, 1778 from North­umberland County, Pennsylvania for three months in Capt. Samuel Young's Company, Col. Dougherty's Regiment; in 1781 for three months in Capt. James Montgomery's Company, Col. William Campbell's Regiment, Virginia troops; and again in 1782 for three months with the same officers. 
Pension: S32621(Penn. Va); Pension Census, Marion County, Illinois, June 1, 1840, age 80, residing with Matthew Young, head of family. 
Marker: A marker was placed by Prairie State Chapter DAR, Centralia in 1941. 
Sources: DAR, NSDAR, PI, PENSION, W 

ZOLL, JACOB 
Born: February 16, 1743 
Died: November 16, 1819
Buried: Franklin County, Illinois 
Spouse: Elizabeth Markle 
Service: Patriotic Service; Pennsylvania 
ZOLL, JACOB, SR. (Christopher)
Born: November 30, 1754 in Maryland
Died: February 14,1841
Buried: Zoll Cemetery, near Fairview, Fulton County, Illinois
Children: Henry
Service: Soldier; Maryland. He served from January to March 1776 in the Frederick County, Maryland Militia in Capt. Ross Key's Company, Col. Norman Bruce's Regiment with Major James Shields.
Pension: R11922 (Md); Residence Township 5, Fulton County, Illinois, Pension Census of June 1, 1840, age 88, residing with Henry 2011, head of family.
Sources: PENSION, W, County Record
